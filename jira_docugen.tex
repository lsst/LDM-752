%%%%%%%%%%%%%%%%%%%%%%%%%%%%%%%%%%%%%%%%%%%%%%%%%%%%%%%%%%%%%%%%%%%%%%%%%%%%%%%%%%%%%%%%%%%%%%%
% generated from JIRA project LVV
% using template at /usr/local/lib/python3.7/site-packages/docsteady/templates/dm-ve.latex.jinja2.
% using docsteady version 1.2rc24
% Please do not edit -- update information in Jira instead
%%%%%%%%%%%%%%%%%%%%%%%%%%%%%%%%%%%%%%%%%%%%%%%%%%%%%%%%%%%%%%%%%%%%%%%%%%%%%%%%%%%%%%%%%%%%%%%

\section{ DM Science Verification Elements }
\label{sec:ves}

The following is the list of verification elements defined in the context of the Science component of the DM subsystem.


\subsection{[LVV-3] DMS-REQ-0002-V-01: Transient Alert Distribution }\label{lvv-3}

\begin{longtable}{cccc}
\hline
\textbf{Jira Link} & \textbf{Assignee} & \textbf{Status} & \textbf{Test Cases}\\ \hline
\href{https://jira.lsstcorp.org/browse/LVV-3}{LVV-3} &
Eric Bellm & Not Covered &
\begin{tabular}{c}
LVV-T101 \\
LVV-T217 \\
\end{tabular}
\\
\hline
\end{longtable}

\textbf{Verification Element Description:} \\
With precursor data, do L1 processing and issue alerts to a
standards-based broker.

{\footnotesize
\begin{longtable}{p{2.5cm}p{13.5cm}}
\hline
\multicolumn{2}{c}{\textbf{Requirement Details}}\\ \hline
Requirement ID & DMS-REQ-0002 \\ \cdashline{1-2}
Requirement Description &
\begin{minipage}[]{13cm}
\textbf{Specification:} Identified transient events shall be made
available to end-users in the form of alerts, which shall be published
to community alert distribution networks using community-standard
protocols, to be determined during the LSST construction phase as
community standards evolve.
\end{minipage}
\\ \cdashline{1-2}
Requirement Priority & 1b \\ \cdashline{1-2}
Upper Level Requirement &
\begin{tabular}{cl}
OSS-REQ-0184 & Transient Alert Publication \\
OSS-REQ-0127 & Level 1 Data Product Availability \\
\end{tabular}
\\ \hline
\end{longtable}
}


\subsubsection{Test Cases Summary}
\begin{longtable}{p{3cm}p{2.5cm}p{2.5cm}p{3cm}p{4cm}}
\toprule
\href{https://jira.lsstcorp.org/secure/Tests.jspa\#/testCase/LVV-T101}{LVV-T101} & \multicolumn{4}{p{12cm}}{ Verify implementation of Transient Alert Distribution } \\ \hline
\textbf{Owner} & \textbf{Status} & \textbf{Version} & \textbf{Critical Event} & \textbf{Verification Type} \\ \hline
Kian-Tat Lim & Draft & 1 & false & Test \\ \hline
\end{longtable}
{\scriptsize
\textbf{Objective:}\\
Precursor or simulated data, execute AP, observe distribution to
simulated clients using standard protocols
}
\begin{longtable}{p{3cm}p{2.5cm}p{2.5cm}p{3cm}p{4cm}}
\toprule
\href{https://jira.lsstcorp.org/secure/Tests.jspa\#/testCase/LVV-T217}{LVV-T217} & \multicolumn{4}{p{12cm}}{ Full Stream Alert Distribution } \\ \hline
\textbf{Owner} & \textbf{Status} & \textbf{Version} & \textbf{Critical Event} & \textbf{Verification Type} \\ \hline
Eric Bellm & Approved & 1 & false & Test \\ \hline
\end{longtable}
{\scriptsize
\textbf{Objective:}\\
This test will check that the full stream of LSST alerts can be
distributed to end users.\\[2\baselineskip]Specifically, this will
demonstrate that:

\begin{itemize}
\tightlist
\item
  Serialized alert packets can be loaded into the alert distribution
  system at LSST-relevant scales (10,000 alerts every 39 seconds);
\item
  Alert packets can be retrieved from the queue system at LSST-relevant
  scales.
\end{itemize}
}
  
 \newpage 
\subsection{[LVV-6] DMS-REQ-0009-V-01: Simulated Data }\label{lvv-6}

\begin{longtable}{cccc}
\hline
\textbf{Jira Link} & \textbf{Assignee} & \textbf{Status} & \textbf{Test Cases}\\ \hline
\href{https://jira.lsstcorp.org/browse/LVV-6}{LVV-6} &
Jim Bosch & Not Covered &
\begin{tabular}{c}
LVV-T125 \\
\end{tabular}
\\
\hline
\end{longtable}

\textbf{Verification Element Description:} \\
Show that artificial sources can be injected into data streams and
recovered. Show that processing of simulated data recovers sources to
the completeness required.

{\footnotesize
\begin{longtable}{p{2.5cm}p{13.5cm}}
\hline
\multicolumn{2}{c}{\textbf{Requirement Details}}\\ \hline
Requirement ID & DMS-REQ-0009 \\ \cdashline{1-2}
Requirement Description &
\begin{minipage}[]{13cm}
\textbf{Specification:} The DMS shall provide the ability to inject
artificial or simulated data into data products to assess the functional
and temporal performance of the production processing software.
\end{minipage}
\\ \cdashline{1-2}
Requirement Priority & 1b \\ \cdashline{1-2}
Upper Level Requirement &
\begin{tabular}{cl}
OSS-REQ-0353 & Difference Source Spuriousness Threshold - Transients \\
DMS-REQ-0007 & Pipeline Infrastructure \\
OSS-REQ-0351 & Difference Source Spurious Probability Metric \\
OSS-REQ-0354 & Difference Source Spuriousness Threshold - MOPS \\
\end{tabular}
\\ \hline
\end{longtable}
}


\subsubsection{Test Cases Summary}
\begin{longtable}{p{3cm}p{2.5cm}p{2.5cm}p{3cm}p{4cm}}
\toprule
\href{https://jira.lsstcorp.org/secure/Tests.jspa\#/testCase/LVV-T125}{LVV-T125} & \multicolumn{4}{p{12cm}}{ Verify implementation of Simulated Data } \\ \hline
\textbf{Owner} & \textbf{Status} & \textbf{Version} & \textbf{Critical Event} & \textbf{Verification Type} \\ \hline
Robert Lupton & Approved & 1 & false & Test \\ \hline
\end{longtable}
{\scriptsize
\textbf{Objective:}\\
Verify that the DMS can inject simulated data into data products for
testing.
}
  
 \newpage 
\subsection{[LVV-7] DMS-REQ-0010-V-01: Difference Exposures }\label{lvv-7}

\begin{longtable}{cccc}
\hline
\textbf{Jira Link} & \textbf{Assignee} & \textbf{Status} & \textbf{Test Cases}\\ \hline
\href{https://jira.lsstcorp.org/browse/LVV-7}{LVV-7} &
Eric Bellm & Not Covered &
\begin{tabular}{c}
LVV-T18 \\
LVV-T20 \\
LVV-T36 \\
\end{tabular}
\\
\hline
\end{longtable}

\textbf{Verification Element Description:} \\
No requirement for quality of difference processing. No requirement this
is tested as part of a full L1 end to end test. Just requires a
processed image and a template: demonstrate that a difference exposure
is created.

{\footnotesize
\begin{longtable}{p{2.5cm}p{13.5cm}}
\hline
\multicolumn{2}{c}{\textbf{Requirement Details}}\\ \hline
Requirement ID & DMS-REQ-0010 \\ \cdashline{1-2}
Requirement Description &
\begin{minipage}[]{13cm}
\textbf{Specification:} The DMS shall create a Difference Exposure from
each Processed Visit Image by subtracting a re-projected, scaled,
PSF-matched Template Image in the same passband.
\end{minipage}
\\ \cdashline{1-2}
Requirement Discussion &
\begin{minipage}[]{13cm}
\textbf{Discussion:} Difference Exposures are not archived, and are
retained for only a limited time to facilitate Alert processing. They
can be re-generated for users on-demand.
\end{minipage}
\\ \cdashline{1-2}
Requirement Priority & 1b \\ \cdashline{1-2}
Upper Level Requirement &
\begin{tabular}{cl}
DMS-REQ-0011 & Produce Difference Sources \\
DMS-REQ-0033 & Provide Source Detection Software \\
OSS-REQ-0129 & Exposures (Level 1) \\
\end{tabular}
\\ \hline
\end{longtable}
}


\subsubsection{Test Cases Summary}
\begin{longtable}{p{3cm}p{2.5cm}p{2.5cm}p{3cm}p{4cm}}
\toprule
\href{https://jira.lsstcorp.org/secure/Tests.jspa\#/testCase/LVV-T18}{LVV-T18} & \multicolumn{4}{p{12cm}}{ AG-00-05: Alert Generation Produces Required Data Products } \\ \hline
\textbf{Owner} & \textbf{Status} & \textbf{Version} & \textbf{Critical Event} & \textbf{Verification Type} \\ \hline
Eric Bellm & Deprecated & 1 & false & Test \\ \hline
\end{longtable}
{\scriptsize
\textbf{Objective:}\\
This test will check that the basic data products produced by Alert
Generation are generated by execution of the science payload.\\
These products will include:

\begin{itemize}
\tightlist
\item
  Processed visit images (PVIs; DMS-REQ-0069);
\item
  Difference Exposures (DMS-REQ-0010);
\item
  DIASource catalogs (DMS-REQ-0269);
\item
  DIAObject catalogs (DMS-REQ-0271);
\end{itemize}
}
\begin{longtable}{p{3cm}p{2.5cm}p{2.5cm}p{3cm}p{4cm}}
\toprule
\href{https://jira.lsstcorp.org/secure/Tests.jspa\#/testCase/LVV-T20}{LVV-T20} & \multicolumn{4}{p{12cm}}{ AG-00-15: Scientific Verification of Difference Images } \\ \hline
\textbf{Owner} & \textbf{Status} & \textbf{Version} & \textbf{Critical Event} & \textbf{Verification Type} \\ \hline
Eric Bellm & Deprecated & 1 & false & Test \\ \hline
\end{longtable}
{\scriptsize
\textbf{Objective:}\\
This test will check that the difference images delivered by the Alert
Generation science pay- load meet the requirements laid down by
\citeds{LSE-61}.\\
Specifically, this will demonstrate that:

\begin{itemize}
\tightlist
\item
  Difference images have been generated and persisted during payload
  execution;
\item
  Each difference image includes information about the identity of the
  input exposures, and metadata such as a representation of the PSF
  matching kernel (DMS-REQ-0074);
\item
  Masks are correctly propagated from the input images.
\end{itemize}

This test does not include quantitative targets for the science quality
criteria.
}
\begin{longtable}{p{3cm}p{2.5cm}p{2.5cm}p{3cm}p{4cm}}
\toprule
\href{https://jira.lsstcorp.org/secure/Tests.jspa\#/testCase/LVV-T36}{LVV-T36} & \multicolumn{4}{p{12cm}}{ Verify implementation of Difference Exposures } \\ \hline
\textbf{Owner} & \textbf{Status} & \textbf{Version} & \textbf{Critical Event} & \textbf{Verification Type} \\ \hline
Eric Bellm & Draft & 1 & false & Test \\ \hline
\end{longtable}
{\scriptsize
\textbf{Objective:}\\
Verify successful creation of a\\
1. PSF-matched template image for a given Processed Visit Image\\
2. Difference Exposure from each Processed Visit Image
}
  
 \newpage 
\subsection{[LVV-8] DMS-REQ-0018-V-01: Raw Science Image Data Acquisition }\label{lvv-8}

\begin{longtable}{cccc}
\hline
\textbf{Jira Link} & \textbf{Assignee} & \textbf{Status} & \textbf{Test Cases}\\ \hline
\href{https://jira.lsstcorp.org/browse/LVV-8}{LVV-8} &
Robert Gruendl & Not Covered &
\begin{tabular}{c}
LVV-T29 \\
LVV-T283 \\
LVV-T284 \\
LVV-T1549 \\
LVV-T1550 \\
LVV-T1556 \\
LVV-T1934 \\
\end{tabular}
\\
\hline
\end{longtable}

\textbf{Verification Element Description:} \\
This requires a DAQ and OCS. We test in all known operating,
calibration, and engineering modes. We verify that the pixels are the
same as provided by the DAQ client library. We do not take
responsibility for corruption between the DAQ and the client. Set up lab
to simulate the summit.

{\footnotesize
\begin{longtable}{p{2.5cm}p{13.5cm}}
\hline
\multicolumn{2}{c}{\textbf{Requirement Details}}\\ \hline
Requirement ID & DMS-REQ-0018 \\ \cdashline{1-2}
Requirement Description &
\begin{minipage}[]{13cm}
\textbf{Specification:} The DMS shall acquire raw Exposure data from the
Camera science sensors during normal operations, calibration data
collection, and in any other required engineering modes.
\end{minipage}
\\ \cdashline{1-2}
Requirement Discussion &
\begin{minipage}[]{13cm}
\textbf{Discussion:} The manner of data acquisition is a matter for the
DM-Camera ICDs, \citeds{LSE-69} and \citeds{LSE-68}, in this area.
\end{minipage}
\\ \cdashline{1-2}
Requirement Priority & 1a \\ \cdashline{1-2}
Upper Level Requirement &
\begin{tabular}{cl}
OSS-REQ-0114 & Acquisition of Science Sensor data \\
\end{tabular}
\\ \hline
\end{longtable}
}


\subsubsection{Test Cases Summary}
\begin{longtable}{p{3cm}p{2.5cm}p{2.5cm}p{3cm}p{4cm}}
\toprule
\href{https://jira.lsstcorp.org/secure/Tests.jspa\#/testCase/LVV-T29}{LVV-T29} & \multicolumn{4}{p{12cm}}{ Verify implementation of Raw Science Image Data Acquisition } \\ \hline
\textbf{Owner} & \textbf{Status} & \textbf{Version} & \textbf{Critical Event} & \textbf{Verification Type} \\ \hline
Kian-Tat Lim & Defined & 1 & false & Test \\ \hline
\end{longtable}
{\scriptsize
\textbf{Objective:}\\
Verify acquisition of raw data from L1 Test Stand DAQ while simulating
all modes
}
\begin{longtable}{p{3cm}p{2.5cm}p{2.5cm}p{3cm}p{4cm}}
\toprule
\href{https://jira.lsstcorp.org/secure/Tests.jspa\#/testCase/LVV-T283}{LVV-T283} & \multicolumn{4}{p{12cm}}{ RAS-00-00: Writing well-formed raw image } \\ \hline
\textbf{Owner} & \textbf{Status} & \textbf{Version} & \textbf{Critical Event} & \textbf{Verification Type} \\ \hline
Michelle Butler & Approved & 1 & false & Test \\ \hline
\end{longtable}
{\scriptsize
\textbf{Objective:}\\
This test will check:\\

\begin{itemize}
\tightlist
\item
  The successful integration of the Pathfinder components with the DM
  Header Service and the Level 1 Archiver;
\item
  That the raw images are well-formed and meet specifications in
  change-controlled documents \citeds{LSE-61};
\end{itemize}

~This Test Case shall be repeated for each of the different cameras
(ATScam, LSSTCam) and sensors (Science, Wavefront, and Guider)
combination.
}
\begin{longtable}{p{3cm}p{2.5cm}p{2.5cm}p{3cm}p{4cm}}
\toprule
\href{https://jira.lsstcorp.org/secure/Tests.jspa\#/testCase/LVV-T284}{LVV-T284} & \multicolumn{4}{p{12cm}}{ RAS-00-05: (LDM-503-8b) Writing data from CCOB to the DBB for further
data processing } \\ \hline
\textbf{Owner} & \textbf{Status} & \textbf{Version} & \textbf{Critical Event} & \textbf{Verification Type} \\ \hline
Michelle Butler & Draft & 1 & false & Test \\ \hline
\end{longtable}
{\scriptsize
\textbf{Objective:}\\
This test will check:

\begin{itemize}
\tightlist
\item
  The successful integration of the DAQ archiver components with the
  CCOB
\item
  That the file can then be ingested into the DBB and be retrieved for
  further analysis
\end{itemize}
}
\begin{longtable}{p{3cm}p{2.5cm}p{2.5cm}p{3cm}p{4cm}}
\toprule
\href{https://jira.lsstcorp.org/secure/Tests.jspa\#/testCase/LVV-T1549}{LVV-T1549} & \multicolumn{4}{p{12cm}}{ LDM-503-6 Comcam verification readiness } \\ \hline
\textbf{Owner} & \textbf{Status} & \textbf{Version} & \textbf{Critical Event} & \textbf{Verification Type} \\ \hline
Michelle Butler & Approved & 1 & false & Demonstration \\ \hline
\end{longtable}
{\scriptsize
\textbf{Objective:}\\
Verify that ComCam has all the services running and verified working for
retrieving an image from the ComCam DAQ and store it on file systems at
the LDF for viewing by RSP. ~
}
\begin{longtable}{p{3cm}p{2.5cm}p{2.5cm}p{3cm}p{4cm}}
\toprule
\href{https://jira.lsstcorp.org/secure/Tests.jspa\#/testCase/LVV-T1550}{LVV-T1550} & \multicolumn{4}{p{12cm}}{ LDM-503-10 DAQ Validation } \\ \hline
\textbf{Owner} & \textbf{Status} & \textbf{Version} & \textbf{Critical Event} & \textbf{Verification Type} \\ \hline
Michelle Butler & Approved & 1 & false & Demonstration \\ \hline
\end{longtable}
{\scriptsize
\textbf{Objective:}\\
Verify that the DAQ can talk to test machines at the BDC through the
DWDM network.~
}
\begin{longtable}{p{3cm}p{2.5cm}p{2.5cm}p{3cm}p{4cm}}
\toprule
\href{https://jira.lsstcorp.org/secure/Tests.jspa\#/testCase/LVV-T1556}{LVV-T1556} & \multicolumn{4}{p{12cm}}{ LDM-503-10B Large Scale CCOB Data Access } \\ \hline
\textbf{Owner} & \textbf{Status} & \textbf{Version} & \textbf{Critical Event} & \textbf{Verification Type} \\ \hline
Michelle Butler & Draft & 1 & false & Demonstration \\ \hline
\end{longtable}
{\scriptsize
\textbf{Objective:}\\
Demonstrate the ability to transfer data from the SLAC test stand or
CCOB with 21 rafts from SLAC and ingested at NCSA and make available
through an instance of the RSP
}
\begin{longtable}{p{3cm}p{2.5cm}p{2.5cm}p{3cm}p{4cm}}
\toprule
\href{https://jira.lsstcorp.org/secure/Tests.jspa\#/testCase/LVV-T1934}{LVV-T1934} & \multicolumn{4}{p{12cm}}{ ComCam Data Transfer and Ingestion } \\ \hline
\textbf{Owner} & \textbf{Status} & \textbf{Version} & \textbf{Critical Event} & \textbf{Verification Type} \\ \hline
Robert Gruendl & Draft & 1 & true & Inspection \\ \hline
\end{longtable}
{\scriptsize
\textbf{Objective:}\\
Verify that ComCam Archiver data taken are transferred to NCSA Data
BackBone endpoint and Ingested
}
  
 \newpage 
\subsection{[LVV-9] DMS-REQ-0020-V-01: Wavefront Sensor Data Acquisition }\label{lvv-9}

\begin{longtable}{cccc}
\hline
\textbf{Jira Link} & \textbf{Assignee} & \textbf{Status} & \textbf{Test Cases}\\ \hline
\href{https://jira.lsstcorp.org/browse/LVV-9}{LVV-9} &
Gregory Dubois-Felsmann & Not Covered &
\begin{tabular}{c}
LVV-T30 \\
LVV-T283 \\
LVV-T284 \\
LVV-T1549 \\
LVV-T1556 \\
\end{tabular}
\\
\hline
\end{longtable}

\textbf{Verification Element Description:} \\
Simulated camera DAQ acquiring wavefront data. Data backbone archiving
the data. The final sentence in the discussion is negated by
DMS-REQ-0265.

{\footnotesize
\begin{longtable}{p{2.5cm}p{13.5cm}}
\hline
\multicolumn{2}{c}{\textbf{Requirement Details}}\\ \hline
Requirement ID & DMS-REQ-0020 \\ \cdashline{1-2}
Requirement Description &
\begin{minipage}[]{13cm}
\textbf{Specification:} The DMS shall acquire raw exposure data from the
Camera wavefront sensors, during normal survey operations and in any
other required operating modes.
\end{minipage}
\\ \cdashline{1-2}
Requirement Discussion &
\begin{minipage}[]{13cm}
\textbf{Discussion:} The details of this are a matter for the DM-Camera
ICD in this area. However, these data should be identical in format and
in mode of acquisition to the raw science sensor data.\\
There is no currently established requirement for the acquisition or
archiving of any raw guider sensor data.
\end{minipage}
\\ \cdashline{1-2}
Requirement Priority & 1a \\ \cdashline{1-2}
Upper Level Requirement &
\begin{tabular}{cl}
OSS-REQ-0316 & Wavefront Sensor Data \\
\end{tabular}
\\ \hline
\end{longtable}
}


\subsubsection{Test Cases Summary}
\begin{longtable}{p{3cm}p{2.5cm}p{2.5cm}p{3cm}p{4cm}}
\toprule
\href{https://jira.lsstcorp.org/secure/Tests.jspa\#/testCase/LVV-T30}{LVV-T30} & \multicolumn{4}{p{12cm}}{ Verify implementation of Wavefront Sensor Data Acquisition } \\ \hline
\textbf{Owner} & \textbf{Status} & \textbf{Version} & \textbf{Critical Event} & \textbf{Verification Type} \\ \hline
Kian-Tat Lim & Defined & 1 & false & Test \\ \hline
\end{longtable}
{\scriptsize
\textbf{Objective:}\\
Verify successful ingestion of wavefront sensor data from L1 Test Stand
DAQ while simulating all modes.
}
\begin{longtable}{p{3cm}p{2.5cm}p{2.5cm}p{3cm}p{4cm}}
\toprule
\href{https://jira.lsstcorp.org/secure/Tests.jspa\#/testCase/LVV-T283}{LVV-T283} & \multicolumn{4}{p{12cm}}{ RAS-00-00: Writing well-formed raw image } \\ \hline
\textbf{Owner} & \textbf{Status} & \textbf{Version} & \textbf{Critical Event} & \textbf{Verification Type} \\ \hline
Michelle Butler & Approved & 1 & false & Test \\ \hline
\end{longtable}
{\scriptsize
\textbf{Objective:}\\
This test will check:\\

\begin{itemize}
\tightlist
\item
  The successful integration of the Pathfinder components with the DM
  Header Service and the Level 1 Archiver;
\item
  That the raw images are well-formed and meet specifications in
  change-controlled documents \citeds{LSE-61};
\end{itemize}

~This Test Case shall be repeated for each of the different cameras
(ATScam, LSSTCam) and sensors (Science, Wavefront, and Guider)
combination.
}
\begin{longtable}{p{3cm}p{2.5cm}p{2.5cm}p{3cm}p{4cm}}
\toprule
\href{https://jira.lsstcorp.org/secure/Tests.jspa\#/testCase/LVV-T284}{LVV-T284} & \multicolumn{4}{p{12cm}}{ RAS-00-05: (LDM-503-8b) Writing data from CCOB to the DBB for further
data processing } \\ \hline
\textbf{Owner} & \textbf{Status} & \textbf{Version} & \textbf{Critical Event} & \textbf{Verification Type} \\ \hline
Michelle Butler & Draft & 1 & false & Test \\ \hline
\end{longtable}
{\scriptsize
\textbf{Objective:}\\
This test will check:

\begin{itemize}
\tightlist
\item
  The successful integration of the DAQ archiver components with the
  CCOB
\item
  That the file can then be ingested into the DBB and be retrieved for
  further analysis
\end{itemize}
}
\begin{longtable}{p{3cm}p{2.5cm}p{2.5cm}p{3cm}p{4cm}}
\toprule
\href{https://jira.lsstcorp.org/secure/Tests.jspa\#/testCase/LVV-T1549}{LVV-T1549} & \multicolumn{4}{p{12cm}}{ LDM-503-6 Comcam verification readiness } \\ \hline
\textbf{Owner} & \textbf{Status} & \textbf{Version} & \textbf{Critical Event} & \textbf{Verification Type} \\ \hline
Michelle Butler & Approved & 1 & false & Demonstration \\ \hline
\end{longtable}
{\scriptsize
\textbf{Objective:}\\
Verify that ComCam has all the services running and verified working for
retrieving an image from the ComCam DAQ and store it on file systems at
the LDF for viewing by RSP. ~
}
\begin{longtable}{p{3cm}p{2.5cm}p{2.5cm}p{3cm}p{4cm}}
\toprule
\href{https://jira.lsstcorp.org/secure/Tests.jspa\#/testCase/LVV-T1556}{LVV-T1556} & \multicolumn{4}{p{12cm}}{ LDM-503-10B Large Scale CCOB Data Access } \\ \hline
\textbf{Owner} & \textbf{Status} & \textbf{Version} & \textbf{Critical Event} & \textbf{Verification Type} \\ \hline
Michelle Butler & Draft & 1 & false & Demonstration \\ \hline
\end{longtable}
{\scriptsize
\textbf{Objective:}\\
Demonstrate the ability to transfer data from the SLAC test stand or
CCOB with 21 rafts from SLAC and ingested at NCSA and make available
through an instance of the RSP
}
  
 \newpage 
\subsection{[LVV-11] DMS-REQ-0024-V-01: Raw Image Assembly }\label{lvv-11}

\begin{longtable}{cccc}
\hline
\textbf{Jira Link} & \textbf{Assignee} & \textbf{Status} & \textbf{Test Cases}\\ \hline
\href{https://jira.lsstcorp.org/browse/LVV-11}{LVV-11} &
Gregory Dubois-Felsmann & Not Covered &
\begin{tabular}{c}
LVV-T32 \\
LVV-T283 \\
LVV-T284 \\
LVV-T1549 \\
LVV-T1550 \\
LVV-T1556 \\
LVV-T1934 \\
\end{tabular}
\\
\hline
\end{longtable}

\textbf{Verification Element Description:} \\
Requires a simulated DAQ and OCS. Files are verified against the
relevant DM specification for raw metadata content and pixel values.

{\footnotesize
\begin{longtable}{p{2.5cm}p{13.5cm}}
\hline
\multicolumn{2}{c}{\textbf{Requirement Details}}\\ \hline
Requirement ID & DMS-REQ-0024 \\ \cdashline{1-2}
Requirement Description &
\begin{minipage}[]{13cm}
\textbf{Specification:} The DMS shall assemble the combination of raw
exposure data from all the readout channels from a single Sensor to form
a single image for that sensor. The image data and relevant exposure
metadata shall be integrated into a standard format suitable for
down-stream processing, archiving, and distribution to the user
community.
\end{minipage}
\\ \cdashline{1-2}
Requirement Discussion &
\begin{minipage}[]{13cm}
\textbf{Discussion:} Relevant exposure metadata are those that define
the observing context, telescope and instrument configuration, and
provenance.
\end{minipage}
\\ \cdashline{1-2}
Requirement Priority & 1a \\ \cdashline{1-2}
Upper Level Requirement &
\begin{tabular}{cl}
OSS-REQ-0114 & Acquisition of Science Sensor data \\
OSS-REQ-0129 & Exposures (Level 1) \\
\end{tabular}
\\ \hline
\end{longtable}
}


\subsubsection{Test Cases Summary}
\begin{longtable}{p{3cm}p{2.5cm}p{2.5cm}p{3cm}p{4cm}}
\toprule
\href{https://jira.lsstcorp.org/secure/Tests.jspa\#/testCase/LVV-T32}{LVV-T32} & \multicolumn{4}{p{12cm}}{ Verify implementation of Raw Image Assembly } \\ \hline
\textbf{Owner} & \textbf{Status} & \textbf{Version} & \textbf{Critical Event} & \textbf{Verification Type} \\ \hline
Kian-Tat Lim & Defined & 1 & false & Test \\ \hline
\end{longtable}
{\scriptsize
\textbf{Objective:}\\
Verify that the raw exposure data from all readout channels in a sensor
can be assembled into a single image, and that all required/relevant
metadata are associated with the image data.~
}
\begin{longtable}{p{3cm}p{2.5cm}p{2.5cm}p{3cm}p{4cm}}
\toprule
\href{https://jira.lsstcorp.org/secure/Tests.jspa\#/testCase/LVV-T283}{LVV-T283} & \multicolumn{4}{p{12cm}}{ RAS-00-00: Writing well-formed raw image } \\ \hline
\textbf{Owner} & \textbf{Status} & \textbf{Version} & \textbf{Critical Event} & \textbf{Verification Type} \\ \hline
Michelle Butler & Approved & 1 & false & Test \\ \hline
\end{longtable}
{\scriptsize
\textbf{Objective:}\\
This test will check:\\

\begin{itemize}
\tightlist
\item
  The successful integration of the Pathfinder components with the DM
  Header Service and the Level 1 Archiver;
\item
  That the raw images are well-formed and meet specifications in
  change-controlled documents \citeds{LSE-61};
\end{itemize}

~This Test Case shall be repeated for each of the different cameras
(ATScam, LSSTCam) and sensors (Science, Wavefront, and Guider)
combination.
}
\begin{longtable}{p{3cm}p{2.5cm}p{2.5cm}p{3cm}p{4cm}}
\toprule
\href{https://jira.lsstcorp.org/secure/Tests.jspa\#/testCase/LVV-T284}{LVV-T284} & \multicolumn{4}{p{12cm}}{ RAS-00-05: (LDM-503-8b) Writing data from CCOB to the DBB for further
data processing } \\ \hline
\textbf{Owner} & \textbf{Status} & \textbf{Version} & \textbf{Critical Event} & \textbf{Verification Type} \\ \hline
Michelle Butler & Draft & 1 & false & Test \\ \hline
\end{longtable}
{\scriptsize
\textbf{Objective:}\\
This test will check:

\begin{itemize}
\tightlist
\item
  The successful integration of the DAQ archiver components with the
  CCOB
\item
  That the file can then be ingested into the DBB and be retrieved for
  further analysis
\end{itemize}
}
\begin{longtable}{p{3cm}p{2.5cm}p{2.5cm}p{3cm}p{4cm}}
\toprule
\href{https://jira.lsstcorp.org/secure/Tests.jspa\#/testCase/LVV-T1549}{LVV-T1549} & \multicolumn{4}{p{12cm}}{ LDM-503-6 Comcam verification readiness } \\ \hline
\textbf{Owner} & \textbf{Status} & \textbf{Version} & \textbf{Critical Event} & \textbf{Verification Type} \\ \hline
Michelle Butler & Approved & 1 & false & Demonstration \\ \hline
\end{longtable}
{\scriptsize
\textbf{Objective:}\\
Verify that ComCam has all the services running and verified working for
retrieving an image from the ComCam DAQ and store it on file systems at
the LDF for viewing by RSP. ~
}
\begin{longtable}{p{3cm}p{2.5cm}p{2.5cm}p{3cm}p{4cm}}
\toprule
\href{https://jira.lsstcorp.org/secure/Tests.jspa\#/testCase/LVV-T1550}{LVV-T1550} & \multicolumn{4}{p{12cm}}{ LDM-503-10 DAQ Validation } \\ \hline
\textbf{Owner} & \textbf{Status} & \textbf{Version} & \textbf{Critical Event} & \textbf{Verification Type} \\ \hline
Michelle Butler & Approved & 1 & false & Demonstration \\ \hline
\end{longtable}
{\scriptsize
\textbf{Objective:}\\
Verify that the DAQ can talk to test machines at the BDC through the
DWDM network.~
}
\begin{longtable}{p{3cm}p{2.5cm}p{2.5cm}p{3cm}p{4cm}}
\toprule
\href{https://jira.lsstcorp.org/secure/Tests.jspa\#/testCase/LVV-T1556}{LVV-T1556} & \multicolumn{4}{p{12cm}}{ LDM-503-10B Large Scale CCOB Data Access } \\ \hline
\textbf{Owner} & \textbf{Status} & \textbf{Version} & \textbf{Critical Event} & \textbf{Verification Type} \\ \hline
Michelle Butler & Draft & 1 & false & Demonstration \\ \hline
\end{longtable}
{\scriptsize
\textbf{Objective:}\\
Demonstrate the ability to transfer data from the SLAC test stand or
CCOB with 21 rafts from SLAC and ingested at NCSA and make available
through an instance of the RSP
}
\begin{longtable}{p{3cm}p{2.5cm}p{2.5cm}p{3cm}p{4cm}}
\toprule
\href{https://jira.lsstcorp.org/secure/Tests.jspa\#/testCase/LVV-T1934}{LVV-T1934} & \multicolumn{4}{p{12cm}}{ ComCam Data Transfer and Ingestion } \\ \hline
\textbf{Owner} & \textbf{Status} & \textbf{Version} & \textbf{Critical Event} & \textbf{Verification Type} \\ \hline
Robert Gruendl & Draft & 1 & true & Inspection \\ \hline
\end{longtable}
{\scriptsize
\textbf{Objective:}\\
Verify that ComCam Archiver data taken are transferred to NCSA Data
BackBone endpoint and Ingested
}
  
 \newpage 
\subsection{[LVV-12] DMS-REQ-0029-V-01: Generate Photometric Zeropoint for Visit Image }\label{lvv-12}

\begin{longtable}{cccc}
\hline
\textbf{Jira Link} & \textbf{Assignee} & \textbf{Status} & \textbf{Test Cases}\\ \hline
\href{https://jira.lsstcorp.org/browse/LVV-12}{LVV-12} &
Jim Bosch & Not Covered &
\begin{tabular}{c}
LVV-T15 \\
LVV-T19 \\
LVV-T39 \\
\end{tabular}
\\
\hline
\end{longtable}

\textbf{Verification Element Description:} \\
Check that a zeropoint is present in output data files from
DMS-REQ-0069. Does not check that the value is reasonable.

{\footnotesize
\begin{longtable}{p{2.5cm}p{13.5cm}}
\hline
\multicolumn{2}{c}{\textbf{Requirement Details}}\\ \hline
Requirement ID & DMS-REQ-0029 \\ \cdashline{1-2}
Requirement Description &
\begin{minipage}[]{13cm}
\textbf{Specification:} The DMS shall derive and persist a photometric
zeropoint for each visit image, per CCD.
\end{minipage}
\\ \cdashline{1-2}
Requirement Priority & 1b \\ \cdashline{1-2}
Upper Level Requirement &
\begin{tabular}{cl}
DMS-REQ-0090 & Generate Alerts \\
OSS-REQ-0056 & System Monitoring \& Diagnostics \\
OSS-REQ-0152 & Level 1 Photometric Zero Point Error \\
\end{tabular}
\\ \hline
\end{longtable}
}


\subsubsection{Test Cases Summary}
\begin{longtable}{p{3cm}p{2.5cm}p{2.5cm}p{3cm}p{4cm}}
\toprule
\href{https://jira.lsstcorp.org/secure/Tests.jspa\#/testCase/LVV-T15}{LVV-T15} & \multicolumn{4}{p{12cm}}{ DRP-00-30: Scientific Verification of Processed Visit Images } \\ \hline
\textbf{Owner} & \textbf{Status} & \textbf{Version} & \textbf{Critical Event} & \textbf{Verification Type} \\ \hline
Jim Bosch & Deprecated & 1 & false & Test \\ \hline
\end{longtable}
{\scriptsize
\textbf{Objective:}\\
This test will check that the Processed Visit Images (PVIs) delivered by
the DRP science payload meet the requirements laid down by \citeds{LSE-61}.\\
Specifically, this will demonstrate that:

\begin{itemize}
\tightlist
\item
  Processed visit images have been generated and persisted during
  payload execution;
\item
  Each PVI includes a background model (DMS-REQ-0327), photometric
  zero-point (DMS- REQ-0029), spatially-varying PSF (DMS-REQ-0070) and
  WCS (DMS-REQ-0030).
\item
  Saturated pixels are correctly masked.
\item
  Pixels affected by cosmic rays are correctly masked.
\item
  The background is not oversubtracted around bright objects.
\end{itemize}

This test does not include quantitative targets for the science quality
criteria; we instead re- quire for each test that we be able to quickly
construct a plot or display summary images that allow such a target can
be visualized.
}
\begin{longtable}{p{3cm}p{2.5cm}p{2.5cm}p{3cm}p{4cm}}
\toprule
\href{https://jira.lsstcorp.org/secure/Tests.jspa\#/testCase/LVV-T19}{LVV-T19} & \multicolumn{4}{p{12cm}}{ AG-00-10: Scientific Verification of Processed Visit Images } \\ \hline
\textbf{Owner} & \textbf{Status} & \textbf{Version} & \textbf{Critical Event} & \textbf{Verification Type} \\ \hline
Eric Bellm & Deprecated & 1 & false & Test \\ \hline
\end{longtable}
{\scriptsize
\textbf{Objective:}\\
This test will check that the Processed Visit Images (PVIs) delivered by
the alert generation science payload meet the requirements laid down by
\citeds{LSE-61}.\\
Specifically, this will demonstrate that:

\begin{itemize}
\tightlist
\item
  Processed visit images have been generated and persisted during
  payload execution;
\item
  Each PVI includes a science pixel array, a mask array, and a variance
  array. (DMS-REQ-0072).
\item
  Each PVI includes a background model (DMS-REQ-0327), photometric
  zero-point (DMS- REQ-0029), spatially-varying PSF (DMS-REQ-0070) and
  WCS (DMS-REQ-0030).
\item
  Saturated pixels are correctly masked.
\item
  Pixels affected by cosmic rays are correctly masked.
\item
  The background is not oversubtracted around bright objects.
\end{itemize}

This test does not include quantitative targets for the science quality
criteria.
}
\begin{longtable}{p{3cm}p{2.5cm}p{2.5cm}p{3cm}p{4cm}}
\toprule
\href{https://jira.lsstcorp.org/secure/Tests.jspa\#/testCase/LVV-T39}{LVV-T39} & \multicolumn{4}{p{12cm}}{ Verify implementation of Generate Photometric Zeropoint for Visit Image } \\ \hline
\textbf{Owner} & \textbf{Status} & \textbf{Version} & \textbf{Critical Event} & \textbf{Verification Type} \\ \hline
Jim Bosch & Approved & 1 & false & Test \\ \hline
\end{longtable}
{\scriptsize
\textbf{Objective:}\\
Verify that Processed Visit Image data products produced by the DRP and
AP pipelines include the parameters of a model that relates the observed
flux on the image to physical flux units.
}
  
 \newpage 
\subsection{[LVV-13] DMS-REQ-0030-V-01: Absolute accuracy of WCS }\label{lvv-13}

\begin{longtable}{cccc}
\hline
\textbf{Jira Link} & \textbf{Assignee} & \textbf{Status} & \textbf{Test Cases}\\ \hline
\href{https://jira.lsstcorp.org/browse/LVV-13}{LVV-13} &
Jim Bosch & Not Covered &
\begin{tabular}{c}
LVV-T15 \\
LVV-T19 \\
LVV-T40 \\
\end{tabular}
\\
\hline
\end{longtable}

\textbf{Verification Element Description:} \\
See Nidever/Economou document Section 3.2. Note terminology in this
requirement is not consistent with LSR. Can be tested with existing
survey data. Also needs to be tested with real LSST data.

Associated element
(\href{https://jira.lsstcorp.org/browse/LVV-9741}{LVV-9741}) satisfies
the minimum number of available astrometric standards per CCD.

{\footnotesize
\begin{longtable}{p{2.5cm}p{13.5cm}}
\hline
\multicolumn{2}{c}{\textbf{Requirement Details}}\\ \hline
Requirement ID & DMS-REQ-0030 \\ \cdashline{1-2}
Requirement Description &
\begin{minipage}[]{13cm}
\textbf{Specification:} The DMS shall generate and persist a WCS for
each visit image. The absolute accuracy of the WCS shall be at least
\textbf{astrometricAccuracy} in all areas of the image, provided that
there are at least \textbf{astrometricMinStandards} astrometric
standards available in each CCD.
\end{minipage}
\\ \cdashline{1-2}
Requirement Parameters & {[}\textbf{astrometricAccuracy = 50{{[}milliarcsecond{]}}} Absolute
accuracy of the WCS across the focal plane (approximately one-quarter of
a pixel)., \textbf{astrometricMinStandards = 5{{[}integer{]}}} Minimum
number of astrometric standards per CCD.{]} \\ \cdashline{1-2}
Requirement Discussion &
\begin{minipage}[]{13cm}
\textbf{Discussion:} The World Coordinate System for visits will be
expressed in terms of a FITS Standard representation, which provides for
named metadata to be interpreted as coefficients of one of a finite set
of coordinate transformations.
\end{minipage}
\\ \cdashline{1-2}
Requirement Priority & 1a \\ \cdashline{1-2}
Upper Level Requirement &
\begin{tabular}{cl}
DMS-REQ-0090 & Generate Alerts \\
DMS-REQ-0104 & Produce Co-Added Exposures \\
OSS-REQ-0149 & Level 1 Catalog Precision \\
OSS-REQ-0162 & Level 2 Catalog Accuracy \\
\end{tabular}
\\ \hline
\end{longtable}
}


\subsubsection{Test Cases Summary}
\begin{longtable}{p{3cm}p{2.5cm}p{2.5cm}p{3cm}p{4cm}}
\toprule
\href{https://jira.lsstcorp.org/secure/Tests.jspa\#/testCase/LVV-T15}{LVV-T15} & \multicolumn{4}{p{12cm}}{ DRP-00-30: Scientific Verification of Processed Visit Images } \\ \hline
\textbf{Owner} & \textbf{Status} & \textbf{Version} & \textbf{Critical Event} & \textbf{Verification Type} \\ \hline
Jim Bosch & Deprecated & 1 & false & Test \\ \hline
\end{longtable}
{\scriptsize
\textbf{Objective:}\\
This test will check that the Processed Visit Images (PVIs) delivered by
the DRP science payload meet the requirements laid down by \citeds{LSE-61}.\\
Specifically, this will demonstrate that:

\begin{itemize}
\tightlist
\item
  Processed visit images have been generated and persisted during
  payload execution;
\item
  Each PVI includes a background model (DMS-REQ-0327), photometric
  zero-point (DMS- REQ-0029), spatially-varying PSF (DMS-REQ-0070) and
  WCS (DMS-REQ-0030).
\item
  Saturated pixels are correctly masked.
\item
  Pixels affected by cosmic rays are correctly masked.
\item
  The background is not oversubtracted around bright objects.
\end{itemize}

This test does not include quantitative targets for the science quality
criteria; we instead re- quire for each test that we be able to quickly
construct a plot or display summary images that allow such a target can
be visualized.
}
\begin{longtable}{p{3cm}p{2.5cm}p{2.5cm}p{3cm}p{4cm}}
\toprule
\href{https://jira.lsstcorp.org/secure/Tests.jspa\#/testCase/LVV-T19}{LVV-T19} & \multicolumn{4}{p{12cm}}{ AG-00-10: Scientific Verification of Processed Visit Images } \\ \hline
\textbf{Owner} & \textbf{Status} & \textbf{Version} & \textbf{Critical Event} & \textbf{Verification Type} \\ \hline
Eric Bellm & Deprecated & 1 & false & Test \\ \hline
\end{longtable}
{\scriptsize
\textbf{Objective:}\\
This test will check that the Processed Visit Images (PVIs) delivered by
the alert generation science payload meet the requirements laid down by
\citeds{LSE-61}.\\
Specifically, this will demonstrate that:

\begin{itemize}
\tightlist
\item
  Processed visit images have been generated and persisted during
  payload execution;
\item
  Each PVI includes a science pixel array, a mask array, and a variance
  array. (DMS-REQ-0072).
\item
  Each PVI includes a background model (DMS-REQ-0327), photometric
  zero-point (DMS- REQ-0029), spatially-varying PSF (DMS-REQ-0070) and
  WCS (DMS-REQ-0030).
\item
  Saturated pixels are correctly masked.
\item
  Pixels affected by cosmic rays are correctly masked.
\item
  The background is not oversubtracted around bright objects.
\end{itemize}

This test does not include quantitative targets for the science quality
criteria.
}
\begin{longtable}{p{3cm}p{2.5cm}p{2.5cm}p{3cm}p{4cm}}
\toprule
\href{https://jira.lsstcorp.org/secure/Tests.jspa\#/testCase/LVV-T40}{LVV-T40} & \multicolumn{4}{p{12cm}}{ Verify implementation of Generate WCS for Visit Images } \\ \hline
\textbf{Owner} & \textbf{Status} & \textbf{Version} & \textbf{Critical Event} & \textbf{Verification Type} \\ \hline
Jim Bosch & Approved & 1 & false & Test \\ \hline
\end{longtable}
{\scriptsize
\textbf{Objective:}\\
Verify that Processed Visit Images produced by the AP and DRP pipelines
include FITS WCS accurate to specified \textbf{astrometricAccuracy} over
the bounds of the image.
}
  
 \newpage 
\subsection{[LVV-14] DMS-REQ-0032-V-01: Image Differencing }\label{lvv-14}

\begin{longtable}{cccc}
\hline
\textbf{Jira Link} & \textbf{Assignee} & \textbf{Status} & \textbf{Test Cases}\\ \hline
\href{https://jira.lsstcorp.org/browse/LVV-14}{LVV-14} &
Eric Bellm & Not Covered &
\begin{tabular}{c}
LVV-T126 \\
\end{tabular}
\\
\hline
\end{longtable}

\textbf{Verification Element Description:} \\
Verified as part of L1 processing.

{\footnotesize
\begin{longtable}{p{2.5cm}p{13.5cm}}
\hline
\multicolumn{2}{c}{\textbf{Requirement Details}}\\ \hline
Requirement ID & DMS-REQ-0032 \\ \cdashline{1-2}
Requirement Description &
\begin{minipage}[]{13cm}
\textbf{Specification:} The DMS shall provide software to perform image
differencing, generating Difference Exposures from the comparison of
single exposures and/or coadded images.
\end{minipage}
\\ \cdashline{1-2}
Requirement Priority & 1b \\ \cdashline{1-2}
Upper Level Requirement &
\begin{tabular}{cl}
OSS-REQ-0121 & Open Source, Open Configuration \\
OSS-REQ-0129 & Exposures (Level 1) \\
\end{tabular}
\\ \hline
\end{longtable}
}


\subsubsection{Test Cases Summary}
\begin{longtable}{p{3cm}p{2.5cm}p{2.5cm}p{3cm}p{4cm}}
\toprule
\href{https://jira.lsstcorp.org/secure/Tests.jspa\#/testCase/LVV-T126}{LVV-T126} & \multicolumn{4}{p{12cm}}{ Verify implementation of Image Differencing } \\ \hline
\textbf{Owner} & \textbf{Status} & \textbf{Version} & \textbf{Critical Event} & \textbf{Verification Type} \\ \hline
Eric Bellm & Defined & 1 & false & Test \\ \hline
\end{longtable}
{\scriptsize
\textbf{Objective:}\\
Verify that the DMS can perform image differencing from single exposures
and coadds.
}
  
 \newpage 
\subsection{[LVV-15] DMS-REQ-0033-V-01: Provide Source Detection Software }\label{lvv-15}

\begin{longtable}{cccc}
\hline
\textbf{Jira Link} & \textbf{Assignee} & \textbf{Status} & \textbf{Test Cases}\\ \hline
\href{https://jira.lsstcorp.org/browse/LVV-15}{LVV-15} &
Jim Bosch & Not Covered &
\begin{tabular}{c}
LVV-T127 \\
LVV-T362 \\
\end{tabular}
\\
\hline
\end{longtable}

\textbf{Verification Element Description:} \\
Given reference (possible simulated) difference images and coadd images,
generate catalog and compare with known values.

{\footnotesize
\begin{longtable}{p{2.5cm}p{13.5cm}}
\hline
\multicolumn{2}{c}{\textbf{Requirement Details}}\\ \hline
Requirement ID & DMS-REQ-0033 \\ \cdashline{1-2}
Requirement Description &
\begin{minipage}[]{13cm}
\textbf{Specification:} The DMS shall provide software for the detection
of sources in a calibrated image, which may be a Difference Image or a
Co-Add image.
\end{minipage}
\\ \cdashline{1-2}
Requirement Priority & 1a \\ \cdashline{1-2}
Upper Level Requirement &
\begin{tabular}{cl}
OSS-REQ-0130 & Catalogs (Level 1) \\
OSS-REQ-0137 & Catalogs (Level 2) \\
OSS-REQ-0121 & Open Source, Open Configuration \\
DMS-REQ-0080 & Difference Sources Available within 24 hours \\
\end{tabular}
\\ \hline
\end{longtable}
}


\subsubsection{Test Cases Summary}
\begin{longtable}{p{3cm}p{2.5cm}p{2.5cm}p{3cm}p{4cm}}
\toprule
\href{https://jira.lsstcorp.org/secure/Tests.jspa\#/testCase/LVV-T127}{LVV-T127} & \multicolumn{4}{p{12cm}}{ Verify implementation of Provide Source Detection Software } \\ \hline
\textbf{Owner} & \textbf{Status} & \textbf{Version} & \textbf{Critical Event} & \textbf{Verification Type} \\ \hline
Robert Lupton & Defined & 1 & false & Test \\ \hline
\end{longtable}
{\scriptsize
\textbf{Objective:}\\
Verify that the DMS provides source detection software that can be
applied to calibrated images, including both difference images and
coadds. This will be verified using simulated data, but could also be
done by inserting artificial sources into existing datasets.
}
\begin{longtable}{p{3cm}p{2.5cm}p{2.5cm}p{3cm}p{4cm}}
\toprule
\href{https://jira.lsstcorp.org/secure/Tests.jspa\#/testCase/LVV-T362}{LVV-T362} & \multicolumn{4}{p{12cm}}{ Installation of the LSST Science Pipelines Payloads } \\ \hline
\textbf{Owner} & \textbf{Status} & \textbf{Version} & \textbf{Critical Event} & \textbf{Verification Type} \\ \hline
John Swinbank & Approved & 1 & false & Test \\ \hline
\end{longtable}
{\scriptsize
\textbf{Objective:}\\
This test will check that:

\begin{itemize}
\tightlist
\item
  The Alert Production Pipeline payload is available for installation
  from documented channels;
\item
  The Data Release Production Pipeline payload is available for
  installation from documented channels;
\item
  The Calibration Products Production Pipeline payload is available for
  installation from documented channels;
\item
  These payloads can be installed on systems at the LSST Data Facility
  following available documentation;
\item
  The installed pipeline payloads are capable of successfully executing
  basic integration tests.
\end{itemize}

Note that this test assumes a 2018-era packaging of the Science
Pipelines software, in which all the above payloads are represented by a
single ``meta-package'', lsst\_distrib.
}
  
 \newpage 
\subsection{[LVV-16] DMS-REQ-0034-V-01: Associate Sources to Objects }\label{lvv-16}

\begin{longtable}{cccc}
\hline
\textbf{Jira Link} & \textbf{Assignee} & \textbf{Status} & \textbf{Test Cases}\\ \hline
\href{https://jira.lsstcorp.org/browse/LVV-16}{LVV-16} &
Jim Bosch & Not Covered &
\begin{tabular}{c}
LVV-T61 \\
\end{tabular}
\\
\hline
\end{longtable}

\textbf{Verification Element Description:} \\
Precursor data. Different filters, sky positions and epochs. L2
processing. Verify object association.

{\footnotesize
\begin{longtable}{p{2.5cm}p{13.5cm}}
\hline
\multicolumn{2}{c}{\textbf{Requirement Details}}\\ \hline
Requirement ID & DMS-REQ-0034 \\ \cdashline{1-2}
Requirement Description &
\begin{minipage}[]{13cm}
\textbf{Specification:} The DMS shall associate Sources measured at
different times and in different passbands with entries in the Object
catalog.
\end{minipage}
\\ \cdashline{1-2}
Requirement Discussion &
\begin{minipage}[]{13cm}
\textbf{Discussion:} The task of association is to relate Sources from
different times, filters, and sky positions, to the corresponding
Objects. Having made these associations, further measurements can be
made on the full object data to generate astronomically useful
quantities.
\end{minipage}
\\ \cdashline{1-2}
Requirement Priority & 1a \\ \cdashline{1-2}
Upper Level Requirement &
\begin{tabular}{cl}
DMS-REQ-0081 & Produce Object Catalog \\
OSS-REQ-0339 & Level 2 Source-Object Association Quality \\
\end{tabular}
\\ \hline
\end{longtable}
}


\subsubsection{Test Cases Summary}
\begin{longtable}{p{3cm}p{2.5cm}p{2.5cm}p{3cm}p{4cm}}
\toprule
\href{https://jira.lsstcorp.org/secure/Tests.jspa\#/testCase/LVV-T61}{LVV-T61} & \multicolumn{4}{p{12cm}}{ Verify implementation of Associate Sources to Objects } \\ \hline
\textbf{Owner} & \textbf{Status} & \textbf{Version} & \textbf{Critical Event} & \textbf{Verification Type} \\ \hline
Jim Bosch & Defined & 1 & false & Test \\ \hline
\end{longtable}
{\scriptsize
\textbf{Objective:}\\
Verify that each Source record contains an ID that associates it with a
best guess at the Object it corresponds to.
}
  
 \newpage 
\subsection{[LVV-17] DMS-REQ-0042-V-01: Provide Astrometric Model }\label{lvv-17}

\begin{longtable}{cccc}
\hline
\textbf{Jira Link} & \textbf{Assignee} & \textbf{Status} & \textbf{Test Cases}\\ \hline
\href{https://jira.lsstcorp.org/browse/LVV-17}{LVV-17} &
Jim Bosch & Not Covered &
\begin{tabular}{c}
LVV-T128 \\
\end{tabular}
\\
\hline
\end{longtable}

\textbf{Verification Element Description:} \\
Precursor data covering a range of epochs and show that proper motion
and parallax has been calculated. The requirement does not specify an
accuracy for these calculations.

{\footnotesize
\begin{longtable}{p{2.5cm}p{13.5cm}}
\hline
\multicolumn{2}{c}{\textbf{Requirement Details}}\\ \hline
Requirement ID & DMS-REQ-0042 \\ \cdashline{1-2}
Requirement Description &
\begin{minipage}[]{13cm}
\textbf{Specification:} An astrometric model shall be provided for every
Object and DIAObject which specifies at least the proper motion and
parallax, and the estimated uncertainties on these quantities.
\end{minipage}
\\ \cdashline{1-2}
Requirement Priority & 1b \\ \cdashline{1-2}
Upper Level Requirement &
\begin{tabular}{cl}
OSS-REQ-0153 & World Coordinate System Accuracy \\
OSS-REQ-0149 & Level 1 Catalog Precision \\
OSS-REQ-0160 & Level 1 Difference Source - Difference Object Association Quality \\
OSS-REQ-0162 & Level 2 Catalog Accuracy \\
\end{tabular}
\\ \hline
\end{longtable}
}


\subsubsection{Test Cases Summary}
\begin{longtable}{p{3cm}p{2.5cm}p{2.5cm}p{3cm}p{4cm}}
\toprule
\href{https://jira.lsstcorp.org/secure/Tests.jspa\#/testCase/LVV-T128}{LVV-T128} & \multicolumn{4}{p{12cm}}{ Verify implementation Provide Astrometric Model } \\ \hline
\textbf{Owner} & \textbf{Status} & \textbf{Version} & \textbf{Critical Event} & \textbf{Verification Type} \\ \hline
Colin Slater & Draft & 1 & false & Test \\ \hline
\end{longtable}
{\scriptsize
\textbf{Objective:}\\
Verify that an astrometric model is available for Objects and
DIAObjects.
}
  
 \newpage 
\subsection{[LVV-18] DMS-REQ-0043-V-01: Provide Calibrated Photometry }\label{lvv-18}

\begin{longtable}{cccc}
\hline
\textbf{Jira Link} & \textbf{Assignee} & \textbf{Status} & \textbf{Test Cases}\\ \hline
\href{https://jira.lsstcorp.org/browse/LVV-18}{LVV-18} &
Jim Bosch & Not Covered &
\begin{tabular}{c}
LVV-T21 \\
LVV-T22 \\
LVV-T129 \\
\end{tabular}
\\
\hline
\end{longtable}

\textbf{Verification Element Description:} \\
Test with precursor data and show that AB magnitudes are calculated.
This functional requirement does not include a test that these
magnitudes are accurate.

{\footnotesize
\begin{longtable}{p{2.5cm}p{13.5cm}}
\hline
\multicolumn{2}{c}{\textbf{Requirement Details}}\\ \hline
Requirement ID & DMS-REQ-0043 \\ \cdashline{1-2}
Requirement Description &
\begin{minipage}[]{13cm}
\textbf{Specification:} The DMS shall provide calibrated photometry in
each observed passband for all measured entities (e.g., DIASources,
Sources, Objects), measuring the AB magnitude of the equivalent flat-SED
source, above the atmosphere. Fluxes shall be calculated for all
measured entities.
\end{minipage}
\\ \cdashline{1-2}
Requirement Discussion &
\begin{minipage}[]{13cm}
\textbf{Discussion:} Note that the SED is only assumed to be flat within
the passband of the measurement.
\end{minipage}
\\ \cdashline{1-2}
Requirement Priority & 1a \\ \cdashline{1-2}
Upper Level Requirement &
\begin{tabular}{cl}
OSS-REQ-0130 & Catalogs (Level 1) \\
OSS-REQ-0275 & Calibration Processing Performance Allocations \\
OSS-REQ-0137 & Catalogs (Level 2) \\
\end{tabular}
\\ \hline
\end{longtable}
}


\subsubsection{Test Cases Summary}
\begin{longtable}{p{3cm}p{2.5cm}p{2.5cm}p{3cm}p{4cm}}
\toprule
\href{https://jira.lsstcorp.org/secure/Tests.jspa\#/testCase/LVV-T21}{LVV-T21} & \multicolumn{4}{p{12cm}}{ AG-00-20: Scientific Verification of DIASource Catalog } \\ \hline
\textbf{Owner} & \textbf{Status} & \textbf{Version} & \textbf{Critical Event} & \textbf{Verification Type} \\ \hline
Eric Bellm & Deprecated & 1 & false & Test \\ \hline
\end{longtable}
{\scriptsize
\textbf{Objective:}\\
This test will check that the difference image source catalogs delivered
by the Alert Generation science payload meet the requirements laid down
by \citeds{LSE-61}.

\begin{itemize}
\tightlist
\item
  Specifically, this will demonstrate that:
\item
  Measurements in the catalog are presented in flux units
  (DMS-REQ-0347);
\item
  Each DIASource record contains an appropriate subset of the attributes
  required by DMS-REQ-0269. In particular, the LDM-503-3-era pipeline is
  expected to provide DIASource positions (sky and focal plane), fluxes,
  and flags indicative of issues encountered during processing.
\item
  Faint DIASources satisfying additional criteria are stored
  (DMS-REQ-0270).
\item
  Derived quantities are provided in pre-computed columns
  (DMS-REQ-0331);
\end{itemize}

This test does not include quantitative targets for the science quality
criteria.\\[2\baselineskip]
}
\begin{longtable}{p{3cm}p{2.5cm}p{2.5cm}p{3cm}p{4cm}}
\toprule
\href{https://jira.lsstcorp.org/secure/Tests.jspa\#/testCase/LVV-T22}{LVV-T22} & \multicolumn{4}{p{12cm}}{ AG-00-25: Scientific Verification of DIAObject Catalog } \\ \hline
\textbf{Owner} & \textbf{Status} & \textbf{Version} & \textbf{Critical Event} & \textbf{Verification Type} \\ \hline
Eric Bellm & Deprecated & 1 & false & Test \\ \hline
\end{longtable}
{\scriptsize
\textbf{Objective:}\\
This test will check that the DIAObject catalogs delivered by the Alert
Generation science pay- load meet the requirements laid down by
\citeds{LSE-61}.\\
Specifically, this will demonstrate that:

\begin{itemize}
\tightlist
\item
  DIAObjects are recorded with unique identifiers (DMS-REQ-0271);
\item
  Measurements in the catalog are presented in flux units
  (DMS-REQ-0347);
\item
  EachDIAObjectrecordcontainscontainsanappropriatesetofsummaryattributes(DMS-
  REQ-0271 and DMS-REQ-0272). Note:

  \begin{itemize}
  \tightlist
  \item
    This test is executed independently of the Data Release Production
    system. Hence, DIAObjects are not associated to Objects, and the
    association metadata specified by DMS-REQ-0271 is not expected to be
    available.
  \item
    TheLDM-503-3erapipelineisnotexpectedtocalculateorpersistallattributesspec-
    ified by DMS-REQ-0272 requirement.
  \end{itemize}
\item
  Relevant derived quantities are provided in pre-computed columns
  (DMS-REQ-0331);~
\end{itemize}

This test does not include quantitative targets for the science quality
criteria.
}
\begin{longtable}{p{3cm}p{2.5cm}p{2.5cm}p{3cm}p{4cm}}
\toprule
\href{https://jira.lsstcorp.org/secure/Tests.jspa\#/testCase/LVV-T129}{LVV-T129} & \multicolumn{4}{p{12cm}}{ Verify implementation of Provide Calibrated Photometry } \\ \hline
\textbf{Owner} & \textbf{Status} & \textbf{Version} & \textbf{Critical Event} & \textbf{Verification Type} \\ \hline
Robert Lupton & Defined & 1 & false & Test \\ \hline
\end{longtable}
{\scriptsize
\textbf{Objective:}\\
Verify that the DMS provides photometry calibrated in AB mags and fluxes
(in nJy) for all measured objects and sources. Must be tested for both
DRP and AP products.
}
  
 \newpage 
\subsection{[LVV-19] DMS-REQ-0046-V-01: Provide Photometric Redshifts of Galaxies }\label{lvv-19}

\begin{longtable}{cccc}
\hline
\textbf{Jira Link} & \textbf{Assignee} & \textbf{Status} & \textbf{Test Cases}\\ \hline
\href{https://jira.lsstcorp.org/browse/LVV-19}{LVV-19} &
Jim Bosch & Not Covered &
\begin{tabular}{c}
LVV-T68 \\
\end{tabular}
\\
\hline
\end{longtable}

\textbf{Verification Element Description:} \\
Verify that the Object table has a photometric redshift for each object.

{\footnotesize
\begin{longtable}{p{2.5cm}p{13.5cm}}
\hline
\multicolumn{2}{c}{\textbf{Requirement Details}}\\ \hline
Requirement ID & DMS-REQ-0046 \\ \cdashline{1-2}
Requirement Description &
\begin{minipage}[]{13cm}
\textbf{Specification:} The DMS shall compute a photometric redshift for
all detected Objects.
\end{minipage}
\\ \cdashline{1-2}
Requirement Priority & 2 \\ \cdashline{1-2}
Upper Level Requirement &
\begin{tabular}{cl}
OSS-REQ-0133 & Level 2 Data Products \\
DMS-REQ-0040 & Enable BAO Analysis \\
\end{tabular}
\\ \hline
\end{longtable}
}


\subsubsection{Test Cases Summary}
\begin{longtable}{p{3cm}p{2.5cm}p{2.5cm}p{3cm}p{4cm}}
\toprule
\href{https://jira.lsstcorp.org/secure/Tests.jspa\#/testCase/LVV-T68}{LVV-T68} & \multicolumn{4}{p{12cm}}{ Verify implementation of Provide Photometric Redshifts of Galaxies } \\ \hline
\textbf{Owner} & \textbf{Status} & \textbf{Version} & \textbf{Critical Event} & \textbf{Verification Type} \\ \hline
Jim Bosch & Draft & 1 & false & Test \\ \hline
\end{longtable}
{\scriptsize
\textbf{Objective:}\\
Verify that Object catalogs produced by the DRP Pipeline include
photometric redshift information.
}
  
 \newpage 
\subsection{[LVV-20] DMS-REQ-0047-V-01: Provide PSF for Coadded Images }\label{lvv-20}

\begin{longtable}{cccc}
\hline
\textbf{Jira Link} & \textbf{Assignee} & \textbf{Status} & \textbf{Test Cases}\\ \hline
\href{https://jira.lsstcorp.org/browse/LVV-20}{LVV-20} &
Jim Bosch & Not Covered &
\begin{tabular}{c}
LVV-T16 \\
LVV-T62 \\
LVV-T62 \\
\end{tabular}
\\
\hline
\end{longtable}

\textbf{Verification Element Description:} \\
From a coadd, request the PSF from every pixel. Does not require that
the PSF varies.

{\footnotesize
\begin{longtable}{p{2.5cm}p{13.5cm}}
\hline
\multicolumn{2}{c}{\textbf{Requirement Details}}\\ \hline
Requirement ID & DMS-REQ-0047 \\ \cdashline{1-2}
Requirement Description &
\begin{minipage}[]{13cm}
\textbf{Specification:} The DMS shall determine a characterization of
the PSF for any specified location in coadded images.
\end{minipage}
\\ \cdashline{1-2}
Requirement Discussion &
\begin{minipage}[]{13cm}
\textbf{Discussion:} The PSF model will be primarily used to perform
initial object characterization and bootstrapping of multi-epoch object
characterization (e.g., Multifit).
\end{minipage}
\\ \cdashline{1-2}
Requirement Priority & 1b \\ \cdashline{1-2}
Upper Level Requirement &
\begin{tabular}{cl}
OSS-REQ-0153 & World Coordinate System Accuracy \\
DMS-REQ-0041 & Measure Intrinsic Ellipticities of Small Galaxies \\
OSS-REQ-0136 & Co-added Exposures \\
OSS-REQ-0316 & Wavefront Sensor Data \\
\end{tabular}
\\ \hline
\end{longtable}
}


\subsubsection{Test Cases Summary}
\begin{longtable}{p{3cm}p{2.5cm}p{2.5cm}p{3cm}p{4cm}}
\toprule
\href{https://jira.lsstcorp.org/secure/Tests.jspa\#/testCase/LVV-T16}{LVV-T16} & \multicolumn{4}{p{12cm}}{ DRP-00-35: Scientific Verification of Coadd Images } \\ \hline
\textbf{Owner} & \textbf{Status} & \textbf{Version} & \textbf{Critical Event} & \textbf{Verification Type} \\ \hline
Jim Bosch & Deprecated & 1 & false & Test \\ \hline
\end{longtable}
{\scriptsize
\textbf{Objective:}\\
This test will check that the coadded images delivered by the DRP
science payload meet the requirements laid down by \citeds{LSE-61}.\\
Specifically, this will demonstrate that:

\begin{itemize}
\tightlist
\item
  Coadds have been generated and persisted during payload execution;~
\item
  Each coadd provides a spatially varying PSF model (DMS-REQ-0047).
\item
  Saturated pixels are correctly masked.
\item
  Pixels affected by satellite trails and ghosts are rejected from the
  coadd.
\item
  The background is not oversubtracted around bright objects.
\end{itemize}

This test does not include quantitative targets for the science quality
criteria; we instead require for each test that we be able to quickly
construct a plot or display summary images that allow such a target can
be visualized.\\[2\baselineskip]
}
\begin{longtable}{p{3cm}p{2.5cm}p{2.5cm}p{3cm}p{4cm}}
\toprule
\href{https://jira.lsstcorp.org/secure/Tests.jspa\#/testCase/LVV-T62}{LVV-T62} & \multicolumn{4}{p{12cm}}{ Verify implementation of Provide PSF for Coadded Images } \\ \hline
\textbf{Owner} & \textbf{Status} & \textbf{Version} & \textbf{Critical Event} & \textbf{Verification Type} \\ \hline
Jim Bosch & Approved & 2 & false & Test \\ \hline
\end{longtable}
{\scriptsize
\textbf{Objective:}\\
Verify that all coadd images produced by the DRP pipelines include a
model from which an image of the PSF at any point on the coadd can be
obtained.
}
\begin{longtable}{p{3cm}p{2.5cm}p{2.5cm}p{3cm}p{4cm}}
\toprule
\href{https://jira.lsstcorp.org/secure/Tests.jspa\#/testCase/LVV-T62}{LVV-T62} & \multicolumn{4}{p{12cm}}{ Verify implementation of Provide PSF for Coadded Images } \\ \hline
\textbf{Owner} & \textbf{Status} & \textbf{Version} & \textbf{Critical Event} & \textbf{Verification Type} \\ \hline
Jim Bosch & Approved & 2 & false & Test \\ \hline
\end{longtable}
{\scriptsize
\textbf{Objective:}\\
Verify that all coadd images produced by the DRP pipelines include a
model from which an image of the PSF at any point on the coadd can be
obtained.
}
  
 \newpage 
\subsection{[LVV-21] DMS-REQ-0052-V-01: Enable a Range of Shape Measurement Approaches }\label{lvv-21}

\begin{longtable}{cccc}
\hline
\textbf{Jira Link} & \textbf{Assignee} & \textbf{Status} & \textbf{Test Cases}\\ \hline
\href{https://jira.lsstcorp.org/browse/LVV-21}{LVV-21} &
Jim Bosch & Not Covered &
\begin{tabular}{c}
LVV-T130 \\
\end{tabular}
\\
\hline
\end{longtable}

\textbf{Verification Element Description:} \\
Demonstrate that the results of multiple shape models are available from
Sources, Objects and ForcedSources and that this information can be
obtained simultaneously using data from multiple exposures.

{\footnotesize
\begin{longtable}{p{2.5cm}p{13.5cm}}
\hline
\multicolumn{2}{c}{\textbf{Requirement Details}}\\ \hline
Requirement ID & DMS-REQ-0052 \\ \cdashline{1-2}
Requirement Description &
\begin{minipage}[]{13cm}
\textbf{Specification:} The DMS shall provide for the use of a variety
of shape models on multiple kinds of input data to measure sources:
measurement on coadds; measurement on coadds using information (e.g.,
PSFs) extracted from the individual exposures; measurement based on all
the information from the individual Exposures simultaneously.
\end{minipage}
\\ \cdashline{1-2}
Requirement Discussion &
\begin{minipage}[]{13cm}
\textbf{Discussion:} The most appropriate measurement model to apply
depends upon the nature of the composite source.
\end{minipage}
\\ \cdashline{1-2}
Requirement Priority & 1b \\ \cdashline{1-2}
Upper Level Requirement &
\begin{tabular}{cl}
OSS-REQ-0137 & Catalogs (Level 2) \\
\end{tabular}
\\ \hline
\end{longtable}
}


\subsubsection{Test Cases Summary}
\begin{longtable}{p{3cm}p{2.5cm}p{2.5cm}p{3cm}p{4cm}}
\toprule
\href{https://jira.lsstcorp.org/secure/Tests.jspa\#/testCase/LVV-T130}{LVV-T130} & \multicolumn{4}{p{12cm}}{ Verify implementation of Enable a Range of Shape Measurement Approaches } \\ \hline
\textbf{Owner} & \textbf{Status} & \textbf{Version} & \textbf{Critical Event} & \textbf{Verification Type} \\ \hline
Colin Slater & Draft & 1 & false & Test \\ \hline
\end{longtable}
{\scriptsize
\textbf{Objective:}\\
Verify that multiple shape measurement algorithms can be used.
}
  
 \newpage 
\subsection{[LVV-22] DMS-REQ-0059-V-01: Bad Pixel Map }\label{lvv-22}

\begin{longtable}{cccc}
\hline
\textbf{Jira Link} & \textbf{Assignee} & \textbf{Status} & \textbf{Test Cases}\\ \hline
\href{https://jira.lsstcorp.org/browse/LVV-22}{LVV-22} &
Jim Bosch & Not Covered &
\begin{tabular}{c}
LVV-T83 \\
\end{tabular}
\\
\hline
\end{longtable}

\textbf{Verification Element Description:} \\
32bits is a minimum requirement. To verify we need to check that it is
at least 32-bit. The product is an image file in unspecified format.
(May want an additional requirement that these data can also be
visualized directly on a web page as part of SUIT). Request the map for
any date, compare with camera team understanding.

{\footnotesize
\begin{longtable}{p{2.5cm}p{13.5cm}}
\hline
\multicolumn{2}{c}{\textbf{Requirement Details}}\\ \hline
Requirement ID & DMS-REQ-0059 \\ \cdashline{1-2}
Requirement Description &
\begin{minipage}[]{13cm}
\textbf{Specification:} The DMS shall produce on an as-needed basis a
map of detector pixels that are affected by one or more pathologies,
such as non-responsive pixels, charge traps, and hot pixels. The
particular pathologies shall be bit-encoded in, at least, 32-bit pixel
values, so that additional pathologies may also be recorded in
down-stream processing software.
\end{minipage}
\\ \cdashline{1-2}
Requirement Discussion &
\begin{minipage}[]{13cm}
\textbf{Discussion:} The fraction of bad pixels is expected to be small.
Therefore the Reference Map, while logically equivalent to an image, may
be stored in a more compressible form.
\end{minipage}
\\ \cdashline{1-2}
Requirement Priority & 1a \\ \cdashline{1-2}
Upper Level Requirement &
\begin{tabular}{cl}
OSS-REQ-0271 & Supported Image Types \\
DMS-REQ-0058 & Correct for Instrument Sensitivity Variation \\
OSS-REQ-0129 & Exposures (Level 1) \\
\end{tabular}
\\ \hline
\end{longtable}
}


\subsubsection{Test Cases Summary}
\begin{longtable}{p{3cm}p{2.5cm}p{2.5cm}p{3cm}p{4cm}}
\toprule
\href{https://jira.lsstcorp.org/secure/Tests.jspa\#/testCase/LVV-T83}{LVV-T83} & \multicolumn{4}{p{12cm}}{ Verify implementation of Bad Pixel Map } \\ \hline
\textbf{Owner} & \textbf{Status} & \textbf{Version} & \textbf{Critical Event} & \textbf{Verification Type} \\ \hline
Robert Lupton & Defined & 1 & false & Test \\ \hline
\end{longtable}
{\scriptsize
\textbf{Objective:}\\
Verify that the DMS can produce a map of detector pixels that suffer
from pathologies, and that these pathologies are encoded in at least
32-bit values.
}
  
 \newpage 
\subsection{[LVV-23] DMS-REQ-0060-V-01: Bias Residual Image }\label{lvv-23}

\begin{longtable}{cccc}
\hline
\textbf{Jira Link} & \textbf{Assignee} & \textbf{Status} & \textbf{Test Cases}\\ \hline
\href{https://jira.lsstcorp.org/browse/LVV-23}{LVV-23} &
Jim Bosch & Not Covered &
\begin{tabular}{c}
LVV-T84 \\
LVV-T368 \\
LVV-T368 \\
\end{tabular}
\\
\hline
\end{longtable}

\textbf{Verification Element Description:} \\
Can be done with simulated raw calibration data. Need to define whether
``as-needed'' is manual trigger or automation. Can this be done with the
camera in the lab?

{\footnotesize
\begin{longtable}{p{2.5cm}p{13.5cm}}
\hline
\multicolumn{2}{c}{\textbf{Requirement Details}}\\ \hline
Requirement ID & DMS-REQ-0060 \\ \cdashline{1-2}
Requirement Description &
\begin{minipage}[]{13cm}
\textbf{Specification:} The DMS shall construct on an as-needed basis an
image that corrects for any temporally stable bias structure that
remains after overscan correction. The Bias Residual shall be
constructed from multiple, zero-second exposures where the overscan
correction has been applied.
\end{minipage}
\\ \cdashline{1-2}
Requirement Priority & 1a \\ \cdashline{1-2}
Upper Level Requirement &
\begin{tabular}{cl}
DMS-REQ-0055 & Correct for Camera Bias Structure \\
OSS-REQ-0271 & Supported Image Types \\
OSS-REQ-0046 & Calibration \\
\end{tabular}
\\ \hline
\end{longtable}
}


\subsubsection{Test Cases Summary}
\begin{longtable}{p{3cm}p{2.5cm}p{2.5cm}p{3cm}p{4cm}}
\toprule
\href{https://jira.lsstcorp.org/secure/Tests.jspa\#/testCase/LVV-T84}{LVV-T84} & \multicolumn{4}{p{12cm}}{ Verify implementation of Bias Residual Image } \\ \hline
\textbf{Owner} & \textbf{Status} & \textbf{Version} & \textbf{Critical Event} & \textbf{Verification Type} \\ \hline
Robert Lupton & Defined & 1 & false & Test \\ \hline
\end{longtable}
{\scriptsize
\textbf{Objective:}\\
Verify that DMS can construct a bias residual image that corrects for
temporally-stable bias structures.\\
Verify that DMS can do this on demand.
}
\begin{longtable}{p{3cm}p{2.5cm}p{2.5cm}p{3cm}p{4cm}}
\toprule
\href{https://jira.lsstcorp.org/secure/Tests.jspa\#/testCase/LVV-T368}{LVV-T368} & \multicolumn{4}{p{12cm}}{ Loading and processing Camera test data } \\ \hline
\textbf{Owner} & \textbf{Status} & \textbf{Version} & \textbf{Critical Event} & \textbf{Verification Type} \\ \hline
John Swinbank & Approved & 2 & false & Test \\ \hline
\end{longtable}
{\scriptsize
\textbf{Objective:}\\
This test will check:

\begin{itemize}
\tightlist
\item
  That Camera test data is available for processing in the LSST Data
  Facility, and accessible through the LSST Science Platform;
\item
  That the Data Management I/O abstraction (the ``Data Butler'') can
  load that data into the Science Platform environment;
\item
  That Data Management algorithmic ``tasks'' can be executed to process
  that data;
\item
  That results can be displayed in the Firefly display tool.
\end{itemize}
}
\begin{longtable}{p{3cm}p{2.5cm}p{2.5cm}p{3cm}p{4cm}}
\toprule
\href{https://jira.lsstcorp.org/secure/Tests.jspa\#/testCase/LVV-T368}{LVV-T368} & \multicolumn{4}{p{12cm}}{ Loading and processing Camera test data } \\ \hline
\textbf{Owner} & \textbf{Status} & \textbf{Version} & \textbf{Critical Event} & \textbf{Verification Type} \\ \hline
John Swinbank & Approved & 2 & false & Test \\ \hline
\end{longtable}
{\scriptsize
\textbf{Objective:}\\
This test will check:

\begin{itemize}
\tightlist
\item
  That Camera test data is available for processing in the LSST Data
  Facility, and accessible through the LSST Science Platform;
\item
  That the Data Management I/O abstraction (the ``Data Butler'') can
  load that data into the Science Platform environment;
\item
  That Data Management algorithmic ``tasks'' can be executed to process
  that data;
\item
  That results can be displayed in the Firefly display tool.
\end{itemize}
}
  
 \newpage 
\subsection{[LVV-24] DMS-REQ-0061-V-01: Crosstalk Correction Matrix }\label{lvv-24}

\begin{longtable}{cccc}
\hline
\textbf{Jira Link} & \textbf{Assignee} & \textbf{Status} & \textbf{Test Cases}\\ \hline
\href{https://jira.lsstcorp.org/browse/LVV-24}{LVV-24} &
Jim Bosch & Not Covered &
\begin{tabular}{c}
LVV-T85 \\
\end{tabular}
\\
\hline
\end{longtable}

\textbf{Verification Element Description:} \\
Needs commissioning data to determine ``as-needed'' timeline. Can
demonstrate algorithms prior to commissioning by taking darks in the
lab.

{\footnotesize
\begin{longtable}{p{2.5cm}p{13.5cm}}
\hline
\multicolumn{2}{c}{\textbf{Requirement Details}}\\ \hline
Requirement ID & DMS-REQ-0061 \\ \cdashline{1-2}
Requirement Description &
\begin{minipage}[]{13cm}
\textbf{Specification:} The DMS shall, on an as-needed basis, determine
from appropriate calibration data what fraction of the signal detected
in any given amplifier on each sensor in the focal plane appears in any
other amplifier, and shall record that fraction in a correction matrix.
The applicability of the correction matrix shall be verified in
production processing on science data.
\end{minipage}
\\ \cdashline{1-2}
Requirement Discussion &
\begin{minipage}[]{13cm}
\textbf{Discussion:} The frequency with which the Cross-talk Correction
Matrix must be computed will be determined during Commissioning and
monitored during operations.
\end{minipage}
\\ \cdashline{1-2}
Requirement Priority & 1a \\ \cdashline{1-2}
Upper Level Requirement &
\begin{tabular}{cl}
OSS-REQ-0329 & Crosstalk Accuracy \\
OSS-REQ-0330 & Crosstalk Measureability \\
DMS-REQ-0056 & Correct for Camera Crosstalk \\
OSS-REQ-0349 & Data Release Production Crosstalk Correction \\
\end{tabular}
\\ \hline
\end{longtable}
}


\subsubsection{Test Cases Summary}
\begin{longtable}{p{3cm}p{2.5cm}p{2.5cm}p{3cm}p{4cm}}
\toprule
\href{https://jira.lsstcorp.org/secure/Tests.jspa\#/testCase/LVV-T85}{LVV-T85} & \multicolumn{4}{p{12cm}}{ Verify implementation of Crosstalk Correction Matrix } \\ \hline
\textbf{Owner} & \textbf{Status} & \textbf{Version} & \textbf{Critical Event} & \textbf{Verification Type} \\ \hline
Robert Lupton & Defined & 1 & false & Test \\ \hline
\end{longtable}
{\scriptsize
\textbf{Objective:}\\
Verify that the DMS can generate a cross-talk correction matrix from
appropriate calibration data.\\
Verify that the DMS can measure the effectiveness of the cross-talk
correction matrix.
}
  
 \newpage 
\subsection{[LVV-25] DMS-REQ-0062-V-01: Illumination Correction Frame }\label{lvv-25}

\begin{longtable}{cccc}
\hline
\textbf{Jira Link} & \textbf{Assignee} & \textbf{Status} & \textbf{Test Cases}\\ \hline
\href{https://jira.lsstcorp.org/browse/LVV-25}{LVV-25} &
Jim Bosch & Not Covered &
\begin{tabular}{c}
LVV-T86 \\
\end{tabular}
\\
\hline
\end{longtable}

\textbf{Verification Element Description:} \\
Needs a real camera during commissioning and data taken in the correct
mode. Can possibly be done prior to commissioning with simulated data.

{\footnotesize
\begin{longtable}{p{2.5cm}p{13.5cm}}
\hline
\multicolumn{2}{c}{\textbf{Requirement Details}}\\ \hline
Requirement ID & DMS-REQ-0062 \\ \cdashline{1-2}
Requirement Description &
\begin{minipage}[]{13cm}
\textbf{Specification:} The DMS shall produce on an as-needed basis an
image that corrects for the non-uniform illumination of the flat-field
calibration apparatus on the focal plane. The effectiveness of the
Illumination Correction shall be verified in production processing on
science data.
\end{minipage}
\\ \cdashline{1-2}
Requirement Discussion &
\begin{minipage}[]{13cm}
\textbf{Discussion:} The Illumination correction is anticipated to be
quite stable. Updates to the correction should be no more frequent than
monthly.
\end{minipage}
\\ \cdashline{1-2}
Requirement Priority & 1b \\ \cdashline{1-2}
Upper Level Requirement &
\begin{tabular}{cl}
OSS-REQ-0271 & Supported Image Types \\
OSS-REQ-0046 & Calibration \\
DMS-REQ-0058 & Correct for Instrument Sensitivity Variation \\
\end{tabular}
\\ \hline
\end{longtable}
}


\subsubsection{Test Cases Summary}
\begin{longtable}{p{3cm}p{2.5cm}p{2.5cm}p{3cm}p{4cm}}
\toprule
\href{https://jira.lsstcorp.org/secure/Tests.jspa\#/testCase/LVV-T86}{LVV-T86} & \multicolumn{4}{p{12cm}}{ Verify implementation of Illumination Correction Frame } \\ \hline
\textbf{Owner} & \textbf{Status} & \textbf{Version} & \textbf{Critical Event} & \textbf{Verification Type} \\ \hline
Robert Lupton & Draft & 1 & false & Test \\ \hline
\end{longtable}
{\scriptsize
\textbf{Objective:}\\
Verify that the DMS can produce an illumination correction frame
calibration product.\\
Verify that the DMS can determine the effectiveness of an illumination
correction and determine how often it should be updated.
}
  
 \newpage 
\subsection{[LVV-26] DMS-REQ-0063-V-01: Monochromatic Flatfield Data Cube }\label{lvv-26}

\begin{longtable}{cccc}
\hline
\textbf{Jira Link} & \textbf{Assignee} & \textbf{Status} & \textbf{Test Cases}\\ \hline
\href{https://jira.lsstcorp.org/browse/LVV-26}{LVV-26} &
Jim Bosch & Not Covered &
\begin{tabular}{c}
LVV-T87 \\
\end{tabular}
\\
\hline
\end{longtable}

\textbf{Verification Element Description:} \\
Needs a real camera during commissioning and data taken in the correct
mode. Possibly can be done with simulated data and lab measurements.

{\footnotesize
\begin{longtable}{p{2.5cm}p{13.5cm}}
\hline
\multicolumn{2}{c}{\textbf{Requirement Details}}\\ \hline
Requirement ID & DMS-REQ-0063 \\ \cdashline{1-2}
Requirement Description &
\begin{minipage}[]{13cm}
\textbf{Specification:} The DMS shall produce on an as-needed basis an
image that corrects for the color-dependent, pixel-to-pixel
non-uniformity in the detector response. The images in the cube shall be
constructed from exposures at multiple wavelengths of a uniformly
illuminated source. The effectiveness of the flat-field shall be
verified in production processing on science data.
\end{minipage}
\\ \cdashline{1-2}
Requirement Discussion &
\begin{minipage}[]{13cm}
\textbf{Discussion:} Monochromatic flat-fields are expected to be
produced no more frequently than monthly, owing to the time required to
obtain the exposures.
\end{minipage}
\\ \cdashline{1-2}
Requirement Priority & 1b \\ \cdashline{1-2}
Upper Level Requirement &
\begin{tabular}{cl}
OSS-REQ-0271 & Supported Image Types \\
OSS-REQ-0046 & Calibration \\
DMS-REQ-0058 & Correct for Instrument Sensitivity Variation \\
DMS-REQ-0057 & Correct for Detector Fringing \\
\end{tabular}
\\ \hline
\end{longtable}
}


\subsubsection{Test Cases Summary}
\begin{longtable}{p{3cm}p{2.5cm}p{2.5cm}p{3cm}p{4cm}}
\toprule
\href{https://jira.lsstcorp.org/secure/Tests.jspa\#/testCase/LVV-T87}{LVV-T87} & \multicolumn{4}{p{12cm}}{ Verify implementation of Monochromatic Flatfield Data Cube } \\ \hline
\textbf{Owner} & \textbf{Status} & \textbf{Version} & \textbf{Critical Event} & \textbf{Verification Type} \\ \hline
Robert Lupton & Draft & 1 & false & Test \\ \hline
\end{longtable}
{\scriptsize
\textbf{Objective:}\\
Verify that the DMS can generate a calibration image/cube that corrects
for pixel-to-pixel wavelength-dependent detector response.\\
Verify that the DMS can measure the effectiveness of this monochromatic
flatfield data cube.
}
  
 \newpage 
\subsection{[LVV-27] DMS-REQ-0065-V-01: Provide Image Access Services }\label{lvv-27}

\begin{longtable}{cccc}
\hline
\textbf{Jira Link} & \textbf{Assignee} & \textbf{Status} & \textbf{Test Cases}\\ \hline
\href{https://jira.lsstcorp.org/browse/LVV-27}{LVV-27} &
Gregory Dubois-Felsmann & Not Covered &
\begin{tabular}{c}
LVV-T134 \\
\end{tabular}
\\
\hline
\end{longtable}

\textbf{Verification Element Description:} \\
Could be verified by DMS-REQ-0298. Demonstrate that SIA can be used to
retrieve image data.

{\footnotesize
\begin{longtable}{p{2.5cm}p{13.5cm}}
\hline
\multicolumn{2}{c}{\textbf{Requirement Details}}\\ \hline
Requirement ID & DMS-REQ-0065 \\ \cdashline{1-2}
Requirement Description &
\begin{minipage}[]{13cm}
\textbf{Specification:} The DMS shall provide a service for Image Access
through community data access protocols, to support programmatic search
and retrieval of images or image cut-outs. The service shall support one
or more community standard formats, including the LSST pipeline input
format.
\end{minipage}
\\ \cdashline{1-2}
Requirement Discussion &
\begin{minipage}[]{13cm}
\textbf{Discussion:} At least the FITS image format will be supported
though an IVOA-standard service such as SIAP. Other image formats such
as JPG may be more compatible with education/public outreach needs.
\end{minipage}
\\ \cdashline{1-2}
Requirement Priority & 1b \\ \cdashline{1-2}
Upper Level Requirement &
\begin{tabular}{cl}
OSS-REQ-0180 & Data Products Query and Download Availability \\
OSS-REQ-0176 & Data Access \\
OSS-REQ-0181 & Data Products Query and Download Infrastructure \\
DMS-REQ-0066 & Keep Exposure Archive \\
\end{tabular}
\\ \hline
\end{longtable}
}

\subsubsection{Verified By}
\begin{itemize}
\item . LVV-10004 (\ref{lvv-10004}) DMS-API-REQ-0028-V-01: Access to Image Data in FITS Format\_1
\item . LVV-10016 (\ref{lvv-10016}) DMS-API-REQ-0016-V-01: SIA Service for Image Availability\_1
\item . LVV-10017 (\ref{lvv-10017}) DMS-API-REQ-0018-V-01: Cutout Service\_1
\item . LVV-10018 (\ref{lvv-10018}) DMS-API-REQ-0017-V-01: SODA Service for Image Data\_1
\end{itemize}

\subsubsection{Test Cases Summary}
\begin{longtable}{p{3cm}p{2.5cm}p{2.5cm}p{3cm}p{4cm}}
\toprule
\href{https://jira.lsstcorp.org/secure/Tests.jspa\#/testCase/LVV-T134}{LVV-T134} & \multicolumn{4}{p{12cm}}{ Verify implementation of Provide Image Access Services } \\ \hline
\textbf{Owner} & \textbf{Status} & \textbf{Version} & \textbf{Critical Event} & \textbf{Verification Type} \\ \hline
Gregory Dubois-Felsmann & Draft & 1 & false & Inspection \\ \hline
\end{longtable}
{\scriptsize
\textbf{Objective:}\\
Verify that images can be identified and that images and image cut-outs
can be retrieved using the network interfaces - primarily IVOA
standards-based - and Python APIs provided for image access by science
users.
}
  
 \newpage 
\subsection{[LVV-28] DMS-REQ-0068-V-01: Raw Science Image Metadata }\label{lvv-28}

\begin{longtable}{cccc}
\hline
\textbf{Jira Link} & \textbf{Assignee} & \textbf{Status} & \textbf{Test Cases}\\ \hline
\href{https://jira.lsstcorp.org/browse/LVV-28}{LVV-28} &
Gregory Dubois-Felsmann & Not Covered &
\begin{tabular}{c}
LVV-T33 \\
LVV-T283 \\
LVV-T284 \\
LVV-T286 \\
LVV-T1549 \\
LVV-T1550 \\
LVV-T1556 \\
\end{tabular}
\\
\hline
\end{longtable}

\textbf{Verification Element Description:} \\
This is a more specific restatement of DMS-REQ-0024. Can be done with
simulated camera DAQ and OCS. Compare against ICD. Test that the
metadata placed on the OCS middleware by the simulated OCS is the same
as that stored in the image metadata.

{\footnotesize
\begin{longtable}{p{2.5cm}p{13.5cm}}
\hline
\multicolumn{2}{c}{\textbf{Requirement Details}}\\ \hline
Requirement ID & DMS-REQ-0068 \\ \cdashline{1-2}
Requirement Description &
\begin{minipage}[]{13cm}
\textbf{Specification:} For each raw science image, the DMS shall store
image metadata including at least:

\begin{itemize}
\tightlist
\item
  Time of exposure start and end, referenced to TAI, and DUT1
\item
  Site metadata (site seeing, transparency, weather, observatory
  location)
\item
  Telescope metadata (telescope pointing, active optics state,
  environmental state)
\item
  Camera metadata (shutter trajectory, wavefront sensors, environmental
  state)
\item
  Program metadata (identifier for main survey, deep drilling, etc.)
\item
  Scheduler metadata (visitID, intended number of exposures in the
  visit)
\end{itemize}
\end{minipage}
\\ \cdashline{1-2}
Requirement Discussion &
\begin{minipage}[]{13cm}
\textbf{Discussion:} The program metadata should be sufficient to
associate an image with a specific Special Program so that DMS-REQ-0320
and DMS-REQ-0397 can be satisfied. The scheduler metadata should
sufficiently inform the processing pipelines regarding e.g., deviations
from 2-snap 30 second visits, so that computational resources can be
appropriately allocated, and so that DMS-REQ-0320 can be satisfied.
\end{minipage}
\\ \cdashline{1-2}
Requirement Priority & 1a \\ \cdashline{1-2}
Upper Level Requirement &
\begin{tabular}{cl}
OSS-REQ-0122 & Provenance \\
DMS-REQ-0320 & Processing of Data From Special Programs \\
DMS-REQ-0066 & Keep Exposure Archive \\
OSS-REQ-0171 & Engineering and Facilities Data \\
\end{tabular}
\\ \hline
\end{longtable}
}


\subsubsection{Test Cases Summary}
\begin{longtable}{p{3cm}p{2.5cm}p{2.5cm}p{3cm}p{4cm}}
\toprule
\href{https://jira.lsstcorp.org/secure/Tests.jspa\#/testCase/LVV-T33}{LVV-T33} & \multicolumn{4}{p{12cm}}{ Verify implementation of Raw Science Image Metadata } \\ \hline
\textbf{Owner} & \textbf{Status} & \textbf{Version} & \textbf{Critical Event} & \textbf{Verification Type} \\ \hline
Kian-Tat Lim & Defined & 1 & false & Test \\ \hline
\end{longtable}
{\scriptsize
\textbf{Objective:}\\
Verify successful ingestion of raw data from L1 Test Stand DAQ and that
image metadata is present and queryable.
}
\begin{longtable}{p{3cm}p{2.5cm}p{2.5cm}p{3cm}p{4cm}}
\toprule
\href{https://jira.lsstcorp.org/secure/Tests.jspa\#/testCase/LVV-T283}{LVV-T283} & \multicolumn{4}{p{12cm}}{ RAS-00-00: Writing well-formed raw image } \\ \hline
\textbf{Owner} & \textbf{Status} & \textbf{Version} & \textbf{Critical Event} & \textbf{Verification Type} \\ \hline
Michelle Butler & Approved & 1 & false & Test \\ \hline
\end{longtable}
{\scriptsize
\textbf{Objective:}\\
This test will check:\\

\begin{itemize}
\tightlist
\item
  The successful integration of the Pathfinder components with the DM
  Header Service and the Level 1 Archiver;
\item
  That the raw images are well-formed and meet specifications in
  change-controlled documents \citeds{LSE-61};
\end{itemize}

~This Test Case shall be repeated for each of the different cameras
(ATScam, LSSTCam) and sensors (Science, Wavefront, and Guider)
combination.
}
\begin{longtable}{p{3cm}p{2.5cm}p{2.5cm}p{3cm}p{4cm}}
\toprule
\href{https://jira.lsstcorp.org/secure/Tests.jspa\#/testCase/LVV-T284}{LVV-T284} & \multicolumn{4}{p{12cm}}{ RAS-00-05: (LDM-503-8b) Writing data from CCOB to the DBB for further
data processing } \\ \hline
\textbf{Owner} & \textbf{Status} & \textbf{Version} & \textbf{Critical Event} & \textbf{Verification Type} \\ \hline
Michelle Butler & Draft & 1 & false & Test \\ \hline
\end{longtable}
{\scriptsize
\textbf{Objective:}\\
This test will check:

\begin{itemize}
\tightlist
\item
  The successful integration of the DAQ archiver components with the
  CCOB
\item
  That the file can then be ingested into the DBB and be retrieved for
  further analysis
\end{itemize}
}
\begin{longtable}{p{3cm}p{2.5cm}p{2.5cm}p{3cm}p{4cm}}
\toprule
\href{https://jira.lsstcorp.org/secure/Tests.jspa\#/testCase/LVV-T286}{LVV-T286} & \multicolumn{4}{p{12cm}}{ RAS-00-20: Raw image are part of the permanent record of survey via DBB } \\ \hline
\textbf{Owner} & \textbf{Status} & \textbf{Version} & \textbf{Critical Event} & \textbf{Verification Type} \\ \hline
Michelle Butler & Approved & 1 & false & Test \\ \hline
\end{longtable}
{\scriptsize
\textbf{Objective:}\\
This test will check:\\[2\baselineskip]

\begin{itemize}
\tightlist
\item
  That the handoff of a raw image from the Level 1 Archiver Service to
  the DBB buffer manager is successful;
\item
  That the raw image is ingested into the Data Backbone successfully;
\item
  That the monitoring of the above items is successful;
\end{itemize}

This Test Case shall be repeated for each of the different cameras
(ATScam, LSSTCam) and sensors (Science, Wavefront, and Guider)
combination.\\[2\baselineskip]Note: For a complete check of the various
aspects of what it means for a raw image to be in the Data Backbone, see
the tests for the Data Backbone.
}
\begin{longtable}{p{3cm}p{2.5cm}p{2.5cm}p{3cm}p{4cm}}
\toprule
\href{https://jira.lsstcorp.org/secure/Tests.jspa\#/testCase/LVV-T1549}{LVV-T1549} & \multicolumn{4}{p{12cm}}{ LDM-503-6 Comcam verification readiness } \\ \hline
\textbf{Owner} & \textbf{Status} & \textbf{Version} & \textbf{Critical Event} & \textbf{Verification Type} \\ \hline
Michelle Butler & Approved & 1 & false & Demonstration \\ \hline
\end{longtable}
{\scriptsize
\textbf{Objective:}\\
Verify that ComCam has all the services running and verified working for
retrieving an image from the ComCam DAQ and store it on file systems at
the LDF for viewing by RSP. ~
}
\begin{longtable}{p{3cm}p{2.5cm}p{2.5cm}p{3cm}p{4cm}}
\toprule
\href{https://jira.lsstcorp.org/secure/Tests.jspa\#/testCase/LVV-T1550}{LVV-T1550} & \multicolumn{4}{p{12cm}}{ LDM-503-10 DAQ Validation } \\ \hline
\textbf{Owner} & \textbf{Status} & \textbf{Version} & \textbf{Critical Event} & \textbf{Verification Type} \\ \hline
Michelle Butler & Approved & 1 & false & Demonstration \\ \hline
\end{longtable}
{\scriptsize
\textbf{Objective:}\\
Verify that the DAQ can talk to test machines at the BDC through the
DWDM network.~
}
\begin{longtable}{p{3cm}p{2.5cm}p{2.5cm}p{3cm}p{4cm}}
\toprule
\href{https://jira.lsstcorp.org/secure/Tests.jspa\#/testCase/LVV-T1556}{LVV-T1556} & \multicolumn{4}{p{12cm}}{ LDM-503-10B Large Scale CCOB Data Access } \\ \hline
\textbf{Owner} & \textbf{Status} & \textbf{Version} & \textbf{Critical Event} & \textbf{Verification Type} \\ \hline
Michelle Butler & Draft & 1 & false & Demonstration \\ \hline
\end{longtable}
{\scriptsize
\textbf{Objective:}\\
Demonstrate the ability to transfer data from the SLAC test stand or
CCOB with 21 rafts from SLAC and ingested at NCSA and make available
through an instance of the RSP
}
  
 \newpage 
\subsection{[LVV-29] DMS-REQ-0069-V-01: Processed Visit Images }\label{lvv-29}

\begin{longtable}{cccc}
\hline
\textbf{Jira Link} & \textbf{Assignee} & \textbf{Status} & \textbf{Test Cases}\\ \hline
\href{https://jira.lsstcorp.org/browse/LVV-29}{LVV-29} &
Jim Bosch & Not Covered &
\begin{tabular}{c}
LVV-T15 \\
LVV-T18 \\
LVV-T19 \\
LVV-T38 \\
LVV-T362 \\
\end{tabular}
\\
\hline
\end{longtable}

\textbf{Verification Element Description:} \\
Use simulated raw data in format from DMS-REQ-0024. Run end-to-end test
for one visit. Check that Processed Visit images have been created in
expected format.

{\footnotesize
\begin{longtable}{p{2.5cm}p{13.5cm}}
\hline
\multicolumn{2}{c}{\textbf{Requirement Details}}\\ \hline
Requirement ID & DMS-REQ-0069 \\ \cdashline{1-2}
Requirement Description &
\begin{minipage}[]{13cm}
\textbf{Specification:} The DMS shall produce Processed Visit Images, in
which the corresponding raw sensor array data has been trimmed of
overscan and corrected for instrumental signature, including crosstalk.
Images obtained in pairs during a standard visit are combined.
\end{minipage}
\\ \cdashline{1-2}
Requirement Discussion &
\begin{minipage}[]{13cm}
\textbf{Discussion:} Processed science exposures are not archived, and
are retained for only a limited time to facilitate down-stream
processing. They will be re-generated for users on-demand using the
latest processing software and calibrations. This aspect of the
processing for Special Programs data is specific to each program.
\end{minipage}
\\ \cdashline{1-2}
Requirement Priority & 1a \\ \cdashline{1-2}
Upper Level Requirement &
\begin{tabular}{cl}
OSS-REQ-0129 & Exposures (Level 1) \\
OSS-REQ-0349 & Data Release Production Crosstalk Correction \\
OSS-REQ-0348 & Alert Production Crosstalk Correction \\
OSS-REQ-0328 & Crosstalk Aggressor Limits \\
\end{tabular}
\\ \hline
\end{longtable}
}


\subsubsection{Test Cases Summary}
\begin{longtable}{p{3cm}p{2.5cm}p{2.5cm}p{3cm}p{4cm}}
\toprule
\href{https://jira.lsstcorp.org/secure/Tests.jspa\#/testCase/LVV-T15}{LVV-T15} & \multicolumn{4}{p{12cm}}{ DRP-00-30: Scientific Verification of Processed Visit Images } \\ \hline
\textbf{Owner} & \textbf{Status} & \textbf{Version} & \textbf{Critical Event} & \textbf{Verification Type} \\ \hline
Jim Bosch & Deprecated & 1 & false & Test \\ \hline
\end{longtable}
{\scriptsize
\textbf{Objective:}\\
This test will check that the Processed Visit Images (PVIs) delivered by
the DRP science payload meet the requirements laid down by \citeds{LSE-61}.\\
Specifically, this will demonstrate that:

\begin{itemize}
\tightlist
\item
  Processed visit images have been generated and persisted during
  payload execution;
\item
  Each PVI includes a background model (DMS-REQ-0327), photometric
  zero-point (DMS- REQ-0029), spatially-varying PSF (DMS-REQ-0070) and
  WCS (DMS-REQ-0030).
\item
  Saturated pixels are correctly masked.
\item
  Pixels affected by cosmic rays are correctly masked.
\item
  The background is not oversubtracted around bright objects.
\end{itemize}

This test does not include quantitative targets for the science quality
criteria; we instead re- quire for each test that we be able to quickly
construct a plot or display summary images that allow such a target can
be visualized.
}
\begin{longtable}{p{3cm}p{2.5cm}p{2.5cm}p{3cm}p{4cm}}
\toprule
\href{https://jira.lsstcorp.org/secure/Tests.jspa\#/testCase/LVV-T18}{LVV-T18} & \multicolumn{4}{p{12cm}}{ AG-00-05: Alert Generation Produces Required Data Products } \\ \hline
\textbf{Owner} & \textbf{Status} & \textbf{Version} & \textbf{Critical Event} & \textbf{Verification Type} \\ \hline
Eric Bellm & Deprecated & 1 & false & Test \\ \hline
\end{longtable}
{\scriptsize
\textbf{Objective:}\\
This test will check that the basic data products produced by Alert
Generation are generated by execution of the science payload.\\
These products will include:

\begin{itemize}
\tightlist
\item
  Processed visit images (PVIs; DMS-REQ-0069);
\item
  Difference Exposures (DMS-REQ-0010);
\item
  DIASource catalogs (DMS-REQ-0269);
\item
  DIAObject catalogs (DMS-REQ-0271);
\end{itemize}
}
\begin{longtable}{p{3cm}p{2.5cm}p{2.5cm}p{3cm}p{4cm}}
\toprule
\href{https://jira.lsstcorp.org/secure/Tests.jspa\#/testCase/LVV-T19}{LVV-T19} & \multicolumn{4}{p{12cm}}{ AG-00-10: Scientific Verification of Processed Visit Images } \\ \hline
\textbf{Owner} & \textbf{Status} & \textbf{Version} & \textbf{Critical Event} & \textbf{Verification Type} \\ \hline
Eric Bellm & Deprecated & 1 & false & Test \\ \hline
\end{longtable}
{\scriptsize
\textbf{Objective:}\\
This test will check that the Processed Visit Images (PVIs) delivered by
the alert generation science payload meet the requirements laid down by
\citeds{LSE-61}.\\
Specifically, this will demonstrate that:

\begin{itemize}
\tightlist
\item
  Processed visit images have been generated and persisted during
  payload execution;
\item
  Each PVI includes a science pixel array, a mask array, and a variance
  array. (DMS-REQ-0072).
\item
  Each PVI includes a background model (DMS-REQ-0327), photometric
  zero-point (DMS- REQ-0029), spatially-varying PSF (DMS-REQ-0070) and
  WCS (DMS-REQ-0030).
\item
  Saturated pixels are correctly masked.
\item
  Pixels affected by cosmic rays are correctly masked.
\item
  The background is not oversubtracted around bright objects.
\end{itemize}

This test does not include quantitative targets for the science quality
criteria.
}
\begin{longtable}{p{3cm}p{2.5cm}p{2.5cm}p{3cm}p{4cm}}
\toprule
\href{https://jira.lsstcorp.org/secure/Tests.jspa\#/testCase/LVV-T38}{LVV-T38} & \multicolumn{4}{p{12cm}}{ Verify implementation of Processed Visit Images } \\ \hline
\textbf{Owner} & \textbf{Status} & \textbf{Version} & \textbf{Critical Event} & \textbf{Verification Type} \\ \hline
Eric Bellm & Defined & 1 & false & Test \\ \hline
\end{longtable}
{\scriptsize
\textbf{Objective:}\\
Verify that the DMS\\
1. Successfully produces Processed Visit Images, where the instrument
signature has been removed.\\
2. Successfully combines images obtained during a standard
visit.\\[2\baselineskip]The verification should include confirming that
the images have been trimmed of the overscan, and that correction of the
instrumental signature (including crosstalk) has been applied properly.
}
\begin{longtable}{p{3cm}p{2.5cm}p{2.5cm}p{3cm}p{4cm}}
\toprule
\href{https://jira.lsstcorp.org/secure/Tests.jspa\#/testCase/LVV-T362}{LVV-T362} & \multicolumn{4}{p{12cm}}{ Installation of the LSST Science Pipelines Payloads } \\ \hline
\textbf{Owner} & \textbf{Status} & \textbf{Version} & \textbf{Critical Event} & \textbf{Verification Type} \\ \hline
John Swinbank & Approved & 1 & false & Test \\ \hline
\end{longtable}
{\scriptsize
\textbf{Objective:}\\
This test will check that:

\begin{itemize}
\tightlist
\item
  The Alert Production Pipeline payload is available for installation
  from documented channels;
\item
  The Data Release Production Pipeline payload is available for
  installation from documented channels;
\item
  The Calibration Products Production Pipeline payload is available for
  installation from documented channels;
\item
  These payloads can be installed on systems at the LSST Data Facility
  following available documentation;
\item
  The installed pipeline payloads are capable of successfully executing
  basic integration tests.
\end{itemize}

Note that this test assumes a 2018-era packaging of the Science
Pipelines software, in which all the above payloads are represented by a
single ``meta-package'', lsst\_distrib.
}
  
 \newpage 
\subsection{[LVV-30] DMS-REQ-0070-V-01: Generate PSF for Visit Images }\label{lvv-30}

\begin{longtable}{cccc}
\hline
\textbf{Jira Link} & \textbf{Assignee} & \textbf{Status} & \textbf{Test Cases}\\ \hline
\href{https://jira.lsstcorp.org/browse/LVV-30}{LVV-30} &
Jim Bosch & Not Covered &
\begin{tabular}{c}
LVV-T15 \\
LVV-T19 \\
LVV-T41 \\
\end{tabular}
\\
\hline
\end{longtable}

\textbf{Verification Element Description:} \\
Can be checked with any test data. No requirement on accuracy. Just test
that a PSF model can be retrieved from any location in the Processed
Visit.

{\footnotesize
\begin{longtable}{p{2.5cm}p{13.5cm}}
\hline
\multicolumn{2}{c}{\textbf{Requirement Details}}\\ \hline
Requirement ID & DMS-REQ-0070 \\ \cdashline{1-2}
Requirement Description &
\begin{minipage}[]{13cm}
\textbf{Specification:} The DMS shall determine a characterization of
the PSF for any specified location in Processed Visit Images.
\end{minipage}
\\ \cdashline{1-2}
Requirement Priority & 1b \\ \cdashline{1-2}
Upper Level Requirement &
\begin{tabular}{cl}
OSS-REQ-0056 & System Monitoring \& Diagnostics \\
DMS-REQ-0116 & Extended Object Shape Parameters \\
\end{tabular}
\\ \hline
\end{longtable}
}


\subsubsection{Test Cases Summary}
\begin{longtable}{p{3cm}p{2.5cm}p{2.5cm}p{3cm}p{4cm}}
\toprule
\href{https://jira.lsstcorp.org/secure/Tests.jspa\#/testCase/LVV-T15}{LVV-T15} & \multicolumn{4}{p{12cm}}{ DRP-00-30: Scientific Verification of Processed Visit Images } \\ \hline
\textbf{Owner} & \textbf{Status} & \textbf{Version} & \textbf{Critical Event} & \textbf{Verification Type} \\ \hline
Jim Bosch & Deprecated & 1 & false & Test \\ \hline
\end{longtable}
{\scriptsize
\textbf{Objective:}\\
This test will check that the Processed Visit Images (PVIs) delivered by
the DRP science payload meet the requirements laid down by \citeds{LSE-61}.\\
Specifically, this will demonstrate that:

\begin{itemize}
\tightlist
\item
  Processed visit images have been generated and persisted during
  payload execution;
\item
  Each PVI includes a background model (DMS-REQ-0327), photometric
  zero-point (DMS- REQ-0029), spatially-varying PSF (DMS-REQ-0070) and
  WCS (DMS-REQ-0030).
\item
  Saturated pixels are correctly masked.
\item
  Pixels affected by cosmic rays are correctly masked.
\item
  The background is not oversubtracted around bright objects.
\end{itemize}

This test does not include quantitative targets for the science quality
criteria; we instead re- quire for each test that we be able to quickly
construct a plot or display summary images that allow such a target can
be visualized.
}
\begin{longtable}{p{3cm}p{2.5cm}p{2.5cm}p{3cm}p{4cm}}
\toprule
\href{https://jira.lsstcorp.org/secure/Tests.jspa\#/testCase/LVV-T19}{LVV-T19} & \multicolumn{4}{p{12cm}}{ AG-00-10: Scientific Verification of Processed Visit Images } \\ \hline
\textbf{Owner} & \textbf{Status} & \textbf{Version} & \textbf{Critical Event} & \textbf{Verification Type} \\ \hline
Eric Bellm & Deprecated & 1 & false & Test \\ \hline
\end{longtable}
{\scriptsize
\textbf{Objective:}\\
This test will check that the Processed Visit Images (PVIs) delivered by
the alert generation science payload meet the requirements laid down by
\citeds{LSE-61}.\\
Specifically, this will demonstrate that:

\begin{itemize}
\tightlist
\item
  Processed visit images have been generated and persisted during
  payload execution;
\item
  Each PVI includes a science pixel array, a mask array, and a variance
  array. (DMS-REQ-0072).
\item
  Each PVI includes a background model (DMS-REQ-0327), photometric
  zero-point (DMS- REQ-0029), spatially-varying PSF (DMS-REQ-0070) and
  WCS (DMS-REQ-0030).
\item
  Saturated pixels are correctly masked.
\item
  Pixels affected by cosmic rays are correctly masked.
\item
  The background is not oversubtracted around bright objects.
\end{itemize}

This test does not include quantitative targets for the science quality
criteria.
}
\begin{longtable}{p{3cm}p{2.5cm}p{2.5cm}p{3cm}p{4cm}}
\toprule
\href{https://jira.lsstcorp.org/secure/Tests.jspa\#/testCase/LVV-T41}{LVV-T41} & \multicolumn{4}{p{12cm}}{ Verify implementation of Generate PSF for Visit Images } \\ \hline
\textbf{Owner} & \textbf{Status} & \textbf{Version} & \textbf{Critical Event} & \textbf{Verification Type} \\ \hline
Jim Bosch & Approved & 1 & false & Test \\ \hline
\end{longtable}
{\scriptsize
\textbf{Objective:}\\
Verify that Processed Visit Images produced by the DRP and AP pipelines
are associated with a model from which one can obtain an image of the
PSF given a point on the image.
}
  
 \newpage 
\subsection{[LVV-31] DMS-REQ-0072-V-01: Processed Visit Image Content }\label{lvv-31}

\begin{longtable}{cccc}
\hline
\textbf{Jira Link} & \textbf{Assignee} & \textbf{Status} & \textbf{Test Cases}\\ \hline
\href{https://jira.lsstcorp.org/browse/LVV-31}{LVV-31} &
Jim Bosch & Not Covered &
\begin{tabular}{c}
LVV-T15 \\
LVV-T19 \\
LVV-T42 \\
\end{tabular}
\\
\hline
\end{longtable}

\textbf{Verification Element Description:} \\
Take output from DMS-REQ-0069 and compare against the processed visit
ICD.

{\footnotesize
\begin{longtable}{p{2.5cm}p{13.5cm}}
\hline
\multicolumn{2}{c}{\textbf{Requirement Details}}\\ \hline
Requirement ID & DMS-REQ-0072 \\ \cdashline{1-2}
Requirement Description &
\begin{minipage}[]{13cm}
\textbf{Specification:} Processed visit images shall include the
corrected science pixel array, an integer mask array where each
bit-plane represents a logical statement about whether a particular
detector pathology affects the pixel, a variance array which represents
the expected variance in the corresponding science pixel, and a
representation of the spatially varying PSF that applies over the extent
of the science array. These images shall also contain metadata that map
pixel to world (sky) coordinates (the WCS) as well as metadata from
which photometric measurements can be derived.
\end{minipage}
\\ \cdashline{1-2}
Requirement Priority & 1a \\ \cdashline{1-2}
Upper Level Requirement &
\begin{tabular}{cl}
OSS-REQ-0129 & Exposures (Level 1) \\
DMS-REQ-0066 & Keep Exposure Archive \\
\end{tabular}
\\ \hline
\end{longtable}
}


\subsubsection{Test Cases Summary}
\begin{longtable}{p{3cm}p{2.5cm}p{2.5cm}p{3cm}p{4cm}}
\toprule
\href{https://jira.lsstcorp.org/secure/Tests.jspa\#/testCase/LVV-T15}{LVV-T15} & \multicolumn{4}{p{12cm}}{ DRP-00-30: Scientific Verification of Processed Visit Images } \\ \hline
\textbf{Owner} & \textbf{Status} & \textbf{Version} & \textbf{Critical Event} & \textbf{Verification Type} \\ \hline
Jim Bosch & Deprecated & 1 & false & Test \\ \hline
\end{longtable}
{\scriptsize
\textbf{Objective:}\\
This test will check that the Processed Visit Images (PVIs) delivered by
the DRP science payload meet the requirements laid down by \citeds{LSE-61}.\\
Specifically, this will demonstrate that:

\begin{itemize}
\tightlist
\item
  Processed visit images have been generated and persisted during
  payload execution;
\item
  Each PVI includes a background model (DMS-REQ-0327), photometric
  zero-point (DMS- REQ-0029), spatially-varying PSF (DMS-REQ-0070) and
  WCS (DMS-REQ-0030).
\item
  Saturated pixels are correctly masked.
\item
  Pixels affected by cosmic rays are correctly masked.
\item
  The background is not oversubtracted around bright objects.
\end{itemize}

This test does not include quantitative targets for the science quality
criteria; we instead re- quire for each test that we be able to quickly
construct a plot or display summary images that allow such a target can
be visualized.
}
\begin{longtable}{p{3cm}p{2.5cm}p{2.5cm}p{3cm}p{4cm}}
\toprule
\href{https://jira.lsstcorp.org/secure/Tests.jspa\#/testCase/LVV-T19}{LVV-T19} & \multicolumn{4}{p{12cm}}{ AG-00-10: Scientific Verification of Processed Visit Images } \\ \hline
\textbf{Owner} & \textbf{Status} & \textbf{Version} & \textbf{Critical Event} & \textbf{Verification Type} \\ \hline
Eric Bellm & Deprecated & 1 & false & Test \\ \hline
\end{longtable}
{\scriptsize
\textbf{Objective:}\\
This test will check that the Processed Visit Images (PVIs) delivered by
the alert generation science payload meet the requirements laid down by
\citeds{LSE-61}.\\
Specifically, this will demonstrate that:

\begin{itemize}
\tightlist
\item
  Processed visit images have been generated and persisted during
  payload execution;
\item
  Each PVI includes a science pixel array, a mask array, and a variance
  array. (DMS-REQ-0072).
\item
  Each PVI includes a background model (DMS-REQ-0327), photometric
  zero-point (DMS- REQ-0029), spatially-varying PSF (DMS-REQ-0070) and
  WCS (DMS-REQ-0030).
\item
  Saturated pixels are correctly masked.
\item
  Pixels affected by cosmic rays are correctly masked.
\item
  The background is not oversubtracted around bright objects.
\end{itemize}

This test does not include quantitative targets for the science quality
criteria.
}
\begin{longtable}{p{3cm}p{2.5cm}p{2.5cm}p{3cm}p{4cm}}
\toprule
\href{https://jira.lsstcorp.org/secure/Tests.jspa\#/testCase/LVV-T42}{LVV-T42} & \multicolumn{4}{p{12cm}}{ Verify implementation of Processed Visit Image Content } \\ \hline
\textbf{Owner} & \textbf{Status} & \textbf{Version} & \textbf{Critical Event} & \textbf{Verification Type} \\ \hline
Jim Bosch & Defined & 1 & false & Test \\ \hline
\end{longtable}
{\scriptsize
\textbf{Objective:}\\
Verify that Processed Visit Images produced by the DRP and AP pipelines
include the observed data, a mask array, a variance array, a PSF model,
and a WCS model.
}
  
 \newpage 
\subsection{[LVV-33] DMS-REQ-0075-V-01: Catalog Queries }\label{lvv-33}

\begin{longtable}{cccc}
\hline
\textbf{Jira Link} & \textbf{Assignee} & \textbf{Status} & \textbf{Test Cases}\\ \hline
\href{https://jira.lsstcorp.org/browse/LVV-33}{LVV-33} &
Colin Slater & Not Covered &
\begin{tabular}{c}
LVV-T149 \\
LVV-T1085 \\
LVV-T1086 \\
LVV-T1087 \\
\end{tabular}
\\
\hline
\end{longtable}

\textbf{Verification Element Description:} \\
Using a TAP service, send an ADQL query and verify that the results are
as expected.

{\footnotesize
\begin{longtable}{p{2.5cm}p{13.5cm}}
\hline
\multicolumn{2}{c}{\textbf{Requirement Details}}\\ \hline
Requirement ID & DMS-REQ-0075 \\ \cdashline{1-2}
Requirement Description &
\begin{minipage}[]{13cm}
\textbf{Specification:} The catalogs shall be queryable with a
structured language, such as SQL.
\end{minipage}
\\ \cdashline{1-2}
Requirement Discussion &
\begin{minipage}[]{13cm}
\textbf{Discussion:} Queries are expected to be generated via Science
User Interfaces, and software within and external to DMS, including VO
clients. The queries may be translated to (and optimized for) the native
query language of the DMS database server.
\end{minipage}
\\ \cdashline{1-2}
Requirement Priority & 1a \\ \cdashline{1-2}
Upper Level Requirement &
\begin{tabular}{cl}
DMS-REQ-0076 & Keep Science Data Archive \\
OSS-REQ-0176 & Data Access \\
\end{tabular}
\\ \hline
\end{longtable}
}


\subsubsection{Test Cases Summary}
\begin{longtable}{p{3cm}p{2.5cm}p{2.5cm}p{3cm}p{4cm}}
\toprule
\href{https://jira.lsstcorp.org/secure/Tests.jspa\#/testCase/LVV-T149}{LVV-T149} & \multicolumn{4}{p{12cm}}{ Verify implementation of Catalog Queries } \\ \hline
\textbf{Owner} & \textbf{Status} & \textbf{Version} & \textbf{Critical Event} & \textbf{Verification Type} \\ \hline
Colin Slater & Approved & 1 & false & Test \\ \hline
\end{longtable}
{\scriptsize
\textbf{Objective:}\\
Verify that SQL, or a similar structured language, can be used to query
catalogs.
}
\begin{longtable}{p{3cm}p{2.5cm}p{2.5cm}p{3cm}p{4cm}}
\toprule
\href{https://jira.lsstcorp.org/secure/Tests.jspa\#/testCase/LVV-T1085}{LVV-T1085} & \multicolumn{4}{p{12cm}}{ Short Queries Functional Test } \\ \hline
\textbf{Owner} & \textbf{Status} & \textbf{Version} & \textbf{Critical Event} & \textbf{Verification Type} \\ \hline
Fritz Mueller & Approved & 1 & false & Test \\ \hline
\end{longtable}
{\scriptsize
\textbf{Objective:}\\
The objective of this test is to ensure that the short queries are
performing as expected and establish a timing baseline benchmark for
these types of queries.
}
\begin{longtable}{p{3cm}p{2.5cm}p{2.5cm}p{3cm}p{4cm}}
\toprule
\href{https://jira.lsstcorp.org/secure/Tests.jspa\#/testCase/LVV-T1086}{LVV-T1086} & \multicolumn{4}{p{12cm}}{ Full Table Scans Functional Test } \\ \hline
\textbf{Owner} & \textbf{Status} & \textbf{Version} & \textbf{Critical Event} & \textbf{Verification Type} \\ \hline
Fritz Mueller & Approved & 1 & false & Test \\ \hline
\end{longtable}
{\scriptsize
\textbf{Objective:}\\
The objective of this test is to ensure that the full table scan queries
are performing as expected and establish a timing baseline benchmark for
these types of queries.
}
\begin{longtable}{p{3cm}p{2.5cm}p{2.5cm}p{3cm}p{4cm}}
\toprule
\href{https://jira.lsstcorp.org/secure/Tests.jspa\#/testCase/LVV-T1087}{LVV-T1087} & \multicolumn{4}{p{12cm}}{ Full Table Joins Functional Test } \\ \hline
\textbf{Owner} & \textbf{Status} & \textbf{Version} & \textbf{Critical Event} & \textbf{Verification Type} \\ \hline
Fritz Mueller & Approved & 1 & false & Test \\ \hline
\end{longtable}
{\scriptsize
\textbf{Objective:}\\
The objective of this test is to ensure that the full table join queries
are performing as expected and establish a timing baseline benchmark for
these types of queries.
}
  
 \newpage 
\subsection{[LVV-38] DMS-REQ-0096-V-01: Generate Data Quality Report Within Specified Time }\label{lvv-38}

\begin{longtable}{cccc}
\hline
\textbf{Jira Link} & \textbf{Assignee} & \textbf{Status} & \textbf{Test Cases}\\ \hline
\href{https://jira.lsstcorp.org/browse/LVV-38}{LVV-38} &
Simon Krughoff & Not Covered &
\begin{tabular}{c}
LVV-T103 \\
\end{tabular}
\\
\hline
\end{longtable}

\textbf{Verification Element Description:} \\
Reduce a night of L1 data. Wait for report to appear. Is it on time? Is
it human-readable? ``Machine-readable'' is a database table or a text
file. The clock begins when Prompt Processing ends in the morning.

{\footnotesize
\begin{longtable}{p{2.5cm}p{13.5cm}}
\hline
\multicolumn{2}{c}{\textbf{Requirement Details}}\\ \hline
Requirement ID & DMS-REQ-0096 \\ \cdashline{1-2}
Requirement Description &
\begin{minipage}[]{13cm}
\textbf{Specification:} The DMS shall generate a nightly Data Quality
Report within time \textbf{dqReportComplTime} in both human-readable and
machine-readable forms.
\end{minipage}
\\ \cdashline{1-2}
Requirement Parameters & \textbf{dqReportComplTime = 4{{[}hour{]}}} Latency for producing Level 1
Data Quality Report. \\ \cdashline{1-2}
Requirement Discussion &
\begin{minipage}[]{13cm}
\textbf{Discussion:} The Report must be timely in order to evaluate
whether changes to hardware, software, or procedures are needed for the
following night's observing.
\end{minipage}
\\ \cdashline{1-2}
Requirement Priority & 1a \\ \cdashline{1-2}
Upper Level Requirement &
\begin{tabular}{cl}
OSS-REQ-0131 & Nightly Summary Products \\
\end{tabular}
\\ \hline
\end{longtable}
}


\subsubsection{Test Cases Summary}
\begin{longtable}{p{3cm}p{2.5cm}p{2.5cm}p{3cm}p{4cm}}
\toprule
\href{https://jira.lsstcorp.org/secure/Tests.jspa\#/testCase/LVV-T103}{LVV-T103} & \multicolumn{4}{p{12cm}}{ Verify implementation of Generate Data Quality Report Within Specified
Time } \\ \hline
\textbf{Owner} & \textbf{Status} & \textbf{Version} & \textbf{Critical Event} & \textbf{Verification Type} \\ \hline
Kian-Tat Lim & Defined & 1 & false & Test \\ \hline
\end{longtable}
{\scriptsize
\textbf{Objective:}\\
Verify that the DMS can generate a nightly L1 Data Quality Report within
\textbf{dqReportComplTime = 4{[}hour{]}}, in both human- and
machine-readable formats.
}
  
 \newpage 
\subsection{[LVV-39] DMS-REQ-0097-V-01: Level 1 Data Quality Report Definition }\label{lvv-39}

\begin{longtable}{cccc}
\hline
\textbf{Jira Link} & \textbf{Assignee} & \textbf{Status} & \textbf{Test Cases}\\ \hline
\href{https://jira.lsstcorp.org/browse/LVV-39}{LVV-39} &
Simon Krughoff & Not Covered &
\begin{tabular}{c}
LVV-T45 \\
\end{tabular}
\\
\hline
\end{longtable}

\textbf{Verification Element Description:} \\
Run multiple visits through the L1 pipeline (can start with raw data
files), check that report is created. The report is a dynamic UI as well
as a static document.

{\footnotesize
\begin{longtable}{p{2.5cm}p{13.5cm}}
\hline
\multicolumn{2}{c}{\textbf{Requirement Details}}\\ \hline
Requirement ID & DMS-REQ-0097 \\ \cdashline{1-2}
Requirement Description &
\begin{minipage}[]{13cm}
\textbf{Specification:} The DMS shall produce a Level 1 Data Quality
Report that contains indicators of data quality that result from running
the DMS pipelines, including at least: Photometric zero point vs. time
for each utilized filter; Sky brightness vs. time for each utilized
filter; seeing vs. time for each utilized filter; PSF parameters vs.
time for each utilized filter; detection efficiency for point sources
vs. mag for each utilized filter.
\end{minipage}
\\ \cdashline{1-2}
Requirement Discussion &
\begin{minipage}[]{13cm}
\textbf{Discussion:} The seeing report is intended as a broad-brush
measure of image quality. The PSF parameters provide more detail, as
they include asymmetries and field location dependence.
\end{minipage}
\\ \cdashline{1-2}
Requirement Priority & 1a \\ \cdashline{1-2}
Upper Level Requirement &
\begin{tabular}{cl}
OSS-REQ-0131 & Nightly Summary Products \\
DMS-REQ-0096 & Generate Data Quality Report Within Specified Time \\
\end{tabular}
\\ \hline
\end{longtable}
}


\subsubsection{Test Cases Summary}
\begin{longtable}{p{3cm}p{2.5cm}p{2.5cm}p{3cm}p{4cm}}
\toprule
\href{https://jira.lsstcorp.org/secure/Tests.jspa\#/testCase/LVV-T45}{LVV-T45} & \multicolumn{4}{p{12cm}}{ Verify implementation of Prompt Processing Data Quality Report
Definition } \\ \hline
\textbf{Owner} & \textbf{Status} & \textbf{Version} & \textbf{Critical Event} & \textbf{Verification Type} \\ \hline
Eric Bellm & Defined & 1 & false & Test \\ \hline
\end{longtable}
{\scriptsize
\textbf{Objective:}\\
Verify that the DMS produces a Prompt Processing Data Quality Report.
~Specifically check absolute value and temporal variation of\\
1. Photometric zeropoint\\
2. Sky brightness\\
3. Seeing\\
4. PSF\\
5. Detection efficiency
}
  
 \newpage 
\subsection{[LVV-41] DMS-REQ-0099-V-01: Level 1 Performance Report Definition }\label{lvv-41}

\begin{longtable}{cccc}
\hline
\textbf{Jira Link} & \textbf{Assignee} & \textbf{Status} & \textbf{Test Cases}\\ \hline
\href{https://jira.lsstcorp.org/browse/LVV-41}{LVV-41} &
Robert Gruendl & Not Covered &
\begin{tabular}{c}
LVV-T46 \\
\end{tabular}
\\
\hline
\end{longtable}

\textbf{Verification Element Description:} \\
Run multiple visits through the L1 pipeline (can start with raw data
files; optimally an entire night), check that report is created. The
report is a dynamic UI as well as a static document.

{\footnotesize
\begin{longtable}{p{2.5cm}p{13.5cm}}
\hline
\multicolumn{2}{c}{\textbf{Requirement Details}}\\ \hline
Requirement ID & DMS-REQ-0099 \\ \cdashline{1-2}
Requirement Description &
\begin{minipage}[]{13cm}
\textbf{Specification:} The DMS shall produce a Level 1 Performance
Report that provides indicators of how the DMS has performed in
processing the night's observations, including at least: number of
observations successfully processed through each pipeline; number of
observations for each pipeline that had recoverable failures (with a
record of the failure type and recovery mechanism); number of
observations for each pipeline that had unrecoverable failures; number
of observations archived at each DMS Facility; number of observations
satisfying the science criteria for each active science program.
\end{minipage}
\\ \cdashline{1-2}
Requirement Priority & 1b \\ \cdashline{1-2}
Upper Level Requirement &
\begin{tabular}{cl}
DMS-REQ-0098 & Generate DMS Performance Report Within Specified Time \\
OSS-REQ-0131 & Nightly Summary Products \\
\end{tabular}
\\ \hline
\end{longtable}
}


\subsubsection{Test Cases Summary}
\begin{longtable}{p{3cm}p{2.5cm}p{2.5cm}p{3cm}p{4cm}}
\toprule
\href{https://jira.lsstcorp.org/secure/Tests.jspa\#/testCase/LVV-T46}{LVV-T46} & \multicolumn{4}{p{12cm}}{ Verify implementation of Prompt Processing Performance Report Definition } \\ \hline
\textbf{Owner} & \textbf{Status} & \textbf{Version} & \textbf{Critical Event} & \textbf{Verification Type} \\ \hline
Eric Bellm & Draft & 1 & false & Test \\ \hline
\end{longtable}
{\scriptsize
\textbf{Objective:}\\
Verify that the DMS produces a Prompt Processing Performance Report.
~Specifically check that the number of observations that describe each
of the following:\\
1. Successfully processed, recoverable failures, unrecoverable
failures.\\
2. Archived\\
3. Result in science.\\[2\baselineskip]This is testing more the
processing rather than the observatory system.
}
  
 \newpage 
\subsection{[LVV-43] DMS-REQ-0101-V-01: Level 1 Calibration Report Definition }\label{lvv-43}

\begin{longtable}{cccc}
\hline
\textbf{Jira Link} & \textbf{Assignee} & \textbf{Status} & \textbf{Test Cases}\\ \hline
\href{https://jira.lsstcorp.org/browse/LVV-43}{LVV-43} &
Robert Lupton & Not Covered &
\begin{tabular}{c}
LVV-T47 \\
\end{tabular}
\\
\hline
\end{longtable}

\textbf{Verification Element Description:} \\
Using precursor and simulated calibration data, run the L1 calibration
pipeline and check report. The report is dynamic and triggers alerts if
calibrations go out of range. Check a static report is created.

{\footnotesize
\begin{longtable}{p{2.5cm}p{13.5cm}}
\hline
\multicolumn{2}{c}{\textbf{Requirement Details}}\\ \hline
Requirement ID & DMS-REQ-0101 \\ \cdashline{1-2}
Requirement Description &
\begin{minipage}[]{13cm}
\textbf{Specification:} The DMS shall produce a Level 1 Calibration
Report that provides a summary of significant differences in Calibration
Images that may indicate evolving problems with the telescope or camera,
including a nightly broad-band flat in each filter.
\end{minipage}
\\ \cdashline{1-2}
Requirement Priority & 1a \\ \cdashline{1-2}
Upper Level Requirement &
\begin{tabular}{cl}
OSS-REQ-0131 & Nightly Summary Products \\
DMS-REQ-0100 & Generate Calibration Report Within Specified Time \\
\end{tabular}
\\ \hline
\end{longtable}
}


\subsubsection{Test Cases Summary}
\begin{longtable}{p{3cm}p{2.5cm}p{2.5cm}p{3cm}p{4cm}}
\toprule
\href{https://jira.lsstcorp.org/secure/Tests.jspa\#/testCase/LVV-T47}{LVV-T47} & \multicolumn{4}{p{12cm}}{ Verify implementation of Prompt Processing Calibration Report Definition } \\ \hline
\textbf{Owner} & \textbf{Status} & \textbf{Version} & \textbf{Critical Event} & \textbf{Verification Type} \\ \hline
Eric Bellm & Defined & 1 & false & Test \\ \hline
\end{longtable}
{\scriptsize
\textbf{Objective:}\\
Verify that the DMS produces a Prompt Processing Calibration Report.
~Specifically check that this report is capable of identifying when
aspects of the telescope or camera are changing with time.
}
  
 \newpage 
\subsection{[LVV-46] DMS-REQ-0106-V-01: Coadded Image Provenance }\label{lvv-46}

\begin{longtable}{cccc}
\hline
\textbf{Jira Link} & \textbf{Assignee} & \textbf{Status} & \textbf{Test Cases}\\ \hline
\href{https://jira.lsstcorp.org/browse/LVV-46}{LVV-46} &
Robert Gruendl & Not Covered &
\begin{tabular}{c}
LVV-T11 \\
LVV-T64 \\
\end{tabular}
\\
\hline
\end{longtable}

\textbf{Verification Element Description:} \\
Given a coadd downloaded from the archive. Request provenance
information. Regenerate coadd. Compare download with newly created
coadd. Can this use the L3 system?

{\footnotesize
\begin{longtable}{p{2.5cm}p{13.5cm}}
\hline
\multicolumn{2}{c}{\textbf{Requirement Details}}\\ \hline
Requirement ID & DMS-REQ-0106 \\ \cdashline{1-2}
Requirement Description &
\begin{minipage}[]{13cm}
\textbf{Specification:} For each Coadded Image, DMS shall store: the
list of input images and the pipeline parameters, including software
versions, used to derive it, and a sufficient set of metadata attributes
for users to re-create them in whole or in part.
\end{minipage}
\\ \cdashline{1-2}
Requirement Discussion &
\begin{minipage}[]{13cm}
\textbf{Discussion:} Not all coadded image types will be made available
to end-users or retained for the life of the survey; however, sufficient
metadata will be preserved so that they may be recreated by end-users.
\end{minipage}
\\ \cdashline{1-2}
Requirement Priority & 1b \\ \cdashline{1-2}
Upper Level Requirement &
\begin{tabular}{cl}
OSS-REQ-0122 & Provenance \\
DMS-REQ-0104 & Produce Co-Added Exposures \\
\end{tabular}
\\ \hline
\end{longtable}
}


\subsubsection{Test Cases Summary}
\begin{longtable}{p{3cm}p{2.5cm}p{2.5cm}p{3cm}p{4cm}}
\toprule
\href{https://jira.lsstcorp.org/secure/Tests.jspa\#/testCase/LVV-T11}{LVV-T11} & \multicolumn{4}{p{12cm}}{ DRP-00-05: Execution of the DRP Science Payload by the Batch Production
Service } \\ \hline
\textbf{Owner} & \textbf{Status} & \textbf{Version} & \textbf{Critical Event} & \textbf{Verification Type} \\ \hline
Jim Bosch & Deprecated & 1 & false & Test \\ \hline
\end{longtable}
{\scriptsize
\textbf{Objective:}\\
This test will check that the DRP Science Payload can be executed using
a specific version of the Batch Production Service provided by the LSST
Data Facility. Since the outputs are stored in the Data Backbone, it too
is a component of this test.
}
\begin{longtable}{p{3cm}p{2.5cm}p{2.5cm}p{3cm}p{4cm}}
\toprule
\href{https://jira.lsstcorp.org/secure/Tests.jspa\#/testCase/LVV-T64}{LVV-T64} & \multicolumn{4}{p{12cm}}{ Verify implementation of Coadded Image Provenance } \\ \hline
\textbf{Owner} & \textbf{Status} & \textbf{Version} & \textbf{Critical Event} & \textbf{Verification Type} \\ \hline
Jim Bosch & Draft & 1 & false & Test \\ \hline
\end{longtable}
{\scriptsize
\textbf{Objective:}\\
Verify that all coadd data products produced by the DRP pipelines are
associated with provenance information that includes the set of input
epochs contributing to that coadd as well as any additional information
needed to exactly produce that coadd.
}
  
 \newpage 
\subsection{[LVV-48] DMS-REQ-0120-V-01: Level 3 Data Product Self Consistency }\label{lvv-48}

\begin{longtable}{cccc}
\hline
\textbf{Jira Link} & \textbf{Assignee} & \textbf{Status} & \textbf{Test Cases}\\ \hline
\href{https://jira.lsstcorp.org/browse/LVV-48}{LVV-48} &
Robert Gruendl & Not Covered &
\begin{tabular}{c}
LVV-T118 \\
\end{tabular}
\\
\hline
\end{longtable}

\textbf{Verification Element Description:} \\
This verification is hard. All you can do is inspect the APIs to ensure
that missed DRs can not happen without being explicit, and ensure that
the butler can be configured to access a specific DR.

{\footnotesize
\begin{longtable}{p{2.5cm}p{13.5cm}}
\hline
\multicolumn{2}{c}{\textbf{Requirement Details}}\\ \hline
Requirement ID & DMS-REQ-0120 \\ \cdashline{1-2}
Requirement Description &
\begin{minipage}[]{13cm}
\textbf{Specification:} The DMS shall provide a means for ensuring that
users' Level 3 processing tasks can be carried out on self-consistent
inputs - i.e., catalogs, images, metadata, calibrations, camera
configuration data, etc., that match each other and all arise from
consistent Level 1 and Level 2 processings.
\end{minipage}
\\ \cdashline{1-2}
Requirement Priority & 2 \\ \cdashline{1-2}
Upper Level Requirement &
\begin{tabular}{cl}
OSS-REQ-0120 & Consistency \\
OSS-REQ-0118 & Consistency and Completeness \\
\end{tabular}
\\ \hline
\end{longtable}
}


\subsubsection{Test Cases Summary}
\begin{longtable}{p{3cm}p{2.5cm}p{2.5cm}p{3cm}p{4cm}}
\toprule
\href{https://jira.lsstcorp.org/secure/Tests.jspa\#/testCase/LVV-T118}{LVV-T118} & \multicolumn{4}{p{12cm}}{ Verify implementation of Level 3 Data Product Self Consistency } \\ \hline
\textbf{Owner} & \textbf{Status} & \textbf{Version} & \textbf{Critical Event} & \textbf{Verification Type} \\ \hline
Colin Slater & Draft & 1 & false & Test \\ \hline
\end{longtable}
{\scriptsize
\textbf{Objective:}\\
Verify that user-driven Level 3 processing is conducted on consistent
sets of input data.
}
  
 \newpage 
\subsection{[LVV-49] DMS-REQ-0121-V-01: Provenance for Level 3 processing at DACs }\label{lvv-49}

\begin{longtable}{cccc}
\hline
\textbf{Jira Link} & \textbf{Assignee} & \textbf{Status} & \textbf{Test Cases}\\ \hline
\href{https://jira.lsstcorp.org/browse/LVV-49}{LVV-49} &
Robert Gruendl & Not Covered &
\begin{tabular}{c}
LVV-T119 \\
\end{tabular}
\\
\hline
\end{longtable}

\textbf{Verification Element Description:} \\
Show that an API exists for reading and writing provenance information
in a L3 environment.

{\footnotesize
\begin{longtable}{p{2.5cm}p{13.5cm}}
\hline
\multicolumn{2}{c}{\textbf{Requirement Details}}\\ \hline
Requirement ID & DMS-REQ-0121 \\ \cdashline{1-2}
Requirement Description &
\begin{minipage}[]{13cm}
\textbf{Specification:} The DMS shall provide a means for recording
provenance information for Level 3 processing that is performed at DACs,
covering at least all the DMS-provided inputs to the processing (e.g.,
catalog data used as inputs, dataset metadata, calibrations and camera
data from the EFD).
\end{minipage}
\\ \cdashline{1-2}
Requirement Discussion &
\begin{minipage}[]{13cm}
\textbf{Discussion:} The DMS should also provide an optional means for
Level 3 processing users at DACs to maintain basic provenance
information on their own inputs to a processing task, such as code or
additional calibration data.\\
\textbf{Rationale:} the DMS should facilitate Level 3 processing users
in being able to carry out their work in a reproducible way.
\end{minipage}
\\ \cdashline{1-2}
Requirement Priority & 2 \\ \cdashline{1-2}
Upper Level Requirement &
\begin{tabular}{cl}
OSS-REQ-0122 & Provenance \\
\end{tabular}
\\ \hline
\end{longtable}
}


\subsubsection{Test Cases Summary}
\begin{longtable}{p{3cm}p{2.5cm}p{2.5cm}p{3cm}p{4cm}}
\toprule
\href{https://jira.lsstcorp.org/secure/Tests.jspa\#/testCase/LVV-T119}{LVV-T119} & \multicolumn{4}{p{12cm}}{ Verify implementation of Provenance for Level 3 processing at DACs } \\ \hline
\textbf{Owner} & \textbf{Status} & \textbf{Version} & \textbf{Critical Event} & \textbf{Verification Type} \\ \hline
Colin Slater & Draft & 1 & false & Test \\ \hline
\end{longtable}
{\scriptsize
\textbf{Objective:}\\
Verify that provenance information is recorded and accessible for
user-generated Level 3 products.
}
  
 \newpage 
\subsection{[LVV-52] DMS-REQ-0124-V-01: Federation with external catalogs }\label{lvv-52}

\begin{longtable}{cccc}
\hline
\textbf{Jira Link} & \textbf{Assignee} & \textbf{Status} & \textbf{Test Cases}\\ \hline
\href{https://jira.lsstcorp.org/browse/LVV-52}{LVV-52} &
Gregory Dubois-Felsmann & Not Covered &
\begin{tabular}{c}
LVV-T206 \\
\end{tabular}
\\
\hline
\end{longtable}

\textbf{Verification Element Description:} \\
Show that an external catalog can be combined with L1/2/3 catalogs. Show
that the specification document exists. Show that more that at least one
community standard is supported.

{\footnotesize
\begin{longtable}{p{2.5cm}p{13.5cm}}
\hline
\multicolumn{2}{c}{\textbf{Requirement Details}}\\ \hline
Requirement ID & DMS-REQ-0124 \\ \cdashline{1-2}
Requirement Description &
\begin{minipage}[]{13cm}
\textbf{Specification:} The DMS shall provide a means for federating
Level 1, 2, and 3 catalogs with externally provided catalogs, for joint
analysis. The DMS shall provide specifications for how external data
must be provided in order for this to be achieved. The DMS shall strive
to support community standards in this regard, including, but not
limited to, virtual observatory facilities that may be available during
the project lifetime.
\end{minipage}
\\ \cdashline{1-2}
Requirement Priority & 2 \\ \cdashline{1-2}
Upper Level Requirement &
\begin{tabular}{cl}
OSS-REQ-0140 & Production \\
DMS-REQ-0125 & Software framework for Level 3 catalog processing \\
\end{tabular}
\\ \hline
\end{longtable}
}


\subsubsection{Test Cases Summary}
\begin{longtable}{p{3cm}p{2.5cm}p{2.5cm}p{3cm}p{4cm}}
\toprule
\href{https://jira.lsstcorp.org/secure/Tests.jspa\#/testCase/LVV-T206}{LVV-T206} & \multicolumn{4}{p{12cm}}{ Verify implementation of Federation with external catalogs } \\ \hline
\textbf{Owner} & \textbf{Status} & \textbf{Version} & \textbf{Critical Event} & \textbf{Verification Type} \\ \hline
Colin Slater & Draft & 1 & false & Test \\ \hline
\end{longtable}
{\scriptsize
\textbf{Objective:}\\
Verify that LSST-produced data can be combined with external datasets.
}
  
 \newpage 
\subsection{[LVV-57] DMS-REQ-0130-V-01: Calibration Data Products }\label{lvv-57}

\begin{longtable}{cccc}
\hline
\textbf{Jira Link} & \textbf{Assignee} & \textbf{Status} & \textbf{Test Cases}\\ \hline
\href{https://jira.lsstcorp.org/browse/LVV-57}{LVV-57} &
Robert Lupton & Not Covered &
\begin{tabular}{c}
LVV-T88 \\
\end{tabular}
\\
\hline
\end{longtable}

\textbf{Verification Element Description:} \\
For every calibration mode, prove that the data can be processed. Can be
done with simulated data and that from the auxilliary telescope. Will
need to be redone with real LSST camera data.

{\footnotesize
\begin{longtable}{p{2.5cm}p{13.5cm}}
\hline
\multicolumn{2}{c}{\textbf{Requirement Details}}\\ \hline
Requirement ID & DMS-REQ-0130 \\ \cdashline{1-2}
Requirement Description &
\begin{minipage}[]{13cm}
\textbf{Specification:} The DMS shall produce and archive Calibration
Data Products that capture the signature of the telescope, camera and
detector, including at least: Crosstalk correction matrix, Bias and Dark
correction frames, a set of monochromatic dome flats spanning the
wavelength range, a synthetic broad-band flat per filter, and an
illumination correction frame per filter.
\end{minipage}
\\ \cdashline{1-2}
Requirement Priority & 1a \\ \cdashline{1-2}
Upper Level Requirement &
\begin{tabular}{cl}
DMS-REQ-0076 & Keep Science Data Archive \\
OSS-REQ-0271 & Supported Image Types \\
OSS-REQ-0194 & Calibration Exposures Per Day \\
OSS-REQ-0129 & Exposures (Level 1) \\
\end{tabular}
\\ \hline
\end{longtable}
}


\subsubsection{Test Cases Summary}
\begin{longtable}{p{3cm}p{2.5cm}p{2.5cm}p{3cm}p{4cm}}
\toprule
\href{https://jira.lsstcorp.org/secure/Tests.jspa\#/testCase/LVV-T88}{LVV-T88} & \multicolumn{4}{p{12cm}}{ Verify implementation of Calibration Data Products } \\ \hline
\textbf{Owner} & \textbf{Status} & \textbf{Version} & \textbf{Critical Event} & \textbf{Verification Type} \\ \hline
Robert Lupton & Defined & 1 & false & Test \\ \hline
\end{longtable}
{\scriptsize
\textbf{Objective:}\\
Verify that the DMS can produce and archive the required Calibration
Data Products: cross talk correction, bias, dark, monochromatic dome
flats, broad-band flats, fringe correction, and illumination
corrections.
}
  
 \newpage 
\subsection{[LVV-59] DMS-REQ-0132-V-01: Calibration Image Provenance }\label{lvv-59}

\begin{longtable}{cccc}
\hline
\textbf{Jira Link} & \textbf{Assignee} & \textbf{Status} & \textbf{Test Cases}\\ \hline
\href{https://jira.lsstcorp.org/browse/LVV-59}{LVV-59} &
Robert Lupton & Not Covered &
\begin{tabular}{c}
LVV-T89 \\
\end{tabular}
\\
\hline
\end{longtable}

\textbf{Verification Element Description:} \\
Can be done with precursor or simulated data. Verify that provenance
information is present.

{\footnotesize
\begin{longtable}{p{2.5cm}p{13.5cm}}
\hline
\multicolumn{2}{c}{\textbf{Requirement Details}}\\ \hline
Requirement ID & DMS-REQ-0132 \\ \cdashline{1-2}
Requirement Description &
\begin{minipage}[]{13cm}
\textbf{Specification:} For each Calibration Production data product,
DMS shall record: the list of input exposures and the range of dates
over which they were obtained; the processing parameters; the
calibration products used to derive it; and a set of metadata attributes
including at least: the date of creation; the calibration image type
(e.g. dome flat, superflat, bias, etc); the provenance of the processing
software; and the instrument configuration including the filter in use,
if applicable.
\end{minipage}
\\ \cdashline{1-2}
Requirement Priority & 1a \\ \cdashline{1-2}
Upper Level Requirement &
\begin{tabular}{cl}
OSS-REQ-0122 & Provenance \\
OSS-REQ-0123 & Reproducibility \\
DMS-REQ-0130 & Calibration Data Products \\
\end{tabular}
\\ \hline
\end{longtable}
}


\subsubsection{Test Cases Summary}
\begin{longtable}{p{3cm}p{2.5cm}p{2.5cm}p{3cm}p{4cm}}
\toprule
\href{https://jira.lsstcorp.org/secure/Tests.jspa\#/testCase/LVV-T89}{LVV-T89} & \multicolumn{4}{p{12cm}}{ Verify implementation of Calibration Image Provenance } \\ \hline
\textbf{Owner} & \textbf{Status} & \textbf{Version} & \textbf{Critical Event} & \textbf{Verification Type} \\ \hline
Robert Lupton & Defined & 1 & false & Test \\ \hline
\end{longtable}
{\scriptsize
\textbf{Objective:}\\
Verify that the DMS records the required provenance information for the
Calibration Data Products.
}
  
 \newpage 
\subsection{[LVV-62] DMS-REQ-0158-V-01: Provide Pipeline Construction Services }\label{lvv-62}

\begin{longtable}{cccc}
\hline
\textbf{Jira Link} & \textbf{Assignee} & \textbf{Status} & \textbf{Test Cases}\\ \hline
\href{https://jira.lsstcorp.org/browse/LVV-62}{LVV-62} &
Robert Lupton & Not Covered &
\begin{tabular}{c}
LVV-T11 \\
\end{tabular}
\\
\hline
\end{longtable}

\textbf{Verification Element Description:} \\
Aggregate of
\href{https://jira.lsstcorp.org/browse/LVV-137}{LVV-137}~(DMS-REQ-0306),
\href{https://jira.lsstcorp.org/browse/LVV-136}{LVV-136}~(DMS-REQ-0305),
\href{https://jira.lsstcorp.org/browse/LVV-138}{LVV-138}~(DMS-REQ-0307).

{\footnotesize
\begin{longtable}{p{2.5cm}p{13.5cm}}
\hline
\multicolumn{2}{c}{\textbf{Requirement Details}}\\ \hline
Requirement ID & DMS-REQ-0158 \\ \cdashline{1-2}
Requirement Description &
\begin{minipage}[]{13cm}

\end{minipage}
\\ \cdashline{1-2}
Requirement Discussion &
\begin{minipage}[]{13cm}
(This is a composite requirement in the SysML model, which simply
aggregates its children.)
\end{minipage}
\\ \cdashline{1-2}
Requirement Priority & 1a \\ \cdashline{1-2}
Upper Level Requirement &
\begin{tabular}{cl}
\end{tabular}
\\ \hline
\end{longtable}
}

\subsubsection{Verified By}
\begin{itemize}
\item . LVV-136 (\ref{lvv-136}) DMS-REQ-0305-V-01: Task Specification
\item . LVV-137 (\ref{lvv-137}) DMS-REQ-0306-V-01: Task Configuration
\item . LVV-138 (\ref{lvv-138}) DMS-REQ-0307-V-01: Unique Processing Coverage
\end{itemize}

\subsubsection{Test Cases Summary}
\begin{longtable}{p{3cm}p{2.5cm}p{2.5cm}p{3cm}p{4cm}}
\toprule
\href{https://jira.lsstcorp.org/secure/Tests.jspa\#/testCase/LVV-T11}{LVV-T11} & \multicolumn{4}{p{12cm}}{ DRP-00-05: Execution of the DRP Science Payload by the Batch Production
Service } \\ \hline
\textbf{Owner} & \textbf{Status} & \textbf{Version} & \textbf{Critical Event} & \textbf{Verification Type} \\ \hline
Jim Bosch & Deprecated & 1 & false & Test \\ \hline
\end{longtable}
{\scriptsize
\textbf{Objective:}\\
This test will check that the DRP Science Payload can be executed using
a specific version of the Batch Production Service provided by the LSST
Data Facility. Since the outputs are stored in the Data Backbone, it too
is a component of this test.
}
  
 \newpage 
\subsection{[LVV-64] DMS-REQ-0161-V-01: Optimization of Cost, Reliability and Availability in
Order }\label{lvv-64}

\begin{longtable}{cccc}
\hline
\textbf{Jira Link} & \textbf{Assignee} & \textbf{Status} & \textbf{Test Cases}\\ \hline
\href{https://jira.lsstcorp.org/browse/LVV-64}{LVV-64} &
Robert Gruendl & Not Covered &
\begin{tabular}{c}
LVV-T172 \\
\end{tabular}
\\
\hline
\end{longtable}

\textbf{Verification Element Description:} \\
Inspect resource management policies that devote resources to production
catch-up (when required) over end users.

{\footnotesize
\begin{longtable}{p{2.5cm}p{13.5cm}}
\hline
\multicolumn{2}{c}{\textbf{Requirement Details}}\\ \hline
Requirement ID & DMS-REQ-0161 \\ \cdashline{1-2}
Requirement Description &
\begin{minipage}[]{13cm}
\textbf{Specification:} Within a fixed cost envelope for the Data
Management subsystem, the allocation of processing and storage
facilities will optimize reliability over availability to end users.
\end{minipage}
\\ \cdashline{1-2}
Requirement Priority & 1b \\ \cdashline{1-2}
Upper Level Requirement &
\begin{tabular}{cl}
\end{tabular}
\\ \hline
\end{longtable}
}


\subsubsection{Test Cases Summary}
\begin{longtable}{p{3cm}p{2.5cm}p{2.5cm}p{3cm}p{4cm}}
\toprule
\href{https://jira.lsstcorp.org/secure/Tests.jspa\#/testCase/LVV-T172}{LVV-T172} & \multicolumn{4}{p{12cm}}{ Verify implementation of Optimization of Cost, Reliability and
Availability } \\ \hline
\textbf{Owner} & \textbf{Status} & \textbf{Version} & \textbf{Critical Event} & \textbf{Verification Type} \\ \hline
Robert Gruendl & Draft & 1 & false & Test \\ \hline
\end{longtable}
{\scriptsize
\textbf{Objective:}\\
In matters of cost, system reliability (functioning properly at a given
time) has precedence over system availability (ability to use the system
at a given time). ~ The optimization may be outside the realm of direct
testing as it is more of a system provisioning guideline but on its face
it demands that the Data Management System include failure reporting,
regimented change control, acceptance testing, maintenance and
monitoring.
}
  
 \newpage 
\subsection{[LVV-96] DMS-REQ-0265-V-01: Guider Calibration Data Acquisition }\label{lvv-96}

\begin{longtable}{cccc}
\hline
\textbf{Jira Link} & \textbf{Assignee} & \textbf{Status} & \textbf{Test Cases}\\ \hline
\href{https://jira.lsstcorp.org/browse/LVV-96}{LVV-96} &
Gregory Dubois-Felsmann & Not Covered &
\begin{tabular}{c}
LVV-T34 \\
LVV-T283 \\
LVV-T284 \\
\end{tabular}
\\
\hline
\end{longtable}

\textbf{Verification Element Description:} \\
Needs a simulated DAQ for guider and data backbone. Does not say whether
data are archived or not.

{\footnotesize
\begin{longtable}{p{2.5cm}p{13.5cm}}
\hline
\multicolumn{2}{c}{\textbf{Requirement Details}}\\ \hline
Requirement ID & DMS-REQ-0265 \\ \cdashline{1-2}
Requirement Description &
\begin{minipage}[]{13cm}
\textbf{Specification:} The DMS shall acquire raw, full-frame exposures
from the camera guider sensors during calibration. The DMS shall produce
calibration data products for the guide sensors.
\end{minipage}
\\ \cdashline{1-2}
Requirement Priority & 1a \\ \cdashline{1-2}
Upper Level Requirement &
\begin{tabular}{cl}
OSS-REQ-0194 & Calibration Exposures Per Day \\
\end{tabular}
\\ \hline
\end{longtable}
}


\subsubsection{Test Cases Summary}
\begin{longtable}{p{3cm}p{2.5cm}p{2.5cm}p{3cm}p{4cm}}
\toprule
\href{https://jira.lsstcorp.org/secure/Tests.jspa\#/testCase/LVV-T34}{LVV-T34} & \multicolumn{4}{p{12cm}}{ Verify implementation of Guider Calibration Data Acquisition } \\ \hline
\textbf{Owner} & \textbf{Status} & \textbf{Version} & \textbf{Critical Event} & \textbf{Verification Type} \\ \hline
Kian-Tat Lim & Defined & 1 & false & Test \\ \hline
\end{longtable}
{\scriptsize
\textbf{Objective:}\\
{Verify successful}\\
{~1. Ingestion of calibration frames from L1 Test Stand DAQ}\\
{~2. Execution of CPP payloads}\\
{~3. Availability of observed guider calibration products}
}
\begin{longtable}{p{3cm}p{2.5cm}p{2.5cm}p{3cm}p{4cm}}
\toprule
\href{https://jira.lsstcorp.org/secure/Tests.jspa\#/testCase/LVV-T283}{LVV-T283} & \multicolumn{4}{p{12cm}}{ RAS-00-00: Writing well-formed raw image } \\ \hline
\textbf{Owner} & \textbf{Status} & \textbf{Version} & \textbf{Critical Event} & \textbf{Verification Type} \\ \hline
Michelle Butler & Approved & 1 & false & Test \\ \hline
\end{longtable}
{\scriptsize
\textbf{Objective:}\\
This test will check:\\

\begin{itemize}
\tightlist
\item
  The successful integration of the Pathfinder components with the DM
  Header Service and the Level 1 Archiver;
\item
  That the raw images are well-formed and meet specifications in
  change-controlled documents \citeds{LSE-61};
\end{itemize}

~This Test Case shall be repeated for each of the different cameras
(ATScam, LSSTCam) and sensors (Science, Wavefront, and Guider)
combination.
}
\begin{longtable}{p{3cm}p{2.5cm}p{2.5cm}p{3cm}p{4cm}}
\toprule
\href{https://jira.lsstcorp.org/secure/Tests.jspa\#/testCase/LVV-T284}{LVV-T284} & \multicolumn{4}{p{12cm}}{ RAS-00-05: (LDM-503-8b) Writing data from CCOB to the DBB for further
data processing } \\ \hline
\textbf{Owner} & \textbf{Status} & \textbf{Version} & \textbf{Critical Event} & \textbf{Verification Type} \\ \hline
Michelle Butler & Draft & 1 & false & Test \\ \hline
\end{longtable}
{\scriptsize
\textbf{Objective:}\\
This test will check:

\begin{itemize}
\tightlist
\item
  The successful integration of the DAQ archiver components with the
  CCOB
\item
  That the file can then be ingested into the DBB and be retrieved for
  further analysis
\end{itemize}
}
  
 \newpage 
\subsection{[LVV-97] DMS-REQ-0266-V-01: Exposure Catalog }\label{lvv-97}

\begin{longtable}{cccc}
\hline
\textbf{Jira Link} & \textbf{Assignee} & \textbf{Status} & \textbf{Test Cases}\\ \hline
\href{https://jira.lsstcorp.org/browse/LVV-97}{LVV-97} &
Jim Bosch & Not Covered &
\begin{tabular}{c}
LVV-T48 \\
\end{tabular}
\\
\hline
\end{longtable}

\textbf{Verification Element Description:} \\
This requires a database table to be created with the relevant columns
and for those columns to be verified. Also show that the data stored in
the table is appropriate.

{\footnotesize
\begin{longtable}{p{2.5cm}p{13.5cm}}
\hline
\multicolumn{2}{c}{\textbf{Requirement Details}}\\ \hline
Requirement ID & DMS-REQ-0266 \\ \cdashline{1-2}
Requirement Description &
\begin{minipage}[]{13cm}
\textbf{Specification:} The DMS shall create an Exposure Catalog
containing information for each exposure that includes the exposure
date/time and duration, properties of the filter used, dome and
telescope pointing and orientation, status of calibration apparatus,
airmass and zenith distance, telescope and dome status, environmental
information, and information regarding each sensor including an ID, its
location in the focal plane, electronic configuration, and WCS.
\end{minipage}
\\ \cdashline{1-2}
Requirement Priority & 1a \\ \cdashline{1-2}
Upper Level Requirement &
\begin{tabular}{cl}
OSS-REQ-0130 & Catalogs (Level 1) \\
\end{tabular}
\\ \hline
\end{longtable}
}


\subsubsection{Test Cases Summary}
\begin{longtable}{p{3cm}p{2.5cm}p{2.5cm}p{3cm}p{4cm}}
\toprule
\href{https://jira.lsstcorp.org/secure/Tests.jspa\#/testCase/LVV-T48}{LVV-T48} & \multicolumn{4}{p{12cm}}{ Verify implementation of Exposure Catalog } \\ \hline
\textbf{Owner} & \textbf{Status} & \textbf{Version} & \textbf{Critical Event} & \textbf{Verification Type} \\ \hline
Jim Bosch & Defined & 1 & false & Test \\ \hline
\end{longtable}
{\scriptsize
\textbf{Objective:}\\
Verify that the DMS creates an Exposure Catalog that includes\\
1. Observation datetime, exposure time\\
2. Filter\\
3. Dome, telescope orientation and status\\
4. Calibration status\\
5. Airmass and zenith\\
6. Environmental information\\
7. Per-sensor information
}
  
 \newpage 
\subsection{[LVV-98] DMS-REQ-0267-V-01: Source Catalog }\label{lvv-98}

\begin{longtable}{cccc}
\hline
\textbf{Jira Link} & \textbf{Assignee} & \textbf{Status} & \textbf{Test Cases}\\ \hline
\href{https://jira.lsstcorp.org/browse/LVV-98}{LVV-98} &
Jim Bosch & Not Covered &
\begin{tabular}{c}
LVV-T12 \\
LVV-T13 \\
LVV-T65 \\
LVV-T362 \\
\end{tabular}
\\
\hline
\end{longtable}

\textbf{Verification Element Description:} \\
First L2 requirement. Can be done with precursor data. At minimum two
visits of the same field and filter and one coadd.

{\footnotesize
\begin{longtable}{p{2.5cm}p{13.5cm}}
\hline
\multicolumn{2}{c}{\textbf{Requirement Details}}\\ \hline
Requirement ID & DMS-REQ-0267 \\ \cdashline{1-2}
Requirement Description &
\begin{minipage}[]{13cm}
\textbf{Specification:} The DMS shall create a Catalog containing all
Sources detected in single (standard) visits and on Co-Adds, and will
contain an identifier of the Exposure on which the Source was detected,
as well as measurements of Source Attributes. The measured attributes
(and associated errors) include location on the focal plane; a static
point-source model fit to world coordinates and flux; a centroid and
adaptive moments; and surface brightnesses through elliptical multiple
apertures that are concentric, PSF-homogenized, and logarithmically
spaced in intensity.
\end{minipage}
\\ \cdashline{1-2}
Requirement Priority & 1b \\ \cdashline{1-2}
Upper Level Requirement &
\begin{tabular}{cl}
OSS-REQ-0137 & Catalogs (Level 2) \\
\end{tabular}
\\ \hline
\end{longtable}
}


\subsubsection{Test Cases Summary}
\begin{longtable}{p{3cm}p{2.5cm}p{2.5cm}p{3cm}p{4cm}}
\toprule
\href{https://jira.lsstcorp.org/secure/Tests.jspa\#/testCase/LVV-T12}{LVV-T12} & \multicolumn{4}{p{12cm}}{ DRP-00-10: Data Release Includes Required Data Products } \\ \hline
\textbf{Owner} & \textbf{Status} & \textbf{Version} & \textbf{Critical Event} & \textbf{Verification Type} \\ \hline
Jim Bosch & Deprecated & 1 & false & Test \\ \hline
\end{longtable}
{\scriptsize
\textbf{Objective:}\\
This test will check that the basic data products which should be in an
data release are generated by execution of the science payload.\\
These products will include:

\begin{itemize}
\tightlist
\item
  Source catalogs, derived from PVIs and coadded images (DMS-REQ-0267 \&
  DMS-REQ-0277);
\item
  Forced source catalogs (DMS-REQ-0268);
\item
  Object catalogs (DMS-REQ-0275);
\item
  Processed visit images (PVIs; DMS-REQ-0069);
\item
  Coadded images (DMS-REQ-0279);
\end{itemize}
}
\begin{longtable}{p{3cm}p{2.5cm}p{2.5cm}p{3cm}p{4cm}}
\toprule
\href{https://jira.lsstcorp.org/secure/Tests.jspa\#/testCase/LVV-T13}{LVV-T13} & \multicolumn{4}{p{12cm}}{ DRP-00-15: Scientific Verification of Source Catalog } \\ \hline
\textbf{Owner} & \textbf{Status} & \textbf{Version} & \textbf{Critical Event} & \textbf{Verification Type} \\ \hline
Jim Bosch & Deprecated & 1 & false & Test \\ \hline
\end{longtable}
{\scriptsize
\textbf{Objective:}\\
This test will check that the source catalogs delivered by the DRP
science payload meet the requirements laid down by \citeds{LSE-61}.\\
Specifically, this will demonstrate that:

\begin{itemize}
\tightlist
\item
  Measurements in the catalog are presented in flux units
  (DMS-REQ-0347);
\item
  Derived quantities are provided in pre-computed columns
  (DMS-REQ-0331);
\item
  Aperture corrections for different photometry algorithms are
  consistent.
\item
  Photometry measurements are consistent with reference catalog
  photometry (including sources not used in photometric calibration).
\item
  Astrometry measurements are consistent with reference catalog
  positions (including sources not used in astrometric calibration).
\end{itemize}

This test does not include quantitative targets for the science quality
criteria; we instead require for each test that we be able to quickly
construct a plot in which such a target can be visualized.
}
\begin{longtable}{p{3cm}p{2.5cm}p{2.5cm}p{3cm}p{4cm}}
\toprule
\href{https://jira.lsstcorp.org/secure/Tests.jspa\#/testCase/LVV-T65}{LVV-T65} & \multicolumn{4}{p{12cm}}{ Verify implementation of Source Catalog } \\ \hline
\textbf{Owner} & \textbf{Status} & \textbf{Version} & \textbf{Critical Event} & \textbf{Verification Type} \\ \hline
Jim Bosch & Defined & 1 & false & Test \\ \hline
\end{longtable}
{\scriptsize
\textbf{Objective:}\\
Verify that all Sources produced by the DRP pipelines contain the
entries listed in DMS-REQ-0267.
}
\begin{longtable}{p{3cm}p{2.5cm}p{2.5cm}p{3cm}p{4cm}}
\toprule
\href{https://jira.lsstcorp.org/secure/Tests.jspa\#/testCase/LVV-T362}{LVV-T362} & \multicolumn{4}{p{12cm}}{ Installation of the LSST Science Pipelines Payloads } \\ \hline
\textbf{Owner} & \textbf{Status} & \textbf{Version} & \textbf{Critical Event} & \textbf{Verification Type} \\ \hline
John Swinbank & Approved & 1 & false & Test \\ \hline
\end{longtable}
{\scriptsize
\textbf{Objective:}\\
This test will check that:

\begin{itemize}
\tightlist
\item
  The Alert Production Pipeline payload is available for installation
  from documented channels;
\item
  The Data Release Production Pipeline payload is available for
  installation from documented channels;
\item
  The Calibration Products Production Pipeline payload is available for
  installation from documented channels;
\item
  These payloads can be installed on systems at the LSST Data Facility
  following available documentation;
\item
  The installed pipeline payloads are capable of successfully executing
  basic integration tests.
\end{itemize}

Note that this test assumes a 2018-era packaging of the Science
Pipelines software, in which all the above payloads are represented by a
single ``meta-package'', lsst\_distrib.
}
  
 \newpage 
\subsection{[LVV-99] DMS-REQ-0268-V-01: Forced-Source Catalog }\label{lvv-99}

\begin{longtable}{cccc}
\hline
\textbf{Jira Link} & \textbf{Assignee} & \textbf{Status} & \textbf{Test Cases}\\ \hline
\href{https://jira.lsstcorp.org/browse/LVV-99}{LVV-99} &
Jim Bosch & Not Covered &
\begin{tabular}{c}
LVV-T12 \\
LVV-T66 \\
\end{tabular}
\\
\hline
\end{longtable}

\textbf{Verification Element Description:} \\
With the precursor data verify that forced source table is created from
all calibrated exposures.

{\footnotesize
\begin{longtable}{p{2.5cm}p{13.5cm}}
\hline
\multicolumn{2}{c}{\textbf{Requirement Details}}\\ \hline
Requirement ID & DMS-REQ-0268 \\ \cdashline{1-2}
Requirement Description &
\begin{minipage}[]{13cm}
\textbf{Specification:} The DMS shall create a Forced-Source Catalog,
consisting of measured fluxes for all entries in the Object Catalog on
all Processed Visit Images and Difference Images. Measurements for each
forced-source shall include the object and visit IDs, the modelled flux
and error (given fixed position, shape, and deblending parameters), and
measurement quality flags.
\end{minipage}
\\ \cdashline{1-2}
Requirement Discussion &
\begin{minipage}[]{13cm}
\textbf{Discussion:} The large number of Forced Sources makes it
impractical to measure more attributes than are necessary to construct a
light curve for variability studies.
\end{minipage}
\\ \cdashline{1-2}
Requirement Priority & 1b \\ \cdashline{1-2}
Upper Level Requirement &
\begin{tabular}{cl}
OSS-REQ-0137 & Catalogs (Level 2) \\
\end{tabular}
\\ \hline
\end{longtable}
}


\subsubsection{Test Cases Summary}
\begin{longtable}{p{3cm}p{2.5cm}p{2.5cm}p{3cm}p{4cm}}
\toprule
\href{https://jira.lsstcorp.org/secure/Tests.jspa\#/testCase/LVV-T12}{LVV-T12} & \multicolumn{4}{p{12cm}}{ DRP-00-10: Data Release Includes Required Data Products } \\ \hline
\textbf{Owner} & \textbf{Status} & \textbf{Version} & \textbf{Critical Event} & \textbf{Verification Type} \\ \hline
Jim Bosch & Deprecated & 1 & false & Test \\ \hline
\end{longtable}
{\scriptsize
\textbf{Objective:}\\
This test will check that the basic data products which should be in an
data release are generated by execution of the science payload.\\
These products will include:

\begin{itemize}
\tightlist
\item
  Source catalogs, derived from PVIs and coadded images (DMS-REQ-0267 \&
  DMS-REQ-0277);
\item
  Forced source catalogs (DMS-REQ-0268);
\item
  Object catalogs (DMS-REQ-0275);
\item
  Processed visit images (PVIs; DMS-REQ-0069);
\item
  Coadded images (DMS-REQ-0279);
\end{itemize}
}
\begin{longtable}{p{3cm}p{2.5cm}p{2.5cm}p{3cm}p{4cm}}
\toprule
\href{https://jira.lsstcorp.org/secure/Tests.jspa\#/testCase/LVV-T66}{LVV-T66} & \multicolumn{4}{p{12cm}}{ Verify implementation of Forced-Source Catalog } \\ \hline
\textbf{Owner} & \textbf{Status} & \textbf{Version} & \textbf{Critical Event} & \textbf{Verification Type} \\ \hline
Jim Bosch & Draft & 1 & false & Test \\ \hline
\end{longtable}
{\scriptsize
\textbf{Objective:}\\
Verify that all ForcedSources produced by the DRP pipelines contain
fluxes measured on difference and direct single-epoch images, associated
uncertainties, an Object ID, and a Visit ID.
}
  
 \newpage 
\subsection{[LVV-100] DMS-REQ-0269-V-01: DIASource Catalog }\label{lvv-100}

\begin{longtable}{cccc}
\hline
\textbf{Jira Link} & \textbf{Assignee} & \textbf{Status} & \textbf{Test Cases}\\ \hline
\href{https://jira.lsstcorp.org/browse/LVV-100}{LVV-100} &
Jim Bosch & Not Covered &
\begin{tabular}{c}
LVV-T18 \\
LVV-T21 \\
LVV-T49 \\
\end{tabular}
\\
\hline
\end{longtable}

\textbf{Verification Element Description:} \\
Assume this is verified by performing a difference image processing and
checking that reasonable data automatically appears in the DIASource
table. Verify against DPDD.

{\footnotesize
\begin{longtable}{p{2.5cm}p{13.5cm}}
\hline
\multicolumn{2}{c}{\textbf{Requirement Details}}\\ \hline
Requirement ID & DMS-REQ-0269 \\ \cdashline{1-2}
Requirement Description &
\begin{minipage}[]{13cm}
\textbf{Specification:} The DMS shall construct a catalog of all Sources
detected on Difference Exposures with SNR \textgreater{}
\textbf{transSNR}. For each Difference Source (DIASource), the DMS shall
be able to provide the identity of the Difference Exposure from which it
was derived; the identity of the associated SSObject, if any; the
identity of the parent Source from which this DIASource has been
deblended, if any. The DMS shall also measure and record a set of
attributes for each DIASource including at least: epoch of the
observation, focal plane position centroid and error (pixel), sky
position and associated error (radec), SNR of the detection; calibrated
PS flux and associated error; likelihood of the observed data given the
PS model; calibrated aperture flux and associated error; calibrated flux
and associated error for a trailed source model, and length and angle of
the trail; flux and associated parameters for a dipole model; parameters
of an adaptive shape measurement and associated error; a measure of
source extendedness; the estimated background at the position of the
object in the template image with associated uncertainty; a measure of
spuriousness; and flags indicating problems encountered while computing
the aforementioned attributes. The DMS shall also determine and record
measurements on the Calibrated exposure the following: calibrated flux
and associated error for the source as measured on the Visit image.
\end{minipage}
\\ \cdashline{1-2}
Requirement Parameters & \textbf{transSNR = 5{{[}float{]}}} The signal-to-noise ratio in
single-visit difference images above which all optical transients are to
be reported. \\ \cdashline{1-2}
Requirement Priority & 1b \\ \cdashline{1-2}
Upper Level Requirement &
\begin{tabular}{cl}
OSS-REQ-0130 & Catalogs (Level 1) \\
DMS-REQ-0270 & Faint DIASource Measurements \\
\end{tabular}
\\ \hline
\end{longtable}
}


\subsubsection{Test Cases Summary}
\begin{longtable}{p{3cm}p{2.5cm}p{2.5cm}p{3cm}p{4cm}}
\toprule
\href{https://jira.lsstcorp.org/secure/Tests.jspa\#/testCase/LVV-T18}{LVV-T18} & \multicolumn{4}{p{12cm}}{ AG-00-05: Alert Generation Produces Required Data Products } \\ \hline
\textbf{Owner} & \textbf{Status} & \textbf{Version} & \textbf{Critical Event} & \textbf{Verification Type} \\ \hline
Eric Bellm & Deprecated & 1 & false & Test \\ \hline
\end{longtable}
{\scriptsize
\textbf{Objective:}\\
This test will check that the basic data products produced by Alert
Generation are generated by execution of the science payload.\\
These products will include:

\begin{itemize}
\tightlist
\item
  Processed visit images (PVIs; DMS-REQ-0069);
\item
  Difference Exposures (DMS-REQ-0010);
\item
  DIASource catalogs (DMS-REQ-0269);
\item
  DIAObject catalogs (DMS-REQ-0271);
\end{itemize}
}
\begin{longtable}{p{3cm}p{2.5cm}p{2.5cm}p{3cm}p{4cm}}
\toprule
\href{https://jira.lsstcorp.org/secure/Tests.jspa\#/testCase/LVV-T21}{LVV-T21} & \multicolumn{4}{p{12cm}}{ AG-00-20: Scientific Verification of DIASource Catalog } \\ \hline
\textbf{Owner} & \textbf{Status} & \textbf{Version} & \textbf{Critical Event} & \textbf{Verification Type} \\ \hline
Eric Bellm & Deprecated & 1 & false & Test \\ \hline
\end{longtable}
{\scriptsize
\textbf{Objective:}\\
This test will check that the difference image source catalogs delivered
by the Alert Generation science payload meet the requirements laid down
by \citeds{LSE-61}.

\begin{itemize}
\tightlist
\item
  Specifically, this will demonstrate that:
\item
  Measurements in the catalog are presented in flux units
  (DMS-REQ-0347);
\item
  Each DIASource record contains an appropriate subset of the attributes
  required by DMS-REQ-0269. In particular, the LDM-503-3-era pipeline is
  expected to provide DIASource positions (sky and focal plane), fluxes,
  and flags indicative of issues encountered during processing.
\item
  Faint DIASources satisfying additional criteria are stored
  (DMS-REQ-0270).
\item
  Derived quantities are provided in pre-computed columns
  (DMS-REQ-0331);
\end{itemize}

This test does not include quantitative targets for the science quality
criteria.\\[2\baselineskip]
}
\begin{longtable}{p{3cm}p{2.5cm}p{2.5cm}p{3cm}p{4cm}}
\toprule
\href{https://jira.lsstcorp.org/secure/Tests.jspa\#/testCase/LVV-T49}{LVV-T49} & \multicolumn{4}{p{12cm}}{ Verify implementation of DIASource Catalog } \\ \hline
\textbf{Owner} & \textbf{Status} & \textbf{Version} & \textbf{Critical Event} & \textbf{Verification Type} \\ \hline
Eric Bellm & Draft & 1 & false & Test \\ \hline
\end{longtable}
{\scriptsize
\textbf{Objective:}\\
Verify that the DMS produces a Source catalog from Difference Exposures
with the required attributes.
}
  
 \newpage 
\subsection{[LVV-101] DMS-REQ-0270-V-01: Faint DIASource Measurements }\label{lvv-101}

\begin{longtable}{cccc}
\hline
\textbf{Jira Link} & \textbf{Assignee} & \textbf{Status} & \textbf{Test Cases}\\ \hline
\href{https://jira.lsstcorp.org/browse/LVV-101}{LVV-101} &
Eric Bellm & Not Covered &
\begin{tabular}{c}
LVV-T21 \\
LVV-T50 \\
\end{tabular}
\\
\hline
\end{longtable}

\textbf{Verification Element Description:} \\
We first need to define some criteria. Then we need to work out whether
this is an after burner, triggered after processing, or something
directly integrated into L1 processing and triggered automatically.

{\footnotesize
\begin{longtable}{p{2.5cm}p{13.5cm}}
\hline
\multicolumn{2}{c}{\textbf{Requirement Details}}\\ \hline
Requirement ID & DMS-REQ-0270 \\ \cdashline{1-2}
Requirement Description &
\begin{minipage}[]{13cm}
\textbf{Specification:} The DMS shall be able to measure and store
DIASources fainter than \textbf{transSNR} that satisfy additional
criteria. A limited number of such sources shall be made to enable
monitoring of DIA quality.
\end{minipage}
\\ \cdashline{1-2}
Requirement Parameters & \textbf{transSNR = 5{{[}float{]}}} The signal-to-noise ratio in
single-visit difference images above which all optical transients are to
be reported. \\ \cdashline{1-2}
Requirement Discussion &
\begin{minipage}[]{13cm}
\textbf{Discussion:} Some individual faint sources may be of high
significance, such as a potentially hazardous asteroid.
\end{minipage}
\\ \cdashline{1-2}
Requirement Priority & 2 \\ \cdashline{1-2}
Upper Level Requirement &
\begin{tabular}{cl}
OSS-REQ-0166 & Alert Completeness and Purity \\
\end{tabular}
\\ \hline
\end{longtable}
}


\subsubsection{Test Cases Summary}
\begin{longtable}{p{3cm}p{2.5cm}p{2.5cm}p{3cm}p{4cm}}
\toprule
\href{https://jira.lsstcorp.org/secure/Tests.jspa\#/testCase/LVV-T21}{LVV-T21} & \multicolumn{4}{p{12cm}}{ AG-00-20: Scientific Verification of DIASource Catalog } \\ \hline
\textbf{Owner} & \textbf{Status} & \textbf{Version} & \textbf{Critical Event} & \textbf{Verification Type} \\ \hline
Eric Bellm & Deprecated & 1 & false & Test \\ \hline
\end{longtable}
{\scriptsize
\textbf{Objective:}\\
This test will check that the difference image source catalogs delivered
by the Alert Generation science payload meet the requirements laid down
by \citeds{LSE-61}.

\begin{itemize}
\tightlist
\item
  Specifically, this will demonstrate that:
\item
  Measurements in the catalog are presented in flux units
  (DMS-REQ-0347);
\item
  Each DIASource record contains an appropriate subset of the attributes
  required by DMS-REQ-0269. In particular, the LDM-503-3-era pipeline is
  expected to provide DIASource positions (sky and focal plane), fluxes,
  and flags indicative of issues encountered during processing.
\item
  Faint DIASources satisfying additional criteria are stored
  (DMS-REQ-0270).
\item
  Derived quantities are provided in pre-computed columns
  (DMS-REQ-0331);
\end{itemize}

This test does not include quantitative targets for the science quality
criteria.\\[2\baselineskip]
}
\begin{longtable}{p{3cm}p{2.5cm}p{2.5cm}p{3cm}p{4cm}}
\toprule
\href{https://jira.lsstcorp.org/secure/Tests.jspa\#/testCase/LVV-T50}{LVV-T50} & \multicolumn{4}{p{12cm}}{ Verify implementation of Faint DIASource Measurements } \\ \hline
\textbf{Owner} & \textbf{Status} & \textbf{Version} & \textbf{Critical Event} & \textbf{Verification Type} \\ \hline
Eric Bellm & Draft & 1 & false & Test \\ \hline
\end{longtable}
{\scriptsize
\textbf{Objective:}\\
Verify that the DMS can produces DIASources measurements for sources
below the nominal S/N cutoff that satisfy additional criteria.
}
  
 \newpage 
\subsection{[LVV-102] DMS-REQ-0271-V-01: Max nearby galaxies associated with DIASource }\label{lvv-102}

\begin{longtable}{cccc}
\hline
\textbf{Jira Link} & \textbf{Assignee} & \textbf{Status} & \textbf{Test Cases}\\ \hline
\href{https://jira.lsstcorp.org/browse/LVV-102}{LVV-102} &
Eric Bellm & Not Covered &
\begin{tabular}{c}
LVV-T18 \\
LVV-T22 \\
LVV-T51 \\
\end{tabular}
\\
\hline
\end{longtable}

\textbf{Verification Element Description:} \\
Run multiple visits through image differencing. Run association
pipeline. Verify that DIASources are correctly associated with
DIAObjects and DIAObjects correctly associated with Objects. Can use
precursor data.

~\\
Associated element
(\href{https://jira.lsstcorp.org/browse/LVV-9743}{LVV-9743}) satisfies
the radius within which an Object is considered coincident with a
DIASource.

Associated element
(\href{https://jira.lsstcorp.org/browse/LVV-9742}{LVV-9742}) satisfies
the maximum number of stars that can be associated with a DIASource.

~

{\footnotesize
\begin{longtable}{p{2.5cm}p{13.5cm}}
\hline
\multicolumn{2}{c}{\textbf{Requirement Details}}\\ \hline
Requirement ID & DMS-REQ-0271 \\ \cdashline{1-2}
Requirement Description &
\begin{minipage}[]{13cm}
\textbf{Specification:} The DMS shall construct a catalog of all
astrophysical objects identified through difference image analysis
(DIAObjects). The DIAObject entries shall include metadata attributes
including at least: a unique identifier; the identifiers of the
\textbf{diaNearbyObjMaxStar} nearest stars and
\textbf{diaNearbyObjMaxGalaxy} nearest galaxies in the Object catalog
lying within \textbf{diaNearbyObjRadius}, the probability that the
DIAObject is the same as the nearby Object; and a set of DIAObject
properties.
\end{minipage}
\\ \cdashline{1-2}
Requirement Parameters & {[}\textbf{diaNearbyObjMaxGalaxy = 3{{[}integer{]}}} Maximum number of
nearby galaxies that can be associated with a DIASource.,
\textbf{diaNearbyObjRadius = 60{{[}arcsecond{]}}} Radius within which an
Object is considered to be near, and possibly coincident with, the
DIASource., \textbf{diaNearbyObjMaxStar = 3{{[}integer{]}}} Maximum
number of stars that can be associated with a DIASource.{]} \\ \cdashline{1-2}
Requirement Priority & 1b \\ \cdashline{1-2}
Upper Level Requirement &
\begin{tabular}{cl}
OSS-REQ-0130 & Catalogs (Level 1) \\
\end{tabular}
\\ \hline
\end{longtable}
}


\subsubsection{Test Cases Summary}
\begin{longtable}{p{3cm}p{2.5cm}p{2.5cm}p{3cm}p{4cm}}
\toprule
\href{https://jira.lsstcorp.org/secure/Tests.jspa\#/testCase/LVV-T18}{LVV-T18} & \multicolumn{4}{p{12cm}}{ AG-00-05: Alert Generation Produces Required Data Products } \\ \hline
\textbf{Owner} & \textbf{Status} & \textbf{Version} & \textbf{Critical Event} & \textbf{Verification Type} \\ \hline
Eric Bellm & Deprecated & 1 & false & Test \\ \hline
\end{longtable}
{\scriptsize
\textbf{Objective:}\\
This test will check that the basic data products produced by Alert
Generation are generated by execution of the science payload.\\
These products will include:

\begin{itemize}
\tightlist
\item
  Processed visit images (PVIs; DMS-REQ-0069);
\item
  Difference Exposures (DMS-REQ-0010);
\item
  DIASource catalogs (DMS-REQ-0269);
\item
  DIAObject catalogs (DMS-REQ-0271);
\end{itemize}
}
\begin{longtable}{p{3cm}p{2.5cm}p{2.5cm}p{3cm}p{4cm}}
\toprule
\href{https://jira.lsstcorp.org/secure/Tests.jspa\#/testCase/LVV-T22}{LVV-T22} & \multicolumn{4}{p{12cm}}{ AG-00-25: Scientific Verification of DIAObject Catalog } \\ \hline
\textbf{Owner} & \textbf{Status} & \textbf{Version} & \textbf{Critical Event} & \textbf{Verification Type} \\ \hline
Eric Bellm & Deprecated & 1 & false & Test \\ \hline
\end{longtable}
{\scriptsize
\textbf{Objective:}\\
This test will check that the DIAObject catalogs delivered by the Alert
Generation science pay- load meet the requirements laid down by
\citeds{LSE-61}.\\
Specifically, this will demonstrate that:

\begin{itemize}
\tightlist
\item
  DIAObjects are recorded with unique identifiers (DMS-REQ-0271);
\item
  Measurements in the catalog are presented in flux units
  (DMS-REQ-0347);
\item
  EachDIAObjectrecordcontainscontainsanappropriatesetofsummaryattributes(DMS-
  REQ-0271 and DMS-REQ-0272). Note:

  \begin{itemize}
  \tightlist
  \item
    This test is executed independently of the Data Release Production
    system. Hence, DIAObjects are not associated to Objects, and the
    association metadata specified by DMS-REQ-0271 is not expected to be
    available.
  \item
    TheLDM-503-3erapipelineisnotexpectedtocalculateorpersistallattributesspec-
    ified by DMS-REQ-0272 requirement.
  \end{itemize}
\item
  Relevant derived quantities are provided in pre-computed columns
  (DMS-REQ-0331);~
\end{itemize}

This test does not include quantitative targets for the science quality
criteria.
}
\begin{longtable}{p{3cm}p{2.5cm}p{2.5cm}p{3cm}p{4cm}}
\toprule
\href{https://jira.lsstcorp.org/secure/Tests.jspa\#/testCase/LVV-T51}{LVV-T51} & \multicolumn{4}{p{12cm}}{ Verify implementation of DIAObject Catalog } \\ \hline
\textbf{Owner} & \textbf{Status} & \textbf{Version} & \textbf{Critical Event} & \textbf{Verification Type} \\ \hline
Eric Bellm & Draft & 1 & false & Test \\ \hline
\end{longtable}
{\scriptsize
\textbf{Objective:}\\
Verify that the DIAObject includes a unique ID, identifiers for nearest
stars and nearest galaxies, and probability of matching to static
Object.
}
  
 \newpage 
\subsection{[LVV-103] DMS-REQ-0272-V-01: DIAObject Attributes }\label{lvv-103}

\begin{longtable}{cccc}
\hline
\textbf{Jira Link} & \textbf{Assignee} & \textbf{Status} & \textbf{Test Cases}\\ \hline
\href{https://jira.lsstcorp.org/browse/LVV-103}{LVV-103} &
Eric Bellm & Not Covered &
\begin{tabular}{c}
LVV-T22 \\
LVV-T52 \\
\end{tabular}
\\
\hline
\end{longtable}

\textbf{Verification Element Description:} \\
Compare contents of table populated in DMS-REQ- 0271 with DPDD.

{\footnotesize
\begin{longtable}{p{2.5cm}p{13.5cm}}
\hline
\multicolumn{2}{c}{\textbf{Requirement Details}}\\ \hline
Requirement ID & DMS-REQ-0272 \\ \cdashline{1-2}
Requirement Description &
\begin{minipage}[]{13cm}
\textbf{Specification:} For each DIAObject the DMS shall store summary
attributes including at least: sky position at the time of the
observation; astrometric attributes including proper motion, parallax
and related errors; point-source magnitude in each passband and related
error; weighted mean forced-photometry flux and related error; periodic
and non-periodic variability measures; and flags that encode special
conditions encountered in measuring the above quantities.
\end{minipage}
\\ \cdashline{1-2}
Requirement Priority & 1b \\ \cdashline{1-2}
Upper Level Requirement &
\begin{tabular}{cl}
OSS-REQ-0130 & Catalogs (Level 1) \\
\end{tabular}
\\ \hline
\end{longtable}
}

\subsubsection{Verified By}
\begin{itemize}
\item . LVV-10990 (\ref{lvv-10990}) Explore time domain with summary statistics
\item . LVV-10990 (\ref{lvv-10990}) Explore time domain with summary statistics
\end{itemize}

\subsubsection{Test Cases Summary}
\begin{longtable}{p{3cm}p{2.5cm}p{2.5cm}p{3cm}p{4cm}}
\toprule
\href{https://jira.lsstcorp.org/secure/Tests.jspa\#/testCase/LVV-T22}{LVV-T22} & \multicolumn{4}{p{12cm}}{ AG-00-25: Scientific Verification of DIAObject Catalog } \\ \hline
\textbf{Owner} & \textbf{Status} & \textbf{Version} & \textbf{Critical Event} & \textbf{Verification Type} \\ \hline
Eric Bellm & Deprecated & 1 & false & Test \\ \hline
\end{longtable}
{\scriptsize
\textbf{Objective:}\\
This test will check that the DIAObject catalogs delivered by the Alert
Generation science pay- load meet the requirements laid down by
\citeds{LSE-61}.\\
Specifically, this will demonstrate that:

\begin{itemize}
\tightlist
\item
  DIAObjects are recorded with unique identifiers (DMS-REQ-0271);
\item
  Measurements in the catalog are presented in flux units
  (DMS-REQ-0347);
\item
  EachDIAObjectrecordcontainscontainsanappropriatesetofsummaryattributes(DMS-
  REQ-0271 and DMS-REQ-0272). Note:

  \begin{itemize}
  \tightlist
  \item
    This test is executed independently of the Data Release Production
    system. Hence, DIAObjects are not associated to Objects, and the
    association metadata specified by DMS-REQ-0271 is not expected to be
    available.
  \item
    TheLDM-503-3erapipelineisnotexpectedtocalculateorpersistallattributesspec-
    ified by DMS-REQ-0272 requirement.
  \end{itemize}
\item
  Relevant derived quantities are provided in pre-computed columns
  (DMS-REQ-0331);~
\end{itemize}

This test does not include quantitative targets for the science quality
criteria.
}
\begin{longtable}{p{3cm}p{2.5cm}p{2.5cm}p{3cm}p{4cm}}
\toprule
\href{https://jira.lsstcorp.org/secure/Tests.jspa\#/testCase/LVV-T52}{LVV-T52} & \multicolumn{4}{p{12cm}}{ Verify implementation of DIAObject Attributes } \\ \hline
\textbf{Owner} & \textbf{Status} & \textbf{Version} & \textbf{Critical Event} & \textbf{Verification Type} \\ \hline
Eric Bellm & Draft & 1 & false & Test \\ \hline
\end{longtable}
{\scriptsize
\textbf{Objective:}\\
Verify that the DMS provides summary attributes for each DIAObject,
including periodicity measures.
}
  
 \newpage 
\subsection{[LVV-104] DMS-REQ-0273-V-01: SSObject Catalog }\label{lvv-104}

\begin{longtable}{cccc}
\hline
\textbf{Jira Link} & \textbf{Assignee} & \textbf{Status} & \textbf{Test Cases}\\ \hline
\href{https://jira.lsstcorp.org/browse/LVV-104}{LVV-104} &
Eric Bellm & Not Covered &
\begin{tabular}{c}
LVV-T53 \\
\end{tabular}
\\
\hline
\end{longtable}

\textbf{Verification Element Description:} \\
We might be able to demonstrate this by providing calculated positions
of known asteroids to MOPS and then checking the SSObject table. Better,
use data from precursor surveys. Also use full simulations with injected
asteroids. Final verification requires a mini-survey of LSST.

{\footnotesize
\begin{longtable}{p{2.5cm}p{13.5cm}}
\hline
\multicolumn{2}{c}{\textbf{Requirement Details}}\\ \hline
Requirement ID & DMS-REQ-0273 \\ \cdashline{1-2}
Requirement Description &
\begin{minipage}[]{13cm}
\textbf{Specification:} The DMS shall produce a catalog of all Solar
System Objects (SSObjects) that have been identified via Solar System
Processing. The SSObject catalog shall include for each entry attributes
including at least the following: Osculating orbital elements and
associated uncertainties, minimum orbit intersection distance (MOID),
mean absolute magnitude and slope parameter per band and associated
errors, and flags that describe conditions of the description.
\end{minipage}
\\ \cdashline{1-2}
Requirement Discussion &
\begin{minipage}[]{13cm}
\textbf{Discussion:} The magnitude and angular velocity limits for
identifying SSObjects are TBD. These limits may be driven more by
computational resource constraints than by the raw reach of the
collected data. The software may well be capable of exceeding the
required limits, but at an unacceptable cost. The slope parameter will
be poorly constrained until later in the survey. A baseline algorithm
and acceptance criteria should be developed prior to verification.
\end{minipage}
\\ \cdashline{1-2}
Requirement Priority & 2 \\ \cdashline{1-2}
Upper Level Requirement &
\begin{tabular}{cl}
OSS-REQ-0130 & Catalogs (Level 1) \\
\end{tabular}
\\ \hline
\end{longtable}
}


\subsubsection{Test Cases Summary}
\begin{longtable}{p{3cm}p{2.5cm}p{2.5cm}p{3cm}p{4cm}}
\toprule
\href{https://jira.lsstcorp.org/secure/Tests.jspa\#/testCase/LVV-T53}{LVV-T53} & \multicolumn{4}{p{12cm}}{ Verify implementation of SSObject Catalog } \\ \hline
\textbf{Owner} & \textbf{Status} & \textbf{Version} & \textbf{Critical Event} & \textbf{Verification Type} \\ \hline
Eric Bellm & Draft & 1 & false & Test \\ \hline
\end{longtable}
{\scriptsize
\textbf{Objective:}\\
Verify that the DMS produces a catalog of Solar System Objects identify
from Moving Object Processing.\\
Verify that the SSObject catalog includes orbital elements and
additional related quanitites.
}
  
 \newpage 
\subsection{[LVV-105] DMS-REQ-0274-V-01: Alert Content }\label{lvv-105}

\begin{longtable}{cccc}
\hline
\textbf{Jira Link} & \textbf{Assignee} & \textbf{Status} & \textbf{Test Cases}\\ \hline
\href{https://jira.lsstcorp.org/browse/LVV-105}{LVV-105} &
Eric Bellm & Not Covered &
\begin{tabular}{c}
LVV-T54 \\
\end{tabular}
\\
\hline
\end{longtable}

\textbf{Verification Element Description:} \\
Interpret this as a full end to end test of L1, rather than the ability
to publish alerts from a DIASources catalog. Compare contents of
DIASources catalog with contents of alert stream.

{\footnotesize
\begin{longtable}{p{2.5cm}p{13.5cm}}
\hline
\multicolumn{2}{c}{\textbf{Requirement Details}}\\ \hline
Requirement ID & DMS-REQ-0274 \\ \cdashline{1-2}
Requirement Description &
\begin{minipage}[]{13cm}
\textbf{Specification:} The DMS shall create an Alert for each detected
DIASource, to be broadcast using community protocols, with content that
includes: a unique Alert ID, the Level-1 database ID, the DIASource
record that triggered the alert, the DIAObject (or SSObject) record, 12
months of previous DIASource records corresponding to the object (if
available), and cut-outs of images (from both the template image and the
difference image) of sufficient areal coverage to identify the DIASource
and its immediate surroundings. These cutouts should include WCS, PSF,
variance and mask information. The Alert should also include program
and/or scheduler metadata.
\end{minipage}
\\ \cdashline{1-2}
Requirement Discussion &
\begin{minipage}[]{13cm}
\textbf{Discussion:} The aim for the Alert content is to include
sufficient information to be relatively self-contained, and to minimize
the demand for follow-up queries of the Level-1 database. This approach
will likely increase the speed and efficiency of down-stream object
classifiers. The included program and/or scheduler metadata should be
sufficient to identify whether the image is associated with a Special
Program (such as an in-progress Deep Drilling Field).
\end{minipage}
\\ \cdashline{1-2}
Requirement Priority & 1b \\ \cdashline{1-2}
Upper Level Requirement &
\begin{tabular}{cl}
OSS-REQ-0128 & Alerts \\
\end{tabular}
\\ \hline
\end{longtable}
}


\subsubsection{Test Cases Summary}
\begin{longtable}{p{3cm}p{2.5cm}p{2.5cm}p{3cm}p{4cm}}
\toprule
\href{https://jira.lsstcorp.org/secure/Tests.jspa\#/testCase/LVV-T54}{LVV-T54} & \multicolumn{4}{p{12cm}}{ Verify implementation of Alert Content } \\ \hline
\textbf{Owner} & \textbf{Status} & \textbf{Version} & \textbf{Critical Event} & \textbf{Verification Type} \\ \hline
Eric Bellm & Draft & 1 & false & Test \\ \hline
\end{longtable}
{\scriptsize
\textbf{Objective:}\\
Verify that the DMS creates an Alert for each detected DIASource\\
Verify that this Alert is broadcasted using community protocols\\
Verify that the context of the Alert packet match requirements.
}
  
 \newpage 
\subsection{[LVV-106] DMS-REQ-0275-V-01: Object Catalog }\label{lvv-106}

\begin{longtable}{cccc}
\hline
\textbf{Jira Link} & \textbf{Assignee} & \textbf{Status} & \textbf{Test Cases}\\ \hline
\href{https://jira.lsstcorp.org/browse/LVV-106}{LVV-106} &
Jim Bosch & Not Covered &
\begin{tabular}{c}
LVV-T12 \\
LVV-T14 \\
LVV-T67 \\
\end{tabular}
\\
\hline
\end{longtable}

\textbf{Verification Element Description:} \\
Precursor data spread across multiple epochs. Must contain SSObjects and
DIASources. Must be coaddable. Can be single filter. Must verify Object
catalog

{\footnotesize
\begin{longtable}{p{2.5cm}p{13.5cm}}
\hline
\multicolumn{2}{c}{\textbf{Requirement Details}}\\ \hline
Requirement ID & DMS-REQ-0275 \\ \cdashline{1-2}
Requirement Description &
\begin{minipage}[]{13cm}
\textbf{Specification:} The DMS shall create an Object Catalog, based on
sources deblended based on knowledge of CoaddSource, DIASource,
DIAObject, and SSObject Catalogs, after multi-epoch spatial association
and characterization.
\end{minipage}
\\ \cdashline{1-2}
Requirement Priority & 1b \\ \cdashline{1-2}
Upper Level Requirement &
\begin{tabular}{cl}
OSS-REQ-0137 & Catalogs (Level 2) \\
\end{tabular}
\\ \hline
\end{longtable}
}

\subsubsection{Verified By}
\begin{itemize}
\item . DM-9953 (\ref{dm-9953}) Pixels rejected from coaddition and CCD are not masked on coadds
\item . DM-13058 (\ref{dm-13058}) Inconsistent aperture corrections in W44 reprocessing of HSC RC1
\end{itemize}

\subsubsection{Test Cases Summary}
\begin{longtable}{p{3cm}p{2.5cm}p{2.5cm}p{3cm}p{4cm}}
\toprule
\href{https://jira.lsstcorp.org/secure/Tests.jspa\#/testCase/LVV-T12}{LVV-T12} & \multicolumn{4}{p{12cm}}{ DRP-00-10: Data Release Includes Required Data Products } \\ \hline
\textbf{Owner} & \textbf{Status} & \textbf{Version} & \textbf{Critical Event} & \textbf{Verification Type} \\ \hline
Jim Bosch & Deprecated & 1 & false & Test \\ \hline
\end{longtable}
{\scriptsize
\textbf{Objective:}\\
This test will check that the basic data products which should be in an
data release are generated by execution of the science payload.\\
These products will include:

\begin{itemize}
\tightlist
\item
  Source catalogs, derived from PVIs and coadded images (DMS-REQ-0267 \&
  DMS-REQ-0277);
\item
  Forced source catalogs (DMS-REQ-0268);
\item
  Object catalogs (DMS-REQ-0275);
\item
  Processed visit images (PVIs; DMS-REQ-0069);
\item
  Coadded images (DMS-REQ-0279);
\end{itemize}
}
\begin{longtable}{p{3cm}p{2.5cm}p{2.5cm}p{3cm}p{4cm}}
\toprule
\href{https://jira.lsstcorp.org/secure/Tests.jspa\#/testCase/LVV-T14}{LVV-T14} & \multicolumn{4}{p{12cm}}{ DRP-00-25: Scientific Verification of Object Catalog } \\ \hline
\textbf{Owner} & \textbf{Status} & \textbf{Version} & \textbf{Critical Event} & \textbf{Verification Type} \\ \hline
Jim Bosch & Deprecated & 1 & false & Test \\ \hline
\end{longtable}
{\scriptsize
\textbf{Objective:}\\
This test will check that the object catalogs delivered by the DRP
science payload meet the requirements laid down by \citeds{LSE-61}.\\
Specifically, this will demonstrate that:

\begin{itemize}
\tightlist
\item
  Measurements in the catalog are presented in flux units
  (DMS-REQ-0347);
\item
  Derived quantities are provided in pre-computed columns
  (DMS-REQ-0331);
\item
  Aperture corrections for different photometry algorithms are
  consistent.
\item
  PSF models correctly predict the ellipticities of stars over each
  tract.
\item
  Photometry measurements are consistent with reference catalog
  photometry (including sources not used in photometric calibration).
\item
  Astrometry measurements are consistent with reference catalog
  positions (including sources not used in astrometric calibration).
\item
  Forced and unforced photometry measurements are consistent.
\item
  The slope of the stellar locus in color-color space is not a function
  of position on the sky.
\end{itemize}

This test does not include quantitative targets for the science quality
criteria; we instead re- quire for each test that we be able to quickly
construct a plot in which such a target can be visualized.\\
All science quality tests in this section shall distinguish between
blended and isolated objects.
}
\begin{longtable}{p{3cm}p{2.5cm}p{2.5cm}p{3cm}p{4cm}}
\toprule
\href{https://jira.lsstcorp.org/secure/Tests.jspa\#/testCase/LVV-T67}{LVV-T67} & \multicolumn{4}{p{12cm}}{ Verify implementation of Object Catalog } \\ \hline
\textbf{Owner} & \textbf{Status} & \textbf{Version} & \textbf{Critical Event} & \textbf{Verification Type} \\ \hline
Jim Bosch & Draft & 1 & false & Test \\ \hline
\end{longtable}
{\scriptsize
\textbf{Objective:}\\
Verify that the DRP pipelines produce an Object catalog derived from
detections made on both coadded images and difference images and
measurements performed on coadds and possibly overlapping single-epoch
images.
}
  
 \newpage 
\subsection{[LVV-107] DMS-REQ-0276-V-01: Object Characterization }\label{lvv-107}

\begin{longtable}{cccc}
\hline
\textbf{Jira Link} & \textbf{Assignee} & \textbf{Status} & \textbf{Test Cases}\\ \hline
\href{https://jira.lsstcorp.org/browse/LVV-107}{LVV-107} &
Jim Bosch & Not Covered &
\begin{tabular}{c}
LVV-T69 \\
\end{tabular}
\\
\hline
\end{longtable}

\textbf{Verification Element Description:} \\
For each object in DMS-REQ-0275 verify that the characterization
measures are defined.

{\footnotesize
\begin{longtable}{p{2.5cm}p{13.5cm}}
\hline
\multicolumn{2}{c}{\textbf{Requirement Details}}\\ \hline
Requirement ID & DMS-REQ-0276 \\ \cdashline{1-2}
Requirement Description &
\begin{minipage}[]{13cm}
\textbf{Specification:} Each entry in the Object Catalog shall include
the following characterization measures: a point-source model fit, a
bulge-disk model fit, standard colors, a centroid, adaptive moments,
Petrosian and Kron fluxes, surface brightness at multiple apertures,
proper motion and parallax, and a variability characterization.
\end{minipage}
\\ \cdashline{1-2}
Requirement Discussion &
\begin{minipage}[]{13cm}
\textbf{Discussion:} These measurements are intended to enable LSST
``static sky'' science.
\end{minipage}
\\ \cdashline{1-2}
Requirement Priority & 1b \\ \cdashline{1-2}
Upper Level Requirement &
\begin{tabular}{cl}
OSS-REQ-0137 & Catalogs (Level 2) \\
\end{tabular}
\\ \hline
\end{longtable}
}


\subsubsection{Test Cases Summary}
\begin{longtable}{p{3cm}p{2.5cm}p{2.5cm}p{3cm}p{4cm}}
\toprule
\href{https://jira.lsstcorp.org/secure/Tests.jspa\#/testCase/LVV-T69}{LVV-T69} & \multicolumn{4}{p{12cm}}{ Verify implementation of Object Characterization } \\ \hline
\textbf{Owner} & \textbf{Status} & \textbf{Version} & \textbf{Critical Event} & \textbf{Verification Type} \\ \hline
Jim Bosch & Draft & 1 & false & Test \\ \hline
\end{longtable}
{\scriptsize
\textbf{Objective:}\\
Verify that Object catalogs produced by the DRP pipeline include all
measurements listed in DMS-REQ-0276: a point-source model fit, a
bulge-disk model fit, standard colors, a centroid, adap- tive moments,
Petrosian and Kron fluxes, surface brightness at multiple apertures,
proper motion and parallax, and a variability characterization.
}
  
 \newpage 
\subsection{[LVV-108] DMS-REQ-0277-V-01: Coadd Source Catalog }\label{lvv-108}

\begin{longtable}{cccc}
\hline
\textbf{Jira Link} & \textbf{Assignee} & \textbf{Status} & \textbf{Test Cases}\\ \hline
\href{https://jira.lsstcorp.org/browse/LVV-108}{LVV-108} &
Jim Bosch & Not Covered &
\begin{tabular}{c}
\end{tabular}
\\
\hline
\end{longtable}

\textbf{Verification Element Description:} \\
Precursor data. Do a miniDRP and verify that a source catalog is created
at that threshold. It's not clear why we have a requirement for a
transient internal catalog.

{\footnotesize
\begin{longtable}{p{2.5cm}p{13.5cm}}
\hline
\multicolumn{2}{c}{\textbf{Requirement Details}}\\ \hline
Requirement ID & DMS-REQ-0277 \\ \cdashline{1-2}
Requirement Description &
\begin{minipage}[]{13cm}
\textbf{Specification:} The DMS shall, in the course of creating the
master Source Catalog, create a catalog from the coadds of all sources
detected in each passband with a SNR \textgreater{}
\textbf{coaddDetectThresh}.
\end{minipage}
\\ \cdashline{1-2}
Requirement Parameters & \textbf{coaddDetectThresh = 5{{[}float{]}}} S/N threshold for detecting
sources in Co-Add images for building the Source Catalog. \\ \cdashline{1-2}
Requirement Discussion &
\begin{minipage}[]{13cm}
\textbf{Discussion:} CoaddSources are in general composites of
overlapping astrophysical objects. This catalog is an intermediate
product in DR production, and will not be permanently archived nor
released to end-users.
\end{minipage}
\\ \cdashline{1-2}
Requirement Priority & 1b \\ \cdashline{1-2}
Upper Level Requirement &
\begin{tabular}{cl}
OSS-REQ-0137 & Catalogs (Level 2) \\
DMS-REQ-0267 & Source Catalog \\
\end{tabular}
\\ \hline
\end{longtable}
}


  
 \newpage 
\subsection{[LVV-109] DMS-REQ-0278-V-01: Coadd Image Method Constraints }\label{lvv-109}

\begin{longtable}{cccc}
\hline
\textbf{Jira Link} & \textbf{Assignee} & \textbf{Status} & \textbf{Test Cases}\\ \hline
\href{https://jira.lsstcorp.org/browse/LVV-109}{LVV-109} &
Jim Bosch & Not Covered &
\begin{tabular}{c}
LVV-T16 \\
LVV-T72 \\
\end{tabular}
\\
\hline
\end{longtable}

\textbf{Verification Element Description:} \\
This is like DMS-REQ-0279 but specifically for overlapping spatial
visits and describing HOW it should be done. Verify that the images are
on the required output grid.

{\footnotesize
\begin{longtable}{p{2.5cm}p{13.5cm}}
\hline
\multicolumn{2}{c}{\textbf{Requirement Details}}\\ \hline
Requirement ID & DMS-REQ-0278 \\ \cdashline{1-2}
Requirement Description &
\begin{minipage}[]{13cm}
\textbf{Specification:} Coadd Images shall be created by combining
spatially overlapping Processed Visit Images (on which bad pixels and
transient sources have been masked), where the contributing Processed
Visit Images have been re-projected to a common reference geometry, and
matched to a common background level which best approximates the
astrophysical background.
\end{minipage}
\\ \cdashline{1-2}
Requirement Discussion &
\begin{minipage}[]{13cm}
\textbf{Discussion:} It is expected that coadded images will be produced
for all observed regions of the sky, not just the main survey area.
\end{minipage}
\\ \cdashline{1-2}
Requirement Priority & 1b \\ \cdashline{1-2}
Upper Level Requirement &
\begin{tabular}{cl}
OSS-REQ-0136 & Co-added Exposures \\
\end{tabular}
\\ \hline
\end{longtable}
}


\subsubsection{Test Cases Summary}
\begin{longtable}{p{3cm}p{2.5cm}p{2.5cm}p{3cm}p{4cm}}
\toprule
\href{https://jira.lsstcorp.org/secure/Tests.jspa\#/testCase/LVV-T16}{LVV-T16} & \multicolumn{4}{p{12cm}}{ DRP-00-35: Scientific Verification of Coadd Images } \\ \hline
\textbf{Owner} & \textbf{Status} & \textbf{Version} & \textbf{Critical Event} & \textbf{Verification Type} \\ \hline
Jim Bosch & Deprecated & 1 & false & Test \\ \hline
\end{longtable}
{\scriptsize
\textbf{Objective:}\\
This test will check that the coadded images delivered by the DRP
science payload meet the requirements laid down by \citeds{LSE-61}.\\
Specifically, this will demonstrate that:

\begin{itemize}
\tightlist
\item
  Coadds have been generated and persisted during payload execution;~
\item
  Each coadd provides a spatially varying PSF model (DMS-REQ-0047).
\item
  Saturated pixels are correctly masked.
\item
  Pixels affected by satellite trails and ghosts are rejected from the
  coadd.
\item
  The background is not oversubtracted around bright objects.
\end{itemize}

This test does not include quantitative targets for the science quality
criteria; we instead require for each test that we be able to quickly
construct a plot or display summary images that allow such a target can
be visualized.\\[2\baselineskip]
}
\begin{longtable}{p{3cm}p{2.5cm}p{2.5cm}p{3cm}p{4cm}}
\toprule
\href{https://jira.lsstcorp.org/secure/Tests.jspa\#/testCase/LVV-T72}{LVV-T72} & \multicolumn{4}{p{12cm}}{ Verify implementation of Coadd Image Method Constraints } \\ \hline
\textbf{Owner} & \textbf{Status} & \textbf{Version} & \textbf{Critical Event} & \textbf{Verification Type} \\ \hline
Jim Bosch & Draft & 1 & false & Test \\ \hline
\end{longtable}
{\scriptsize
\textbf{Objective:}\\
Verify the implementation of how Coadd images are created.
}
  
 \newpage 
\subsection{[LVV-110] DMS-REQ-0279-V-01: Deep Detection Coadds }\label{lvv-110}

\begin{longtable}{cccc}
\hline
\textbf{Jira Link} & \textbf{Assignee} & \textbf{Status} & \textbf{Test Cases}\\ \hline
\href{https://jira.lsstcorp.org/browse/LVV-110}{LVV-110} &
Jim Bosch & Not Covered &
\begin{tabular}{c}
LVV-T12 \\
LVV-T16 \\
LVV-T73 \\
\end{tabular}
\\
\hline
\end{longtable}

\textbf{Verification Element Description:} \\
Precursor data. Multi filter. System should automatically trigger co-add
processing and filter out poor data. Timescale for this should be
configurable. Add more data and verify coadd has been changed.

{\footnotesize
\begin{longtable}{p{2.5cm}p{13.5cm}}
\hline
\multicolumn{2}{c}{\textbf{Requirement Details}}\\ \hline
Requirement ID & DMS-REQ-0279 \\ \cdashline{1-2}
Requirement Description &
\begin{minipage}[]{13cm}
\textbf{Specification:} The DMS shall periodicaly create Co-added Images
in each of the u,g,r,i,z,y passbands by combining all archived exposures
taken of the same region of sky and in the same passband that meet
specified quality conditions.
\end{minipage}
\\ \cdashline{1-2}
Requirement Discussion &
\begin{minipage}[]{13cm}
\textbf{Discussion:} Quality attributes may include thresholds on
seeing, sky brightness, wavefront quality, PSF shape and spatial
variability, or date of exposure.
\end{minipage}
\\ \cdashline{1-2}
Requirement Priority & 1b \\ \cdashline{1-2}
Upper Level Requirement &
\begin{tabular}{cl}
OSS-REQ-0136 & Co-added Exposures \\
\end{tabular}
\\ \hline
\end{longtable}
}


\subsubsection{Test Cases Summary}
\begin{longtable}{p{3cm}p{2.5cm}p{2.5cm}p{3cm}p{4cm}}
\toprule
\href{https://jira.lsstcorp.org/secure/Tests.jspa\#/testCase/LVV-T12}{LVV-T12} & \multicolumn{4}{p{12cm}}{ DRP-00-10: Data Release Includes Required Data Products } \\ \hline
\textbf{Owner} & \textbf{Status} & \textbf{Version} & \textbf{Critical Event} & \textbf{Verification Type} \\ \hline
Jim Bosch & Deprecated & 1 & false & Test \\ \hline
\end{longtable}
{\scriptsize
\textbf{Objective:}\\
This test will check that the basic data products which should be in an
data release are generated by execution of the science payload.\\
These products will include:

\begin{itemize}
\tightlist
\item
  Source catalogs, derived from PVIs and coadded images (DMS-REQ-0267 \&
  DMS-REQ-0277);
\item
  Forced source catalogs (DMS-REQ-0268);
\item
  Object catalogs (DMS-REQ-0275);
\item
  Processed visit images (PVIs; DMS-REQ-0069);
\item
  Coadded images (DMS-REQ-0279);
\end{itemize}
}
\begin{longtable}{p{3cm}p{2.5cm}p{2.5cm}p{3cm}p{4cm}}
\toprule
\href{https://jira.lsstcorp.org/secure/Tests.jspa\#/testCase/LVV-T16}{LVV-T16} & \multicolumn{4}{p{12cm}}{ DRP-00-35: Scientific Verification of Coadd Images } \\ \hline
\textbf{Owner} & \textbf{Status} & \textbf{Version} & \textbf{Critical Event} & \textbf{Verification Type} \\ \hline
Jim Bosch & Deprecated & 1 & false & Test \\ \hline
\end{longtable}
{\scriptsize
\textbf{Objective:}\\
This test will check that the coadded images delivered by the DRP
science payload meet the requirements laid down by \citeds{LSE-61}.\\
Specifically, this will demonstrate that:

\begin{itemize}
\tightlist
\item
  Coadds have been generated and persisted during payload execution;~
\item
  Each coadd provides a spatially varying PSF model (DMS-REQ-0047).
\item
  Saturated pixels are correctly masked.
\item
  Pixels affected by satellite trails and ghosts are rejected from the
  coadd.
\item
  The background is not oversubtracted around bright objects.
\end{itemize}

This test does not include quantitative targets for the science quality
criteria; we instead require for each test that we be able to quickly
construct a plot or display summary images that allow such a target can
be visualized.\\[2\baselineskip]
}
\begin{longtable}{p{3cm}p{2.5cm}p{2.5cm}p{3cm}p{4cm}}
\toprule
\href{https://jira.lsstcorp.org/secure/Tests.jspa\#/testCase/LVV-T73}{LVV-T73} & \multicolumn{4}{p{12cm}}{ Verify implementation of Deep Detection Coadds } \\ \hline
\textbf{Owner} & \textbf{Status} & \textbf{Version} & \textbf{Critical Event} & \textbf{Verification Type} \\ \hline
Jim Bosch & Draft & 1 & false & Test \\ \hline
\end{longtable}
{\scriptsize
\textbf{Objective:}\\
Verify that the DRP pipelines produce a suite of per-band coadded images
that are optimized for depth.
}
  
 \newpage 
\subsection{[LVV-111] DMS-REQ-0280-V-01: Template Coadds }\label{lvv-111}

\begin{longtable}{cccc}
\hline
\textbf{Jira Link} & \textbf{Assignee} & \textbf{Status} & \textbf{Test Cases}\\ \hline
\href{https://jira.lsstcorp.org/browse/LVV-111}{LVV-111} &
Eric Bellm & Not Covered &
\begin{tabular}{c}
LVV-T74 \\
\end{tabular}
\\
\hline
\end{longtable}

\textbf{Verification Element Description:} \\
Precursor data. Not obvious this has to be demonstrated with all
filters. Is ``periodic'' manual or automated? Much like DMS-REQ-0279
with different constraints. Demonstrate that templates are created with
appropriate bins.

{\footnotesize
\begin{longtable}{p{2.5cm}p{13.5cm}}
\hline
\multicolumn{2}{c}{\textbf{Requirement Details}}\\ \hline
Requirement ID & DMS-REQ-0280 \\ \cdashline{1-2}
Requirement Description &
\begin{minipage}[]{13cm}
\textbf{Specification:} The DMS shall periodically create Template
Images in each of the u,g,r,i,z,y passbands. Templates may be
constructed as part of executing the Data Release Production payload, or
by a separate execution of the Template Generation payload. Prior to
their availability from Data Releases these coadds shall be created
incrementally when sufficient data passing relevant quality criteria is
available.
\end{minipage}
\\ \cdashline{1-2}
Requirement Discussion &
\begin{minipage}[]{13cm}
\textbf{Discussion:} Image Templates are used by the Image Difference
pipeline in the course of identifying transient or variable sources. The
temporal range of epochs may be limited to avoid confusing slowly moving
sources (such as high proper motion stars) with genuine transients.
Incremental template building enables Alert Production when no Data
Release template is yet available. It is anticipated that incremental
template generation could be run nightly, but once a template is
produced for a sky position and filter it will not be replaced until the
next Data Release to avoid repeated baseline changes. To enable artifact
rejection and to comply with OSS-REQ-0158, incremental templates will be
built with at least three images.
\end{minipage}
\\ \cdashline{1-2}
Requirement Priority & 1b \\ \cdashline{1-2}
Upper Level Requirement &
\begin{tabular}{cl}
OSS-REQ-0158 & Coaddition for Templates for Subtraction \\
OSS-REQ-0136 & Co-added Exposures \\
\end{tabular}
\\ \hline
\end{longtable}
}


\subsubsection{Test Cases Summary}
\begin{longtable}{p{3cm}p{2.5cm}p{2.5cm}p{3cm}p{4cm}}
\toprule
\href{https://jira.lsstcorp.org/secure/Tests.jspa\#/testCase/LVV-T74}{LVV-T74} & \multicolumn{4}{p{12cm}}{ Verify implementation of Template Coadds } \\ \hline
\textbf{Owner} & \textbf{Status} & \textbf{Version} & \textbf{Critical Event} & \textbf{Verification Type} \\ \hline
Eric Bellm & Draft & 1 & false & Test \\ \hline
\end{longtable}
{\scriptsize
\textbf{Objective:}\\
Verify that the DMS can produce Template Coadds for DIA processing.
}
  
 \newpage 
\subsection{[LVV-112] DMS-REQ-0281-V-01: Multi-band Coadds }\label{lvv-112}

\begin{longtable}{cccc}
\hline
\textbf{Jira Link} & \textbf{Assignee} & \textbf{Status} & \textbf{Test Cases}\\ \hline
\href{https://jira.lsstcorp.org/browse/LVV-112}{LVV-112} &
Jim Bosch & Not Covered &
\begin{tabular}{c}
LVV-T75 \\
\end{tabular}
\\
\hline
\end{longtable}

\textbf{Verification Element Description:} \\
Like DMS-REQ-0279 with different constraints.

{\footnotesize
\begin{longtable}{p{2.5cm}p{13.5cm}}
\hline
\multicolumn{2}{c}{\textbf{Requirement Details}}\\ \hline
Requirement ID & DMS-REQ-0281 \\ \cdashline{1-2}
Requirement Description &
\begin{minipage}[]{13cm}
\textbf{Specification:} The DMS shall periodically create Multi-band
Coadd images which are constructed similarly to Deep Detection Coadds,
but where all passbands are combined.
\end{minipage}
\\ \cdashline{1-2}
Requirement Discussion &
\begin{minipage}[]{13cm}
\textbf{Discussion:} The multi-color Coadds are intended for very deep
detection.
\end{minipage}
\\ \cdashline{1-2}
Requirement Priority & 1b \\ \cdashline{1-2}
Upper Level Requirement &
\begin{tabular}{cl}
OSS-REQ-0136 & Co-added Exposures \\
\end{tabular}
\\ \hline
\end{longtable}
}


\subsubsection{Test Cases Summary}
\begin{longtable}{p{3cm}p{2.5cm}p{2.5cm}p{3cm}p{4cm}}
\toprule
\href{https://jira.lsstcorp.org/secure/Tests.jspa\#/testCase/LVV-T75}{LVV-T75} & \multicolumn{4}{p{12cm}}{ Verify implementation of Multi-band Coadds } \\ \hline
\textbf{Owner} & \textbf{Status} & \textbf{Version} & \textbf{Critical Event} & \textbf{Verification Type} \\ \hline
Jim Bosch & Draft & 1 & false & Test \\ \hline
\end{longtable}
{\scriptsize
\textbf{Objective:}\\
Verify that the DRP pipelines produce multi-band coadds for detection
purposes.
}
  
 \newpage 
\subsection{[LVV-113] DMS-REQ-0282-V-01: Dark Current Correction Frame Creation }\label{lvv-113}

\begin{longtable}{cccc}
\hline
\textbf{Jira Link} & \textbf{Assignee} & \textbf{Status} & \textbf{Test Cases}\\ \hline
\href{https://jira.lsstcorp.org/browse/LVV-113}{LVV-113} &
Robert Lupton & Not Covered &
\begin{tabular}{c}
LVV-T90 \\
\end{tabular}
\\
\hline
\end{longtable}

\textbf{Verification Element Description:} \\
Can demonstrate dark processing with camera in lab and with simulated
dark data.

{\footnotesize
\begin{longtable}{p{2.5cm}p{13.5cm}}
\hline
\multicolumn{2}{c}{\textbf{Requirement Details}}\\ \hline
Requirement ID & DMS-REQ-0282 \\ \cdashline{1-2}
Requirement Description &
\begin{minipage}[]{13cm}
\textbf{Specification:} The DMS shall produce on an as-needed basis a
dark current correction image, which is constructed from multiple,
closed-shutter exposures of appropriate duration. The effectiveness of
the Dark Correction shall be verified in production processing on
science data.
\end{minipage}
\\ \cdashline{1-2}
Requirement Discussion &
\begin{minipage}[]{13cm}
\textbf{Discussion:} The need for a dark current correction will have to
be quantified during Commissioning. Collecting closed-dome dark
exposures may be deemed necessary to monitor the health of the
detectors, even if not used in calibration processing.
\end{minipage}
\\ \cdashline{1-2}
Requirement Priority & 1a \\ \cdashline{1-2}
Upper Level Requirement &
\begin{tabular}{cl}
OSS-REQ-0271 & Supported Image Types \\
OSS-REQ-0046 & Calibration \\
\end{tabular}
\\ \hline
\end{longtable}
}


\subsubsection{Test Cases Summary}
\begin{longtable}{p{3cm}p{2.5cm}p{2.5cm}p{3cm}p{4cm}}
\toprule
\href{https://jira.lsstcorp.org/secure/Tests.jspa\#/testCase/LVV-T90}{LVV-T90} & \multicolumn{4}{p{12cm}}{ Verify implementation of Dark Current Correction Frame } \\ \hline
\textbf{Owner} & \textbf{Status} & \textbf{Version} & \textbf{Critical Event} & \textbf{Verification Type} \\ \hline
Robert Lupton & Defined & 1 & false & Test \\ \hline
\end{longtable}
{\scriptsize
\textbf{Objective:}\\
Verify that the DMS can produce a dark correction frame calibration
product.
}
  
 \newpage 
\subsection{[LVV-114] DMS-REQ-0283-V-01: Fringe Correction Frame }\label{lvv-114}

\begin{longtable}{cccc}
\hline
\textbf{Jira Link} & \textbf{Assignee} & \textbf{Status} & \textbf{Test Cases}\\ \hline
\href{https://jira.lsstcorp.org/browse/LVV-114}{LVV-114} &
Robert Lupton & Not Covered &
\begin{tabular}{c}
LVV-T91 \\
\end{tabular}
\\
\hline
\end{longtable}

\textbf{Verification Element Description:} \\
Needs a real camera during commissioning and data taken in the correct
mode. Can possibly be done prior to commissioning with simulated data.

{\footnotesize
\begin{longtable}{p{2.5cm}p{13.5cm}}
\hline
\multicolumn{2}{c}{\textbf{Requirement Details}}\\ \hline
Requirement ID & DMS-REQ-0283 \\ \cdashline{1-2}
Requirement Description &
\begin{minipage}[]{13cm}
\textbf{Specification:} The DMS shall produce on an as-needed basis an
image that corrects for detector fringing. The effectiveness of the
Fringe Correction shall be verified in production processing on science
data.
\end{minipage}
\\ \cdashline{1-2}
Requirement Discussion &
\begin{minipage}[]{13cm}
\textbf{Discussion:} Fringing is likely to affect only the reddest
filters, where the CCD substrate becomes semi-transparent to incident
light.
\end{minipage}
\\ \cdashline{1-2}
Requirement Priority & 1b \\ \cdashline{1-2}
Upper Level Requirement &
\begin{tabular}{cl}
OSS-REQ-0271 & Supported Image Types \\
OSS-REQ-0046 & Calibration \\
\end{tabular}
\\ \hline
\end{longtable}
}


\subsubsection{Test Cases Summary}
\begin{longtable}{p{3cm}p{2.5cm}p{2.5cm}p{3cm}p{4cm}}
\toprule
\href{https://jira.lsstcorp.org/secure/Tests.jspa\#/testCase/LVV-T91}{LVV-T91} & \multicolumn{4}{p{12cm}}{ Verify implementation of Fringe Correction Frame } \\ \hline
\textbf{Owner} & \textbf{Status} & \textbf{Version} & \textbf{Critical Event} & \textbf{Verification Type} \\ \hline
Robert Lupton & Draft & 1 & false & Test \\ \hline
\end{longtable}
{\scriptsize
\textbf{Objective:}\\
Verify that the DMS can produce an fringe-correction frame calibration
product.\\
Verify that the DMS can determine the effectiveness of the
fringe-correction frame and determine how often it should be updated.
}
  
 \newpage 
\subsection{[LVV-116] DMS-REQ-0285-V-01: Level 1 Source Association }\label{lvv-116}

\begin{longtable}{cccc}
\hline
\textbf{Jira Link} & \textbf{Assignee} & \textbf{Status} & \textbf{Test Cases}\\ \hline
\href{https://jira.lsstcorp.org/browse/LVV-116}{LVV-116} &
Eric Bellm & Not Covered &
\begin{tabular}{c}
LVV-T22 \\
LVV-T108 \\
LVV-T550 \\
\end{tabular}
\\
\hline
\end{longtable}

\textbf{Verification Element Description:} \\
How is this not just DMS-REQ-0271 rewritten? Is ``clusters'' important
here? Night of precursor L1 data processing should result in DIAObject
and SSObject association.

{\footnotesize
\begin{longtable}{p{2.5cm}p{13.5cm}}
\hline
\multicolumn{2}{c}{\textbf{Requirement Details}}\\ \hline
Requirement ID & DMS-REQ-0285 \\ \cdashline{1-2}
Requirement Description &
\begin{minipage}[]{13cm}
\textbf{Specification:} The DMS shall associate clusters of DIASources
detected on multiple visits taken at different times with either a
DIAObject or an SSObject.
\end{minipage}
\\ \cdashline{1-2}
Requirement Discussion &
\begin{minipage}[]{13cm}
\textbf{Discussion:} The association will represent the underlying
astrophysical phenomenon.
\end{minipage}
\\ \cdashline{1-2}
Requirement Priority & 1b \\ \cdashline{1-2}
Upper Level Requirement &
\begin{tabular}{cl}
OSS-REQ-0130 & Catalogs (Level 1) \\
OSS-REQ-0160 & Level 1 Difference Source - Difference Object Association Quality \\
OSS-REQ-0159 & Level 1 Solar System Object Quality \\
\end{tabular}
\\ \hline
\end{longtable}
}


\subsubsection{Test Cases Summary}
\begin{longtable}{p{3cm}p{2.5cm}p{2.5cm}p{3cm}p{4cm}}
\toprule
\href{https://jira.lsstcorp.org/secure/Tests.jspa\#/testCase/LVV-T22}{LVV-T22} & \multicolumn{4}{p{12cm}}{ AG-00-25: Scientific Verification of DIAObject Catalog } \\ \hline
\textbf{Owner} & \textbf{Status} & \textbf{Version} & \textbf{Critical Event} & \textbf{Verification Type} \\ \hline
Eric Bellm & Deprecated & 1 & false & Test \\ \hline
\end{longtable}
{\scriptsize
\textbf{Objective:}\\
This test will check that the DIAObject catalogs delivered by the Alert
Generation science pay- load meet the requirements laid down by
\citeds{LSE-61}.\\
Specifically, this will demonstrate that:

\begin{itemize}
\tightlist
\item
  DIAObjects are recorded with unique identifiers (DMS-REQ-0271);
\item
  Measurements in the catalog are presented in flux units
  (DMS-REQ-0347);
\item
  EachDIAObjectrecordcontainscontainsanappropriatesetofsummaryattributes(DMS-
  REQ-0271 and DMS-REQ-0272). Note:

  \begin{itemize}
  \tightlist
  \item
    This test is executed independently of the Data Release Production
    system. Hence, DIAObjects are not associated to Objects, and the
    association metadata specified by DMS-REQ-0271 is not expected to be
    available.
  \item
    TheLDM-503-3erapipelineisnotexpectedtocalculateorpersistallattributesspec-
    ified by DMS-REQ-0272 requirement.
  \end{itemize}
\item
  Relevant derived quantities are provided in pre-computed columns
  (DMS-REQ-0331);~
\end{itemize}

This test does not include quantitative targets for the science quality
criteria.
}
\begin{longtable}{p{3cm}p{2.5cm}p{2.5cm}p{3cm}p{4cm}}
\toprule
\href{https://jira.lsstcorp.org/secure/Tests.jspa\#/testCase/LVV-T108}{LVV-T108} & \multicolumn{4}{p{12cm}}{ Verify implementation of Level 1 Source Association } \\ \hline
\textbf{Owner} & \textbf{Status} & \textbf{Version} & \textbf{Critical Event} & \textbf{Verification Type} \\ \hline
Eric Bellm & Draft & 1 & false & Test \\ \hline
\end{longtable}
{\scriptsize
\textbf{Objective:}\\
Verify that the DMS associates DIASources into a DIAObject or SSObject.
}
\begin{longtable}{p{3cm}p{2.5cm}p{2.5cm}p{3cm}p{4cm}}
\toprule
\href{https://jira.lsstcorp.org/secure/Tests.jspa\#/testCase/LVV-T550}{LVV-T550} & \multicolumn{4}{p{12cm}}{ MOPS -- orbit association completeness } \\ \hline
\textbf{Owner} & \textbf{Status} & \textbf{Version} & \textbf{Critical Event} & \textbf{Verification Type} \\ \hline
Scott Daniel & Defined & 1 & true & Test \\ \hline
\end{longtable}
{\scriptsize
\textbf{Objective:}\\
Test completeness of orbit association using simulated data
}
  
 \newpage 
\subsection{[LVV-117] DMS-REQ-0286-V-01: SSObject Precovery }\label{lvv-117}

\begin{longtable}{cccc}
\hline
\textbf{Jira Link} & \textbf{Assignee} & \textbf{Status} & \textbf{Test Cases}\\ \hline
\href{https://jira.lsstcorp.org/browse/LVV-117}{LVV-117} &
Eric Bellm & Not Covered &
\begin{tabular}{c}
LVV-T109 \\
\end{tabular}
\\
\hline
\end{longtable}

\textbf{Verification Element Description:} \\
Carefully craft an input dataset from precursor data that ensures that
precovery will only be triggered later in the processing. Check that
precovery occurs and object association is done.

{\footnotesize
\begin{longtable}{p{2.5cm}p{13.5cm}}
\hline
\multicolumn{2}{c}{\textbf{Requirement Details}}\\ \hline
Requirement ID & DMS-REQ-0286 \\ \cdashline{1-2}
Requirement Description &
\begin{minipage}[]{13cm}
\textbf{Specification:} Upon identifying a new SSObject, the DMS shall
associate additional DIAObjects that are consistent with the orbital
parameters (precovery), and update DIAObject entries so associated.
\end{minipage}
\\ \cdashline{1-2}
Requirement Priority & 2 \\ \cdashline{1-2}
Upper Level Requirement &
\begin{tabular}{cl}
OSS-REQ-0159 & Level 1 Solar System Object Quality \\
\end{tabular}
\\ \hline
\end{longtable}
}


\subsubsection{Test Cases Summary}
\begin{longtable}{p{3cm}p{2.5cm}p{2.5cm}p{3cm}p{4cm}}
\toprule
\href{https://jira.lsstcorp.org/secure/Tests.jspa\#/testCase/LVV-T109}{LVV-T109} & \multicolumn{4}{p{12cm}}{ Verify implementation of SSObject Precovery } \\ \hline
\textbf{Owner} & \textbf{Status} & \textbf{Version} & \textbf{Critical Event} & \textbf{Verification Type} \\ \hline
Eric Bellm & Draft & 1 & false & Test \\ \hline
\end{longtable}
{\scriptsize
\textbf{Objective:}\\
Verify that the DMS associates additional DIAObjects (both forward and
back in time) with objects classified as SSObjects.
}
  
 \newpage 
\subsection{[LVV-118] DMS-REQ-0287-V-01: Max look-back time for precovery }\label{lvv-118}

\begin{longtable}{cccc}
\hline
\textbf{Jira Link} & \textbf{Assignee} & \textbf{Status} & \textbf{Test Cases}\\ \hline
\href{https://jira.lsstcorp.org/browse/LVV-118}{LVV-118} &
Eric Bellm & Not Covered &
\begin{tabular}{c}
LVV-T110 \\
\end{tabular}
\\
\hline
\end{longtable}

\textbf{Verification Element Description:} \\
Precursor or simulated L1 data covering precoveryWindow plus a few days.
Detect DIASource towards end of window, ensure, at minimum,
precoveryWindow forced photometry is performed.

Associated element
(\href{https://jira.lsstcorp.org/browse/LVV-9747}{LVV-9747}) satisfies
the lifetime of cached L1 data products.

Associated element
(\href{https://jira.lsstcorp.org/browse/LVV-9746}{LVV-9746}) satisfies
the time in which L1 data products shall be publicly released.

{\footnotesize
\begin{longtable}{p{2.5cm}p{13.5cm}}
\hline
\multicolumn{2}{c}{\textbf{Requirement Details}}\\ \hline
Requirement ID & DMS-REQ-0287 \\ \cdashline{1-2}
Requirement Description &
\begin{minipage}[]{13cm}
\textbf{Specification:} For all DIASources not associated with either
DIAObjects or SSObjects, the DMS shall perform forced photometry at the
location of the new source (precovery) on all Difference Exposures
obtained in the prior \textbf{precoveryWindow}, and make the results
publicly available within \textbf{L1PublicT}.
\end{minipage}
\\ \cdashline{1-2}
Requirement Parameters & {[}\textbf{precoveryWindow = 30{{[}day{]}}} Maximum look-back time for
precovery measurments on prior Exposures., \textbf{l1CacheLifetime =
30{{[}day{]}}} Lifetime in the cache of un-archived Level-1 data
products., \textbf{L1PublicT = 24{{[}hour{]}}} Maximum time from the
acquisition of science data to the release of associated Level 1 Data
Products (except alerts){]} \\ \cdashline{1-2}
Requirement Discussion &
\begin{minipage}[]{13cm}
\textbf{Discussion:} The \textbf{precoveryWindow} is intended to satisfy
the most common scientific use cases (e.g., Supernovae), without placing
an undue burden on the processing infrastructure. For reasons of
practicality and efficiency, \textbf{precoveryWindow} \textless{}=
l*1CacheLifetime*.
\end{minipage}
\\ \cdashline{1-2}
Requirement Priority & 1b \\ \cdashline{1-2}
Upper Level Requirement &
\begin{tabular}{cl}
OSS-REQ-0130 & Catalogs (Level 1) \\
\end{tabular}
\\ \hline
\end{longtable}
}


\subsubsection{Test Cases Summary}
\begin{longtable}{p{3cm}p{2.5cm}p{2.5cm}p{3cm}p{4cm}}
\toprule
\href{https://jira.lsstcorp.org/secure/Tests.jspa\#/testCase/LVV-T110}{LVV-T110} & \multicolumn{4}{p{12cm}}{ Verify implementation of DIASource Precovery } \\ \hline
\textbf{Owner} & \textbf{Status} & \textbf{Version} & \textbf{Critical Event} & \textbf{Verification Type} \\ \hline
Eric Bellm & Draft & 1 & false & Test \\ \hline
\end{longtable}
{\scriptsize
\textbf{Objective:}\\
Verify that DMS performs forced photometry for new DIAObjects at all
available images within the precoveryWindow.
}
  
 \newpage 
\subsection{[LVV-119] DMS-REQ-0288-V-01: Use of External Orbit Catalogs }\label{lvv-119}

\begin{longtable}{cccc}
\hline
\textbf{Jira Link} & \textbf{Assignee} & \textbf{Status} & \textbf{Test Cases}\\ \hline
\href{https://jira.lsstcorp.org/browse/LVV-119}{LVV-119} &
Eric Bellm & Not Covered &
\begin{tabular}{c}
LVV-T111 \\
\end{tabular}
\\
\hline
\end{longtable}

\textbf{Verification Element Description:} \\
Either demonstrate an external catalog being used in MOPS, or show the
code that would use the external catalog. Former preferred.

{\footnotesize
\begin{longtable}{p{2.5cm}p{13.5cm}}
\hline
\multicolumn{2}{c}{\textbf{Requirement Details}}\\ \hline
Requirement ID & DMS-REQ-0288 \\ \cdashline{1-2}
Requirement Description &
\begin{minipage}[]{13cm}
\textbf{Specification:} It shall be possible for DMS to make use of
approved external catalogs and observations to improve the
identification of SSObjects, and therefore increase the purity of the
transient Alert stream in nightly processing.
\end{minipage}
\\ \cdashline{1-2}
Requirement Priority & 2 \\ \cdashline{1-2}
Upper Level Requirement &
\begin{tabular}{cl}
OSS-REQ-0159 & Level 1 Solar System Object Quality \\
\end{tabular}
\\ \hline
\end{longtable}
}


\subsubsection{Test Cases Summary}
\begin{longtable}{p{3cm}p{2.5cm}p{2.5cm}p{3cm}p{4cm}}
\toprule
\href{https://jira.lsstcorp.org/secure/Tests.jspa\#/testCase/LVV-T111}{LVV-T111} & \multicolumn{4}{p{12cm}}{ Verify implementation of Use of External Orbit Catalogs } \\ \hline
\textbf{Owner} & \textbf{Status} & \textbf{Version} & \textbf{Critical Event} & \textbf{Verification Type} \\ \hline
Eric Bellm & Draft & 1 & false & Test \\ \hline
\end{longtable}
{\scriptsize
\textbf{Objective:}\\
Verify that the DMS can make use of external catalogs to improve
identification of SSObjects.
}
  
 \newpage 
\subsection{[LVV-120] DMS-REQ-0289-V-01: Calibration Production Processing }\label{lvv-120}

\begin{longtable}{cccc}
\hline
\textbf{Jira Link} & \textbf{Assignee} & \textbf{Status} & \textbf{Test Cases}\\ \hline
\href{https://jira.lsstcorp.org/browse/LVV-120}{LVV-120} &
Robert Lupton & Not Covered &
\begin{tabular}{c}
LVV-T115 \\
LVV-T1935 \\
\end{tabular}
\\
\hline
\end{longtable}

\textbf{Verification Element Description:} \\
Vague. DM needs to be able to take any calibration data and reduce them.
This requirement does not cover decisions on when to take calibrations.
Show that CPP is in place and can reduce data.

{\footnotesize
\begin{longtable}{p{2.5cm}p{13.5cm}}
\hline
\multicolumn{2}{c}{\textbf{Requirement Details}}\\ \hline
Requirement ID & DMS-REQ-0289 \\ \cdashline{1-2}
Requirement Description &
\begin{minipage}[]{13cm}
\textbf{Specification:} The DMS shall be capable of producing
calibration data products on an as-needed basis, consistent with
monitoring the health and performance of the instrument, the
availability of raw calibration exposures, the temporal stability of the
calibrations, and of the SRD requirements for calibration accuracy.
\end{minipage}
\\ \cdashline{1-2}
Requirement Priority & 1a \\ \cdashline{1-2}
Upper Level Requirement &
\begin{tabular}{cl}
OSS-REQ-0004 & The Archive Facility \\
OSS-REQ-0170 & Calibration Data \\
\end{tabular}
\\ \hline
\end{longtable}
}


\subsubsection{Test Cases Summary}
\begin{longtable}{p{3cm}p{2.5cm}p{2.5cm}p{3cm}p{4cm}}
\toprule
\href{https://jira.lsstcorp.org/secure/Tests.jspa\#/testCase/LVV-T115}{LVV-T115} & \multicolumn{4}{p{12cm}}{ Verify implementation of Calibration Production Processing } \\ \hline
\textbf{Owner} & \textbf{Status} & \textbf{Version} & \textbf{Critical Event} & \textbf{Verification Type} \\ \hline
Kian-Tat Lim & Defined & 1 & false & Test \\ \hline
\end{longtable}
{\scriptsize
\textbf{Objective:}\\
Execute CPP on a variety of representative cadences, and verify that the
calibration pipeline correctly produces necessary calibration products.
}
\begin{longtable}{p{3cm}p{2.5cm}p{2.5cm}p{3cm}p{4cm}}
\toprule
\href{https://jira.lsstcorp.org/secure/Tests.jspa\#/testCase/LVV-T1935}{LVV-T1935} & \multicolumn{4}{p{12cm}}{ Demonstrate ComCam Data Processing Capability } \\ \hline
\textbf{Owner} & \textbf{Status} & \textbf{Version} & \textbf{Critical Event} & \textbf{Verification Type} \\ \hline
Robert Gruendl & Draft & 1 & true & Demonstration \\ \hline
\end{longtable}
{\scriptsize
\textbf{Objective:}\\
To process raw ComCam data and demonstrate that the results are
available either in the shared DM development environment/repository or
in the RSP.
}
  
 \newpage 
\subsection{[LVV-121] DMS-REQ-0290-V-01: Level 3 Data Import }\label{lvv-121}

\begin{longtable}{cccc}
\hline
\textbf{Jira Link} & \textbf{Assignee} & \textbf{Status} & \textbf{Test Cases}\\ \hline
\href{https://jira.lsstcorp.org/browse/LVV-121}{LVV-121} &
Colin Slater & Not Covered &
\begin{tabular}{c}
LVV-T122 \\
\end{tabular}
\\
\hline
\end{longtable}

\textbf{Verification Element Description:} \\
Requires a fixed list of import formats. L3 user uploads catalog into L3
system and can then do queries upon it.

{\footnotesize
\begin{longtable}{p{2.5cm}p{13.5cm}}
\hline
\multicolumn{2}{c}{\textbf{Requirement Details}}\\ \hline
Requirement ID & DMS-REQ-0290 \\ \cdashline{1-2}
Requirement Description &
\begin{minipage}[]{13cm}
\textbf{Specification:} The DMS shall be able to ingest tables from
common file formats (e.g. FITS tables, CSV files with supporting
metadata) to facilitate the loading of external catalogs and the
production of Level-3 data products.
\end{minipage}
\\ \cdashline{1-2}
Requirement Priority & 2 \\ \cdashline{1-2}
Upper Level Requirement &
\begin{tabular}{cl}
OSS-REQ-0140 & Production \\
\end{tabular}
\\ \hline
\end{longtable}
}


\subsubsection{Test Cases Summary}
\begin{longtable}{p{3cm}p{2.5cm}p{2.5cm}p{3cm}p{4cm}}
\toprule
\href{https://jira.lsstcorp.org/secure/Tests.jspa\#/testCase/LVV-T122}{LVV-T122} & \multicolumn{4}{p{12cm}}{ Verify implementation of Level 3 Data Import } \\ \hline
\textbf{Owner} & \textbf{Status} & \textbf{Version} & \textbf{Critical Event} & \textbf{Verification Type} \\ \hline
Colin Slater & Draft & 1 & false & Test \\ \hline
\end{longtable}
{\scriptsize
\textbf{Objective:}\\
Verify that the Science Platform can ingest data from community-standard
file formats.
}
  
 \newpage 
\subsection{[LVV-122] DMS-REQ-0291-V-01: Query Repeatability }\label{lvv-122}

\begin{longtable}{cccc}
\hline
\textbf{Jira Link} & \textbf{Assignee} & \textbf{Status} & \textbf{Test Cases}\\ \hline
\href{https://jira.lsstcorp.org/browse/LVV-122}{LVV-122} &
Colin Slater & Not Covered &
\begin{tabular}{c}
LVV-T96 \\
\end{tabular}
\\
\hline
\end{longtable}

\textbf{Verification Element Description:} \\
Can be verified prior to commissioning with processed precursor test
data along with a defined set of queries. Query on previous DR run is
verified to work even when newer DR is the default.

{\footnotesize
\begin{longtable}{p{2.5cm}p{13.5cm}}
\hline
\multicolumn{2}{c}{\textbf{Requirement Details}}\\ \hline
Requirement ID & DMS-REQ-0291 \\ \cdashline{1-2}
Requirement Description &
\begin{minipage}[]{13cm}
\textbf{Specification:} The DMS shall ensure that any query executed at
a particular point in time against any DMS delivered database shall be
repeatable at a later date, and produce results that are either
identical or include additional results (owing to updates from Level-1
processing).
\end{minipage}
\\ \cdashline{1-2}
Requirement Discussion &
\begin{minipage}[]{13cm}
\textbf{Discussion:} It would be desirable to have the ability to
``save'' a query such that the date or data release would be included
explicitly. Additionally, the ability to associate this query with a DOI
would allow queries to be shared and included in scientific papers
without requiring a large copy and paste.
\end{minipage}
\\ \cdashline{1-2}
Requirement Priority & 1b \\ \cdashline{1-2}
Upper Level Requirement &
\begin{tabular}{cl}
OSS-REQ-0181 & Data Products Query and Download Infrastructure \\
\end{tabular}
\\ \hline
\end{longtable}
}


\subsubsection{Test Cases Summary}
\begin{longtable}{p{3cm}p{2.5cm}p{2.5cm}p{3cm}p{4cm}}
\toprule
\href{https://jira.lsstcorp.org/secure/Tests.jspa\#/testCase/LVV-T96}{LVV-T96} & \multicolumn{4}{p{12cm}}{ Verify implementation of Query Repeatability } \\ \hline
\textbf{Owner} & \textbf{Status} & \textbf{Version} & \textbf{Critical Event} & \textbf{Verification Type} \\ \hline
Colin Slater & Draft & 1 & false & Test \\ \hline
\end{longtable}
{\scriptsize
\textbf{Objective:}\\
Verify that prior queries can be rerun with identical results, or with
new additional data for live (Alert Production) databases.
}
  
 \newpage 
\subsection{[LVV-123] DMS-REQ-0292-V-01: Uniqueness of IDs Across Data Releases }\label{lvv-123}

\begin{longtable}{cccc}
\hline
\textbf{Jira Link} & \textbf{Assignee} & \textbf{Status} & \textbf{Test Cases}\\ \hline
\href{https://jira.lsstcorp.org/browse/LVV-123}{LVV-123} &
Colin Slater & Not Covered &
\begin{tabular}{c}
LVV-T97 \\
\end{tabular}
\\
\hline
\end{longtable}

\textbf{Verification Element Description:} \\
Simple: Inspect the ID generation code and confirm that DR number is
encoded in each ID. Better: With carefully selected precursor data, do
multiple DRP runs and verify that IDs are not reused.

{\footnotesize
\begin{longtable}{p{2.5cm}p{13.5cm}}
\hline
\multicolumn{2}{c}{\textbf{Requirement Details}}\\ \hline
Requirement ID & DMS-REQ-0292 \\ \cdashline{1-2}
Requirement Description &
\begin{minipage}[]{13cm}
\textbf{Specification:} To reduce the likelihood for confusion, all IDs
shall be unique across databases and database versions, other than those
corresponding to uniquely identifiable entities (i.e., IDs of
exposures).
\end{minipage}
\\ \cdashline{1-2}
Requirement Discussion &
\begin{minipage}[]{13cm}
\textbf{Discussion:} For example, DR4 and DR5 (or any other) release
will share no identical Object, Source, DIAObject or DIASource IDs.
\end{minipage}
\\ \cdashline{1-2}
Requirement Priority & 1a \\ \cdashline{1-2}
Upper Level Requirement &
\begin{tabular}{cl}
OSS-REQ-0130 & Catalogs (Level 1) \\
OSS-REQ-0137 & Catalogs (Level 2) \\
\end{tabular}
\\ \hline
\end{longtable}
}


\subsubsection{Test Cases Summary}
\begin{longtable}{p{3cm}p{2.5cm}p{2.5cm}p{3cm}p{4cm}}
\toprule
\href{https://jira.lsstcorp.org/secure/Tests.jspa\#/testCase/LVV-T97}{LVV-T97} & \multicolumn{4}{p{12cm}}{ Verify implementation of Uniqueness of IDs Across Data Releases } \\ \hline
\textbf{Owner} & \textbf{Status} & \textbf{Version} & \textbf{Critical Event} & \textbf{Verification Type} \\ \hline
Kian-Tat Lim & Defined & 1 & false & Test \\ \hline
\end{longtable}
{\scriptsize
\textbf{Objective:}\\
Verify that the IDs of Objects, Sources, DIAObjects, and DIASources from
different Data Releases are unique.
}
  
 \newpage 
\subsection{[LVV-124] DMS-REQ-0293-V-01: Selection of Datasets }\label{lvv-124}

\begin{longtable}{cccc}
\hline
\textbf{Jira Link} & \textbf{Assignee} & \textbf{Status} & \textbf{Test Cases}\\ \hline
\href{https://jira.lsstcorp.org/browse/LVV-124}{LVV-124} &
Jim Bosch & Not Covered &
\begin{tabular}{c}
LVV-T11 \\
LVV-T98 \\
\end{tabular}
\\
\hline
\end{longtable}

\textbf{Verification Element Description:} \\
Demonstrate that composites can be assembled in the butler for a
reasonable sampling of dataset types.

{\footnotesize
\begin{longtable}{p{2.5cm}p{13.5cm}}
\hline
\multicolumn{2}{c}{\textbf{Requirement Details}}\\ \hline
Requirement ID & DMS-REQ-0293 \\ \cdashline{1-2}
Requirement Description &
\begin{minipage}[]{13cm}
\textbf{Specification:} A Dataset may consist of one or more pixel
images, a set of records in a file or database, or any other grouping of
data that are processed or produced as a logical unit. The DMS shall be
able to identify and retrieve complete, consistent datasets for
processing.
\end{minipage}
\\ \cdashline{1-2}
Requirement Discussion &
\begin{minipage}[]{13cm}
\textbf{Discussion:} Logical groupings might be pairs of Exposures in a
Visit, along with supporting metadata and provenance information, or
might be groupings defined in the context of Level-3 processing.
\end{minipage}
\\ \cdashline{1-2}
Requirement Priority & 1a \\ \cdashline{1-2}
Upper Level Requirement &
\begin{tabular}{cl}
OSS-REQ-0176 & Data Access \\
OSS-REQ-0118 & Consistency and Completeness \\
\end{tabular}
\\ \hline
\end{longtable}
}


\subsubsection{Test Cases Summary}
\begin{longtable}{p{3cm}p{2.5cm}p{2.5cm}p{3cm}p{4cm}}
\toprule
\href{https://jira.lsstcorp.org/secure/Tests.jspa\#/testCase/LVV-T11}{LVV-T11} & \multicolumn{4}{p{12cm}}{ DRP-00-05: Execution of the DRP Science Payload by the Batch Production
Service } \\ \hline
\textbf{Owner} & \textbf{Status} & \textbf{Version} & \textbf{Critical Event} & \textbf{Verification Type} \\ \hline
Jim Bosch & Deprecated & 1 & false & Test \\ \hline
\end{longtable}
{\scriptsize
\textbf{Objective:}\\
This test will check that the DRP Science Payload can be executed using
a specific version of the Batch Production Service provided by the LSST
Data Facility. Since the outputs are stored in the Data Backbone, it too
is a component of this test.
}
\begin{longtable}{p{3cm}p{2.5cm}p{2.5cm}p{3cm}p{4cm}}
\toprule
\href{https://jira.lsstcorp.org/secure/Tests.jspa\#/testCase/LVV-T98}{LVV-T98} & \multicolumn{4}{p{12cm}}{ Verify implementation of Selection of Datasets } \\ \hline
\textbf{Owner} & \textbf{Status} & \textbf{Version} & \textbf{Critical Event} & \textbf{Verification Type} \\ \hline
Kian-Tat Lim & Defined & 1 & false & Test \\ \hline
\end{longtable}
{\scriptsize
\textbf{Objective:}\\
Verify that the DMS can identify and retrieve datasets consisting of
logical groupings of Exposures, metadata, provenance, etc., or other
groupings that are processed or produced as a logical unit.
}
  
 \newpage 
\subsection{[LVV-125] DMS-REQ-0294-V-01: Processing of Datasets }\label{lvv-125}

\begin{longtable}{cccc}
\hline
\textbf{Jira Link} & \textbf{Assignee} & \textbf{Status} & \textbf{Test Cases}\\ \hline
\href{https://jira.lsstcorp.org/browse/LVV-125}{LVV-125} &
Robert Lupton & Not Covered &
\begin{tabular}{c}
LVV-T12 \\
LVV-T99 \\
\end{tabular}
\\
\hline
\end{longtable}

\textbf{Verification Element Description:} \\
The intent is (at least partially) that (1) no datasets requested to be
processed will be inadvertently omitted and that (2) no duplicate
results will be produced that need to be de-duplicated downstream. As
written, the requirement could also be read to mean that the status of
all datasets that have been attempted to process must be recorded.

Â~

This requirement needs clarification and is impossible to verify as
written.

{\footnotesize
\begin{longtable}{p{2.5cm}p{13.5cm}}
\hline
\multicolumn{2}{c}{\textbf{Requirement Details}}\\ \hline
Requirement ID & DMS-REQ-0294 \\ \cdashline{1-2}
Requirement Description &
\begin{minipage}[]{13cm}
\textbf{Specification:} The DMS shall process all requested datasets
until either a successful result is recorded or a permanent failure is
recognized. If any dataset is processed, in part or in whole, more than
once, only one of the wholly processed results will be recorded for
further processing.
\end{minipage}
\\ \cdashline{1-2}
Requirement Discussion &
\begin{minipage}[]{13cm}
\textbf{Discussion:} The criteria may be specified by DMS processing
software, or by a scientist end-user for Level-3 production.
\end{minipage}
\\ \cdashline{1-2}
Requirement Priority & 1b \\ \cdashline{1-2}
Upper Level Requirement &
\begin{tabular}{cl}
OSS-REQ-0120 & Consistency \\
OSS-REQ-0119 & Completeness \\
OSS-REQ-0118 & Consistency and Completeness \\
OSS-REQ-0117 & Automated Production \\
\end{tabular}
\\ \hline
\end{longtable}
}


\subsubsection{Test Cases Summary}
\begin{longtable}{p{3cm}p{2.5cm}p{2.5cm}p{3cm}p{4cm}}
\toprule
\href{https://jira.lsstcorp.org/secure/Tests.jspa\#/testCase/LVV-T12}{LVV-T12} & \multicolumn{4}{p{12cm}}{ DRP-00-10: Data Release Includes Required Data Products } \\ \hline
\textbf{Owner} & \textbf{Status} & \textbf{Version} & \textbf{Critical Event} & \textbf{Verification Type} \\ \hline
Jim Bosch & Deprecated & 1 & false & Test \\ \hline
\end{longtable}
{\scriptsize
\textbf{Objective:}\\
This test will check that the basic data products which should be in an
data release are generated by execution of the science payload.\\
These products will include:

\begin{itemize}
\tightlist
\item
  Source catalogs, derived from PVIs and coadded images (DMS-REQ-0267 \&
  DMS-REQ-0277);
\item
  Forced source catalogs (DMS-REQ-0268);
\item
  Object catalogs (DMS-REQ-0275);
\item
  Processed visit images (PVIs; DMS-REQ-0069);
\item
  Coadded images (DMS-REQ-0279);
\end{itemize}
}
\begin{longtable}{p{3cm}p{2.5cm}p{2.5cm}p{3cm}p{4cm}}
\toprule
\href{https://jira.lsstcorp.org/secure/Tests.jspa\#/testCase/LVV-T99}{LVV-T99} & \multicolumn{4}{p{12cm}}{ Verify implementation of Processing of Datasets } \\ \hline
\textbf{Owner} & \textbf{Status} & \textbf{Version} & \textbf{Critical Event} & \textbf{Verification Type} \\ \hline
Kian-Tat Lim & Draft & 1 & false & Test \\ \hline
\end{longtable}
{\scriptsize
\textbf{Objective:}\\
Execute AP and DRP, simulate failures, observe correct processing
}
  
 \newpage 
\subsection{[LVV-127] DMS-REQ-0296-V-01: Pre-cursor, and Real Data }\label{lvv-127}

\begin{longtable}{cccc}
\hline
\textbf{Jira Link} & \textbf{Assignee} & \textbf{Status} & \textbf{Test Cases}\\ \hline
\href{https://jira.lsstcorp.org/browse/LVV-127}{LVV-127} &
Simon Krughoff & Not Covered &
\begin{tabular}{c}
LVV-T132 \\
LVV-T362 \\
\end{tabular}
\\
\hline
\end{longtable}

\textbf{Verification Element Description:} \\
Proven by reducing precursor data.

{\footnotesize
\begin{longtable}{p{2.5cm}p{13.5cm}}
\hline
\multicolumn{2}{c}{\textbf{Requirement Details}}\\ \hline
Requirement ID & DMS-REQ-0296 \\ \cdashline{1-2}
Requirement Description &
\begin{minipage}[]{13cm}
\textbf{Specification:} The DMS shall provide for the ability to process
data from other electronic, pixel-oriented astronomical imaging cameras.
\end{minipage}
\\ \cdashline{1-2}
Requirement Discussion &
\begin{minipage}[]{13cm}
\textbf{Discussion:} A comparison of DMS products to that produced by
similar systems for other cameras provides an essential validation of
DMS algorithms and techniques.
\end{minipage}
\\ \cdashline{1-2}
Requirement Priority & 1a \\ \cdashline{1-2}
Upper Level Requirement &
\begin{tabular}{cl}
\end{tabular}
\\ \hline
\end{longtable}
}


\subsubsection{Test Cases Summary}
\begin{longtable}{p{3cm}p{2.5cm}p{2.5cm}p{3cm}p{4cm}}
\toprule
\href{https://jira.lsstcorp.org/secure/Tests.jspa\#/testCase/LVV-T132}{LVV-T132} & \multicolumn{4}{p{12cm}}{ Verify implementation of Pre-cursor and Real Data } \\ \hline
\textbf{Owner} & \textbf{Status} & \textbf{Version} & \textbf{Critical Event} & \textbf{Verification Type} \\ \hline
Robert Gruendl & Approved & 1 & false & Test \\ \hline
\end{longtable}
{\scriptsize
\textbf{Objective:}\\
Demonstrate that pixel-oriented data from astronomical imaging cameras
(precursor or otherwise) can be processed using LSST Science Algorithms
and organized for access through the Data Butler Access Client. ~
}
\begin{longtable}{p{3cm}p{2.5cm}p{2.5cm}p{3cm}p{4cm}}
\toprule
\href{https://jira.lsstcorp.org/secure/Tests.jspa\#/testCase/LVV-T362}{LVV-T362} & \multicolumn{4}{p{12cm}}{ Installation of the LSST Science Pipelines Payloads } \\ \hline
\textbf{Owner} & \textbf{Status} & \textbf{Version} & \textbf{Critical Event} & \textbf{Verification Type} \\ \hline
John Swinbank & Approved & 1 & false & Test \\ \hline
\end{longtable}
{\scriptsize
\textbf{Objective:}\\
This test will check that:

\begin{itemize}
\tightlist
\item
  The Alert Production Pipeline payload is available for installation
  from documented channels;
\item
  The Data Release Production Pipeline payload is available for
  installation from documented channels;
\item
  The Calibration Products Production Pipeline payload is available for
  installation from documented channels;
\item
  These payloads can be installed on systems at the LSST Data Facility
  following available documentation;
\item
  The installed pipeline payloads are capable of successfully executing
  basic integration tests.
\end{itemize}

Note that this test assumes a 2018-era packaging of the Science
Pipelines software, in which all the above payloads are represented by a
single ``meta-package'', lsst\_distrib.
}
  
 \newpage 
\subsection{[LVV-130] DMS-REQ-0299-V-01: Data Product Ingest }\label{lvv-130}

\begin{longtable}{cccc}
\hline
\textbf{Jira Link} & \textbf{Assignee} & \textbf{Status} & \textbf{Test Cases}\\ \hline
\href{https://jira.lsstcorp.org/browse/LVV-130}{LVV-130} &
Jim Bosch & Not Covered &
\begin{tabular}{c}
LVV-T137 \\
LVV-T374 \\
LVV-T1934 \\
LVV-T1935 \\
\end{tabular}
\\
\hline
\end{longtable}

\textbf{Verification Element Description:} \\
Verify by running a mini-DRP (L1 and L2) and running the ingest phase
and checking that all items appear in the archive.

{\footnotesize
\begin{longtable}{p{2.5cm}p{13.5cm}}
\hline
\multicolumn{2}{c}{\textbf{Requirement Details}}\\ \hline
Requirement ID & DMS-REQ-0299 \\ \cdashline{1-2}
Requirement Description &
\begin{minipage}[]{13cm}
\textbf{Specification:} The DMS shall provide software to ingest data
products into the Science Data Archive.
\end{minipage}
\\ \cdashline{1-2}
Requirement Priority & 1a \\ \cdashline{1-2}
Upper Level Requirement &
\begin{tabular}{cl}
OSS-REQ-0141 & Storage \\
OSS-REQ-0004 & The Archive Facility \\
\end{tabular}
\\ \hline
\end{longtable}
}


\subsubsection{Test Cases Summary}
\begin{longtable}{p{3cm}p{2.5cm}p{2.5cm}p{3cm}p{4cm}}
\toprule
\href{https://jira.lsstcorp.org/secure/Tests.jspa\#/testCase/LVV-T137}{LVV-T137} & \multicolumn{4}{p{12cm}}{ Verify implementation of Data Product Ingest } \\ \hline
\textbf{Owner} & \textbf{Status} & \textbf{Version} & \textbf{Critical Event} & \textbf{Verification Type} \\ \hline
Colin Slater & Defined & 1 & false & Test \\ \hline
\end{longtable}
{\scriptsize
\textbf{Objective:}\\
Verify that data products can be ingested.
}
\begin{longtable}{p{3cm}p{2.5cm}p{2.5cm}p{3cm}p{4cm}}
\toprule
\href{https://jira.lsstcorp.org/secure/Tests.jspa\#/testCase/LVV-T374}{LVV-T374} & \multicolumn{4}{p{12cm}}{ Ingesting Camera test data } \\ \hline
\textbf{Owner} & \textbf{Status} & \textbf{Version} & \textbf{Critical Event} & \textbf{Verification Type} \\ \hline
John Swinbank & Approved & 1 & false & Test \\ \hline
\end{longtable}
{\scriptsize
\textbf{Objective:}\\
This test will check:

\begin{itemize}
\tightlist
\item
  That raw Camera test data is available on a filesystem in the LSST
  Data Facility;
\item
  That raw Camera test data can be ingested and made available through
  the Data Management I/O abstraction (the ``Data Butler'').
\end{itemize}
}
\begin{longtable}{p{3cm}p{2.5cm}p{2.5cm}p{3cm}p{4cm}}
\toprule
\href{https://jira.lsstcorp.org/secure/Tests.jspa\#/testCase/LVV-T1934}{LVV-T1934} & \multicolumn{4}{p{12cm}}{ ComCam Data Transfer and Ingestion } \\ \hline
\textbf{Owner} & \textbf{Status} & \textbf{Version} & \textbf{Critical Event} & \textbf{Verification Type} \\ \hline
Robert Gruendl & Draft & 1 & true & Inspection \\ \hline
\end{longtable}
{\scriptsize
\textbf{Objective:}\\
Verify that ComCam Archiver data taken are transferred to NCSA Data
BackBone endpoint and Ingested
}
\begin{longtable}{p{3cm}p{2.5cm}p{2.5cm}p{3cm}p{4cm}}
\toprule
\href{https://jira.lsstcorp.org/secure/Tests.jspa\#/testCase/LVV-T1935}{LVV-T1935} & \multicolumn{4}{p{12cm}}{ Demonstrate ComCam Data Processing Capability } \\ \hline
\textbf{Owner} & \textbf{Status} & \textbf{Version} & \textbf{Critical Event} & \textbf{Verification Type} \\ \hline
Robert Gruendl & Draft & 1 & true & Demonstration \\ \hline
\end{longtable}
{\scriptsize
\textbf{Objective:}\\
To process raw ComCam data and demonstrate that the results are
available either in the shared DM development environment/repository or
in the RSP.
}
  
 \newpage 
\subsection{[LVV-132] DMS-REQ-0301-V-01: Control of Level-1 Production }\label{lvv-132}

\begin{longtable}{cccc}
\hline
\textbf{Jira Link} & \textbf{Assignee} & \textbf{Status} & \textbf{Test Cases}\\ \hline
\href{https://jira.lsstcorp.org/browse/LVV-132}{LVV-132} &
Eric Bellm & Not Covered &
\begin{tabular}{c}
LVV-T147 \\
\end{tabular}
\\
\hline
\end{longtable}

\textbf{Verification Element Description:} \\
Run a test night of L1 data.

{\footnotesize
\begin{longtable}{p{2.5cm}p{13.5cm}}
\hline
\multicolumn{2}{c}{\textbf{Requirement Details}}\\ \hline
Requirement ID & DMS-REQ-0301 \\ \cdashline{1-2}
Requirement Description &
\begin{minipage}[]{13cm}
\textbf{Specification:} The DMS shall contain a component to control all
Level-1 Data Product production.
\end{minipage}
\\ \cdashline{1-2}
Requirement Discussion &
\begin{minipage}[]{13cm}
\textbf{Discussion:} This specifically addresses the need to control the
Alert Production across all DMS facilities.
\end{minipage}
\\ \cdashline{1-2}
Requirement Priority & 1b \\ \cdashline{1-2}
Upper Level Requirement &
\begin{tabular}{cl}
OSS-REQ-0044 & Standard Operating States \\
\end{tabular}
\\ \hline
\end{longtable}
}


\subsubsection{Test Cases Summary}
\begin{longtable}{p{3cm}p{2.5cm}p{2.5cm}p{3cm}p{4cm}}
\toprule
\href{https://jira.lsstcorp.org/secure/Tests.jspa\#/testCase/LVV-T147}{LVV-T147} & \multicolumn{4}{p{12cm}}{ Verify implementation of Control of Level-1 Production } \\ \hline
\textbf{Owner} & \textbf{Status} & \textbf{Version} & \textbf{Critical Event} & \textbf{Verification Type} \\ \hline
Robert Gruendl & Draft & 1 & false & Test \\ \hline
\end{longtable}
{\scriptsize
\textbf{Objective:}\\
Demonstrate that the DMS can control all Prompt Processing across DMS
facilities.
}
  
 \newpage 
\subsection{[LVV-138] DMS-REQ-0307-V-01: Unique Processing Coverage }\label{lvv-138}

\begin{longtable}{cccc}
\hline
\textbf{Jira Link} & \textbf{Assignee} & \textbf{Status} & \textbf{Test Cases}\\ \hline
\href{https://jira.lsstcorp.org/browse/LVV-138}{LVV-138} &
Jim Bosch & Not Covered &
\begin{tabular}{c}
LVV-T148 \\
\end{tabular}
\\
\hline
\end{longtable}

\textbf{Verification Element Description:} \\
Similar to DMS-REQ-0125. I don't know how to control this. Is an
iterator interface enough to verify this?

{\footnotesize
\begin{longtable}{p{2.5cm}p{13.5cm}}
\hline
\multicolumn{2}{c}{\textbf{Requirement Details}}\\ \hline
Requirement ID & DMS-REQ-0307 \\ \cdashline{1-2}
Requirement Description &
\begin{minipage}[]{13cm}
\textbf{Specification:} The DMS shall enable applications to process
every record in a table meeting user-specified criteria exactly once.\\
\textbf{~}\\
\textbf{Commentary:} The ``exactly once'' constraint can be confusing to
some readers and would benefit from clarification in the discussion.
\end{minipage}
\\ \cdashline{1-2}
Requirement Priority & 2 \\ \cdashline{1-2}
Upper Level Requirement &
\begin{tabular}{cl}
OSS-REQ-0120 & Consistency \\
OSS-REQ-0118 & Consistency and Completeness \\
\end{tabular}
\\ \hline
\end{longtable}
}


\subsubsection{Test Cases Summary}
\begin{longtable}{p{3cm}p{2.5cm}p{2.5cm}p{3cm}p{4cm}}
\toprule
\href{https://jira.lsstcorp.org/secure/Tests.jspa\#/testCase/LVV-T148}{LVV-T148} & \multicolumn{4}{p{12cm}}{ Verify implementation of Unique Processing Coverage } \\ \hline
\textbf{Owner} & \textbf{Status} & \textbf{Version} & \textbf{Critical Event} & \textbf{Verification Type} \\ \hline
Colin Slater & Draft & 1 & false & Test \\ \hline
\end{longtable}
{\scriptsize
\textbf{Objective:}\\
Verify that a user-specified criterion can be used to process each
record in a table exactly once.
}
  
 \newpage 
\subsection{[LVV-142] DMS-REQ-0311-V-01: Regenerate Un-archived Data Products }\label{lvv-142}

\begin{longtable}{cccc}
\hline
\textbf{Jira Link} & \textbf{Assignee} & \textbf{Status} & \textbf{Test Cases}\\ \hline
\href{https://jira.lsstcorp.org/browse/LVV-142}{LVV-142} &
Robert Gruendl & Not Covered &
\begin{tabular}{c}
LVV-T156 \\
\end{tabular}
\\
\hline
\end{longtable}

\textbf{Verification Element Description:} \\
Run a small processing job. Download the unarchived data products. From
information in the provenance of those data products, request a new
processing and compare. Required that the baseline software is updated
before this test is performed so that the provenance system is forced to
use an older build.

{\footnotesize
\begin{longtable}{p{2.5cm}p{13.5cm}}
\hline
\multicolumn{2}{c}{\textbf{Requirement Details}}\\ \hline
Requirement ID & DMS-REQ-0311 \\ \cdashline{1-2}
Requirement Description &
\begin{minipage}[]{13cm}
\textbf{Specification:} The DMS shall be able to regenerate unarchived
data products to within scientifically reasonable tolerances.
\end{minipage}
\\ \cdashline{1-2}
Requirement Discussion &
\begin{minipage}[]{13cm}
\textbf{Discussion:} Unarchived data products currently include
Processed Visit Images for single visits, some Coadds, and Difference
Images. Scientifically reasonable tolerances means well within the
formal uncertainties of the data product, given the same production
software, calibrations, and compute platform, all of which are expected
to change (and improve) during the course of the survey.
\end{minipage}
\\ \cdashline{1-2}
Requirement Priority & 1b \\ \cdashline{1-2}
Upper Level Requirement &
\begin{tabular}{cl}
OSS-REQ-0129 & Exposures (Level 1) \\
\end{tabular}
\\ \hline
\end{longtable}
}


\subsubsection{Test Cases Summary}
\begin{longtable}{p{3cm}p{2.5cm}p{2.5cm}p{3cm}p{4cm}}
\toprule
\href{https://jira.lsstcorp.org/secure/Tests.jspa\#/testCase/LVV-T156}{LVV-T156} & \multicolumn{4}{p{12cm}}{ Verify implementation of Regenerate Un-archived Data Products } \\ \hline
\textbf{Owner} & \textbf{Status} & \textbf{Version} & \textbf{Critical Event} & \textbf{Verification Type} \\ \hline
Simon Krughoff & Draft & 1 & false & Test \\ \hline
\end{longtable}
{\scriptsize
\textbf{Objective:}\\
Not all of the ancillary data products produced by a data release will
be archived permanently. ~These ancillary products have been promised as
accessible to the community.~ Show that these products can be produced
from an archived data release after the fact.
}
  
 \newpage 
\subsection{[LVV-148] DMS-REQ-0317-V-01: DIAForcedSource Catalog }\label{lvv-148}

\begin{longtable}{cccc}
\hline
\textbf{Jira Link} & \textbf{Assignee} & \textbf{Status} & \textbf{Test Cases}\\ \hline
\href{https://jira.lsstcorp.org/browse/LVV-148}{LVV-148} &
Eric Bellm & Not Covered &
\begin{tabular}{c}
LVV-T55 \\
\end{tabular}
\\
\hline
\end{longtable}

\textbf{Verification Element Description:} \\
From precursor data reduced with difference imaging, calculate forced
sources and insert into table. Verify content of table against DPDD.

{\footnotesize
\begin{longtable}{p{2.5cm}p{13.5cm}}
\hline
\multicolumn{2}{c}{\textbf{Requirement Details}}\\ \hline
Requirement ID & DMS-REQ-0317 \\ \cdashline{1-2}
Requirement Description &
\begin{minipage}[]{13cm}
\textbf{Specification:} The DMS shall create a DIAForcedSource Catalog,
consisting of measured fluxes for entries in the DIAObject Catalog on
Difference Exposures. Measurements for each forced-source shall include
the DIAObject and visit IDs, the modeled flux and error (given fixed
position, shape, and deblending parameters), and measurement quality
flags.
\end{minipage}
\\ \cdashline{1-2}
Requirement Discussion &
\begin{minipage}[]{13cm}
\textbf{Discussion:} The large number of such forced sources makes it
impractical to measure more attributes than are necessary to construct a
light curve for variability studies.
\end{minipage}
\\ \cdashline{1-2}
Requirement Priority & 2 \\ \cdashline{1-2}
Upper Level Requirement &
\begin{tabular}{cl}
OSS-REQ-0130 & Catalogs (Level 1) \\
\end{tabular}
\\ \hline
\end{longtable}
}


\subsubsection{Test Cases Summary}
\begin{longtable}{p{3cm}p{2.5cm}p{2.5cm}p{3cm}p{4cm}}
\toprule
\href{https://jira.lsstcorp.org/secure/Tests.jspa\#/testCase/LVV-T55}{LVV-T55} & \multicolumn{4}{p{12cm}}{ Verify implementation of DIAForcedSource Catalog } \\ \hline
\textbf{Owner} & \textbf{Status} & \textbf{Version} & \textbf{Critical Event} & \textbf{Verification Type} \\ \hline
Eric Bellm & Draft & 1 & false & Test \\ \hline
\end{longtable}
{\scriptsize
\textbf{Objective:}\\
Verify that the DMS produces a DIAForcedSource Catalog and that the
catalog contains measured fluxes for DIAObjects.
}
  
 \newpage 
\subsection{[LVV-150] DMS-REQ-0319-V-01: Characterizing Variability }\label{lvv-150}

\begin{longtable}{cccc}
\hline
\textbf{Jira Link} & \textbf{Assignee} & \textbf{Status} & \textbf{Test Cases}\\ \hline
\href{https://jira.lsstcorp.org/browse/LVV-150}{LVV-150} &
Eric Bellm & Not Covered &
\begin{tabular}{c}
LVV-T56 \\
\end{tabular}
\\
\hline
\end{longtable}

\textbf{Verification Element Description:} \\
Using a DIAObject database populated with a simulated 2 year history,
run a simulated image through the alert production system and test that
the issued alerts use the correct range of data for characterization.

{\footnotesize
\begin{longtable}{p{2.5cm}p{13.5cm}}
\hline
\multicolumn{2}{c}{\textbf{Requirement Details}}\\ \hline
Requirement ID & DMS-REQ-0319 \\ \cdashline{1-2}
Requirement Description &
\begin{minipage}[]{13cm}
\textbf{Specification:} For alert production, DIAObject variability
characterization shall include data collected during the time period
from the present to at least \textbf{diaCharacterizationCutoff} in the
past.
\end{minipage}
\\ \cdashline{1-2}
Requirement Parameters & \textbf{diaCharacterizationCutoff = 1{{[}year{]}}} Time-period to use
for characterizing variability in L1 system. \\ \cdashline{1-2}
Requirement Discussion &
\begin{minipage}[]{13cm}
\textbf{Discussion:} These measurements can come from the live L1
database. For level 1 processing during Data Release Production, all
data should be used for characterization.
\end{minipage}
\\ \cdashline{1-2}
Requirement Priority & 1b \\ \cdashline{1-2}
Upper Level Requirement &
\begin{tabular}{cl}
OSS-REQ-0126 & Level 1 Data Products \\
\end{tabular}
\\ \hline
\end{longtable}
}


\subsubsection{Test Cases Summary}
\begin{longtable}{p{3cm}p{2.5cm}p{2.5cm}p{3cm}p{4cm}}
\toprule
\href{https://jira.lsstcorp.org/secure/Tests.jspa\#/testCase/LVV-T56}{LVV-T56} & \multicolumn{4}{p{12cm}}{ Verify implementation of Characterizing Variability } \\ \hline
\textbf{Owner} & \textbf{Status} & \textbf{Version} & \textbf{Critical Event} & \textbf{Verification Type} \\ \hline
Eric Bellm & Draft & 1 & false & Test \\ \hline
\end{longtable}
{\scriptsize
\textbf{Objective:}\\
Verify that the variability characterization in the DIAObject catalog
includes data collected within previous ``diaCharacterizationCutoff''
period of time.
}
  
 \newpage 
\subsection{[LVV-152] DMS-REQ-0321-V-01: Level 1 Processing of Special Programs Data }\label{lvv-152}

\begin{longtable}{cccc}
\hline
\textbf{Jira Link} & \textbf{Assignee} & \textbf{Status} & \textbf{Test Cases}\\ \hline
\href{https://jira.lsstcorp.org/browse/LVV-152}{LVV-152} &
Melissa Graham & Not Covered &
\begin{tabular}{c}
LVV-T93 \\
\end{tabular}
\\
\hline
\end{longtable}

\textbf{Verification Element Description:} \\
Process some representative special programs style data and demonstrate
that a full night could be reduced in time.

{\footnotesize
\begin{longtable}{p{2.5cm}p{13.5cm}}
\hline
\multicolumn{2}{c}{\textbf{Requirement Details}}\\ \hline
Requirement ID & DMS-REQ-0321 \\ \cdashline{1-2}
Requirement Description &
\begin{minipage}[]{13cm}
\textbf{Specification:} All Level 1 processing from special programs
shall be completed before data arrives from the following night's
observations.
\end{minipage}
\\ \cdashline{1-2}
Requirement Discussion &
\begin{minipage}[]{13cm}
\textbf{Discussion:} Only Special Programs data that can be incorporated
into the prompt pipeline (i.e., standard visit images, or non-standard
visit images that can be shown to result in quality DIA products), will
be processed with the prompt pipeline and contribute to the Alert
Stream.
\end{minipage}
\\ \cdashline{1-2}
Requirement Priority & 2 \\ \cdashline{1-2}
Upper Level Requirement &
\begin{tabular}{cl}
OSS-REQ-0392 & Data Products Handling for Special Programs \\
\end{tabular}
\\ \hline
\end{longtable}
}


\subsubsection{Test Cases Summary}
\begin{longtable}{p{3cm}p{2.5cm}p{2.5cm}p{3cm}p{4cm}}
\toprule
\href{https://jira.lsstcorp.org/secure/Tests.jspa\#/testCase/LVV-T93}{LVV-T93} & \multicolumn{4}{p{12cm}}{ Verify implementation of Level 1 Processing of Special Programs Data } \\ \hline
\textbf{Owner} & \textbf{Status} & \textbf{Version} & \textbf{Critical Event} & \textbf{Verification Type} \\ \hline
Melissa Graham & Draft & 1 & false & Test \\ \hline
\end{longtable}
{\scriptsize
\textbf{Objective:}\\
Execute multi-day operations rehearsal. Observe whether Prompt
Processing data products generated in time and confirm whether
processing has completed before the start of the next simulated night.~
}
  
 \newpage 
\subsection{[LVV-154] DMS-REQ-0323-V-01: Calculating SSObject Parameters }\label{lvv-154}

\begin{longtable}{cccc}
\hline
\textbf{Jira Link} & \textbf{Assignee} & \textbf{Status} & \textbf{Test Cases}\\ \hline
\href{https://jira.lsstcorp.org/browse/LVV-154}{LVV-154} &
Eric Bellm & Not Covered &
\begin{tabular}{c}
LVV-T57 \\
\end{tabular}
\\
\hline
\end{longtable}

\textbf{Verification Element Description:} \\
Use the APIs to calculate the required parameters for a sample of
different categories of SSObjects.

{\footnotesize
\begin{longtable}{p{2.5cm}p{13.5cm}}
\hline
\multicolumn{2}{c}{\textbf{Requirement Details}}\\ \hline
Requirement ID & DMS-REQ-0323 \\ \cdashline{1-2}
Requirement Description &
\begin{minipage}[]{13cm}
\textbf{Specification:} The LSST database shall supply functions or
tables to provide, for every SSObject, at least the phase angle for
every observation, and the reduced and absolute asteroid magnitudes in
all LSST bands.
\end{minipage}
\\ \cdashline{1-2}
Requirement Priority & 3 \\ \cdashline{1-2}
Upper Level Requirement &
\begin{tabular}{cl}
OSS-REQ-0126 & Level 1 Data Products \\
\end{tabular}
\\ \hline
\end{longtable}
}


\subsubsection{Test Cases Summary}
\begin{longtable}{p{3cm}p{2.5cm}p{2.5cm}p{3cm}p{4cm}}
\toprule
\href{https://jira.lsstcorp.org/secure/Tests.jspa\#/testCase/LVV-T57}{LVV-T57} & \multicolumn{4}{p{12cm}}{ Verify implementation of Calculating SSObject Parameters } \\ \hline
\textbf{Owner} & \textbf{Status} & \textbf{Version} & \textbf{Critical Event} & \textbf{Verification Type} \\ \hline
Eric Bellm & Draft & 1 & false & Test \\ \hline
\end{longtable}
{\scriptsize
\textbf{Objective:}\\
Verify that the DMS database provides functions to compute phase angles
and magnitudes in LSST bands for every SSObject.
}
  
 \newpage 
\subsection{[LVV-155] DMS-REQ-0324-V-01: Matching DIASources to Objects }\label{lvv-155}

\begin{longtable}{cccc}
\hline
\textbf{Jira Link} & \textbf{Assignee} & \textbf{Status} & \textbf{Test Cases}\\ \hline
\href{https://jira.lsstcorp.org/browse/LVV-155}{LVV-155} &
Eric Bellm & Not Covered &
\begin{tabular}{c}
LVV-T58 \\
\end{tabular}
\\
\hline
\end{longtable}

\textbf{Verification Element Description:} \\
Do a mini data release production run, search for an Object and request
the associated DIASources.

{\footnotesize
\begin{longtable}{p{2.5cm}p{13.5cm}}
\hline
\multicolumn{2}{c}{\textbf{Requirement Details}}\\ \hline
Requirement ID & DMS-REQ-0324 \\ \cdashline{1-2}
Requirement Description &
\begin{minipage}[]{13cm}
\textbf{Specification:} A L1 DIASource to L2 Object positional
cross-match table or database view shall be made available.
\end{minipage}
\\ \cdashline{1-2}
Requirement Discussion &
\begin{minipage}[]{13cm}
\textbf{Discussion:} Care should be taken to note that this is purely a
cross-match based on separation on the sky and does not imply the
DIASource and Object are physically the same.
\end{minipage}
\\ \cdashline{1-2}
Requirement Priority & 1b \\ \cdashline{1-2}
Upper Level Requirement &
\begin{tabular}{cl}
OSS-REQ-0126 & Level 1 Data Products \\
\end{tabular}
\\ \hline
\end{longtable}
}


\subsubsection{Test Cases Summary}
\begin{longtable}{p{3cm}p{2.5cm}p{2.5cm}p{3cm}p{4cm}}
\toprule
\href{https://jira.lsstcorp.org/secure/Tests.jspa\#/testCase/LVV-T58}{LVV-T58} & \multicolumn{4}{p{12cm}}{ Verify implementation of Matching DIASources to Objects } \\ \hline
\textbf{Owner} & \textbf{Status} & \textbf{Version} & \textbf{Critical Event} & \textbf{Verification Type} \\ \hline
Eric Bellm & Draft & 1 & false & Test \\ \hline
\end{longtable}
{\scriptsize
\textbf{Objective:}\\
Verify that a cross-match table is available between DIASources and
Objects.
}
  
 \newpage 
\subsection{[LVV-156] DMS-REQ-0325-V-01: Regenerating L1 Data Products During Data Release
Processing }\label{lvv-156}

\begin{longtable}{cccc}
\hline
\textbf{Jira Link} & \textbf{Assignee} & \textbf{Status} & \textbf{Test Cases}\\ \hline
\href{https://jira.lsstcorp.org/browse/LVV-156}{LVV-156} &
Jim Bosch & Not Covered &
\begin{tabular}{c}
LVV-T59 \\
\end{tabular}
\\
\hline
\end{longtable}

\textbf{Verification Element Description:} \\
Do a mini data release production run and show that L1 data products
were regenerated.

{\footnotesize
\begin{longtable}{p{2.5cm}p{13.5cm}}
\hline
\multicolumn{2}{c}{\textbf{Requirement Details}}\\ \hline
Requirement ID & DMS-REQ-0325 \\ \cdashline{1-2}
Requirement Description &
\begin{minipage}[]{13cm}
\textbf{Specification:} During Data Release Processing, all the Level 1
data products shall be regenerated using the current best algorithms.
\end{minipage}
\\ \cdashline{1-2}
Requirement Discussion &
\begin{minipage}[]{13cm}
\textbf{Discussion:} Variability characterization will use the full
light curve history.
\end{minipage}
\\ \cdashline{1-2}
Requirement Priority & 2 \\ \cdashline{1-2}
Upper Level Requirement &
\begin{tabular}{cl}
OSS-REQ-0135 & Uniformly calibrated and processed versions of Level 1 Data Products \\
\end{tabular}
\\ \hline
\end{longtable}
}


\subsubsection{Test Cases Summary}
\begin{longtable}{p{3cm}p{2.5cm}p{2.5cm}p{3cm}p{4cm}}
\toprule
\href{https://jira.lsstcorp.org/secure/Tests.jspa\#/testCase/LVV-T59}{LVV-T59} & \multicolumn{4}{p{12cm}}{ Verify implementation of Regenerating L1 Data Products During Data
Release Processing } \\ \hline
\textbf{Owner} & \textbf{Status} & \textbf{Version} & \textbf{Critical Event} & \textbf{Verification Type} \\ \hline
Kian-Tat Lim & Draft & 1 & false & Test \\ \hline
\end{longtable}
{\scriptsize
\textbf{Objective:}\\
Verify that the Prompt Processing data products are regenerated during
DRP.
}
  
 \newpage 
\subsection{[LVV-157] DMS-REQ-0326-V-01: Storing Approximations of Per-pixel Metadata }\label{lvv-157}

\begin{longtable}{cccc}
\hline
\textbf{Jira Link} & \textbf{Assignee} & \textbf{Status} & \textbf{Test Cases}\\ \hline
\href{https://jira.lsstcorp.org/browse/LVV-157}{LVV-157} &
Simon Krughoff & Not Covered &
\begin{tabular}{c}
LVV-T23 \\
\end{tabular}
\\
\hline
\end{longtable}

\textbf{Verification Element Description:} \\
Generate a coadd and inspect the output file to verify that parametrized
forms of are available.

{\footnotesize
\begin{longtable}{p{2.5cm}p{13.5cm}}
\hline
\multicolumn{2}{c}{\textbf{Requirement Details}}\\ \hline
Requirement ID & DMS-REQ-0326 \\ \cdashline{1-2}
Requirement Description &
\begin{minipage}[]{13cm}
\textbf{Specification:} Image depth and mask information shall be
available in a parametrized approximate form in addition to a full
per-pixel form.
\end{minipage}
\\ \cdashline{1-2}
Requirement Discussion &
\begin{minipage}[]{13cm}
\textbf{Discussion:} This parametrization could be in formats such as
MOC, Mangle polygons, or STC regions. ~Note that, under requirements
DMS-REQ-0383 and DMS-REQ-0379, MOCs for the survey coverage as a simple
Boolean map are required to be generated; the present requirement covers
maps providing additional information as a function of sky position.
\end{minipage}
\\ \cdashline{1-2}
Requirement Priority & 2 \\ \cdashline{1-2}
Upper Level Requirement &
\begin{tabular}{cl}
OSS-REQ-0391 & Data Product Conventions \\
\end{tabular}
\\ \hline
\end{longtable}
}


\subsubsection{Test Cases Summary}
\begin{longtable}{p{3cm}p{2.5cm}p{2.5cm}p{3cm}p{4cm}}
\toprule
\href{https://jira.lsstcorp.org/secure/Tests.jspa\#/testCase/LVV-T23}{LVV-T23} & \multicolumn{4}{p{12cm}}{ Verify implementation of Storing Approximations of Per-pixel Metadata } \\ \hline
\textbf{Owner} & \textbf{Status} & \textbf{Version} & \textbf{Critical Event} & \textbf{Verification Type} \\ \hline
Simon Krughoff & Draft & 1 & false & Test \\ \hline
\end{longtable}
{\scriptsize
\textbf{Objective:}\\
\textbf{Test Items}\\[2\baselineskip]Show that the compressed form depth
and mask maps adequately represents the exact version of the same
information.
}
  
 \newpage 
\subsection{[LVV-158] DMS-REQ-0327-V-01: Background Model Calculation }\label{lvv-158}

\begin{longtable}{cccc}
\hline
\textbf{Jira Link} & \textbf{Assignee} & \textbf{Status} & \textbf{Test Cases}\\ \hline
\href{https://jira.lsstcorp.org/browse/LVV-158}{LVV-158} &
Robert Lupton & Not Covered &
\begin{tabular}{c}
LVV-T15 \\
LVV-T19 \\
LVV-T43 \\
\end{tabular}
\\
\hline
\end{longtable}

\textbf{Verification Element Description:} \\
Process a visit. Retrieve that visit from the output repository and
verify that a background model is available.

{\footnotesize
\begin{longtable}{p{2.5cm}p{13.5cm}}
\hline
\multicolumn{2}{c}{\textbf{Requirement Details}}\\ \hline
Requirement ID & DMS-REQ-0327 \\ \cdashline{1-2}
Requirement Description &
\begin{minipage}[]{13cm}
\textbf{Specification:} The DMS shall derive and persist a background
model (both due to night sky and astrophysical) for each visit image,
per CCD.
\end{minipage}
\\ \cdashline{1-2}
Requirement Priority & 1b \\ \cdashline{1-2}
Upper Level Requirement &
\begin{tabular}{cl}
OSS-REQ-0056 & System Monitoring \& Diagnostics \\
\end{tabular}
\\ \hline
\end{longtable}
}


\subsubsection{Test Cases Summary}
\begin{longtable}{p{3cm}p{2.5cm}p{2.5cm}p{3cm}p{4cm}}
\toprule
\href{https://jira.lsstcorp.org/secure/Tests.jspa\#/testCase/LVV-T15}{LVV-T15} & \multicolumn{4}{p{12cm}}{ DRP-00-30: Scientific Verification of Processed Visit Images } \\ \hline
\textbf{Owner} & \textbf{Status} & \textbf{Version} & \textbf{Critical Event} & \textbf{Verification Type} \\ \hline
Jim Bosch & Deprecated & 1 & false & Test \\ \hline
\end{longtable}
{\scriptsize
\textbf{Objective:}\\
This test will check that the Processed Visit Images (PVIs) delivered by
the DRP science payload meet the requirements laid down by \citeds{LSE-61}.\\
Specifically, this will demonstrate that:

\begin{itemize}
\tightlist
\item
  Processed visit images have been generated and persisted during
  payload execution;
\item
  Each PVI includes a background model (DMS-REQ-0327), photometric
  zero-point (DMS- REQ-0029), spatially-varying PSF (DMS-REQ-0070) and
  WCS (DMS-REQ-0030).
\item
  Saturated pixels are correctly masked.
\item
  Pixels affected by cosmic rays are correctly masked.
\item
  The background is not oversubtracted around bright objects.
\end{itemize}

This test does not include quantitative targets for the science quality
criteria; we instead re- quire for each test that we be able to quickly
construct a plot or display summary images that allow such a target can
be visualized.
}
\begin{longtable}{p{3cm}p{2.5cm}p{2.5cm}p{3cm}p{4cm}}
\toprule
\href{https://jira.lsstcorp.org/secure/Tests.jspa\#/testCase/LVV-T19}{LVV-T19} & \multicolumn{4}{p{12cm}}{ AG-00-10: Scientific Verification of Processed Visit Images } \\ \hline
\textbf{Owner} & \textbf{Status} & \textbf{Version} & \textbf{Critical Event} & \textbf{Verification Type} \\ \hline
Eric Bellm & Deprecated & 1 & false & Test \\ \hline
\end{longtable}
{\scriptsize
\textbf{Objective:}\\
This test will check that the Processed Visit Images (PVIs) delivered by
the alert generation science payload meet the requirements laid down by
\citeds{LSE-61}.\\
Specifically, this will demonstrate that:

\begin{itemize}
\tightlist
\item
  Processed visit images have been generated and persisted during
  payload execution;
\item
  Each PVI includes a science pixel array, a mask array, and a variance
  array. (DMS-REQ-0072).
\item
  Each PVI includes a background model (DMS-REQ-0327), photometric
  zero-point (DMS- REQ-0029), spatially-varying PSF (DMS-REQ-0070) and
  WCS (DMS-REQ-0030).
\item
  Saturated pixels are correctly masked.
\item
  Pixels affected by cosmic rays are correctly masked.
\item
  The background is not oversubtracted around bright objects.
\end{itemize}

This test does not include quantitative targets for the science quality
criteria.
}
\begin{longtable}{p{3cm}p{2.5cm}p{2.5cm}p{3cm}p{4cm}}
\toprule
\href{https://jira.lsstcorp.org/secure/Tests.jspa\#/testCase/LVV-T43}{LVV-T43} & \multicolumn{4}{p{12cm}}{ Verify implementation of Background Model Calculation } \\ \hline
\textbf{Owner} & \textbf{Status} & \textbf{Version} & \textbf{Critical Event} & \textbf{Verification Type} \\ \hline
Jim Bosch & Approved & 1 & false & Test \\ \hline
\end{longtable}
{\scriptsize
\textbf{Objective:}\\
Verify that Processed Visit Images produced by the DRP and AP pipelines
have had a model of the background subtracted, and that this model is
persisted in a way that permits the background subtracted from any CCD
to be retrieved along with the image for that CCD.
}
  
 \newpage 
\subsection{[LVV-159] DMS-REQ-0328-V-01: Documenting Image Characterization }\label{lvv-159}

\begin{longtable}{cccc}
\hline
\textbf{Jira Link} & \textbf{Assignee} & \textbf{Status} & \textbf{Test Cases}\\ \hline
\href{https://jira.lsstcorp.org/browse/LVV-159}{LVV-159} &
Robert Lupton & Not Covered &
\begin{tabular}{c}
LVV-T44 \\
\end{tabular}
\\
\hline
\end{longtable}

\textbf{Verification Element Description:} \\
Verify existence of documentation. Compare file contents with document
descriptions.

{\footnotesize
\begin{longtable}{p{2.5cm}p{13.5cm}}
\hline
\multicolumn{2}{c}{\textbf{Requirement Details}}\\ \hline
Requirement ID & DMS-REQ-0328 \\ \cdashline{1-2}
Requirement Description &
\begin{minipage}[]{13cm}
\textbf{Specification:} The persisted format for Processed Visit Images
shall be fully documented, and shall include a description of all image
characterization data products.
\end{minipage}
\\ \cdashline{1-2}
Requirement Discussion &
\begin{minipage}[]{13cm}
\textbf{Discussion:} This will allow the community to use them to
increase understanding of LSST images and derived LSST catalogs.
\end{minipage}
\\ \cdashline{1-2}
Requirement Priority & 1b \\ \cdashline{1-2}
Upper Level Requirement &
\begin{tabular}{cl}
OSS-REQ-0391 & Data Product Conventions \\
\end{tabular}
\\ \hline
\end{longtable}
}


\subsubsection{Test Cases Summary}
\begin{longtable}{p{3cm}p{2.5cm}p{2.5cm}p{3cm}p{4cm}}
\toprule
\href{https://jira.lsstcorp.org/secure/Tests.jspa\#/testCase/LVV-T44}{LVV-T44} & \multicolumn{4}{p{12cm}}{ Verify implementation of Documenting Image Characterization } \\ \hline
\textbf{Owner} & \textbf{Status} & \textbf{Version} & \textbf{Critical Event} & \textbf{Verification Type} \\ \hline
Jim Bosch & Draft & 1 & false & Test \\ \hline
\end{longtable}
{\scriptsize
\textbf{Objective:}\\
Verify that the persisted format for Processed Visit Images and
associated instrument-signature-removal data products is documented.
}
  
 \newpage 
\subsection{[LVV-160] DMS-REQ-0329-V-01: All-Sky Visualization of Data Releases }\label{lvv-160}

\begin{longtable}{cccc}
\hline
\textbf{Jira Link} & \textbf{Assignee} & \textbf{Status} & \textbf{Test Cases}\\ \hline
\href{https://jira.lsstcorp.org/browse/LVV-160}{LVV-160} &
Simon Krughoff & Not Covered &
\begin{tabular}{c}
LVV-T76 \\
\end{tabular}
\\
\hline
\end{longtable}

\textbf{Verification Element Description:} \\
Test that generated images can be displayed in all sky tool. The exact
details of that format are TBD.

{\footnotesize
\begin{longtable}{p{2.5cm}p{13.5cm}}
\hline
\multicolumn{2}{c}{\textbf{Requirement Details}}\\ \hline
Requirement ID & DMS-REQ-0329 \\ \cdashline{1-2}
Requirement Description &
\begin{minipage}[]{13cm}
\textbf{Specification:} Data Release Processing shall generate co-adds
suitable for use in all-sky visualization tools, allowing panning and
zooming of the entire data release.
\end{minipage}
\\ \cdashline{1-2}
Requirement Discussion &
\begin{minipage}[]{13cm}
\textbf{Discussion:} For example, this could mean HEALPix tiles suitable
for use in a HiPS server. The exact technology choice has to be
confirmed before understanding which format is required.
\end{minipage}
\\ \cdashline{1-2}
Requirement Priority & 2 \\ \cdashline{1-2}
Upper Level Requirement &
\begin{tabular}{cl}
OSS-REQ-0136 & Co-added Exposures \\
\end{tabular}
\\ \hline
\end{longtable}
}


\subsubsection{Test Cases Summary}
\begin{longtable}{p{3cm}p{2.5cm}p{2.5cm}p{3cm}p{4cm}}
\toprule
\href{https://jira.lsstcorp.org/secure/Tests.jspa\#/testCase/LVV-T76}{LVV-T76} & \multicolumn{4}{p{12cm}}{ Verify implementation of All-Sky Visualization of Data Releases } \\ \hline
\textbf{Owner} & \textbf{Status} & \textbf{Version} & \textbf{Critical Event} & \textbf{Verification Type} \\ \hline
Simon Krughoff & Draft & 1 & false & Test \\ \hline
\end{longtable}
{\scriptsize
\textbf{Objective:}\\
Show that it's possible to produce large area visualizations from Data
Release data products.
}
  
 \newpage 
\subsection{[LVV-161] DMS-REQ-0330-V-01: Best Seeing Coadds }\label{lvv-161}

\begin{longtable}{cccc}
\hline
\textbf{Jira Link} & \textbf{Assignee} & \textbf{Status} & \textbf{Test Cases}\\ \hline
\href{https://jira.lsstcorp.org/browse/LVV-161}{LVV-161} &
Jim Bosch & Not Covered &
\begin{tabular}{c}
LVV-T77 \\
\end{tabular}
\\
\hline
\end{longtable}

\textbf{Verification Element Description:} \\
Using a suitable test dataset, form a query specifying a seeing range
and submit a job to create a coadd from the resulting images.

{\footnotesize
\begin{longtable}{p{2.5cm}p{13.5cm}}
\hline
\multicolumn{2}{c}{\textbf{Requirement Details}}\\ \hline
Requirement ID & DMS-REQ-0330 \\ \cdashline{1-2}
Requirement Description &
\begin{minipage}[]{13cm}
\textbf{Specification:} Best seeing coadds shall be made for each band
(including multi-color).
\end{minipage}
\\ \cdashline{1-2}
Requirement Discussion &
\begin{minipage}[]{13cm}
\textbf{Discussion:} DMS-REQ-0279 states that seeing-based co-adds
should be possible. This requirement states that they shall be made.
\end{minipage}
\\ \cdashline{1-2}
Requirement Priority & 2 \\ \cdashline{1-2}
Upper Level Requirement &
\begin{tabular}{cl}
OSS-REQ-0136 & Co-added Exposures \\
\end{tabular}
\\ \hline
\end{longtable}
}


\subsubsection{Test Cases Summary}
\begin{longtable}{p{3cm}p{2.5cm}p{2.5cm}p{3cm}p{4cm}}
\toprule
\href{https://jira.lsstcorp.org/secure/Tests.jspa\#/testCase/LVV-T77}{LVV-T77} & \multicolumn{4}{p{12cm}}{ Verify implementation of Best Seeing Coadds } \\ \hline
\textbf{Owner} & \textbf{Status} & \textbf{Version} & \textbf{Critical Event} & \textbf{Verification Type} \\ \hline
Jim Bosch & Draft & 1 & false & Test \\ \hline
\end{longtable}
{\scriptsize
\textbf{Objective:}\\
Verify that the DRP pipelines produce a suite of per-band coadds with
input images filtered to optimize the size of the effective PSF on the
coadd.
}
  
 \newpage 
\subsection{[LVV-162] DMS-REQ-0331-V-01: Computing Derived Quantities }\label{lvv-162}

\begin{longtable}{cccc}
\hline
\textbf{Jira Link} & \textbf{Assignee} & \textbf{Status} & \textbf{Test Cases}\\ \hline
\href{https://jira.lsstcorp.org/browse/LVV-162}{LVV-162} &
Melissa Graham & Not Covered &
\begin{tabular}{c}
LVV-T13 \\
LVV-T14 \\
LVV-T21 \\
LVV-T22 \\
LVV-T24 \\
\end{tabular}
\\
\hline
\end{longtable}

\textbf{Verification Element Description:} \\
Verify that derived quantities have been stored in the database. The
exact list of items is TBD.

{\footnotesize
\begin{longtable}{p{2.5cm}p{13.5cm}}
\hline
\multicolumn{2}{c}{\textbf{Requirement Details}}\\ \hline
Requirement ID & DMS-REQ-0331 \\ \cdashline{1-2}
Requirement Description &
\begin{minipage}[]{13cm}
\textbf{Specification:} Common derived quantities shall be made
available to end-users by either providing pre-computed columns or
providing functions that can be used dynamically in queries. These
should at least include the ability to calculate the reduced chi-squared
of fitted models and make it as easy as possible to calculate
color-color diagrams.
\end{minipage}
\\ \cdashline{1-2}
Requirement Discussion &
\begin{minipage}[]{13cm}
\textbf{Discussion:} Example quantities include those used to assess
model fit quality or those required for calculating color-magnitude
diagrams. Care should be taken to name the derived columns in a clear
unambiguous way.
\end{minipage}
\\ \cdashline{1-2}
Requirement Priority & 1b \\ \cdashline{1-2}
Upper Level Requirement &
\begin{tabular}{cl}
OSS-REQ-0391 & Data Product Conventions \\
\end{tabular}
\\ \hline
\end{longtable}
}

\subsubsection{Verified By}
\begin{itemize}
\item . DM-9953 (\ref{dm-9953}) Pixels rejected from coaddition and CCD are not masked on coadds
\item . DM-13058 (\ref{dm-13058}) Inconsistent aperture corrections in W44 reprocessing of HSC RC1
\end{itemize}

\subsubsection{Test Cases Summary}
\begin{longtable}{p{3cm}p{2.5cm}p{2.5cm}p{3cm}p{4cm}}
\toprule
\href{https://jira.lsstcorp.org/secure/Tests.jspa\#/testCase/LVV-T13}{LVV-T13} & \multicolumn{4}{p{12cm}}{ DRP-00-15: Scientific Verification of Source Catalog } \\ \hline
\textbf{Owner} & \textbf{Status} & \textbf{Version} & \textbf{Critical Event} & \textbf{Verification Type} \\ \hline
Jim Bosch & Deprecated & 1 & false & Test \\ \hline
\end{longtable}
{\scriptsize
\textbf{Objective:}\\
This test will check that the source catalogs delivered by the DRP
science payload meet the requirements laid down by \citeds{LSE-61}.\\
Specifically, this will demonstrate that:

\begin{itemize}
\tightlist
\item
  Measurements in the catalog are presented in flux units
  (DMS-REQ-0347);
\item
  Derived quantities are provided in pre-computed columns
  (DMS-REQ-0331);
\item
  Aperture corrections for different photometry algorithms are
  consistent.
\item
  Photometry measurements are consistent with reference catalog
  photometry (including sources not used in photometric calibration).
\item
  Astrometry measurements are consistent with reference catalog
  positions (including sources not used in astrometric calibration).
\end{itemize}

This test does not include quantitative targets for the science quality
criteria; we instead require for each test that we be able to quickly
construct a plot in which such a target can be visualized.
}
\begin{longtable}{p{3cm}p{2.5cm}p{2.5cm}p{3cm}p{4cm}}
\toprule
\href{https://jira.lsstcorp.org/secure/Tests.jspa\#/testCase/LVV-T14}{LVV-T14} & \multicolumn{4}{p{12cm}}{ DRP-00-25: Scientific Verification of Object Catalog } \\ \hline
\textbf{Owner} & \textbf{Status} & \textbf{Version} & \textbf{Critical Event} & \textbf{Verification Type} \\ \hline
Jim Bosch & Deprecated & 1 & false & Test \\ \hline
\end{longtable}
{\scriptsize
\textbf{Objective:}\\
This test will check that the object catalogs delivered by the DRP
science payload meet the requirements laid down by \citeds{LSE-61}.\\
Specifically, this will demonstrate that:

\begin{itemize}
\tightlist
\item
  Measurements in the catalog are presented in flux units
  (DMS-REQ-0347);
\item
  Derived quantities are provided in pre-computed columns
  (DMS-REQ-0331);
\item
  Aperture corrections for different photometry algorithms are
  consistent.
\item
  PSF models correctly predict the ellipticities of stars over each
  tract.
\item
  Photometry measurements are consistent with reference catalog
  photometry (including sources not used in photometric calibration).
\item
  Astrometry measurements are consistent with reference catalog
  positions (including sources not used in astrometric calibration).
\item
  Forced and unforced photometry measurements are consistent.
\item
  The slope of the stellar locus in color-color space is not a function
  of position on the sky.
\end{itemize}

This test does not include quantitative targets for the science quality
criteria; we instead re- quire for each test that we be able to quickly
construct a plot in which such a target can be visualized.\\
All science quality tests in this section shall distinguish between
blended and isolated objects.
}
\begin{longtable}{p{3cm}p{2.5cm}p{2.5cm}p{3cm}p{4cm}}
\toprule
\href{https://jira.lsstcorp.org/secure/Tests.jspa\#/testCase/LVV-T21}{LVV-T21} & \multicolumn{4}{p{12cm}}{ AG-00-20: Scientific Verification of DIASource Catalog } \\ \hline
\textbf{Owner} & \textbf{Status} & \textbf{Version} & \textbf{Critical Event} & \textbf{Verification Type} \\ \hline
Eric Bellm & Deprecated & 1 & false & Test \\ \hline
\end{longtable}
{\scriptsize
\textbf{Objective:}\\
This test will check that the difference image source catalogs delivered
by the Alert Generation science payload meet the requirements laid down
by \citeds{LSE-61}.

\begin{itemize}
\tightlist
\item
  Specifically, this will demonstrate that:
\item
  Measurements in the catalog are presented in flux units
  (DMS-REQ-0347);
\item
  Each DIASource record contains an appropriate subset of the attributes
  required by DMS-REQ-0269. In particular, the LDM-503-3-era pipeline is
  expected to provide DIASource positions (sky and focal plane), fluxes,
  and flags indicative of issues encountered during processing.
\item
  Faint DIASources satisfying additional criteria are stored
  (DMS-REQ-0270).
\item
  Derived quantities are provided in pre-computed columns
  (DMS-REQ-0331);
\end{itemize}

This test does not include quantitative targets for the science quality
criteria.\\[2\baselineskip]
}
\begin{longtable}{p{3cm}p{2.5cm}p{2.5cm}p{3cm}p{4cm}}
\toprule
\href{https://jira.lsstcorp.org/secure/Tests.jspa\#/testCase/LVV-T22}{LVV-T22} & \multicolumn{4}{p{12cm}}{ AG-00-25: Scientific Verification of DIAObject Catalog } \\ \hline
\textbf{Owner} & \textbf{Status} & \textbf{Version} & \textbf{Critical Event} & \textbf{Verification Type} \\ \hline
Eric Bellm & Deprecated & 1 & false & Test \\ \hline
\end{longtable}
{\scriptsize
\textbf{Objective:}\\
This test will check that the DIAObject catalogs delivered by the Alert
Generation science pay- load meet the requirements laid down by
\citeds{LSE-61}.\\
Specifically, this will demonstrate that:

\begin{itemize}
\tightlist
\item
  DIAObjects are recorded with unique identifiers (DMS-REQ-0271);
\item
  Measurements in the catalog are presented in flux units
  (DMS-REQ-0347);
\item
  EachDIAObjectrecordcontainscontainsanappropriatesetofsummaryattributes(DMS-
  REQ-0271 and DMS-REQ-0272). Note:

  \begin{itemize}
  \tightlist
  \item
    This test is executed independently of the Data Release Production
    system. Hence, DIAObjects are not associated to Objects, and the
    association metadata specified by DMS-REQ-0271 is not expected to be
    available.
  \item
    TheLDM-503-3erapipelineisnotexpectedtocalculateorpersistallattributesspec-
    ified by DMS-REQ-0272 requirement.
  \end{itemize}
\item
  Relevant derived quantities are provided in pre-computed columns
  (DMS-REQ-0331);~
\end{itemize}

This test does not include quantitative targets for the science quality
criteria.
}
\begin{longtable}{p{3cm}p{2.5cm}p{2.5cm}p{3cm}p{4cm}}
\toprule
\href{https://jira.lsstcorp.org/secure/Tests.jspa\#/testCase/LVV-T24}{LVV-T24} & \multicolumn{4}{p{12cm}}{ Verify implementation of Computing Derived Quantities } \\ \hline
\textbf{Owner} & \textbf{Status} & \textbf{Version} & \textbf{Critical Event} & \textbf{Verification Type} \\ \hline
Melissa Graham & Draft & 1 & false & Test \\ \hline
\end{longtable}
{\scriptsize
\textbf{Objective:}\\
To confirm that common derived quantities (apparent magnitude, FWHM in
arcsec, ellipticity) are available to an end-user by, e.g., ensuring a
color-color diagram is easy to construction, fitting functions to
derived data, or generating other common scientific derivatives.
}
  
 \newpage 
\subsection{[LVV-163] DMS-REQ-0332-V-01: Denormalizing Database Tables }\label{lvv-163}

\begin{longtable}{cccc}
\hline
\textbf{Jira Link} & \textbf{Assignee} & \textbf{Status} & \textbf{Test Cases}\\ \hline
\href{https://jira.lsstcorp.org/browse/LVV-163}{LVV-163} &
Colin Slater & Not Covered &
\begin{tabular}{c}
LVV-T25 \\
\end{tabular}
\\
\hline
\end{longtable}

\textbf{Verification Element Description:} \\
Show that some tables have been denormalized. This requirement needs
some more explicit phrasing.

{\footnotesize
\begin{longtable}{p{2.5cm}p{13.5cm}}
\hline
\multicolumn{2}{c}{\textbf{Requirement Details}}\\ \hline
Requirement ID & DMS-REQ-0332 \\ \cdashline{1-2}
Requirement Description &
\begin{minipage}[]{13cm}
\textbf{Specification:} The database tables shall contain views
presented to the users that will be appropriately denormalized for ease
of use.
\end{minipage}
\\ \cdashline{1-2}
Requirement Priority & 2 \\ \cdashline{1-2}
Upper Level Requirement &
\begin{tabular}{cl}
OSS-REQ-0133 & Level 2 Data Products \\
\end{tabular}
\\ \hline
\end{longtable}
}


\subsubsection{Test Cases Summary}
\begin{longtable}{p{3cm}p{2.5cm}p{2.5cm}p{3cm}p{4cm}}
\toprule
\href{https://jira.lsstcorp.org/secure/Tests.jspa\#/testCase/LVV-T25}{LVV-T25} & \multicolumn{4}{p{12cm}}{ Verify implementation of Denormalizing Database Tables } \\ \hline
\textbf{Owner} & \textbf{Status} & \textbf{Version} & \textbf{Critical Event} & \textbf{Verification Type} \\ \hline
Colin Slater & Draft & 1 & false & Test \\ \hline
\end{longtable}
{\scriptsize
\textbf{Objective:}\\
Verify that commonly useful views of data are easy to obtain through the
Science Platform.
}
  
 \newpage 
\subsection{[LVV-164] DMS-REQ-0333-V-01: Maximum Likelihood Values and Covariances }\label{lvv-164}

\begin{longtable}{cccc}
\hline
\textbf{Jira Link} & \textbf{Assignee} & \textbf{Status} & \textbf{Test Cases}\\ \hline
\href{https://jira.lsstcorp.org/browse/LVV-164}{LVV-164} &
Jim Bosch & Not Covered &
\begin{tabular}{c}
LVV-T26 \\
\end{tabular}
\\
\hline
\end{longtable}

\textbf{Verification Element Description:} \\
Inspect the tables and show that maximum likelihood values and
covariances have been calculated.

{\footnotesize
\begin{longtable}{p{2.5cm}p{13.5cm}}
\hline
\multicolumn{2}{c}{\textbf{Requirement Details}}\\ \hline
Requirement ID & DMS-REQ-0333 \\ \cdashline{1-2}
Requirement Description &
\begin{minipage}[]{13cm}
\textbf{Specification:} Quantities delivered by all measurement
algorithms shall include maximum likelihood values and covariances.
\end{minipage}
\\ \cdashline{1-2}
Requirement Discussion &
\begin{minipage}[]{13cm}
\textbf{Discussion:} Algorithms for which such values are impossible,
will be documented explicitly to declare that the values are
unavailable.
\end{minipage}
\\ \cdashline{1-2}
Requirement Priority & 1b \\ \cdashline{1-2}
Upper Level Requirement &
\begin{tabular}{cl}
OSS-REQ-0391 & Data Product Conventions \\
\end{tabular}
\\ \hline
\end{longtable}
}


\subsubsection{Test Cases Summary}
\begin{longtable}{p{3cm}p{2.5cm}p{2.5cm}p{3cm}p{4cm}}
\toprule
\href{https://jira.lsstcorp.org/secure/Tests.jspa\#/testCase/LVV-T26}{LVV-T26} & \multicolumn{4}{p{12cm}}{ Verify implementation of Maximum Likelihood Values and Covariances } \\ \hline
\textbf{Owner} & \textbf{Status} & \textbf{Version} & \textbf{Critical Event} & \textbf{Verification Type} \\ \hline
Jim Bosch & Draft & 1 & false & Test \\ \hline
\end{longtable}
{\scriptsize
\textbf{Objective:}\\
\begin{itemize}
\tightlist
\item
  Check that all measurements in source and object schemas include
  columns containing uncertainties, including covariances between
  jointly-measured quantities.
\item
  Check that all model-fit measurements in source and object schemas
  include columns that report goodness-of-fit.
\item
  Check that most sources and objects with successful measurements
  report finite uncertainty values for those measurements.
\item
  Check that most sources and objects with successful model-fit
  measurements report finite goodness-of-fit values.
\end{itemize}
}
  
 \newpage 
\subsection{[LVV-166] DMS-REQ-0335-V-01: PSF-Matched Coadds }\label{lvv-166}

\begin{longtable}{cccc}
\hline
\textbf{Jira Link} & \textbf{Assignee} & \textbf{Status} & \textbf{Test Cases}\\ \hline
\href{https://jira.lsstcorp.org/browse/LVV-166}{LVV-166} &
Jim Bosch & Not Covered &
\begin{tabular}{c}
LVV-T79 \\
\end{tabular}
\\
\hline
\end{longtable}

\textbf{Verification Element Description:} \\
Do a mini data release production. Demonstrate that a PSF-matched coadd
was created and inspect the archive to confirm that the file is not
present.

{\footnotesize
\begin{longtable}{p{2.5cm}p{13.5cm}}
\hline
\multicolumn{2}{c}{\textbf{Requirement Details}}\\ \hline
Requirement ID & DMS-REQ-0335 \\ \cdashline{1-2}
Requirement Description &
\begin{minipage}[]{13cm}
\textbf{Specification:} One (ugrizy plus multi-band) set of PSF-matched
coadds shall be made but shall not be archived.
\end{minipage}
\\ \cdashline{1-2}
Requirement Discussion &
\begin{minipage}[]{13cm}
\textbf{Discussion:} These are used to measure colors and shapes of
objects at ``standard'' seeing. Sufficient provenance information will
be made available to allow these coadds to be recreated by Level 3
users.
\end{minipage}
\\ \cdashline{1-2}
Requirement Priority & 1b \\ \cdashline{1-2}
Upper Level Requirement &
\begin{tabular}{cl}
OSS-REQ-0133 & Level 2 Data Products \\
\end{tabular}
\\ \hline
\end{longtable}
}


\subsubsection{Test Cases Summary}
\begin{longtable}{p{3cm}p{2.5cm}p{2.5cm}p{3cm}p{4cm}}
\toprule
\href{https://jira.lsstcorp.org/secure/Tests.jspa\#/testCase/LVV-T79}{LVV-T79} & \multicolumn{4}{p{12cm}}{ Verify implementation of PSF-Matched Coadds } \\ \hline
\textbf{Owner} & \textbf{Status} & \textbf{Version} & \textbf{Critical Event} & \textbf{Verification Type} \\ \hline
Jim Bosch & Draft & 1 & false & Test \\ \hline
\end{longtable}
{\scriptsize
\textbf{Objective:}\\
Verify that the DRP pipelines produce PSF matched coadds.
}
  
 \newpage 
\subsection{[LVV-167] DMS-REQ-0336-V-01: Regenerating Data Products from Previous Data
Releases }\label{lvv-167}

\begin{longtable}{cccc}
\hline
\textbf{Jira Link} & \textbf{Assignee} & \textbf{Status} & \textbf{Test Cases}\\ \hline
\href{https://jira.lsstcorp.org/browse/LVV-167}{LVV-167} &
Robert Lupton & Not Covered &
\begin{tabular}{c}
LVV-T159 \\
\end{tabular}
\\
\hline
\end{longtable}

\textbf{Verification Element Description:} \\
Generate a data product on demand using an old version of the software.
The general problem of demonstrating that a DR1 product generated at the
time of DR1 is reproducible at the time of DR11 is hard to verify.

{\footnotesize
\begin{longtable}{p{2.5cm}p{13.5cm}}
\hline
\multicolumn{2}{c}{\textbf{Requirement Details}}\\ \hline
Requirement ID & DMS-REQ-0336 \\ \cdashline{1-2}
Requirement Description &
\begin{minipage}[]{13cm}
\textbf{Specification:} The DMS shall be able to regenerate data
products from previous data releases to within scientifically reasonable
tolerances.
\end{minipage}
\\ \cdashline{1-2}
Requirement Discussion &
\begin{minipage}[]{13cm}
\textbf{Discussion:} This is similar to DMS-REQ-0311, but covering prior
data releases. The intent is for the software to be runnable in the same
environment as was used for the original data release without the
software having to be ported to a modern operating system.
\end{minipage}
\\ \cdashline{1-2}
Requirement Priority & 1b \\ \cdashline{1-2}
Upper Level Requirement &
\begin{tabular}{cl}
LSR-REQ-0049 & Data Product Archiving \\
\end{tabular}
\\ \hline
\end{longtable}
}


\subsubsection{Test Cases Summary}
\begin{longtable}{p{3cm}p{2.5cm}p{2.5cm}p{3cm}p{4cm}}
\toprule
\href{https://jira.lsstcorp.org/secure/Tests.jspa\#/testCase/LVV-T159}{LVV-T159} & \multicolumn{4}{p{12cm}}{ Verify implementation of Regenerating Data Products from Previous Data
Releases } \\ \hline
\textbf{Owner} & \textbf{Status} & \textbf{Version} & \textbf{Critical Event} & \textbf{Verification Type} \\ \hline
Simon Krughoff & Draft & 1 & false & Test \\ \hline
\end{longtable}
{\scriptsize
\textbf{Objective:}\\
Show that un-archived data products from previous data releases can be
generated using through the LSST Science Platform.
}
  
 \newpage 
\subsection{[LVV-168] DMS-REQ-0337-V-01: Detecting faint variable objects }\label{lvv-168}

\begin{longtable}{cccc}
\hline
\textbf{Jira Link} & \textbf{Assignee} & \textbf{Status} & \textbf{Test Cases}\\ \hline
\href{https://jira.lsstcorp.org/browse/LVV-168}{LVV-168} &
Melissa Graham & Not Covered &
\begin{tabular}{c}
LVV-T80 \\
\end{tabular}
\\
\hline
\end{longtable}

\textbf{Verification Element Description:} \\
Given a suitable dataset, process it in such a way as to detect more
faint sources.

{\footnotesize
\begin{longtable}{p{2.5cm}p{13.5cm}}
\hline
\multicolumn{2}{c}{\textbf{Requirement Details}}\\ \hline
Requirement ID & DMS-REQ-0337 \\ \cdashline{1-2}
Requirement Description &
\begin{minipage}[]{13cm}
\textbf{Specification:} The DMS shall be able to detect faint objects
showing long-term variability, or nearby object with high proper
motions.
\end{minipage}
\\ \cdashline{1-2}
Requirement Discussion &
\begin{minipage}[]{13cm}
\textbf{Discussion:} For example, this could be implemented using
short-period (yearly) coadds.
\end{minipage}
\\ \cdashline{1-2}
Requirement Priority & 2 \\ \cdashline{1-2}
Upper Level Requirement &
\begin{tabular}{cl}
OSS-REQ-0136 & Co-added Exposures \\
\end{tabular}
\\ \hline
\end{longtable}
}


\subsubsection{Test Cases Summary}
\begin{longtable}{p{3cm}p{2.5cm}p{2.5cm}p{3cm}p{4cm}}
\toprule
\href{https://jira.lsstcorp.org/secure/Tests.jspa\#/testCase/LVV-T80}{LVV-T80} & \multicolumn{4}{p{12cm}}{ Verify implementation of Detecting faint variable objects } \\ \hline
\textbf{Owner} & \textbf{Status} & \textbf{Version} & \textbf{Critical Event} & \textbf{Verification Type} \\ \hline
Melissa Graham & Draft & 1 & false & Test \\ \hline
\end{longtable}
{\scriptsize
\textbf{Objective:}\\
To verify that the Data Release Production pipeline will be able to
detect faint sources with long-term variability (e.g., quasars, proper
motion stars) via, e.g., shorter timescale coadds (month to a few
months).
}
  
 \newpage 
\subsection{[LVV-169] DMS-REQ-0338-V-01: Targeted Coadds }\label{lvv-169}

\begin{longtable}{cccc}
\hline
\textbf{Jira Link} & \textbf{Assignee} & \textbf{Status} & \textbf{Test Cases}\\ \hline
\href{https://jira.lsstcorp.org/browse/LVV-169}{LVV-169} &
Robert Lupton & Not Covered &
\begin{tabular}{c}
LVV-T81 \\
\end{tabular}
\\
\hline
\end{longtable}

\textbf{Verification Element Description:} \\
Show procedure for persisting cutouts from a coadd. Show user interface
for retrieving the history of cutouts for a specific location.

{\footnotesize
\begin{longtable}{p{2.5cm}p{13.5cm}}
\hline
\multicolumn{2}{c}{\textbf{Requirement Details}}\\ \hline
Requirement ID & DMS-REQ-0338 \\ \cdashline{1-2}
Requirement Description &
\begin{minipage}[]{13cm}
\textbf{Specification:} It shall be possible to retain small sections of
all generated coadds.
\end{minipage}
\\ \cdashline{1-2}
Requirement Discussion &
\begin{minipage}[]{13cm}
\textbf{Discussion:} This supports quality assessment and targeted
science.
\end{minipage}
\\ \cdashline{1-2}
Requirement Priority & 2 \\ \cdashline{1-2}
Upper Level Requirement &
\begin{tabular}{cl}
LSR-REQ-0040 & Data Quality Monitoring \\
OSS-REQ-0136 & Co-added Exposures \\
\end{tabular}
\\ \hline
\end{longtable}
}


\subsubsection{Test Cases Summary}
\begin{longtable}{p{3cm}p{2.5cm}p{2.5cm}p{3cm}p{4cm}}
\toprule
\href{https://jira.lsstcorp.org/secure/Tests.jspa\#/testCase/LVV-T81}{LVV-T81} & \multicolumn{4}{p{12cm}}{ Verify implementation of Targeted Coadds } \\ \hline
\textbf{Owner} & \textbf{Status} & \textbf{Version} & \textbf{Critical Event} & \textbf{Verification Type} \\ \hline
Jim Bosch & Draft & 1 & false & Test \\ \hline
\end{longtable}
{\scriptsize
\textbf{Objective:}\\
Verify that small sections of any coadd produced by the DRP pipelines
can be retained, even if the full coadd is not.
}
  
 \newpage 
\subsection{[LVV-170] DMS-REQ-0339-V-01: Tracking Characterization Changes Between Data
Releases }\label{lvv-170}

\begin{longtable}{cccc}
\hline
\textbf{Jira Link} & \textbf{Assignee} & \textbf{Status} & \textbf{Test Cases}\\ \hline
\href{https://jira.lsstcorp.org/browse/LVV-170}{LVV-170} &
Colin Slater & Not Covered &
\begin{tabular}{c}
LVV-T82 \\
\end{tabular}
\\
\hline
\end{longtable}

\textbf{Verification Element Description:} \\
Show procedure for selecting samples for long term persistence.
Demonstrate that some data can be moved from a data release to a
separate store.

{\footnotesize
\begin{longtable}{p{2.5cm}p{13.5cm}}
\hline
\multicolumn{2}{c}{\textbf{Requirement Details}}\\ \hline
Requirement ID & DMS-REQ-0339 \\ \cdashline{1-2}
Requirement Description &
\begin{minipage}[]{13cm}
\textbf{Specification:} Small, overlapping, samples of data from older
releases shall be kept loaded in the database.
\end{minipage}
\\ \cdashline{1-2}
Requirement Discussion &
\begin{minipage}[]{13cm}
\textbf{Discussion:} This enables a comparison of how current data
releases relate to previous data releases and to improve data quality
monitoring.
\end{minipage}
\\ \cdashline{1-2}
Requirement Priority & 1a \\ \cdashline{1-2}
Upper Level Requirement &
\begin{tabular}{cl}
LSR-REQ-0040 & Data Quality Monitoring \\
\end{tabular}
\\ \hline
\end{longtable}
}


\subsubsection{Test Cases Summary}
\begin{longtable}{p{3cm}p{2.5cm}p{2.5cm}p{3cm}p{4cm}}
\toprule
\href{https://jira.lsstcorp.org/secure/Tests.jspa\#/testCase/LVV-T82}{LVV-T82} & \multicolumn{4}{p{12cm}}{ Verify implementation of Tracking Characterization Changes Between Data
Releases } \\ \hline
\textbf{Owner} & \textbf{Status} & \textbf{Version} & \textbf{Critical Event} & \textbf{Verification Type} \\ \hline
Jim Bosch & Defined & 1 & false & Test \\ \hline
\end{longtable}
{\scriptsize
\textbf{Objective:}\\
Verify that small-area subsets of a DR can be retained when most of that
DR is retired, for comparison with future DRs.
}
  
 \newpage 
\subsection{[LVV-178] DMS-REQ-0347-V-01: Measurements in catalogs }\label{lvv-178}

\begin{longtable}{cccc}
\hline
\textbf{Jira Link} & \textbf{Assignee} & \textbf{Status} & \textbf{Test Cases}\\ \hline
\href{https://jira.lsstcorp.org/browse/LVV-178}{LVV-178} &
Colin Slater & Not Covered &
\begin{tabular}{c}
LVV-T13 \\
LVV-T14 \\
LVV-T21 \\
LVV-T22 \\
LVV-T28 \\
LVV-T1946 \\
LVV-T1947 \\
\end{tabular}
\\
\hline
\end{longtable}

\textbf{Verification Element Description:} \\
Inspect the schema for each table and ensure that measurement columns
use appropriate units.

{\footnotesize
\begin{longtable}{p{2.5cm}p{13.5cm}}
\hline
\multicolumn{2}{c}{\textbf{Requirement Details}}\\ \hline
Requirement ID & DMS-REQ-0347 \\ \cdashline{1-2}
Requirement Description &
\begin{minipage}[]{13cm}
\textbf{Specification:} All catalogs shall record source measurements in
fluxes, reported in nanojansky.
\end{minipage}
\\ \cdashline{1-2}
Requirement Discussion &
\begin{minipage}[]{13cm}
\textbf{Discussion:} Difference measurements can go negative and in
multi-epoch surveys averaging of fluxes rather than magnitudes is
required. This requirement does not preclude making magnitudes available
where appropriate. The rationale for the use of nanojanskys is presented
in Document-27758.
\end{minipage}
\\ \cdashline{1-2}
Requirement Priority & 1b \\ \cdashline{1-2}
Upper Level Requirement &
\begin{tabular}{cl}
OSS-REQ-0391 & Data Product Conventions \\
\end{tabular}
\\ \hline
\end{longtable}
}

\subsubsection{Verified By}
\begin{itemize}
\item . DM-9953 (\ref{dm-9953}) Pixels rejected from coaddition and CCD are not masked on coadds
\item . DM-13058 (\ref{dm-13058}) Inconsistent aperture corrections in W44 reprocessing of HSC RC1
\end{itemize}

\subsubsection{Test Cases Summary}
\begin{longtable}{p{3cm}p{2.5cm}p{2.5cm}p{3cm}p{4cm}}
\toprule
\href{https://jira.lsstcorp.org/secure/Tests.jspa\#/testCase/LVV-T13}{LVV-T13} & \multicolumn{4}{p{12cm}}{ DRP-00-15: Scientific Verification of Source Catalog } \\ \hline
\textbf{Owner} & \textbf{Status} & \textbf{Version} & \textbf{Critical Event} & \textbf{Verification Type} \\ \hline
Jim Bosch & Deprecated & 1 & false & Test \\ \hline
\end{longtable}
{\scriptsize
\textbf{Objective:}\\
This test will check that the source catalogs delivered by the DRP
science payload meet the requirements laid down by \citeds{LSE-61}.\\
Specifically, this will demonstrate that:

\begin{itemize}
\tightlist
\item
  Measurements in the catalog are presented in flux units
  (DMS-REQ-0347);
\item
  Derived quantities are provided in pre-computed columns
  (DMS-REQ-0331);
\item
  Aperture corrections for different photometry algorithms are
  consistent.
\item
  Photometry measurements are consistent with reference catalog
  photometry (including sources not used in photometric calibration).
\item
  Astrometry measurements are consistent with reference catalog
  positions (including sources not used in astrometric calibration).
\end{itemize}

This test does not include quantitative targets for the science quality
criteria; we instead require for each test that we be able to quickly
construct a plot in which such a target can be visualized.
}
\begin{longtable}{p{3cm}p{2.5cm}p{2.5cm}p{3cm}p{4cm}}
\toprule
\href{https://jira.lsstcorp.org/secure/Tests.jspa\#/testCase/LVV-T14}{LVV-T14} & \multicolumn{4}{p{12cm}}{ DRP-00-25: Scientific Verification of Object Catalog } \\ \hline
\textbf{Owner} & \textbf{Status} & \textbf{Version} & \textbf{Critical Event} & \textbf{Verification Type} \\ \hline
Jim Bosch & Deprecated & 1 & false & Test \\ \hline
\end{longtable}
{\scriptsize
\textbf{Objective:}\\
This test will check that the object catalogs delivered by the DRP
science payload meet the requirements laid down by \citeds{LSE-61}.\\
Specifically, this will demonstrate that:

\begin{itemize}
\tightlist
\item
  Measurements in the catalog are presented in flux units
  (DMS-REQ-0347);
\item
  Derived quantities are provided in pre-computed columns
  (DMS-REQ-0331);
\item
  Aperture corrections for different photometry algorithms are
  consistent.
\item
  PSF models correctly predict the ellipticities of stars over each
  tract.
\item
  Photometry measurements are consistent with reference catalog
  photometry (including sources not used in photometric calibration).
\item
  Astrometry measurements are consistent with reference catalog
  positions (including sources not used in astrometric calibration).
\item
  Forced and unforced photometry measurements are consistent.
\item
  The slope of the stellar locus in color-color space is not a function
  of position on the sky.
\end{itemize}

This test does not include quantitative targets for the science quality
criteria; we instead re- quire for each test that we be able to quickly
construct a plot in which such a target can be visualized.\\
All science quality tests in this section shall distinguish between
blended and isolated objects.
}
\begin{longtable}{p{3cm}p{2.5cm}p{2.5cm}p{3cm}p{4cm}}
\toprule
\href{https://jira.lsstcorp.org/secure/Tests.jspa\#/testCase/LVV-T21}{LVV-T21} & \multicolumn{4}{p{12cm}}{ AG-00-20: Scientific Verification of DIASource Catalog } \\ \hline
\textbf{Owner} & \textbf{Status} & \textbf{Version} & \textbf{Critical Event} & \textbf{Verification Type} \\ \hline
Eric Bellm & Deprecated & 1 & false & Test \\ \hline
\end{longtable}
{\scriptsize
\textbf{Objective:}\\
This test will check that the difference image source catalogs delivered
by the Alert Generation science payload meet the requirements laid down
by \citeds{LSE-61}.

\begin{itemize}
\tightlist
\item
  Specifically, this will demonstrate that:
\item
  Measurements in the catalog are presented in flux units
  (DMS-REQ-0347);
\item
  Each DIASource record contains an appropriate subset of the attributes
  required by DMS-REQ-0269. In particular, the LDM-503-3-era pipeline is
  expected to provide DIASource positions (sky and focal plane), fluxes,
  and flags indicative of issues encountered during processing.
\item
  Faint DIASources satisfying additional criteria are stored
  (DMS-REQ-0270).
\item
  Derived quantities are provided in pre-computed columns
  (DMS-REQ-0331);
\end{itemize}

This test does not include quantitative targets for the science quality
criteria.\\[2\baselineskip]
}
\begin{longtable}{p{3cm}p{2.5cm}p{2.5cm}p{3cm}p{4cm}}
\toprule
\href{https://jira.lsstcorp.org/secure/Tests.jspa\#/testCase/LVV-T22}{LVV-T22} & \multicolumn{4}{p{12cm}}{ AG-00-25: Scientific Verification of DIAObject Catalog } \\ \hline
\textbf{Owner} & \textbf{Status} & \textbf{Version} & \textbf{Critical Event} & \textbf{Verification Type} \\ \hline
Eric Bellm & Deprecated & 1 & false & Test \\ \hline
\end{longtable}
{\scriptsize
\textbf{Objective:}\\
This test will check that the DIAObject catalogs delivered by the Alert
Generation science pay- load meet the requirements laid down by
\citeds{LSE-61}.\\
Specifically, this will demonstrate that:

\begin{itemize}
\tightlist
\item
  DIAObjects are recorded with unique identifiers (DMS-REQ-0271);
\item
  Measurements in the catalog are presented in flux units
  (DMS-REQ-0347);
\item
  EachDIAObjectrecordcontainscontainsanappropriatesetofsummaryattributes(DMS-
  REQ-0271 and DMS-REQ-0272). Note:

  \begin{itemize}
  \tightlist
  \item
    This test is executed independently of the Data Release Production
    system. Hence, DIAObjects are not associated to Objects, and the
    association metadata specified by DMS-REQ-0271 is not expected to be
    available.
  \item
    TheLDM-503-3erapipelineisnotexpectedtocalculateorpersistallattributesspec-
    ified by DMS-REQ-0272 requirement.
  \end{itemize}
\item
  Relevant derived quantities are provided in pre-computed columns
  (DMS-REQ-0331);~
\end{itemize}

This test does not include quantitative targets for the science quality
criteria.
}
\begin{longtable}{p{3cm}p{2.5cm}p{2.5cm}p{3cm}p{4cm}}
\toprule
\href{https://jira.lsstcorp.org/secure/Tests.jspa\#/testCase/LVV-T28}{LVV-T28} & \multicolumn{4}{p{12cm}}{ Verify implementation of measurements in catalogs from PVIs } \\ \hline
\textbf{Owner} & \textbf{Status} & \textbf{Version} & \textbf{Critical Event} & \textbf{Verification Type} \\ \hline
Colin Slater & Approved & 1 & false & Test \\ \hline
\end{longtable}
{\scriptsize
\textbf{Objective:}\\
Verify that source measurements in catalogs containing measurements from
processed visit images are in flux units.
}
\begin{longtable}{p{3cm}p{2.5cm}p{2.5cm}p{3cm}p{4cm}}
\toprule
\href{https://jira.lsstcorp.org/secure/Tests.jspa\#/testCase/LVV-T1946}{LVV-T1946} & \multicolumn{4}{p{12cm}}{ Verify implementation of measurements in catalogs from coadds } \\ \hline
\textbf{Owner} & \textbf{Status} & \textbf{Version} & \textbf{Critical Event} & \textbf{Verification Type} \\ \hline
Jeffrey Carlin & Approved & 1 & false & Test \\ \hline
\end{longtable}
{\scriptsize
\textbf{Objective:}\\
Verify that source measurements in catalogs containing measurements from
coadd images are in flux units.
}
\begin{longtable}{p{3cm}p{2.5cm}p{2.5cm}p{3cm}p{4cm}}
\toprule
\href{https://jira.lsstcorp.org/secure/Tests.jspa\#/testCase/LVV-T1947}{LVV-T1947} & \multicolumn{4}{p{12cm}}{ Verify implementation of measurements in catalogs from difference images } \\ \hline
\textbf{Owner} & \textbf{Status} & \textbf{Version} & \textbf{Critical Event} & \textbf{Verification Type} \\ \hline
Jeffrey Carlin & Approved & 1 & false & Test \\ \hline
\end{longtable}
{\scriptsize
\textbf{Objective:}\\
Verify that source measurements in catalogs containing measurements from
difference images are in flux units.
}
  
 \newpage 
\subsection{[LVV-179] DMS-REQ-0348-V-01: Pre-defined alert filters }\label{lvv-179}

\begin{longtable}{cccc}
\hline
\textbf{Jira Link} & \textbf{Assignee} & \textbf{Status} & \textbf{Test Cases}\\ \hline
\href{https://jira.lsstcorp.org/browse/LVV-179}{LVV-179} &
Eric Bellm & Not Covered &
\begin{tabular}{c}
LVV-T114 \\
LVV-T218 \\
\end{tabular}
\\
\hline
\end{longtable}

\textbf{Verification Element Description:} \\
Create a filter from a restricted set of predefined filters.

{\footnotesize
\begin{longtable}{p{2.5cm}p{13.5cm}}
\hline
\multicolumn{2}{c}{\textbf{Requirement Details}}\\ \hline
Requirement ID & DMS-REQ-0348 \\ \cdashline{1-2}
Requirement Description &
\begin{minipage}[]{13cm}
\textbf{Specification:} Users of the LSST Alert Filtering Service shall
be able to use a predefined set of simple filters.
\end{minipage}
\\ \cdashline{1-2}
Requirement Discussion &
\begin{minipage}[]{13cm}
\textbf{Discussion:} See LSR-REQ-0026
\end{minipage}
\\ \cdashline{1-2}
Requirement Priority & 2 \\ \cdashline{1-2}
Upper Level Requirement &
\begin{tabular}{cl}
LSR-REQ-0026 & Predefined Transient Filters \\
\end{tabular}
\\ \hline
\end{longtable}
}


\subsubsection{Test Cases Summary}
\begin{longtable}{p{3cm}p{2.5cm}p{2.5cm}p{3cm}p{4cm}}
\toprule
\href{https://jira.lsstcorp.org/secure/Tests.jspa\#/testCase/LVV-T114}{LVV-T114} & \multicolumn{4}{p{12cm}}{ Verify implementation of Pre-defined alert filters } \\ \hline
\textbf{Owner} & \textbf{Status} & \textbf{Version} & \textbf{Critical Event} & \textbf{Verification Type} \\ \hline
Eric Bellm & Defined & 1 & false & Test \\ \hline
\end{longtable}
{\scriptsize
\textbf{Objective:}\\
Verify that users of the Alert Filtering service can use a predefined
set of filters.
}
\begin{longtable}{p{3cm}p{2.5cm}p{2.5cm}p{3cm}p{4cm}}
\toprule
\href{https://jira.lsstcorp.org/secure/Tests.jspa\#/testCase/LVV-T218}{LVV-T218} & \multicolumn{4}{p{12cm}}{ Simple Filtering of the LSST Alert Stream } \\ \hline
\textbf{Owner} & \textbf{Status} & \textbf{Version} & \textbf{Critical Event} & \textbf{Verification Type} \\ \hline
Eric Bellm & Approved & 1 & false & Test \\ \hline
\end{longtable}
{\scriptsize
\textbf{Objective:}\\
This test will demonstrate the LSST Alert Filtering Service that returns
a subset of alerts from the full stream identified by user-provided
filters.\\[2\baselineskip]Specifically, this will demonstrate that:\\

\begin{itemize}
\tightlist
\item
  The filtering service can retrieve alerts from the full alert stream
  and filter them according to their contents; ~ ~
\item
  The filtered subset can be delivered to science users.
\end{itemize}
}
  
 \newpage 
\subsection{[LVV-180] DMS-REQ-0349-V-01: Detecting extended low surface brightness objects }\label{lvv-180}

\begin{longtable}{cccc}
\hline
\textbf{Jira Link} & \textbf{Assignee} & \textbf{Status} & \textbf{Test Cases}\\ \hline
\href{https://jira.lsstcorp.org/browse/LVV-180}{LVV-180} &
Jim Bosch & Not Covered &
\begin{tabular}{c}
LVV-T71 \\
\end{tabular}
\\
\hline
\end{longtable}

\textbf{Verification Element Description:} \\
From a suitable dataset, using LSST code, post process it and detect low
surface brightness objects.

{\footnotesize
\begin{longtable}{p{2.5cm}p{13.5cm}}
\hline
\multicolumn{2}{c}{\textbf{Requirement Details}}\\ \hline
Requirement ID & DMS-REQ-0349 \\ \cdashline{1-2}
Requirement Description &
\begin{minipage}[]{13cm}
\textbf{Specification:} It shall be possible to detect extended low
surface brightness objects in coadds.
\end{minipage}
\\ \cdashline{1-2}
Requirement Discussion &
\begin{minipage}[]{13cm}
\textbf{Discussion:} For example, this could be done by using the binned
detection algorithm from SDSS.
\end{minipage}
\\ \cdashline{1-2}
Requirement Priority & 2 \\ \cdashline{1-2}
Upper Level Requirement &
\begin{tabular}{cl}
OSS-REQ-0133 & Level 2 Data Products \\
\end{tabular}
\\ \hline
\end{longtable}
}


\subsubsection{Test Cases Summary}
\begin{longtable}{p{3cm}p{2.5cm}p{2.5cm}p{3cm}p{4cm}}
\toprule
\href{https://jira.lsstcorp.org/secure/Tests.jspa\#/testCase/LVV-T71}{LVV-T71} & \multicolumn{4}{p{12cm}}{ Verify implementation of Detecting extended low surface brightness
objects } \\ \hline
\textbf{Owner} & \textbf{Status} & \textbf{Version} & \textbf{Critical Event} & \textbf{Verification Type} \\ \hline
Jim Bosch & Draft & 1 & false & Test \\ \hline
\end{longtable}
{\scriptsize
\textbf{Objective:}\\
Verify that low-surface brightness objects (including those whose PSF
S/N is lower than the detection threshold) are detected in coadds.
}
  
 \newpage 
\subsection{[LVV-181] DMS-REQ-0350-V-01: Associating Objects across data releases }\label{lvv-181}

\begin{longtable}{cccc}
\hline
\textbf{Jira Link} & \textbf{Assignee} & \textbf{Status} & \textbf{Test Cases}\\ \hline
\href{https://jira.lsstcorp.org/browse/LVV-181}{LVV-181} &
Colin Slater & Not Covered &
\begin{tabular}{c}
LVV-T116 \\
\end{tabular}
\\
\hline
\end{longtable}

\textbf{Verification Element Description:} \\
Do two mini data release production runs on a single dataset that covers
a shared area multiple times. Query the second data release's Object
table and request an association with the previous data release. Do this
with the previous data release being inaccessible.

{\footnotesize
\begin{longtable}{p{2.5cm}p{13.5cm}}
\hline
\multicolumn{2}{c}{\textbf{Requirement Details}}\\ \hline
Requirement ID & DMS-REQ-0350 \\ \cdashline{1-2}
Requirement Description &
\begin{minipage}[]{13cm}
\textbf{Specification:} It shall be possible to associate an Object in
one data release to the most likely match in the Object table from
another data release. This shall be possible without the previous data
releases being online.
\end{minipage}
\\ \cdashline{1-2}
Requirement Discussion &
\begin{minipage}[]{13cm}
\textbf{Discussion:} This could be done with a database table mapping
every Object in one data release to the matched Object in every other
data release.
\end{minipage}
\\ \cdashline{1-2}
Requirement Priority & 2 \\ \cdashline{1-2}
Upper Level Requirement &
\begin{tabular}{cl}
\end{tabular}
\\ \hline
\end{longtable}
}


\subsubsection{Test Cases Summary}
\begin{longtable}{p{3cm}p{2.5cm}p{2.5cm}p{3cm}p{4cm}}
\toprule
\href{https://jira.lsstcorp.org/secure/Tests.jspa\#/testCase/LVV-T116}{LVV-T116} & \multicolumn{4}{p{12cm}}{ Verify implementation of Associating Objects across data releases } \\ \hline
\textbf{Owner} & \textbf{Status} & \textbf{Version} & \textbf{Critical Event} & \textbf{Verification Type} \\ \hline
Kian-Tat Lim & Draft & 1 & false & Test \\ \hline
\end{longtable}
{\scriptsize
\textbf{Objective:}\\
Load DR, observe queryable association
}
  
 \newpage 
\subsection{[LVV-182] DMS-REQ-0351-V-01: Provide Beam Projector Coordinate Calculation
Software }\label{lvv-182}

\begin{longtable}{cccc}
\hline
\textbf{Jira Link} & \textbf{Assignee} & \textbf{Status} & \textbf{Test Cases}\\ \hline
\href{https://jira.lsstcorp.org/browse/LVV-182}{LVV-182} &
Robert Lupton & Not Covered &
\begin{tabular}{c}
LVV-T133 \\
\end{tabular}
\\
\hline
\end{longtable}

\textbf{Verification Element Description:} \\
Convert some coordinates using the transformation code and compare with
expectations.

{\footnotesize
\begin{longtable}{p{2.5cm}p{13.5cm}}
\hline
\multicolumn{2}{c}{\textbf{Requirement Details}}\\ \hline
Requirement ID & DMS-REQ-0351 \\ \cdashline{1-2}
Requirement Description &
\begin{minipage}[]{13cm}
\textbf{Specification:} The DMS shall provide software to represent the
coordinate transformations relating the collimated beam projector
position and telescope pupil position to the illumination position on
the telescope optical elements and focal plane.
\end{minipage}
\\ \cdashline{1-2}
Requirement Priority & 1a \\ \cdashline{1-2}
Upper Level Requirement &
\begin{tabular}{cl}
OSS-REQ-0383 & Beam Projector Coordinate Relationship \\
\end{tabular}
\\ \hline
\end{longtable}
}


\subsubsection{Test Cases Summary}
\begin{longtable}{p{3cm}p{2.5cm}p{2.5cm}p{3cm}p{4cm}}
\toprule
\href{https://jira.lsstcorp.org/secure/Tests.jspa\#/testCase/LVV-T133}{LVV-T133} & \multicolumn{4}{p{12cm}}{ Verify implementation of Provide Beam Projector Coordinate Calculation
Software } \\ \hline
\textbf{Owner} & \textbf{Status} & \textbf{Version} & \textbf{Critical Event} & \textbf{Verification Type} \\ \hline
Robert Lupton & Defined & 1 & false & Test \\ \hline
\end{longtable}
{\scriptsize
\textbf{Objective:}\\
Verify that the DMS provides software to calculate coordinates relating
the collimated beam projector position and telescope pupil position to
the illumination position on the telescope optical elements and focal
plane.
}
  
 \newpage 
\subsection{[LVV-3399] DMS-REQ-0378-V-01: Simultaneous Image Access Performance }\label{lvv-3399}

\begin{longtable}{cccc}
\hline
\textbf{Jira Link} & \textbf{Assignee} & \textbf{Status} & \textbf{Test Cases}\\ \hline
\href{https://jira.lsstcorp.org/browse/LVV-3399}{LVV-3399} &
Leanne Guy & Not Covered &
\begin{tabular}{c}
\end{tabular}
\\
\hline
\end{longtable}

\textbf{Verification Element Description:} \\
Undefined

{\footnotesize
\begin{longtable}{p{2.5cm}p{13.5cm}}
\hline
\multicolumn{2}{c}{\textbf{Requirement Details}}\\ \hline
Requirement ID & DMS-REQ-0378 \\ \cdashline{1-2}
Requirement Description &
\begin{minipage}[]{13cm}
\textbf{Specification:} All the enclosed performance metrics shall be
met simultaneously.
\end{minipage}
\\ \cdashline{1-2}
Requirement Discussion &
\begin{minipage}[]{13cm}
\textbf{Discussion:} While these image access requirements specify
maximum timings and minimum capacities for retrieval by a particular
mechanism (VO services), it should be noted that simultaneous usage of
other access mechanisms will in practice increase timings and/or reduce
available capacity.
\end{minipage}
\\ \cdashline{1-2}
Requirement Priority & 2 \\ \cdashline{1-2}
Upper Level Requirement &
\begin{tabular}{cl}
OSS-REQ-0181 & Data Products Query and Download Infrastructure \\
\end{tabular}
\\ \hline
\end{longtable}
}


  
 \newpage 
\subsection{[LVV-3401] DMS-REQ-0359-V-01: RMS photometric repeatability in uzy }\label{lvv-3401}

\begin{longtable}{cccc}
\hline
\textbf{Jira Link} & \textbf{Assignee} & \textbf{Status} & \textbf{Test Cases}\\ \hline
\href{https://jira.lsstcorp.org/browse/LVV-3401}{LVV-3401} &
Leanne Guy & Not Covered &
\begin{tabular}{c}
LVV-T1756 \\
\end{tabular}
\\
\hline
\end{longtable}

\textbf{Verification Element Description:} \\
The RMS photometric repeatability of bright non-saturated unresolved
point sources in the u, z, and y filters shall be less than
\textbf{PA1uzy = 7.5 millimagnitudes}.

Associated element DMS-REQ-0359-V-02
(\href{https://jira.lsstcorp.org/browse/LVV-9751}{LVV-9751}) satisfies
the requirement on the maximum fraction of sensors with scientifically
unusable pixels.

Associated element DMS-REQ-0359-V-03
(\href{https://jira.lsstcorp.org/browse/LVV-9752}{LVV-9752}) satisfies
the constraint on maximum fraction of outliers among non-saturated point
sources.

Associated element DMS-REQ-0359-V-04
(\href{https://jira.lsstcorp.org/browse/LVV-9753}{LVV-9753}) satisfies
the accuracy of zero-point for colors that use the u-band.

Associated element DMS-REQ-0359-V-05
(\href{https://jira.lsstcorp.org/browse/LVV-9754}{LVV-9754}) satisfies
the repeatability outlier limit in g, r, and i-bands.

Associated element DMS-REQ-0359-V-06
(\href{https://jira.lsstcorp.org/browse/LVV-9755}{LVV-9755}) satisfies
the constraint on the accuracy of the transformation from internal to
physical photometric scales.

Associated element DMS-REQ-0359-V-07
(\href{https://jira.lsstcorp.org/browse/LVV-9756}{LVV-9756}) satisfies
the rms width of the internal photometric zero-point in u-band.

Associated element DMS-REQ-0359-V-08
(\href{https://jira.lsstcorp.org/browse/LVV-9757}{LVV-9757}) satisfies
the maximum local significance of imperfect crosstalk corrections.

Associated element DMS-REQ-0359-V-09
(\href{https://jira.lsstcorp.org/browse/LVV-9758}{LVV-9758}) satisfies
the repeatability outlier limit in u, z, and y-bands.

Associated element DMS-REQ-0359-V-10
(\href{https://jira.lsstcorp.org/browse/LVV-9759}{LVV-9759}) satisfies
the rms photometric repeatability in g, r, and i-bands.

Associated element DMS-REQ-0359-V-11
(\href{https://jira.lsstcorp.org/browse/LVV-9760}{LVV-9760}) satisfies
the fraction of zero-point errors that can exceed the outlier limit.

Associated element DMS-REQ-0359-V-12
(\href{https://jira.lsstcorp.org/browse/LVV-9761}{LVV-9761}) satisfies
the maximum fraction of unusable pixels per sensor.

Associated element DMS-REQ-0359-V-13
(\href{https://jira.lsstcorp.org/browse/LVV-9762}{LVV-9762}) satisfies
the maximum allowable precision in the sky brightness determination.

Associated element DMS-REQ-0359-V-14
(\href{https://jira.lsstcorp.org/browse/LVV-9763}{LVV-9763}) satisfies
the rms width of the internal photometric zero-point in g, r, i, z, and
y-bands.

Associated element DMS-REQ-0359-V-15
(\href{https://jira.lsstcorp.org/browse/LVV-9764}{LVV-9764}) satisfies
the percentage of the image area affected by ghosts that exceed the
threshold.

Associated element DMS-REQ-0359-V-16
(\href{https://jira.lsstcorp.org/browse/LVV-9765}{LVV-9765}) satisfies
the accuracy of zero-point for colors that do not include the u-band.

Associated element DMS-REQ-0359-V-17
(\href{https://jira.lsstcorp.org/browse/LVV-9766}{LVV-9766}) satisfies
the maximum RMS of the ratio of the flux measurement error between
resolved/unresolved sources.

{\footnotesize
\begin{longtable}{p{2.5cm}p{13.5cm}}
\hline
\multicolumn{2}{c}{\textbf{Requirement Details}}\\ \hline
Requirement ID & DMS-REQ-0359 \\ \cdashline{1-2}
Requirement Description &
\begin{minipage}[]{13cm}
\textbf{Specification:} The DMS shall include software to enable the
calculation of the photometric performance metrics defined in
OSS-REQ-0387.
\end{minipage}
\\ \cdashline{1-2}
Requirement Parameters & {[}\textbf{GhostAF = 1{{[}percent{]}}} Percentage of image area that can
have ghosts with surface brightness gradient amplitude of more than 1/3
of the sky noise over 1 arcsec., \textbf{PF1 = 10{{[}percent{]}}} The
maximum fraction of isolated non-saturated point source measurements
exceeding the outlier limit., \textbf{PA1gri = 5{{[}millimagnitude{]}}}
The RMS photometric repeatability of bright non-saturated unresolved
point sources in the g, r, and i filters., \textbf{PA3 =
10{{[}millimagnitude{]}}} RMS width of internal photometric zero-point
(precision of system uniformity across the sky) for all bands except
u-band., \textbf{PA4 = 15{{[}millimagnitude{]}}} The zero point error
outlier limit., \textbf{PF2 = 10{{[}percent{]}}} Fraction of zeropoint
errors that can exceed the zero point error outlier limit.,
\textbf{PixFrac = 1{{[}percent{]}}} The maximum fraction of pixels
scientifically unusable per sensor out of the total allowable fraction
of sensors meeting this performance., \textbf{PA1uzy =
7.5{{[}millimagnitude{]}}} The RMS photometric repeatability of bright
non-saturated unresolved point sources in the u, z, and y filters.,
\textbf{PA6 = 10{{[}millimagnitude{]}}} Accuracy of the transformation
of the internal LSST photometry to a physical scale (e.g. AB
magnitudes)., \textbf{Xtalk = 3{{[}sigma{]}}} The maximum local
significance integrated over the PSF of imperfect crosstalk
corrections., \textbf{PA5u = 10{{[}millimagnitude{]}}} Accuracy of
absolute band-to-band color zero-point for colors constructed using the
u-band., \textbf{ResSource = 2{{[}unitless{]}}} Maximum RMS of the ratio
of the error in integrated flux measurement between bright, isolated,
resolved sources less than 10 arcsec in diameter and bright, isolated
unresolved point sources., \textbf{PA2uzy = 22.5{{[}millimagnitude{]}}}
Repeatability outlier limit for isolated bright non-saturated point
sources in the u, z, and y filters., \textbf{SensorFraction =
15{{[}percent{]}}} The maximum allowable fraction of sensors with
PixFrac scientifically unusable pixels., \textbf{PA3u =
20{{[}millimagnitude{]}}} RMS width of internal photometric zero-point
(precision of system uniformity across the sky) in the u-band.,
\textbf{PA2gri = 15{{[}millimagnitude{]}}} Repeatability outlier limit
for isolated bright non-saturated point sources in the g, r, and i
filters., \textbf{SBPrec = 1{{[}percent{]}}} The maximum error in the
precision of the sky brightness determination., \textbf{PA5 =
5{{[}millimagnitude{]}}} Accuracy of absolute band-to-band color
zero-point for all colors constructed from any filter pair, excluding
the u-band.{]} \\ \cdashline{1-2}
Requirement Discussion &
\begin{minipage}[]{13cm}
\textbf{Discussion:} The relevant metrics are listed in the table
photometricPerformance below. The values in the tables are the target
values for LSST but are not verified as part of this requirement.
\end{minipage}
\\ \cdashline{1-2}
Requirement Priority & 1a \\ \cdashline{1-2}
Upper Level Requirement &
\begin{tabular}{cl}
OSS-REQ-0387 & Photometric Performance \\
\end{tabular}
\\ \hline
\end{longtable}
}


\subsubsection{Test Cases Summary}
\begin{longtable}{p{3cm}p{2.5cm}p{2.5cm}p{3cm}p{4cm}}
\toprule
\href{https://jira.lsstcorp.org/secure/Tests.jspa\#/testCase/LVV-T1756}{LVV-T1756} & \multicolumn{4}{p{12cm}}{ Verify calculation of photometric repeatability in uzy filters } \\ \hline
\textbf{Owner} & \textbf{Status} & \textbf{Version} & \textbf{Critical Event} & \textbf{Verification Type} \\ \hline
Jeffrey Carlin & Approved & 1 & false & Test \\ \hline
\end{longtable}
{\scriptsize
\textbf{Objective:}\\
Verify that the DM system has provided the code to calculate the RMS
photometric repeatability of bright non-saturated unresolved point
sources in the u, z, and y filters, and assess whether it meets the
requirement that it shall be less than \textbf{PA1uzy = 7.5
millimagnitudes}.
}
  
 \newpage 
\subsection{[LVV-3402] DMS-REQ-0360-V-01: Median astrometric error on 20 arcmin scales }\label{lvv-3402}

\begin{longtable}{cccc}
\hline
\textbf{Jira Link} & \textbf{Assignee} & \textbf{Status} & \textbf{Test Cases}\\ \hline
\href{https://jira.lsstcorp.org/browse/LVV-3402}{LVV-3402} &
Leanne Guy & Not Covered &
\begin{tabular}{c}
LVV-T363 \\
LVV-T1745 \\
\end{tabular}
\\
\hline
\end{longtable}

\textbf{Verification Element Description:} \\
Median relative astrometric measurement error on 20 arcminute scales
shall be no more than~\textbf{AM2 = 10~milliarcseconds.}

Associated element DMS-REQ-0360-V-02
(\href{https://jira.lsstcorp.org/browse/LVV-9767}{LVV-9767})~satisfies~the
maximum fraction of astrometric outliers on 5 arcminute scales.

Associated element
DMS-REQ-0360-V-03~\href{https://jira.lsstcorp.org/browse/LVV-9768}{(LVV-9768)}~satisfies~the
median astrometric error on 5 arcminute scales.

Associated element
DMS-REQ-0360-V-04~\href{https://jira.lsstcorp.org/browse/LVV-9769}{(LVV-9769)}~satisfies~the
median astrometric error in absolute positions.

Associated element
DMS-REQ-0360-V-05~\href{https://jira.lsstcorp.org/browse/LVV-9770}{(LVV-9770)}~satisfies~the
astrometric outlier limit on 20 arcminute scales.

Associated element
DMS-REQ-0360-V-06~\href{https://jira.lsstcorp.org/browse/LVV-9771}{(LVV-9771)}~satisfies~the
color difference outlier limit relative to r-band.

Associated element
DMS-REQ-0360-V-07~\href{https://jira.lsstcorp.org/browse/LVV-9773}{(LVV-9773)}~satisfies
the astrometric outlier limit on 5 arcminute scales.

Associated element
DMS-REQ-0360-V-08~\href{https://jira.lsstcorp.org/browse/LVV-9774}{(LVV-9774)}~satisfies~the
median astrometric error on 200 arcminute scales.

Associated element
DMS-REQ-0360-V-09~\href{https://jira.lsstcorp.org/browse/LVV-9775}{(LVV-9775)}~satisfies
the astrometric outlier limit on 200 arcminute scales.

Associated element
DMS-REQ-0360-V-10~\href{https://jira.lsstcorp.org/browse/LVV-9776}{(LVV-9776)}~satisfies
the maximum fraction of astrometric outliers on 20 arcminute scales.

Associated element
DMS-REQ-0360-V-11~\href{https://jira.lsstcorp.org/browse/LVV-9777}{(LVV-9777)}~satisfies
the maximum fraction of r-band color difference outliers.

Associated element
DMS-REQ-0360-V-12~\href{https://jira.lsstcorp.org/browse/LVV-9778}{(LVV-9778)}~satisfies~the
RMS difference between separations measured in the r-band and those
measured in any other filter.

Associated element
DMS-REQ-0360-V-13~\href{https://jira.lsstcorp.org/browse/LVV-9779}{(LVV-9779)}~satisfies
the maximum fraction of astrometric outliers on 200 arcminute scales.

{\footnotesize
\begin{longtable}{p{2.5cm}p{13.5cm}}
\hline
\multicolumn{2}{c}{\textbf{Requirement Details}}\\ \hline
Requirement ID & DMS-REQ-0360 \\ \cdashline{1-2}
Requirement Description &
\begin{minipage}[]{13cm}
\textbf{Specification:} The DMS shall include software to enable the
calculation of the astrometric performance metrics defined in
OSS-REQ-0388.
\end{minipage}
\\ \cdashline{1-2}
Requirement Parameters & {[}\textbf{AM2 = 10{{[}milliarcsecond{]}}} Median relative astrometric
measurement error on 20 arcminute scales., \textbf{AM1 =
10{{[}milliarcsecond{]}}} Median relative astrometric measurement error
on 5 arcminute scales shall be less than AM1., \textbf{AM3 =
15{{[}milliarcsecond{]}}} Median relative astrometric measurement error
on 200 arcminute scales., \textbf{AA1 = 50{{[}milliarcsecond{]}}} Median
error in absolute position for each axis, RA and DEC, shall be less than
AA1., \textbf{AF1 = 10{{[}percent{]}}} The maximum fraction of relative
astrometric measurements on 5 arcminute scales to exceed 5 arcminute
outlier limit., \textbf{AD3 = 30{{[}milliarcsecond{]}}} 200 arcminute
outlier limit., \textbf{AB1 = 10{{[}milliarcsecond{]}}} RMS difference
between separations measured in the r-band and those measured in any
other filter., \textbf{AD1 = 20{{[}milliarcsecond{]}}} 5 arcminute
outlier limit., \textbf{AB2 = 20{{[}milliarcsecond{]}}} The color
difference outlier limit for separations measured relative the r-band
filter in any other filter., \textbf{AD2 = 20{{[}milliarcsecond{]}}} 20
arcminute outlier limit., \textbf{ABF1 = 10{{[}percent{]}}} Fraction of
separations measured relative to the r-band that can exceed the color
difference outlier limit., \textbf{AF3 = 10{{[}percent{]}}} Fraction of
relative astrometric measurements on 200 arcminute scales to exceed 200
arcminute outlier limit., \textbf{AF2 = 10{{[}percent{]}}} The maximum
fraction of relative astrometric measurements on 20 arcminute scales to
exceed 20 arcminute outlier limit.{]} \\ \cdashline{1-2}
Requirement Discussion &
\begin{minipage}[]{13cm}
\textbf{Discussion:} The relevant metrics are listed in the table below.
The values in the tables are the target values for LSST but are not
verified as part of this requirement.
\end{minipage}
\\ \cdashline{1-2}
Requirement Priority & 1a \\ \cdashline{1-2}
Upper Level Requirement &
\begin{tabular}{cl}
OSS-REQ-0388 & Astrometric Performance \\
\end{tabular}
\\ \hline
\end{longtable}
}


\subsubsection{Test Cases Summary}
\begin{longtable}{p{3cm}p{2.5cm}p{2.5cm}p{3cm}p{4cm}}
\toprule
\href{https://jira.lsstcorp.org/secure/Tests.jspa\#/testCase/LVV-T363}{LVV-T363} & \multicolumn{4}{p{12cm}}{ Science Pipelines Release Documentation } \\ \hline
\textbf{Owner} & \textbf{Status} & \textbf{Version} & \textbf{Critical Event} & \textbf{Verification Type} \\ \hline
John Swinbank & Approved & 1 & false & Inspection \\ \hline
\end{longtable}
{\scriptsize
\textbf{Objective:}\\
This test will check:

\begin{itemize}
\tightlist
\item
  That a particular Science Pipelines release is adequately described by
  documentation at the https://pipelines.lsst.io/ site;
\item
  That the Science Pipelines release is accompanied by a
  characterization report which describes its scientific performance.
\end{itemize}
}
\begin{longtable}{p{3cm}p{2.5cm}p{2.5cm}p{3cm}p{4cm}}
\toprule
\href{https://jira.lsstcorp.org/secure/Tests.jspa\#/testCase/LVV-T1745}{LVV-T1745} & \multicolumn{4}{p{12cm}}{ Verify calculation of median relative astrometric measurement error on
20 arcminute scales } \\ \hline
\textbf{Owner} & \textbf{Status} & \textbf{Version} & \textbf{Critical Event} & \textbf{Verification Type} \\ \hline
Jeffrey Carlin & Approved & 1 & false & Test \\ \hline
\end{longtable}
{\scriptsize
\textbf{Objective:}\\
Verify that the DM system has provided the code to calculate the median
relative astrometric measurement error on 20 arcminute scales and assess
whether it meets the requirement that it shall be no more than AM2 = 10
milliarcseconds.
}
  
 \newpage 
\subsection{[LVV-3404] DMS-REQ-0362-V-01: Median residual PSF ellipticity correlations on 5
arcmin scales }\label{lvv-3404}

\begin{longtable}{cccc}
\hline
\textbf{Jira Link} & \textbf{Assignee} & \textbf{Status} & \textbf{Test Cases}\\ \hline
\href{https://jira.lsstcorp.org/browse/LVV-3404}{LVV-3404} &
Leanne Guy & Not Covered &
\begin{tabular}{c}
LVV-T376 \\
LVV-T1754 \\
\end{tabular}
\\
\hline
\end{longtable}

\textbf{Verification Element Description:} \\
Median residual PSF ellipticity correlations averaged over an arbitrary
field of view for separations less than 5 arcmin shall be no greater
than~\textbf{TE2 = 1.0e-7{{[}arcminuteSeparationCorrelation{]}}.}

Associated element DMS-REQ-0362-V-02
(\href{https://jira.lsstcorp.org/browse/LVV-9780}{LVV-9780}) satisfies
the maximum fraction of ellipticity residuals exceeding the outlier
limits.~

Associated element DMS-REQ-0362-V-03
\href{https://jira.lsstcorp.org/browse/LVV-9781}{(LVV-9781)} satisfies
the outlier limit on the PSF ellipticity correlation residuals on 5
arcmin scales.

Associated element DMS-REQ-0362-V-04
(\href{https://jira.lsstcorp.org/browse/LVV-9782}{LVV-9782}) satisfies
the median residual PSF ellipticity correlations on 1 arcmin scales.

Associated element DMS-REQ-0362-V-05
(\href{https://jira.lsstcorp.org/browse/LVV-9783}{LVV-9783})
satisfies~the outlier limit on the PSF ellipticity correlation residuals
on 1 arcmin scales.

{\footnotesize
\begin{longtable}{p{2.5cm}p{13.5cm}}
\hline
\multicolumn{2}{c}{\textbf{Requirement Details}}\\ \hline
Requirement ID & DMS-REQ-0362 \\ \cdashline{1-2}
Requirement Description &
\begin{minipage}[]{13cm}
\textbf{Specification:} The DMS shall include software to enable the
calculation of the elipticity correlations metrics defined in
OSS-REQ-0403, OSS-REQ-0404, and OSS-REQ-0405.
\end{minipage}
\\ \cdashline{1-2}
Requirement Parameters & {[}\textbf{TE3 = 4.0e-5{{[}unitless (angular correlation){]}}} Per-image
limit on the median residual ellipticity correlations at scales less
than 5 arcmin., \textbf{TE4 = 2.0e-7{{[}unitless (angular
correlation){]}}} Per-image limit on the median residual ellipticity
correlations at scales greater than or equal to 5 arcmin., \textbf{TE2 =
1.0e-7{{[}unitless (angular correlation){]}}} Maximum full-survey median
for residual ellipticity correlations at scales greater than or equal to
5 arcmin., \textbf{TEF = 15{{[}percent{]}}} Maximum fraction of visit
images that may exceed the TE3 or TE4 limits., \textbf{TE1 =
2.0e-5{{[}unitless (angular correlation){]}}} Maximum full-survey median
for residual ellipticity correlations at scales less than or equal to 1
arcmin.{]} \\ \cdashline{1-2}
Requirement Discussion &
\begin{minipage}[]{13cm}
\textbf{Discussion:} The relevant metrics are listed in the table below.
The values in the tables are the target values for LSST but are not
verified as part of this requirement.
\end{minipage}
\\ \cdashline{1-2}
Requirement Priority & 1b \\ \cdashline{1-2}
Upper Level Requirement &
\begin{tabular}{cl}
OSS-REQ-0403 & Ellipticity Correlation Function Distribution per Image \\
OSS-REQ-0404 & Ellipticity Correlation Function Distribution for Full Survey
(medians) \\
OSS-REQ-0405 & Ellipticity Correlation Function Distribution for Full Survey
(continuity) \\
\end{tabular}
\\ \hline
\end{longtable}
}


\subsubsection{Test Cases Summary}
\begin{longtable}{p{3cm}p{2.5cm}p{2.5cm}p{3cm}p{4cm}}
\toprule
\href{https://jira.lsstcorp.org/secure/Tests.jspa\#/testCase/LVV-T376}{LVV-T376} & \multicolumn{4}{p{12cm}}{ Verify the Calculation of Ellipticity Residuals and Correlations } \\ \hline
\textbf{Owner} & \textbf{Status} & \textbf{Version} & \textbf{Critical Event} & \textbf{Verification Type} \\ \hline
Leanne Guy & Approved & 1 & false & Test \\ \hline
\end{longtable}
{\scriptsize
\textbf{Objective:}\\
Verify that the DMS includes software to enable the calculation of the
ellipticity residuals and correlation metrics defined in the OSS.~
}
\begin{longtable}{p{3cm}p{2.5cm}p{2.5cm}p{3cm}p{4cm}}
\toprule
\href{https://jira.lsstcorp.org/secure/Tests.jspa\#/testCase/LVV-T1754}{LVV-T1754} & \multicolumn{4}{p{12cm}}{ Verify calculation of residual PSF ellipticity correlations for
separations less than 5 arcmin } \\ \hline
\textbf{Owner} & \textbf{Status} & \textbf{Version} & \textbf{Critical Event} & \textbf{Verification Type} \\ \hline
Jeffrey Carlin & Approved & 1 & false & Test \\ \hline
\end{longtable}
{\scriptsize
\textbf{Objective:}\\
Verify that the DM system has provided the code to calculate the median
residual PSF ellipticity correlations averaged over an arbitrary field
of view for separations less than 5 arcmin, and assess whether it meets
the requirement that it shall be no greater than \textbf{TE2 =
1.0e-7{[}arcminuteSeparationCorrelation{]}.}
}
  
 \newpage 
\subsection{[LVV-5640] DM-TS-CON-ICD-0011-V-01: Data Format\_DM\_1 }\label{lvv-5640}

\begin{longtable}{cccc}
\hline
\textbf{Jira Link} & \textbf{Assignee} & \textbf{Status} & \textbf{Test Cases}\\ \hline
\href{https://jira.lsstcorp.org/browse/LVV-5640}{LVV-5640} &
Leanne Guy & Not Covered &
\begin{tabular}{c}
\end{tabular}
\\
\hline
\end{longtable}

\textbf{Verification Element Description:} \\
Undefined

{\footnotesize
\begin{longtable}{p{2.5cm}p{13.5cm}}
\hline
\multicolumn{2}{c}{\textbf{Requirement Details}}\\ \hline
Requirement ID & DM-TS-CON-ICD-0011 \\ \cdashline{1-2}
Requirement Description &
\begin{minipage}[]{13cm}
\textbf{Specification}: The WCS solution published by Data Management
shall include the following items: equinox (double, currently 2000.0),
system (string, currently `FK5'), unit (string, currently `deg'), and
then, for each sensor, reference pixel x/y coordinates (two doubles),
reference pixel RA/dec coordinates (two doubles), and rotation and scale
matrix (four doubles).
\end{minipage}
\\ \cdashline{1-2}
Requirement Discussion &
\begin{minipage}[]{13cm}
\textbf{Discussion}: DM does not calculate the WCS for the wavefront
sensors. The idea is then to use the offset and rotation information
from the WCS to define the exact location of the donuts to use. The
distortion will be measured carefully during commissioning. The WCS
solution is also required for the Calibration mode using the science
detectors as wavefront sensors.
\end{minipage}
\\ \cdashline{1-2}
Requirement Priority &  \\ \cdashline{1-2}
Upper Level Requirement &
\begin{tabular}{cl}
\end{tabular}
\\ \hline
\end{longtable}
}


  
 \newpage 
\subsection{[LVV-5641] DM-TS-CON-ICD-0011-V-02: Data Format\_DM\_2 }\label{lvv-5641}

\begin{longtable}{cccc}
\hline
\textbf{Jira Link} & \textbf{Assignee} & \textbf{Status} & \textbf{Test Cases}\\ \hline
\href{https://jira.lsstcorp.org/browse/LVV-5641}{LVV-5641} &
Leanne Guy & Not Covered &
\begin{tabular}{c}
\end{tabular}
\\
\hline
\end{longtable}

\textbf{Verification Element Description:} \\
Undefined

{\footnotesize
\begin{longtable}{p{2.5cm}p{13.5cm}}
\hline
\multicolumn{2}{c}{\textbf{Requirement Details}}\\ \hline
Requirement ID & DM-TS-CON-ICD-0011 \\ \cdashline{1-2}
Requirement Description &
\begin{minipage}[]{13cm}
\textbf{Specification}: The WCS solution published by Data Management
shall include the following items: equinox (double, currently 2000.0),
system (string, currently `FK5'), unit (string, currently `deg'), and
then, for each sensor, reference pixel x/y coordinates (two doubles),
reference pixel RA/dec coordinates (two doubles), and rotation and scale
matrix (four doubles).
\end{minipage}
\\ \cdashline{1-2}
Requirement Discussion &
\begin{minipage}[]{13cm}
\textbf{Discussion}: DM does not calculate the WCS for the wavefront
sensors. The idea is then to use the offset and rotation information
from the WCS to define the exact location of the donuts to use. The
distortion will be measured carefully during commissioning. The WCS
solution is also required for the Calibration mode using the science
detectors as wavefront sensors.
\end{minipage}
\\ \cdashline{1-2}
Requirement Priority &  \\ \cdashline{1-2}
Upper Level Requirement &
\begin{tabular}{cl}
\end{tabular}
\\ \hline
\end{longtable}
}


  
 \newpage 
\subsection{[LVV-5646] DM-TS-CON-ICD-0002-V-01: Timing\_DM\_1 }\label{lvv-5646}

\begin{longtable}{cccc}
\hline
\textbf{Jira Link} & \textbf{Assignee} & \textbf{Status} & \textbf{Test Cases}\\ \hline
\href{https://jira.lsstcorp.org/browse/LVV-5646}{LVV-5646} &
Leanne Guy & Not Covered &
\begin{tabular}{c}
\end{tabular}
\\
\hline
\end{longtable}

\textbf{Verification Element Description:} \\
Undefined

{\footnotesize
\begin{longtable}{p{2.5cm}p{13.5cm}}
\hline
\multicolumn{2}{c}{\textbf{Requirement Details}}\\ \hline
Requirement ID & DM-TS-CON-ICD-0002 \\ \cdashline{1-2}
Requirement Description &
\begin{minipage}[]{13cm}
\textbf{Specification:} Data Management shall provide, for each
exposure, a calculation of the WCS for each sensor including the
wavefront sensors and guider sensors. The solution shall be published as
telemetry within time \textbf{wcsSolutionFeedbackTime} of the close of
data acquisition for the visit.
\end{minipage}
\\ \cdashline{1-2}
Requirement Parameters & \textbf{wcsSolutionFeedbackTime = 60{{[}second{]}}} Time following the
conclusion of readout of an exposure within which DM must provide a WCS
solution for each sensor. \\ \cdashline{1-2}
Requirement Discussion &
\begin{minipage}[]{13cm}
\textbf{Discussion:} The T\&S~and commissioning teams express the need
to know about the WCS coordinates on a 60s timeframe to decrease the
time overhead of the wait during Active Optics System
(AOS)~applications.\\
Note that this is not in the baseline for ComCam. There is no alert
production so that's not in the baseline and potentially has some cost
and schedule impacts.\\
\hspace*{0.333em}
\end{minipage}
\\ \cdashline{1-2}
Requirement Priority &  \\ \cdashline{1-2}
Upper Level Requirement &
\begin{tabular}{cl}
\end{tabular}
\\ \hline
\end{longtable}
}


  
 \newpage 
\subsection{[LVV-5647] DM-TS-CON-ICD-0002-V-02: Timing\_DM\_2 }\label{lvv-5647}

\begin{longtable}{cccc}
\hline
\textbf{Jira Link} & \textbf{Assignee} & \textbf{Status} & \textbf{Test Cases}\\ \hline
\href{https://jira.lsstcorp.org/browse/LVV-5647}{LVV-5647} &
Leanne Guy & Not Covered &
\begin{tabular}{c}
\end{tabular}
\\
\hline
\end{longtable}

\textbf{Verification Element Description:} \\
Undefined

{\footnotesize
\begin{longtable}{p{2.5cm}p{13.5cm}}
\hline
\multicolumn{2}{c}{\textbf{Requirement Details}}\\ \hline
Requirement ID & DM-TS-CON-ICD-0002 \\ \cdashline{1-2}
Requirement Description &
\begin{minipage}[]{13cm}
\textbf{Specification:} Data Management shall provide, for each
exposure, a calculation of the WCS for each sensor including the
wavefront sensors and guider sensors. The solution shall be published as
telemetry within time \textbf{wcsSolutionFeedbackTime} of the close of
data acquisition for the visit.
\end{minipage}
\\ \cdashline{1-2}
Requirement Parameters & \textbf{wcsSolutionFeedbackTime = 60{{[}second{]}}} Time following the
conclusion of readout of an exposure within which DM must provide a WCS
solution for each sensor. \\ \cdashline{1-2}
Requirement Discussion &
\begin{minipage}[]{13cm}
\textbf{Discussion:} The T\&S~and commissioning teams express the need
to know about the WCS coordinates on a 60s timeframe to decrease the
time overhead of the wait during Active Optics System
(AOS)~applications.\\
Note that this is not in the baseline for ComCam. There is no alert
production so that's not in the baseline and potentially has some cost
and schedule impacts.\\
\hspace*{0.333em}
\end{minipage}
\\ \cdashline{1-2}
Requirement Priority &  \\ \cdashline{1-2}
Upper Level Requirement &
\begin{tabular}{cl}
\end{tabular}
\\ \hline
\end{longtable}
}


  
 \newpage 
\subsection{[LVV-9741] DMS-REQ-0030-V-02: Minimum astrometric standards per CCD }\label{lvv-9741}

\begin{longtable}{cccc}
\hline
\textbf{Jira Link} & \textbf{Assignee} & \textbf{Status} & \textbf{Test Cases}\\ \hline
\href{https://jira.lsstcorp.org/browse/LVV-9741}{LVV-9741} &
Leanne Guy & Not Covered &
\begin{tabular}{c}
LVV-T1240 \\
\end{tabular}
\\
\hline
\end{longtable}

\textbf{Verification Element Description:} \\
Verify that the minimum number of astrometric standards available per
CCD for determining the WCS is at least \textbf{astrometricMinStandards
= 5.}

Associated element
(\href{https://jira.lsstcorp.org/browse/LVV-13}{LVV-13}) satisfies the
constraint on absolute accuracy of the WCS.

{\footnotesize
\begin{longtable}{p{2.5cm}p{13.5cm}}
\hline
\multicolumn{2}{c}{\textbf{Requirement Details}}\\ \hline
Requirement ID & DMS-REQ-0030 \\ \cdashline{1-2}
Requirement Description &
\begin{minipage}[]{13cm}
\textbf{Specification:} The DMS shall generate and persist a WCS for
each visit image. The absolute accuracy of the WCS shall be at least
\textbf{astrometricAccuracy} in all areas of the image, provided that
there are at least \textbf{astrometricMinStandards} astrometric
standards available in each CCD.
\end{minipage}
\\ \cdashline{1-2}
Requirement Parameters & {[}\textbf{astrometricAccuracy = 50{{[}milliarcsecond{]}}} Absolute
accuracy of the WCS across the focal plane (approximately one-quarter of
a pixel)., \textbf{astrometricMinStandards = 5{{[}integer{]}}} Minimum
number of astrometric standards per CCD.{]} \\ \cdashline{1-2}
Requirement Discussion &
\begin{minipage}[]{13cm}
\textbf{Discussion:} The World Coordinate System for visits will be
expressed in terms of a FITS Standard representation, which provides for
named metadata to be interpreted as coefficients of one of a finite set
of coordinate transformations.
\end{minipage}
\\ \cdashline{1-2}
Requirement Priority & 1a \\ \cdashline{1-2}
Upper Level Requirement &
\begin{tabular}{cl}
DMS-REQ-0090 & Generate Alerts \\
DMS-REQ-0104 & Produce Co-Added Exposures \\
OSS-REQ-0149 & Level 1 Catalog Precision \\
OSS-REQ-0162 & Level 2 Catalog Accuracy \\
\end{tabular}
\\ \hline
\end{longtable}
}


\subsubsection{Test Cases Summary}
\begin{longtable}{p{3cm}p{2.5cm}p{2.5cm}p{3cm}p{4cm}}
\toprule
\href{https://jira.lsstcorp.org/secure/Tests.jspa\#/testCase/LVV-T1240}{LVV-T1240} & \multicolumn{4}{p{12cm}}{ Verify implementation of minimum astrometric standards per CCD } \\ \hline
\textbf{Owner} & \textbf{Status} & \textbf{Version} & \textbf{Critical Event} & \textbf{Verification Type} \\ \hline
Jim Bosch & Approved & 1 & false & Test \\ \hline
\end{longtable}
{\scriptsize
\textbf{Objective:}\\
Verify that each CCD in a processed dataset had its astrometric solution
determined by at least~\textbf{astrometricMinStandards = 5~}astrometric
standards.
}
  
 \newpage 
\subsection{[LVV-9743] DMS-REQ-0271-V-03: Radius considered nearby }\label{lvv-9743}

\begin{longtable}{cccc}
\hline
\textbf{Jira Link} & \textbf{Assignee} & \textbf{Status} & \textbf{Test Cases}\\ \hline
\href{https://jira.lsstcorp.org/browse/LVV-9743}{LVV-9743} &
Leanne Guy & Not Covered &
\begin{tabular}{c}
\end{tabular}
\\
\hline
\end{longtable}

\textbf{Verification Element Description:} \\
Verify that the radius used to determine coincidence between an Object
and a DIASource is \textbf{diaNearbyObjRadius = 60 arcseconds.}

Associated element
(\href{https://jira.lsstcorp.org/browse/LVV-9742}{LVV-9742}) satisfies
the maximum number of stars that can be associated with a DIASource.

Associated element
(\href{https://jira.lsstcorp.org/browse/LVV-102}{LVV-102}) satisfies the
maximum number of galaxies that can be associated with a DIASource.

Â~

{\footnotesize
\begin{longtable}{p{2.5cm}p{13.5cm}}
\hline
\multicolumn{2}{c}{\textbf{Requirement Details}}\\ \hline
Requirement ID & DMS-REQ-0271 \\ \cdashline{1-2}
Requirement Description &
\begin{minipage}[]{13cm}
\textbf{Specification:} The DMS shall construct a catalog of all
astrophysical objects identified through difference image analysis
(DIAObjects). The DIAObject entries shall include metadata attributes
including at least: a unique identifier; the identifiers of the
\textbf{diaNearbyObjMaxStar} nearest stars and
\textbf{diaNearbyObjMaxGalaxy} nearest galaxies in the Object catalog
lying within \textbf{diaNearbyObjRadius}, the probability that the
DIAObject is the same as the nearby Object; and a set of DIAObject
properties.
\end{minipage}
\\ \cdashline{1-2}
Requirement Parameters & {[}\textbf{diaNearbyObjMaxGalaxy = 3{{[}integer{]}}} Maximum number of
nearby galaxies that can be associated with a DIASource.,
\textbf{diaNearbyObjRadius = 60{{[}arcsecond{]}}} Radius within which an
Object is considered to be near, and possibly coincident with, the
DIASource., \textbf{diaNearbyObjMaxStar = 3{{[}integer{]}}} Maximum
number of stars that can be associated with a DIASource.{]} \\ \cdashline{1-2}
Requirement Priority & 1b \\ \cdashline{1-2}
Upper Level Requirement &
\begin{tabular}{cl}
OSS-REQ-0130 & Catalogs (Level 1) \\
\end{tabular}
\\ \hline
\end{longtable}
}


  
 \newpage 
\subsection{[LVV-9745] DMS-REQ-0131-V-02: Max number of calibs to be processed }\label{lvv-9745}

\begin{longtable}{cccc}
\hline
\textbf{Jira Link} & \textbf{Assignee} & \textbf{Status} & \textbf{Test Cases}\\ \hline
\href{https://jira.lsstcorp.org/browse/LVV-9745}{LVV-9745} &
Leanne Guy & Not Covered &
\begin{tabular}{c}
LVV-T1277 \\
\end{tabular}
\\
\hline
\end{longtable}

\textbf{Verification Element Description:} \\
Verify that \textbf{nCalExpProc = 25}Â~calibration exposures can be
processed simultaneously and made available within the allotted time.

Associated element
(\href{https://jira.lsstcorp.org/browse/LVV-58}{LVV-58}) satisfies the
time allowed for processing calibration exposures.

{\footnotesize
\begin{longtable}{p{2.5cm}p{13.5cm}}
\hline
\multicolumn{2}{c}{\textbf{Requirement Details}}\\ \hline
Requirement ID & DMS-REQ-0131 \\ \cdashline{1-2}
Requirement Description &
\begin{minipage}[]{13cm}
\textbf{Specification:} Calibration products from a group of up to
\textbf{nCalExpProc} related exposures that should be processed
together, shall be available from the DMS image archive within
\textbf{calProcTime} of the end of the acquisition of images/data for
that group.
\end{minipage}
\\ \cdashline{1-2}
Requirement Parameters & {[}\textbf{nCalExpProc = 25{{[}integer{]}}} Maximum number of
calibration exposures that can be processed together within time
calProcTime., \textbf{calProcTime = 1200{{[}second{]}}} Time allowed to
process nCalExpProc calibration exposures and have them available within
the DMS.{]} \\ \cdashline{1-2}
Requirement Discussion &
\begin{minipage}[]{13cm}
\textbf{Discussion:} The motivation here is that calibration images will
be needed at least 1 hour prior to the start of observing and this
requirement allows the calibration observations to be planned
accordingly.
\end{minipage}
\\ \cdashline{1-2}
Requirement Priority & 2 \\ \cdashline{1-2}
Upper Level Requirement &
\begin{tabular}{cl}
OSS-REQ-0046 & Calibration \\
OSS-REQ-0021 & Base Site \\
OSS-REQ-0194 & Calibration Exposures Per Day \\
DMS-REQ-0130 & Calibration Data Products \\
\end{tabular}
\\ \hline
\end{longtable}
}


\subsubsection{Test Cases Summary}
\begin{longtable}{p{3cm}p{2.5cm}p{2.5cm}p{3cm}p{4cm}}
\toprule
\href{https://jira.lsstcorp.org/secure/Tests.jspa\#/testCase/LVV-T1277}{LVV-T1277} & \multicolumn{4}{p{12cm}}{ Verify processing of maximum number of calibration exposures } \\ \hline
\textbf{Owner} & \textbf{Status} & \textbf{Version} & \textbf{Critical Event} & \textbf{Verification Type} \\ \hline
Kian-Tat Lim & Draft & 1 & false & Test \\ \hline
\end{longtable}
{\scriptsize
\textbf{Objective:}\\
Verify that as many as \textbf{nCalExpProc = 25} calibration exposures
can be processed together within time calProcTime.
}
  
 \newpage 
\subsection{[LVV-9746] DMS-REQ-0287-V-02: Max time from acquisition to L1 data release }\label{lvv-9746}

\begin{longtable}{cccc}
\hline
\textbf{Jira Link} & \textbf{Assignee} & \textbf{Status} & \textbf{Test Cases}\\ \hline
\href{https://jira.lsstcorp.org/browse/LVV-9746}{LVV-9746} &
Leanne Guy & Not Covered &
\begin{tabular}{c}
\end{tabular}
\\
\hline
\end{longtable}

\textbf{Verification Element Description:} \\
Verify that L1 associated data products are available within L1PublicT =
24 hours.

Associated element
(\href{https://jira.lsstcorp.org/browse/LVV-9747}{LVV-9747}) satisfies
the lifetime of cached L1 data products.

Associated element
(\href{https://jira.lsstcorp.org/browse/LVV-118}{LVV-118}) satisfies the
maximum look-back time for precovery measurements.

{\footnotesize
\begin{longtable}{p{2.5cm}p{13.5cm}}
\hline
\multicolumn{2}{c}{\textbf{Requirement Details}}\\ \hline
Requirement ID & DMS-REQ-0287 \\ \cdashline{1-2}
Requirement Description &
\begin{minipage}[]{13cm}
\textbf{Specification:} For all DIASources not associated with either
DIAObjects or SSObjects, the DMS shall perform forced photometry at the
location of the new source (precovery) on all Difference Exposures
obtained in the prior \textbf{precoveryWindow}, and make the results
publicly available within \textbf{L1PublicT}.
\end{minipage}
\\ \cdashline{1-2}
Requirement Parameters & {[}\textbf{precoveryWindow = 30{{[}day{]}}} Maximum look-back time for
precovery measurments on prior Exposures., \textbf{l1CacheLifetime =
30{{[}day{]}}} Lifetime in the cache of un-archived Level-1 data
products., \textbf{L1PublicT = 24{{[}hour{]}}} Maximum time from the
acquisition of science data to the release of associated Level 1 Data
Products (except alerts){]} \\ \cdashline{1-2}
Requirement Discussion &
\begin{minipage}[]{13cm}
\textbf{Discussion:} The \textbf{precoveryWindow} is intended to satisfy
the most common scientific use cases (e.g., Supernovae), without placing
an undue burden on the processing infrastructure. For reasons of
practicality and efficiency, \textbf{precoveryWindow} \textless{}=
l*1CacheLifetime*.
\end{minipage}
\\ \cdashline{1-2}
Requirement Priority & 1b \\ \cdashline{1-2}
Upper Level Requirement &
\begin{tabular}{cl}
OSS-REQ-0130 & Catalogs (Level 1) \\
\end{tabular}
\\ \hline
\end{longtable}
}


  
 \newpage 
\subsection{[LVV-9747] DMS-REQ-0287-V-03: Lifetime of archived L1 data products }\label{lvv-9747}

\begin{longtable}{cccc}
\hline
\textbf{Jira Link} & \textbf{Assignee} & \textbf{Status} & \textbf{Test Cases}\\ \hline
\href{https://jira.lsstcorp.org/browse/LVV-9747}{LVV-9747} &
Leanne Guy & Not Covered &
\begin{tabular}{c}
\end{tabular}
\\
\hline
\end{longtable}

\textbf{Verification Element Description:} \\
Verify storage of unarchived Level-1 data products for at least
\textbf{l1CacheLifetime = 30 days.}

Associated element
(\href{https://jira.lsstcorp.org/browse/LVV-9746}{LVV-9746}) satisfies
the time in which L1 data products shall be publicly released.

Associated element
(\href{https://jira.lsstcorp.org/browse/LVV-118}{LVV-118}) satisfies the
maximum look-back time for precovery measurements.

{\footnotesize
\begin{longtable}{p{2.5cm}p{13.5cm}}
\hline
\multicolumn{2}{c}{\textbf{Requirement Details}}\\ \hline
Requirement ID & DMS-REQ-0287 \\ \cdashline{1-2}
Requirement Description &
\begin{minipage}[]{13cm}
\textbf{Specification:} For all DIASources not associated with either
DIAObjects or SSObjects, the DMS shall perform forced photometry at the
location of the new source (precovery) on all Difference Exposures
obtained in the prior \textbf{precoveryWindow}, and make the results
publicly available within \textbf{L1PublicT}.
\end{minipage}
\\ \cdashline{1-2}
Requirement Parameters & {[}\textbf{precoveryWindow = 30{{[}day{]}}} Maximum look-back time for
precovery measurments on prior Exposures., \textbf{l1CacheLifetime =
30{{[}day{]}}} Lifetime in the cache of un-archived Level-1 data
products., \textbf{L1PublicT = 24{{[}hour{]}}} Maximum time from the
acquisition of science data to the release of associated Level 1 Data
Products (except alerts){]} \\ \cdashline{1-2}
Requirement Discussion &
\begin{minipage}[]{13cm}
\textbf{Discussion:} The \textbf{precoveryWindow} is intended to satisfy
the most common scientific use cases (e.g., Supernovae), without placing
an undue burden on the processing infrastructure. For reasons of
practicality and efficiency, \textbf{precoveryWindow} \textless{}=
l*1CacheLifetime*.
\end{minipage}
\\ \cdashline{1-2}
Requirement Priority & 1b \\ \cdashline{1-2}
Upper Level Requirement &
\begin{tabular}{cl}
OSS-REQ-0130 & Catalogs (Level 1) \\
\end{tabular}
\\ \hline
\end{longtable}
}


  
 \newpage 
\subsection{[LVV-9751] DMS-REQ-0359-V-02: Max fraction of sensors with excess unusable pixels }\label{lvv-9751}

\begin{longtable}{cccc}
\hline
\textbf{Jira Link} & \textbf{Assignee} & \textbf{Status} & \textbf{Test Cases}\\ \hline
\href{https://jira.lsstcorp.org/browse/LVV-9751}{LVV-9751} &
Leanne Guy & Not Covered &
\begin{tabular}{c}
LVV-T377 \\
LVV-T1847 \\
\end{tabular}
\\
\hline
\end{longtable}

\textbf{Verification Element Description:} \\
The maximum allowable fraction of sensors with \textbf{PixFrac
\textgreater{} 1} percent scientifically unusable pixels shall be
\textbf{SensorFraction = 15 percent.}

Associated element DMS-REQ-0359-V-01
(\href{https://jira.lsstcorp.org/browse/LVV-3401}{LVV-3401}) satisfies
the requirement on photometric repeatability in the u, z, and y-band
filters.

Associated element DMS-REQ-0359-V-03
(\href{https://jira.lsstcorp.org/browse/LVV-9752}{LVV-9752}) satisfies
the constraint on maximum fraction of outliers among non-saturated point
sources.

Associated element DMS-REQ-0359-V-04
(\href{https://jira.lsstcorp.org/browse/LVV-9753}{LVV-9753}) satisfies
the accuracy of zero-point for colors that use the u-band.

Associated element DMS-REQ-0359-V-05
(\href{https://jira.lsstcorp.org/browse/LVV-9754}{LVV-9754}) satisfies
the repeatability outlier limit in g, r, and i-bands.

Associated element DMS-REQ-0359-V-06
(\href{https://jira.lsstcorp.org/browse/LVV-9755}{LVV-9755}) satisfies
the constraint on the accuracy of the transformation from internal to
physical photometric scales.

Associated element DMS-REQ-0359-V-07
(\href{https://jira.lsstcorp.org/browse/LVV-9756}{LVV-9756}) satisfies
the rms width of the internal photometric zero-point in u-band.

Associated element DMS-REQ-0359-V-08
(\href{https://jira.lsstcorp.org/browse/LVV-9757}{LVV-9757}) satisfies
the maximum local significance of imperfect crosstalk corrections.

Associated element DMS-REQ-0359-V-09
(\href{https://jira.lsstcorp.org/browse/LVV-9758}{LVV-9758}) satisfies
the repeatability outlier limit in u, z, and y-bands.

Associated element DMS-REQ-0359-V-10
(\href{https://jira.lsstcorp.org/browse/LVV-9759}{LVV-9759}) satisfies
the rms photometric repeatability in g, r, and i-bands.

Associated element DMS-REQ-0359-V-11
(\href{https://jira.lsstcorp.org/browse/LVV-9760}{LVV-9760}) satisfies
the fraction of zero-point errors that can exceed the outlier limit.

Associated element DMS-REQ-0359-V-12
(\href{https://jira.lsstcorp.org/browse/LVV-9761}{LVV-9761}) satisfies
the maximum fraction of unusable pixels per sensor.

Associated element DMS-REQ-0359-V-13
(\href{https://jira.lsstcorp.org/browse/LVV-9762}{LVV-9762}) satisfies
the maximum allowable precision in the sky brightness determination.

Associated element DMS-REQ-0359-V-14
(\href{https://jira.lsstcorp.org/browse/LVV-9763}{LVV-9763}) satisfies
the rms width of the internal photometric zero-point in g, r, i, z, and
y-bands.

Associated element DMS-REQ-0359-V-15
(\href{https://jira.lsstcorp.org/browse/LVV-9764}{LVV-9764}) satisfies
the percentage of the image area affected by ghosts that exceed the
threshold.

Associated element DMS-REQ-0359-V-16
(\href{https://jira.lsstcorp.org/browse/LVV-9765}{LVV-9765}) satisfies
the accuracy of zero-point for colors that do not include the u-band.

Associated element DMS-REQ-0359-V-17
(\href{https://jira.lsstcorp.org/browse/LVV-9766}{LVV-9766}) satisfies
the maximum RMS of the ratio of the flux measurement error between
resolved/unresolved sources.

{\footnotesize
\begin{longtable}{p{2.5cm}p{13.5cm}}
\hline
\multicolumn{2}{c}{\textbf{Requirement Details}}\\ \hline
Requirement ID & DMS-REQ-0359 \\ \cdashline{1-2}
Requirement Description &
\begin{minipage}[]{13cm}
\textbf{Specification:} The DMS shall include software to enable the
calculation of the photometric performance metrics defined in
OSS-REQ-0387.
\end{minipage}
\\ \cdashline{1-2}
Requirement Parameters & {[}\textbf{GhostAF = 1{{[}percent{]}}} Percentage of image area that can
have ghosts with surface brightness gradient amplitude of more than 1/3
of the sky noise over 1 arcsec., \textbf{PF1 = 10{{[}percent{]}}} The
maximum fraction of isolated non-saturated point source measurements
exceeding the outlier limit., \textbf{PA1gri = 5{{[}millimagnitude{]}}}
The RMS photometric repeatability of bright non-saturated unresolved
point sources in the g, r, and i filters., \textbf{PA3 =
10{{[}millimagnitude{]}}} RMS width of internal photometric zero-point
(precision of system uniformity across the sky) for all bands except
u-band., \textbf{PA4 = 15{{[}millimagnitude{]}}} The zero point error
outlier limit., \textbf{PF2 = 10{{[}percent{]}}} Fraction of zeropoint
errors that can exceed the zero point error outlier limit.,
\textbf{PixFrac = 1{{[}percent{]}}} The maximum fraction of pixels
scientifically unusable per sensor out of the total allowable fraction
of sensors meeting this performance., \textbf{PA1uzy =
7.5{{[}millimagnitude{]}}} The RMS photometric repeatability of bright
non-saturated unresolved point sources in the u, z, and y filters.,
\textbf{PA6 = 10{{[}millimagnitude{]}}} Accuracy of the transformation
of the internal LSST photometry to a physical scale (e.g. AB
magnitudes)., \textbf{Xtalk = 3{{[}sigma{]}}} The maximum local
significance integrated over the PSF of imperfect crosstalk
corrections., \textbf{PA5u = 10{{[}millimagnitude{]}}} Accuracy of
absolute band-to-band color zero-point for colors constructed using the
u-band., \textbf{ResSource = 2{{[}unitless{]}}} Maximum RMS of the ratio
of the error in integrated flux measurement between bright, isolated,
resolved sources less than 10 arcsec in diameter and bright, isolated
unresolved point sources., \textbf{PA2uzy = 22.5{{[}millimagnitude{]}}}
Repeatability outlier limit for isolated bright non-saturated point
sources in the u, z, and y filters., \textbf{SensorFraction =
15{{[}percent{]}}} The maximum allowable fraction of sensors with
PixFrac scientifically unusable pixels., \textbf{PA3u =
20{{[}millimagnitude{]}}} RMS width of internal photometric zero-point
(precision of system uniformity across the sky) in the u-band.,
\textbf{PA2gri = 15{{[}millimagnitude{]}}} Repeatability outlier limit
for isolated bright non-saturated point sources in the g, r, and i
filters., \textbf{SBPrec = 1{{[}percent{]}}} The maximum error in the
precision of the sky brightness determination., \textbf{PA5 =
5{{[}millimagnitude{]}}} Accuracy of absolute band-to-band color
zero-point for all colors constructed from any filter pair, excluding
the u-band.{]} \\ \cdashline{1-2}
Requirement Discussion &
\begin{minipage}[]{13cm}
\textbf{Discussion:} The relevant metrics are listed in the table
photometricPerformance below. The values in the tables are the target
values for LSST but are not verified as part of this requirement.
\end{minipage}
\\ \cdashline{1-2}
Requirement Priority & 1a \\ \cdashline{1-2}
Upper Level Requirement &
\begin{tabular}{cl}
OSS-REQ-0387 & Photometric Performance \\
\end{tabular}
\\ \hline
\end{longtable}
}


\subsubsection{Test Cases Summary}
\begin{longtable}{p{3cm}p{2.5cm}p{2.5cm}p{3cm}p{4cm}}
\toprule
\href{https://jira.lsstcorp.org/secure/Tests.jspa\#/testCase/LVV-T377}{LVV-T377} & \multicolumn{4}{p{12cm}}{ Verify Calculation of Photometric Performance Metrics } \\ \hline
\textbf{Owner} & \textbf{Status} & \textbf{Version} & \textbf{Critical Event} & \textbf{Verification Type} \\ \hline
Leanne Guy & Approved & 1 & false & Test \\ \hline
\end{longtable}
{\scriptsize
\textbf{Objective:}\\
Verify that the DMS system provides software to calculate photometric
performance metrics, and that the algorithms are properly calculating
the desired quantities. Note that because the DMS requirement is that
the software shall be provided (and not on the actual measured values of
the metrics), we verify all of the requirements via a single test case.
}
\begin{longtable}{p{3cm}p{2.5cm}p{2.5cm}p{3cm}p{4cm}}
\toprule
\href{https://jira.lsstcorp.org/secure/Tests.jspa\#/testCase/LVV-T1847}{LVV-T1847} & \multicolumn{4}{p{12cm}}{ Verify calculation of sensor fraction with unusable pixels } \\ \hline
\textbf{Owner} & \textbf{Status} & \textbf{Version} & \textbf{Critical Event} & \textbf{Verification Type} \\ \hline
Jeffrey Carlin & Draft & 1 & false & Test \\ \hline
\end{longtable}
{\scriptsize
\textbf{Objective:}\\
Verify that the DM system provides software to assess whether the
maximum allowable fraction of sensors with \textbf{PixFrac
\textgreater{} 1} percent scientifically unusable pixels is less
than~\textbf{SensorFraction = 15 percent.}
}
  
 \newpage 
\subsection{[LVV-9752] DMS-REQ-0359-V-03: Max fraction of outliers among non-saturated sources }\label{lvv-9752}

\begin{longtable}{cccc}
\hline
\textbf{Jira Link} & \textbf{Assignee} & \textbf{Status} & \textbf{Test Cases}\\ \hline
\href{https://jira.lsstcorp.org/browse/LVV-9752}{LVV-9752} &
Leanne Guy & Not Covered &
\begin{tabular}{c}
LVV-T1758 \\
LVV-T1759 \\
\end{tabular}
\\
\hline
\end{longtable}

\textbf{Verification Element Description:} \\
The maximum fraction of isolated non-saturated point source measurements
exceeding the outlier limit shall be less than \textbf{PF1 = 10
percent}.

Associated element DMS-REQ-0359-V-01
(\href{https://jira.lsstcorp.org/browse/LVV-3401}{LVV-3401}) satisfies
the requirement on photometric repeatability in the u, z, and y-band
filters.

Associated element DMS-REQ-0359-V-02
(\href{https://jira.lsstcorp.org/browse/LVV-9751}{LVV-9751}) satisfies
the requirement on the maximum fraction of sensors with scientifically
unusable pixels.

Associated element DMS-REQ-0359-V-04
(\href{https://jira.lsstcorp.org/browse/LVV-9753}{LVV-9753}) satisfies
the accuracy of zero-point for colors that use the u-band.

Associated element DMS-REQ-0359-V-05
(\href{https://jira.lsstcorp.org/browse/LVV-9754}{LVV-9754}) satisfies
the repeatability outlier limit in g, r, and i-bands.

Associated element DMS-REQ-0359-V-06
(\href{https://jira.lsstcorp.org/browse/LVV-9755}{LVV-9755}) satisfies
the constraint on the accuracy of the transformation from internal to
physical photometric scales.

Associated element DMS-REQ-0359-V-07
(\href{https://jira.lsstcorp.org/browse/LVV-9756}{LVV-9756}) satisfies
the rms width of the internal photometric zero-point in u-band.

Associated element DMS-REQ-0359-V-08
(\href{https://jira.lsstcorp.org/browse/LVV-9757}{LVV-9757}) satisfies
the maximum local significance of imperfect crosstalk corrections.

Associated element DMS-REQ-0359-V-09
(\href{https://jira.lsstcorp.org/browse/LVV-9758}{LVV-9758}) satisfies
the repeatability outlier limit in u, z, and y-bands.

Associated element DMS-REQ-0359-V-10
(\href{https://jira.lsstcorp.org/browse/LVV-9759}{LVV-9759}) satisfies
the rms photometric repeatability in g, r, and i-bands.

Associated element DMS-REQ-0359-V-11
(\href{https://jira.lsstcorp.org/browse/LVV-9760}{LVV-9760}) satisfies
the fraction of zero-point errors that can exceed the outlier limit.

Associated element DMS-REQ-0359-V-12
(\href{https://jira.lsstcorp.org/browse/LVV-9761}{LVV-9761}) satisfies
the maximum fraction of unusable pixels per sensor.

Associated element DMS-REQ-0359-V-13
(\href{https://jira.lsstcorp.org/browse/LVV-9762}{LVV-9762}) satisfies
the maximum allowable precision in the sky brightness determination.

Associated element DMS-REQ-0359-V-14
(\href{https://jira.lsstcorp.org/browse/LVV-9763}{LVV-9763}) satisfies
the rms width of the internal photometric zero-point in g, r, i, z, and
y-bands.

Associated element DMS-REQ-0359-V-15
(\href{https://jira.lsstcorp.org/browse/LVV-9764}{LVV-9764}) satisfies
the percentage of the image area affected by ghosts that exceed the
threshold.

Associated element DMS-REQ-0359-V-16
(\href{https://jira.lsstcorp.org/browse/LVV-9765}{LVV-9765}) satisfies
the accuracy of zero-point for colors that do not include the u-band.

Associated element DMS-REQ-0359-V-17
(\href{https://jira.lsstcorp.org/browse/LVV-9766}{LVV-9766}) satisfies
the maximum RMS of the ratio of the flux measurement error between
resolved/unresolved sources.

{\footnotesize
\begin{longtable}{p{2.5cm}p{13.5cm}}
\hline
\multicolumn{2}{c}{\textbf{Requirement Details}}\\ \hline
Requirement ID & DMS-REQ-0359 \\ \cdashline{1-2}
Requirement Description &
\begin{minipage}[]{13cm}
\textbf{Specification:} The DMS shall include software to enable the
calculation of the photometric performance metrics defined in
OSS-REQ-0387.
\end{minipage}
\\ \cdashline{1-2}
Requirement Parameters & {[}\textbf{GhostAF = 1{{[}percent{]}}} Percentage of image area that can
have ghosts with surface brightness gradient amplitude of more than 1/3
of the sky noise over 1 arcsec., \textbf{PF1 = 10{{[}percent{]}}} The
maximum fraction of isolated non-saturated point source measurements
exceeding the outlier limit., \textbf{PA1gri = 5{{[}millimagnitude{]}}}
The RMS photometric repeatability of bright non-saturated unresolved
point sources in the g, r, and i filters., \textbf{PA3 =
10{{[}millimagnitude{]}}} RMS width of internal photometric zero-point
(precision of system uniformity across the sky) for all bands except
u-band., \textbf{PA4 = 15{{[}millimagnitude{]}}} The zero point error
outlier limit., \textbf{PF2 = 10{{[}percent{]}}} Fraction of zeropoint
errors that can exceed the zero point error outlier limit.,
\textbf{PixFrac = 1{{[}percent{]}}} The maximum fraction of pixels
scientifically unusable per sensor out of the total allowable fraction
of sensors meeting this performance., \textbf{PA1uzy =
7.5{{[}millimagnitude{]}}} The RMS photometric repeatability of bright
non-saturated unresolved point sources in the u, z, and y filters.,
\textbf{PA6 = 10{{[}millimagnitude{]}}} Accuracy of the transformation
of the internal LSST photometry to a physical scale (e.g. AB
magnitudes)., \textbf{Xtalk = 3{{[}sigma{]}}} The maximum local
significance integrated over the PSF of imperfect crosstalk
corrections., \textbf{PA5u = 10{{[}millimagnitude{]}}} Accuracy of
absolute band-to-band color zero-point for colors constructed using the
u-band., \textbf{ResSource = 2{{[}unitless{]}}} Maximum RMS of the ratio
of the error in integrated flux measurement between bright, isolated,
resolved sources less than 10 arcsec in diameter and bright, isolated
unresolved point sources., \textbf{PA2uzy = 22.5{{[}millimagnitude{]}}}
Repeatability outlier limit for isolated bright non-saturated point
sources in the u, z, and y filters., \textbf{SensorFraction =
15{{[}percent{]}}} The maximum allowable fraction of sensors with
PixFrac scientifically unusable pixels., \textbf{PA3u =
20{{[}millimagnitude{]}}} RMS width of internal photometric zero-point
(precision of system uniformity across the sky) in the u-band.,
\textbf{PA2gri = 15{{[}millimagnitude{]}}} Repeatability outlier limit
for isolated bright non-saturated point sources in the g, r, and i
filters., \textbf{SBPrec = 1{{[}percent{]}}} The maximum error in the
precision of the sky brightness determination., \textbf{PA5 =
5{{[}millimagnitude{]}}} Accuracy of absolute band-to-band color
zero-point for all colors constructed from any filter pair, excluding
the u-band.{]} \\ \cdashline{1-2}
Requirement Discussion &
\begin{minipage}[]{13cm}
\textbf{Discussion:} The relevant metrics are listed in the table
photometricPerformance below. The values in the tables are the target
values for LSST but are not verified as part of this requirement.
\end{minipage}
\\ \cdashline{1-2}
Requirement Priority & 1a \\ \cdashline{1-2}
Upper Level Requirement &
\begin{tabular}{cl}
OSS-REQ-0387 & Photometric Performance \\
\end{tabular}
\\ \hline
\end{longtable}
}


\subsubsection{Test Cases Summary}
\begin{longtable}{p{3cm}p{2.5cm}p{2.5cm}p{3cm}p{4cm}}
\toprule
\href{https://jira.lsstcorp.org/secure/Tests.jspa\#/testCase/LVV-T1758}{LVV-T1758} & \multicolumn{4}{p{12cm}}{ Verify calculation of photometric outliers in uzy bands } \\ \hline
\textbf{Owner} & \textbf{Status} & \textbf{Version} & \textbf{Critical Event} & \textbf{Verification Type} \\ \hline
Jeffrey Carlin & Approved & 1 & false & Test \\ \hline
\end{longtable}
{\scriptsize
\textbf{Objective:}\\
Verify that the DM system has provided the code to calculate the
photometric repeatability in the u, z, and y filters, and assess whether
it meets the requirement that no more than \textbf{PF1 =
10{[}percent{]}} of the repeatability outliers exceed the outlier limit
of \textbf{PA2uzy = 22.5 millimagnitudes}.~
}
\begin{longtable}{p{3cm}p{2.5cm}p{2.5cm}p{3cm}p{4cm}}
\toprule
\href{https://jira.lsstcorp.org/secure/Tests.jspa\#/testCase/LVV-T1759}{LVV-T1759} & \multicolumn{4}{p{12cm}}{ Verify calculation of photometric outliers in gri bands } \\ \hline
\textbf{Owner} & \textbf{Status} & \textbf{Version} & \textbf{Critical Event} & \textbf{Verification Type} \\ \hline
Jeffrey Carlin & Approved & 1 & false & Test \\ \hline
\end{longtable}
{\scriptsize
\textbf{Objective:}\\
Verify that the DM system has provided the code to calculate the
photometric repeatability in the g, r, and i filters, and assess whether
it meets the requirement that no more than \textbf{PF1 =
10{[}percent{]}} of the repeatability outliers exceed the outlier limit
of \textbf{PA2gri = 15 millimagnitudes}.
}
  
 \newpage 
\subsection{[LVV-9753] DMS-REQ-0359-V-04: Accuracy of zero point for colors with u-band }\label{lvv-9753}

\begin{longtable}{cccc}
\hline
\textbf{Jira Link} & \textbf{Assignee} & \textbf{Status} & \textbf{Test Cases}\\ \hline
\href{https://jira.lsstcorp.org/browse/LVV-9753}{LVV-9753} &
Leanne Guy & Not Covered &
\begin{tabular}{c}
LVV-T377 \\
LVV-T1846 \\
\end{tabular}
\\
\hline
\end{longtable}

\textbf{Verification Element Description:} \\
The accuracy of absolute band-to-band color zero-points for all colors
constructed from any filter pair, including the u-band shall be less
than \textbf{PA5u = 10 millimagnitudes}.

Associated element DMS-REQ-0359-V-01
(\href{https://jira.lsstcorp.org/browse/LVV-3401}{LVV-3401}) satisfies
the requirement on photometric repeatability in the u, z, and y-band
filters.

Associated element DMS-REQ-0359-V-02
(\href{https://jira.lsstcorp.org/browse/LVV-9751}{LVV-9751}) satisfies
the requirement on the maximum fraction of sensors with scientifically
unusable pixels.

Associated element DMS-REQ-0359-V-03
(\href{https://jira.lsstcorp.org/browse/LVV-9752}{LVV-9752}) satisfies
the constraint on maximum fraction of outliers among non-saturated point
sources.

Associated element DMS-REQ-0359-V-05
(\href{https://jira.lsstcorp.org/browse/LVV-9754}{LVV-9754}) satisfies
the repeatability outlier limit in g, r, and i-bands.

Associated element DMS-REQ-0359-V-06
(\href{https://jira.lsstcorp.org/browse/LVV-9755}{LVV-9755}) satisfies
the constraint on the accuracy of the transformation from internal to
physical photometric scales.

Associated element DMS-REQ-0359-V-07
(\href{https://jira.lsstcorp.org/browse/LVV-9756}{LVV-9756}) satisfies
the rms width of the internal photometric zero-point in u-band.

Associated element DMS-REQ-0359-V-08
(\href{https://jira.lsstcorp.org/browse/LVV-9757}{LVV-9757}) satisfies
the maximum local significance of imperfect crosstalk corrections.

Associated element DMS-REQ-0359-V-09
(\href{https://jira.lsstcorp.org/browse/LVV-9758}{LVV-9758}) satisfies
the repeatability outlier limit in u, z, and y-bands.

Associated element DMS-REQ-0359-V-10
(\href{https://jira.lsstcorp.org/browse/LVV-9759}{LVV-9759}) satisfies
the rms photometric repeatability in g, r, and i-bands.

Associated element DMS-REQ-0359-V-11
(\href{https://jira.lsstcorp.org/browse/LVV-9760}{LVV-9760}) satisfies
the fraction of zero-point errors that can exceed the outlier limit.

Associated element DMS-REQ-0359-V-12
(\href{https://jira.lsstcorp.org/browse/LVV-9761}{LVV-9761}) satisfies
the maximum fraction of unusable pixels per sensor.

Associated element DMS-REQ-0359-V-13
(\href{https://jira.lsstcorp.org/browse/LVV-9762}{LVV-9762}) satisfies
the maximum allowable precision in the sky brightness determination.

Associated element DMS-REQ-0359-V-14
(\href{https://jira.lsstcorp.org/browse/LVV-9763}{LVV-9763}) satisfies
the rms width of the internal photometric zero-point in g, r, i, z, and
y-bands.

Associated element DMS-REQ-0359-V-15
(\href{https://jira.lsstcorp.org/browse/LVV-9764}{LVV-9764}) satisfies
the percentage of the image area affected by ghosts that exceed the
threshold.

Associated element DMS-REQ-0359-V-16
(\href{https://jira.lsstcorp.org/browse/LVV-9765}{LVV-9765}) satisfies
the accuracy of zero-point for colors that do not include the u-band.

Associated element DMS-REQ-0359-V-17
(\href{https://jira.lsstcorp.org/browse/LVV-9766}{LVV-9766}) satisfies
the maximum RMS of the ratio of the flux measurement error between
resolved/unresolved sources.

{\footnotesize
\begin{longtable}{p{2.5cm}p{13.5cm}}
\hline
\multicolumn{2}{c}{\textbf{Requirement Details}}\\ \hline
Requirement ID & DMS-REQ-0359 \\ \cdashline{1-2}
Requirement Description &
\begin{minipage}[]{13cm}
\textbf{Specification:} The DMS shall include software to enable the
calculation of the photometric performance metrics defined in
OSS-REQ-0387.
\end{minipage}
\\ \cdashline{1-2}
Requirement Parameters & {[}\textbf{GhostAF = 1{{[}percent{]}}} Percentage of image area that can
have ghosts with surface brightness gradient amplitude of more than 1/3
of the sky noise over 1 arcsec., \textbf{PF1 = 10{{[}percent{]}}} The
maximum fraction of isolated non-saturated point source measurements
exceeding the outlier limit., \textbf{PA1gri = 5{{[}millimagnitude{]}}}
The RMS photometric repeatability of bright non-saturated unresolved
point sources in the g, r, and i filters., \textbf{PA3 =
10{{[}millimagnitude{]}}} RMS width of internal photometric zero-point
(precision of system uniformity across the sky) for all bands except
u-band., \textbf{PA4 = 15{{[}millimagnitude{]}}} The zero point error
outlier limit., \textbf{PF2 = 10{{[}percent{]}}} Fraction of zeropoint
errors that can exceed the zero point error outlier limit.,
\textbf{PixFrac = 1{{[}percent{]}}} The maximum fraction of pixels
scientifically unusable per sensor out of the total allowable fraction
of sensors meeting this performance., \textbf{PA1uzy =
7.5{{[}millimagnitude{]}}} The RMS photometric repeatability of bright
non-saturated unresolved point sources in the u, z, and y filters.,
\textbf{PA6 = 10{{[}millimagnitude{]}}} Accuracy of the transformation
of the internal LSST photometry to a physical scale (e.g. AB
magnitudes)., \textbf{Xtalk = 3{{[}sigma{]}}} The maximum local
significance integrated over the PSF of imperfect crosstalk
corrections., \textbf{PA5u = 10{{[}millimagnitude{]}}} Accuracy of
absolute band-to-band color zero-point for colors constructed using the
u-band., \textbf{ResSource = 2{{[}unitless{]}}} Maximum RMS of the ratio
of the error in integrated flux measurement between bright, isolated,
resolved sources less than 10 arcsec in diameter and bright, isolated
unresolved point sources., \textbf{PA2uzy = 22.5{{[}millimagnitude{]}}}
Repeatability outlier limit for isolated bright non-saturated point
sources in the u, z, and y filters., \textbf{SensorFraction =
15{{[}percent{]}}} The maximum allowable fraction of sensors with
PixFrac scientifically unusable pixels., \textbf{PA3u =
20{{[}millimagnitude{]}}} RMS width of internal photometric zero-point
(precision of system uniformity across the sky) in the u-band.,
\textbf{PA2gri = 15{{[}millimagnitude{]}}} Repeatability outlier limit
for isolated bright non-saturated point sources in the g, r, and i
filters., \textbf{SBPrec = 1{{[}percent{]}}} The maximum error in the
precision of the sky brightness determination., \textbf{PA5 =
5{{[}millimagnitude{]}}} Accuracy of absolute band-to-band color
zero-point for all colors constructed from any filter pair, excluding
the u-band.{]} \\ \cdashline{1-2}
Requirement Discussion &
\begin{minipage}[]{13cm}
\textbf{Discussion:} The relevant metrics are listed in the table
photometricPerformance below. The values in the tables are the target
values for LSST but are not verified as part of this requirement.
\end{minipage}
\\ \cdashline{1-2}
Requirement Priority & 1a \\ \cdashline{1-2}
Upper Level Requirement &
\begin{tabular}{cl}
OSS-REQ-0387 & Photometric Performance \\
\end{tabular}
\\ \hline
\end{longtable}
}


\subsubsection{Test Cases Summary}
\begin{longtable}{p{3cm}p{2.5cm}p{2.5cm}p{3cm}p{4cm}}
\toprule
\href{https://jira.lsstcorp.org/secure/Tests.jspa\#/testCase/LVV-T377}{LVV-T377} & \multicolumn{4}{p{12cm}}{ Verify Calculation of Photometric Performance Metrics } \\ \hline
\textbf{Owner} & \textbf{Status} & \textbf{Version} & \textbf{Critical Event} & \textbf{Verification Type} \\ \hline
Leanne Guy & Approved & 1 & false & Test \\ \hline
\end{longtable}
{\scriptsize
\textbf{Objective:}\\
Verify that the DMS system provides software to calculate photometric
performance metrics, and that the algorithms are properly calculating
the desired quantities. Note that because the DMS requirement is that
the software shall be provided (and not on the actual measured values of
the metrics), we verify all of the requirements via a single test case.
}
\begin{longtable}{p{3cm}p{2.5cm}p{2.5cm}p{3cm}p{4cm}}
\toprule
\href{https://jira.lsstcorp.org/secure/Tests.jspa\#/testCase/LVV-T1846}{LVV-T1846} & \multicolumn{4}{p{12cm}}{ Verify calculation of band-to-band color zero-point accuracy including
u-band } \\ \hline
\textbf{Owner} & \textbf{Status} & \textbf{Version} & \textbf{Critical Event} & \textbf{Verification Type} \\ \hline
Jeffrey Carlin & Draft & 1 & false & Test \\ \hline
\end{longtable}
{\scriptsize
\textbf{Objective:}\\
Verify that the DM system provides software to assess whether the
accuracy of absolute band-to-band color zero-points for all colors
constructed from any filter pair, including the u-band, is less than
\textbf{PA5u = 10 millimagnitudes}.
}
  
 \newpage 
\subsection{[LVV-9754] DMS-REQ-0359-V-05: Repeatability outlier limit in gri }\label{lvv-9754}

\begin{longtable}{cccc}
\hline
\textbf{Jira Link} & \textbf{Assignee} & \textbf{Status} & \textbf{Test Cases}\\ \hline
\href{https://jira.lsstcorp.org/browse/LVV-9754}{LVV-9754} &
Leanne Guy & Not Covered &
\begin{tabular}{c}
LVV-T1759 \\
\end{tabular}
\\
\hline
\end{longtable}

\textbf{Verification Element Description:} \\
The repeatability outlier limit for isolated bright non-saturated point
sources in the g, r, and i filters shall be less thanÂ~\textbf{PA2gri =
15 millimagnitudes}.

Associated element DMS-REQ-0359-V-01
(\href{https://jira.lsstcorp.org/browse/LVV-3401}{LVV-3401}) satisfies
the requirement on photometric repeatability in the u, z, and y-band
filters.

Associated element DMS-REQ-0359-V-02
(\href{https://jira.lsstcorp.org/browse/LVV-9751}{LVV-9751}) satisfies
the requirement on the maximum fraction of sensors with scientifically
unusable pixels.

Associated element DMS-REQ-0359-V-03
(\href{https://jira.lsstcorp.org/browse/LVV-9752}{LVV-9752}) satisfies
the constraint on maximum fraction of outliers among non-saturated point
sources.

Associated element DMS-REQ-0359-V-04
(\href{https://jira.lsstcorp.org/browse/LVV-9753}{LVV-9753}) satisfies
the accuracy of zero-point for colors that use the u-band.

Associated element DMS-REQ-0359-V-06
(\href{https://jira.lsstcorp.org/browse/LVV-9755}{LVV-9755}) satisfies
the constraint on the accuracy of the transformation from internal to
physical photometric scales.

Associated element DMS-REQ-0359-V-07
(\href{https://jira.lsstcorp.org/browse/LVV-9756}{LVV-9756}) satisfies
the rms width of the internal photometric zero-point in u-band.

Associated element DMS-REQ-0359-V-08
(\href{https://jira.lsstcorp.org/browse/LVV-9757}{LVV-9757}) satisfies
the maximum local significance of imperfect crosstalk corrections.

Associated element DMS-REQ-0359-V-09
(\href{https://jira.lsstcorp.org/browse/LVV-9758}{LVV-9758}) satisfies
the repeatability outlier limit in u, z, and y-bands.

Associated element DMS-REQ-0359-V-10
(\href{https://jira.lsstcorp.org/browse/LVV-9759}{LVV-9759}) satisfies
the rms photometric repeatability in g, r, and i-bands.

Associated element DMS-REQ-0359-V-11
(\href{https://jira.lsstcorp.org/browse/LVV-9760}{LVV-9760}) satisfies
the fraction of zero-point errors that can exceed the outlier limit.

Associated element DMS-REQ-0359-V-12
(\href{https://jira.lsstcorp.org/browse/LVV-9761}{LVV-9761}) satisfies
the maximum fraction of unusable pixels per sensor.

Associated element DMS-REQ-0359-V-13
(\href{https://jira.lsstcorp.org/browse/LVV-9762}{LVV-9762}) satisfies
the maximum allowable precision in the sky brightness determination.

Associated element DMS-REQ-0359-V-14
(\href{https://jira.lsstcorp.org/browse/LVV-9763}{LVV-9763}) satisfies
the rms width of the internal photometric zero-point in g, r, i, z, and
y-bands.

Associated element DMS-REQ-0359-V-15
(\href{https://jira.lsstcorp.org/browse/LVV-9764}{LVV-9764}) satisfies
the percentage of the image area affected by ghosts that exceed the
threshold.

Associated element DMS-REQ-0359-V-16
(\href{https://jira.lsstcorp.org/browse/LVV-9765}{LVV-9765}) satisfies
the accuracy of zero-point for colors that do not include the u-band.

Associated element DMS-REQ-0359-V-17
(\href{https://jira.lsstcorp.org/browse/LVV-9766}{LVV-9766}) satisfies
the maximum RMS of the ratio of the flux measurement error between
resolved/unresolved sources.

{\footnotesize
\begin{longtable}{p{2.5cm}p{13.5cm}}
\hline
\multicolumn{2}{c}{\textbf{Requirement Details}}\\ \hline
Requirement ID & DMS-REQ-0359 \\ \cdashline{1-2}
Requirement Description &
\begin{minipage}[]{13cm}
\textbf{Specification:} The DMS shall include software to enable the
calculation of the photometric performance metrics defined in
OSS-REQ-0387.
\end{minipage}
\\ \cdashline{1-2}
Requirement Parameters & {[}\textbf{GhostAF = 1{{[}percent{]}}} Percentage of image area that can
have ghosts with surface brightness gradient amplitude of more than 1/3
of the sky noise over 1 arcsec., \textbf{PF1 = 10{{[}percent{]}}} The
maximum fraction of isolated non-saturated point source measurements
exceeding the outlier limit., \textbf{PA1gri = 5{{[}millimagnitude{]}}}
The RMS photometric repeatability of bright non-saturated unresolved
point sources in the g, r, and i filters., \textbf{PA3 =
10{{[}millimagnitude{]}}} RMS width of internal photometric zero-point
(precision of system uniformity across the sky) for all bands except
u-band., \textbf{PA4 = 15{{[}millimagnitude{]}}} The zero point error
outlier limit., \textbf{PF2 = 10{{[}percent{]}}} Fraction of zeropoint
errors that can exceed the zero point error outlier limit.,
\textbf{PixFrac = 1{{[}percent{]}}} The maximum fraction of pixels
scientifically unusable per sensor out of the total allowable fraction
of sensors meeting this performance., \textbf{PA1uzy =
7.5{{[}millimagnitude{]}}} The RMS photometric repeatability of bright
non-saturated unresolved point sources in the u, z, and y filters.,
\textbf{PA6 = 10{{[}millimagnitude{]}}} Accuracy of the transformation
of the internal LSST photometry to a physical scale (e.g. AB
magnitudes)., \textbf{Xtalk = 3{{[}sigma{]}}} The maximum local
significance integrated over the PSF of imperfect crosstalk
corrections., \textbf{PA5u = 10{{[}millimagnitude{]}}} Accuracy of
absolute band-to-band color zero-point for colors constructed using the
u-band., \textbf{ResSource = 2{{[}unitless{]}}} Maximum RMS of the ratio
of the error in integrated flux measurement between bright, isolated,
resolved sources less than 10 arcsec in diameter and bright, isolated
unresolved point sources., \textbf{PA2uzy = 22.5{{[}millimagnitude{]}}}
Repeatability outlier limit for isolated bright non-saturated point
sources in the u, z, and y filters., \textbf{SensorFraction =
15{{[}percent{]}}} The maximum allowable fraction of sensors with
PixFrac scientifically unusable pixels., \textbf{PA3u =
20{{[}millimagnitude{]}}} RMS width of internal photometric zero-point
(precision of system uniformity across the sky) in the u-band.,
\textbf{PA2gri = 15{{[}millimagnitude{]}}} Repeatability outlier limit
for isolated bright non-saturated point sources in the g, r, and i
filters., \textbf{SBPrec = 1{{[}percent{]}}} The maximum error in the
precision of the sky brightness determination., \textbf{PA5 =
5{{[}millimagnitude{]}}} Accuracy of absolute band-to-band color
zero-point for all colors constructed from any filter pair, excluding
the u-band.{]} \\ \cdashline{1-2}
Requirement Discussion &
\begin{minipage}[]{13cm}
\textbf{Discussion:} The relevant metrics are listed in the table
photometricPerformance below. The values in the tables are the target
values for LSST but are not verified as part of this requirement.
\end{minipage}
\\ \cdashline{1-2}
Requirement Priority & 1a \\ \cdashline{1-2}
Upper Level Requirement &
\begin{tabular}{cl}
OSS-REQ-0387 & Photometric Performance \\
\end{tabular}
\\ \hline
\end{longtable}
}


\subsubsection{Test Cases Summary}
\begin{longtable}{p{3cm}p{2.5cm}p{2.5cm}p{3cm}p{4cm}}
\toprule
\href{https://jira.lsstcorp.org/secure/Tests.jspa\#/testCase/LVV-T1759}{LVV-T1759} & \multicolumn{4}{p{12cm}}{ Verify calculation of photometric outliers in gri bands } \\ \hline
\textbf{Owner} & \textbf{Status} & \textbf{Version} & \textbf{Critical Event} & \textbf{Verification Type} \\ \hline
Jeffrey Carlin & Approved & 1 & false & Test \\ \hline
\end{longtable}
{\scriptsize
\textbf{Objective:}\\
Verify that the DM system has provided the code to calculate the
photometric repeatability in the g, r, and i filters, and assess whether
it meets the requirement that no more than \textbf{PF1 =
10{[}percent{]}} of the repeatability outliers exceed the outlier limit
of \textbf{PA2gri = 15 millimagnitudes}.
}
  
 \newpage 
\subsection{[LVV-9755] DMS-REQ-0359-V-06: Accuracy of photometric transformation }\label{lvv-9755}

\begin{longtable}{cccc}
\hline
\textbf{Jira Link} & \textbf{Assignee} & \textbf{Status} & \textbf{Test Cases}\\ \hline
\href{https://jira.lsstcorp.org/browse/LVV-9755}{LVV-9755} &
Leanne Guy & Not Covered &
\begin{tabular}{c}
LVV-T377 \\
LVV-T1845 \\
\end{tabular}
\\
\hline
\end{longtable}

\textbf{Verification Element Description:} \\
The accuracy of the transformation of internal LSST photometry to a
physical scale (e.g. AB magnitudes) shall be less than \textbf{PA6 = 10
millimagnitudes}.

Associated element DMS-REQ-0359-V-01
(\href{https://jira.lsstcorp.org/browse/LVV-3401}{LVV-3401}) satisfies
the requirement on photometric repeatability in the u, z, and y-band
filters.

Associated element DMS-REQ-0359-V-02
(\href{https://jira.lsstcorp.org/browse/LVV-9751}{LVV-9751}) satisfies
the requirement on the maximum fraction of sensors with scientifically
unusable pixels.

Associated element DMS-REQ-0359-V-03
(\href{https://jira.lsstcorp.org/browse/LVV-9752}{LVV-9752}) satisfies
the constraint on maximum fraction of outliers among non-saturated point
sources.

Associated element DMS-REQ-0359-V-04
(\href{https://jira.lsstcorp.org/browse/LVV-9753}{LVV-9753}) satisfies
the accuracy of zero-point for colors that use the u-band.

Associated element DMS-REQ-0359-V-05
(\href{https://jira.lsstcorp.org/browse/LVV-9754}{LVV-9754}) satisfies
the repeatability outlier limit in g, r, and i-bands.

Associated element DMS-REQ-0359-V-07
(\href{https://jira.lsstcorp.org/browse/LVV-9756}{LVV-9756}) satisfies
the rms width of the internal photometric zero-point in u-band.

Associated element DMS-REQ-0359-V-08
(\href{https://jira.lsstcorp.org/browse/LVV-9757}{LVV-9757}) satisfies
the maximum local significance of imperfect crosstalk corrections.

Associated element DMS-REQ-0359-V-09
(\href{https://jira.lsstcorp.org/browse/LVV-9758}{LVV-9758}) satisfies
the repeatability outlier limit in u, z, and y-bands.

Associated element DMS-REQ-0359-V-10
(\href{https://jira.lsstcorp.org/browse/LVV-9759}{LVV-9759}) satisfies
the rms photometric repeatability in g, r, and i-bands.

Associated element DMS-REQ-0359-V-11
(\href{https://jira.lsstcorp.org/browse/LVV-9760}{LVV-9760}) satisfies
the fraction of zero-point errors that can exceed the outlier limit.

Associated element DMS-REQ-0359-V-12
(\href{https://jira.lsstcorp.org/browse/LVV-9761}{LVV-9761}) satisfies
the maximum fraction of unusable pixels per sensor.

Associated element DMS-REQ-0359-V-13
(\href{https://jira.lsstcorp.org/browse/LVV-9762}{LVV-9762}) satisfies
the maximum allowable precision in the sky brightness determination.

Associated element DMS-REQ-0359-V-14
(\href{https://jira.lsstcorp.org/browse/LVV-9763}{LVV-9763}) satisfies
the rms width of the internal photometric zero-point in g, r, i, z, and
y-bands.

Associated element DMS-REQ-0359-V-15
(\href{https://jira.lsstcorp.org/browse/LVV-9764}{LVV-9764}) satisfies
the percentage of the image area affected by ghosts that exceed the
threshold.

Associated element DMS-REQ-0359-V-16
(\href{https://jira.lsstcorp.org/browse/LVV-9765}{LVV-9765}) satisfies
the accuracy of zero-point for colors that do not include the u-band.

Associated element DMS-REQ-0359-V-17
(\href{https://jira.lsstcorp.org/browse/LVV-9766}{LVV-9766}) satisfies
the maximum RMS of the ratio of the flux measurement error between
resolved/unresolved sources.

{\footnotesize
\begin{longtable}{p{2.5cm}p{13.5cm}}
\hline
\multicolumn{2}{c}{\textbf{Requirement Details}}\\ \hline
Requirement ID & DMS-REQ-0359 \\ \cdashline{1-2}
Requirement Description &
\begin{minipage}[]{13cm}
\textbf{Specification:} The DMS shall include software to enable the
calculation of the photometric performance metrics defined in
OSS-REQ-0387.
\end{minipage}
\\ \cdashline{1-2}
Requirement Parameters & {[}\textbf{GhostAF = 1{{[}percent{]}}} Percentage of image area that can
have ghosts with surface brightness gradient amplitude of more than 1/3
of the sky noise over 1 arcsec., \textbf{PF1 = 10{{[}percent{]}}} The
maximum fraction of isolated non-saturated point source measurements
exceeding the outlier limit., \textbf{PA1gri = 5{{[}millimagnitude{]}}}
The RMS photometric repeatability of bright non-saturated unresolved
point sources in the g, r, and i filters., \textbf{PA3 =
10{{[}millimagnitude{]}}} RMS width of internal photometric zero-point
(precision of system uniformity across the sky) for all bands except
u-band., \textbf{PA4 = 15{{[}millimagnitude{]}}} The zero point error
outlier limit., \textbf{PF2 = 10{{[}percent{]}}} Fraction of zeropoint
errors that can exceed the zero point error outlier limit.,
\textbf{PixFrac = 1{{[}percent{]}}} The maximum fraction of pixels
scientifically unusable per sensor out of the total allowable fraction
of sensors meeting this performance., \textbf{PA1uzy =
7.5{{[}millimagnitude{]}}} The RMS photometric repeatability of bright
non-saturated unresolved point sources in the u, z, and y filters.,
\textbf{PA6 = 10{{[}millimagnitude{]}}} Accuracy of the transformation
of the internal LSST photometry to a physical scale (e.g. AB
magnitudes)., \textbf{Xtalk = 3{{[}sigma{]}}} The maximum local
significance integrated over the PSF of imperfect crosstalk
corrections., \textbf{PA5u = 10{{[}millimagnitude{]}}} Accuracy of
absolute band-to-band color zero-point for colors constructed using the
u-band., \textbf{ResSource = 2{{[}unitless{]}}} Maximum RMS of the ratio
of the error in integrated flux measurement between bright, isolated,
resolved sources less than 10 arcsec in diameter and bright, isolated
unresolved point sources., \textbf{PA2uzy = 22.5{{[}millimagnitude{]}}}
Repeatability outlier limit for isolated bright non-saturated point
sources in the u, z, and y filters., \textbf{SensorFraction =
15{{[}percent{]}}} The maximum allowable fraction of sensors with
PixFrac scientifically unusable pixels., \textbf{PA3u =
20{{[}millimagnitude{]}}} RMS width of internal photometric zero-point
(precision of system uniformity across the sky) in the u-band.,
\textbf{PA2gri = 15{{[}millimagnitude{]}}} Repeatability outlier limit
for isolated bright non-saturated point sources in the g, r, and i
filters., \textbf{SBPrec = 1{{[}percent{]}}} The maximum error in the
precision of the sky brightness determination., \textbf{PA5 =
5{{[}millimagnitude{]}}} Accuracy of absolute band-to-band color
zero-point for all colors constructed from any filter pair, excluding
the u-band.{]} \\ \cdashline{1-2}
Requirement Discussion &
\begin{minipage}[]{13cm}
\textbf{Discussion:} The relevant metrics are listed in the table
photometricPerformance below. The values in the tables are the target
values for LSST but are not verified as part of this requirement.
\end{minipage}
\\ \cdashline{1-2}
Requirement Priority & 1a \\ \cdashline{1-2}
Upper Level Requirement &
\begin{tabular}{cl}
OSS-REQ-0387 & Photometric Performance \\
\end{tabular}
\\ \hline
\end{longtable}
}


\subsubsection{Test Cases Summary}
\begin{longtable}{p{3cm}p{2.5cm}p{2.5cm}p{3cm}p{4cm}}
\toprule
\href{https://jira.lsstcorp.org/secure/Tests.jspa\#/testCase/LVV-T377}{LVV-T377} & \multicolumn{4}{p{12cm}}{ Verify Calculation of Photometric Performance Metrics } \\ \hline
\textbf{Owner} & \textbf{Status} & \textbf{Version} & \textbf{Critical Event} & \textbf{Verification Type} \\ \hline
Leanne Guy & Approved & 1 & false & Test \\ \hline
\end{longtable}
{\scriptsize
\textbf{Objective:}\\
Verify that the DMS system provides software to calculate photometric
performance metrics, and that the algorithms are properly calculating
the desired quantities. Note that because the DMS requirement is that
the software shall be provided (and not on the actual measured values of
the metrics), we verify all of the requirements via a single test case.
}
\begin{longtable}{p{3cm}p{2.5cm}p{2.5cm}p{3cm}p{4cm}}
\toprule
\href{https://jira.lsstcorp.org/secure/Tests.jspa\#/testCase/LVV-T1845}{LVV-T1845} & \multicolumn{4}{p{12cm}}{ Verify accuracy of photometric transformation to physical scale } \\ \hline
\textbf{Owner} & \textbf{Status} & \textbf{Version} & \textbf{Critical Event} & \textbf{Verification Type} \\ \hline
Jeffrey Carlin & Draft & 1 & false & Test \\ \hline
\end{longtable}
{\scriptsize
\textbf{Objective:}\\
Verify that the DM system provides software to assess whether the
accuracy of the transformation of internal LSST photometry to a physical
scale (e.g. AB magnitudes) is less than \textbf{PA6 = 10
millimagnitudes}.
}
  
 \newpage 
\subsection{[LVV-9756] DMS-REQ-0359-V-07: RMS width of zero point in u-band }\label{lvv-9756}

\begin{longtable}{cccc}
\hline
\textbf{Jira Link} & \textbf{Assignee} & \textbf{Status} & \textbf{Test Cases}\\ \hline
\href{https://jira.lsstcorp.org/browse/LVV-9756}{LVV-9756} &
Leanne Guy & Not Covered &
\begin{tabular}{c}
LVV-T377 \\
LVV-T1844 \\
\end{tabular}
\\
\hline
\end{longtable}

\textbf{Verification Element Description:} \\
The RMS width of internal photometric zero-point (precision of system
uniformity across the sky) in the u-band shall be less than \textbf{PA3u
= 20 millimagnitudes}.

Associated element DMS-REQ-0359-V-01
(\href{https://jira.lsstcorp.org/browse/LVV-3401}{LVV-3401}) satisfies
the requirement on photometric repeatability in the u, z, and y-band
filters.

Associated element DMS-REQ-0359-V-02
(\href{https://jira.lsstcorp.org/browse/LVV-9751}{LVV-9751}) satisfies
the requirement on the maximum fraction of sensors with scientifically
unusable pixels.

Associated element DMS-REQ-0359-V-03
(\href{https://jira.lsstcorp.org/browse/LVV-9752}{LVV-9752}) satisfies
the constraint on maximum fraction of outliers among non-saturated point
sources.

Associated element DMS-REQ-0359-V-04
(\href{https://jira.lsstcorp.org/browse/LVV-9753}{LVV-9753}) satisfies
the accuracy of zero-point for colors that use the u-band.

Associated element DMS-REQ-0359-V-05
(\href{https://jira.lsstcorp.org/browse/LVV-9754}{LVV-9754}) satisfies
the repeatability outlier limit in g, r, and i-bands.

Associated element DMS-REQ-0359-V-06
(\href{https://jira.lsstcorp.org/browse/LVV-9755}{LVV-9755}) satisfies
the constraint on the accuracy of the transformation from internal to
physical photometric scales.

Associated element DMS-REQ-0359-V-08
(\href{https://jira.lsstcorp.org/browse/LVV-9757}{LVV-9757}) satisfies
the maximum local significance of imperfect crosstalk corrections.

Associated element DMS-REQ-0359-V-09
(\href{https://jira.lsstcorp.org/browse/LVV-9758}{LVV-9758}) satisfies
the repeatability outlier limit in u, z, and y-bands.

Associated element DMS-REQ-0359-V-10
(\href{https://jira.lsstcorp.org/browse/LVV-9759}{LVV-9759}) satisfies
the rms photometric repeatability in g, r, and i-bands.

Associated element DMS-REQ-0359-V-11
(\href{https://jira.lsstcorp.org/browse/LVV-9760}{LVV-9760}) satisfies
the fraction of zero-point errors that can exceed the outlier limit.

Associated element DMS-REQ-0359-V-12
(\href{https://jira.lsstcorp.org/browse/LVV-9761}{LVV-9761}) satisfies
the maximum fraction of unusable pixels per sensor.

Associated element DMS-REQ-0359-V-13
(\href{https://jira.lsstcorp.org/browse/LVV-9762}{LVV-9762}) satisfies
the maximum allowable precision in the sky brightness determination.

Associated element DMS-REQ-0359-V-14
(\href{https://jira.lsstcorp.org/browse/LVV-9763}{LVV-9763}) satisfies
the rms width of the internal photometric zero-point in g, r, i, z, and
y-bands.

Associated element DMS-REQ-0359-V-15
(\href{https://jira.lsstcorp.org/browse/LVV-9764}{LVV-9764}) satisfies
the percentage of the image area affected by ghosts that exceed the
threshold.

Associated element DMS-REQ-0359-V-16
(\href{https://jira.lsstcorp.org/browse/LVV-9765}{LVV-9765}) satisfies
the accuracy of zero-point for colors that do not include the u-band.

Associated element DMS-REQ-0359-V-17
(\href{https://jira.lsstcorp.org/browse/LVV-9766}{LVV-9766}) satisfies
the maximum RMS of the ratio of the flux measurement error between
resolved/unresolved sources.

{\footnotesize
\begin{longtable}{p{2.5cm}p{13.5cm}}
\hline
\multicolumn{2}{c}{\textbf{Requirement Details}}\\ \hline
Requirement ID & DMS-REQ-0359 \\ \cdashline{1-2}
Requirement Description &
\begin{minipage}[]{13cm}
\textbf{Specification:} The DMS shall include software to enable the
calculation of the photometric performance metrics defined in
OSS-REQ-0387.
\end{minipage}
\\ \cdashline{1-2}
Requirement Parameters & {[}\textbf{GhostAF = 1{{[}percent{]}}} Percentage of image area that can
have ghosts with surface brightness gradient amplitude of more than 1/3
of the sky noise over 1 arcsec., \textbf{PF1 = 10{{[}percent{]}}} The
maximum fraction of isolated non-saturated point source measurements
exceeding the outlier limit., \textbf{PA1gri = 5{{[}millimagnitude{]}}}
The RMS photometric repeatability of bright non-saturated unresolved
point sources in the g, r, and i filters., \textbf{PA3 =
10{{[}millimagnitude{]}}} RMS width of internal photometric zero-point
(precision of system uniformity across the sky) for all bands except
u-band., \textbf{PA4 = 15{{[}millimagnitude{]}}} The zero point error
outlier limit., \textbf{PF2 = 10{{[}percent{]}}} Fraction of zeropoint
errors that can exceed the zero point error outlier limit.,
\textbf{PixFrac = 1{{[}percent{]}}} The maximum fraction of pixels
scientifically unusable per sensor out of the total allowable fraction
of sensors meeting this performance., \textbf{PA1uzy =
7.5{{[}millimagnitude{]}}} The RMS photometric repeatability of bright
non-saturated unresolved point sources in the u, z, and y filters.,
\textbf{PA6 = 10{{[}millimagnitude{]}}} Accuracy of the transformation
of the internal LSST photometry to a physical scale (e.g. AB
magnitudes)., \textbf{Xtalk = 3{{[}sigma{]}}} The maximum local
significance integrated over the PSF of imperfect crosstalk
corrections., \textbf{PA5u = 10{{[}millimagnitude{]}}} Accuracy of
absolute band-to-band color zero-point for colors constructed using the
u-band., \textbf{ResSource = 2{{[}unitless{]}}} Maximum RMS of the ratio
of the error in integrated flux measurement between bright, isolated,
resolved sources less than 10 arcsec in diameter and bright, isolated
unresolved point sources., \textbf{PA2uzy = 22.5{{[}millimagnitude{]}}}
Repeatability outlier limit for isolated bright non-saturated point
sources in the u, z, and y filters., \textbf{SensorFraction =
15{{[}percent{]}}} The maximum allowable fraction of sensors with
PixFrac scientifically unusable pixels., \textbf{PA3u =
20{{[}millimagnitude{]}}} RMS width of internal photometric zero-point
(precision of system uniformity across the sky) in the u-band.,
\textbf{PA2gri = 15{{[}millimagnitude{]}}} Repeatability outlier limit
for isolated bright non-saturated point sources in the g, r, and i
filters., \textbf{SBPrec = 1{{[}percent{]}}} The maximum error in the
precision of the sky brightness determination., \textbf{PA5 =
5{{[}millimagnitude{]}}} Accuracy of absolute band-to-band color
zero-point for all colors constructed from any filter pair, excluding
the u-band.{]} \\ \cdashline{1-2}
Requirement Discussion &
\begin{minipage}[]{13cm}
\textbf{Discussion:} The relevant metrics are listed in the table
photometricPerformance below. The values in the tables are the target
values for LSST but are not verified as part of this requirement.
\end{minipage}
\\ \cdashline{1-2}
Requirement Priority & 1a \\ \cdashline{1-2}
Upper Level Requirement &
\begin{tabular}{cl}
OSS-REQ-0387 & Photometric Performance \\
\end{tabular}
\\ \hline
\end{longtable}
}


\subsubsection{Test Cases Summary}
\begin{longtable}{p{3cm}p{2.5cm}p{2.5cm}p{3cm}p{4cm}}
\toprule
\href{https://jira.lsstcorp.org/secure/Tests.jspa\#/testCase/LVV-T377}{LVV-T377} & \multicolumn{4}{p{12cm}}{ Verify Calculation of Photometric Performance Metrics } \\ \hline
\textbf{Owner} & \textbf{Status} & \textbf{Version} & \textbf{Critical Event} & \textbf{Verification Type} \\ \hline
Leanne Guy & Approved & 1 & false & Test \\ \hline
\end{longtable}
{\scriptsize
\textbf{Objective:}\\
Verify that the DMS system provides software to calculate photometric
performance metrics, and that the algorithms are properly calculating
the desired quantities. Note that because the DMS requirement is that
the software shall be provided (and not on the actual measured values of
the metrics), we verify all of the requirements via a single test case.
}
\begin{longtable}{p{3cm}p{2.5cm}p{2.5cm}p{3cm}p{4cm}}
\toprule
\href{https://jira.lsstcorp.org/secure/Tests.jspa\#/testCase/LVV-T1844}{LVV-T1844} & \multicolumn{4}{p{12cm}}{ Verify calculation of u-band photometric zero-point RMS } \\ \hline
\textbf{Owner} & \textbf{Status} & \textbf{Version} & \textbf{Critical Event} & \textbf{Verification Type} \\ \hline
Jeffrey Carlin & Draft & 1 & false & Test \\ \hline
\end{longtable}
{\scriptsize
\textbf{Objective:}\\
Verify that the DM system provides software to assess whether the RMS
width of internal photometric zero-point (precision of system uniformity
across the sky) in the u-band is less than \textbf{PA3u = 20
millimagnitudes}.
}
  
 \newpage 
\subsection{[LVV-9757] DMS-REQ-0359-V-08: Max cross-talk imperfections }\label{lvv-9757}

\begin{longtable}{cccc}
\hline
\textbf{Jira Link} & \textbf{Assignee} & \textbf{Status} & \textbf{Test Cases}\\ \hline
\href{https://jira.lsstcorp.org/browse/LVV-9757}{LVV-9757} &
Leanne Guy & Not Covered &
\begin{tabular}{c}
LVV-T377 \\
LVV-T1843 \\
\end{tabular}
\\
\hline
\end{longtable}

\textbf{Verification Element Description:} \\
The maximum local significance integrated over the PSF of imperfect
crosstalk corrections shall be less than \textbf{Xtalk = 3 sigma}.

Associated element DMS-REQ-0359-V-01
(\href{https://jira.lsstcorp.org/browse/LVV-3401}{LVV-3401}) satisfies
the requirement on photometric repeatability in the u, z, and y-band
filters.

Associated element DMS-REQ-0359-V-02
(\href{https://jira.lsstcorp.org/browse/LVV-9751}{LVV-9751}) satisfies
the requirement on the maximum fraction of sensors with scientifically
unusable pixels.

Associated element DMS-REQ-0359-V-03
(\href{https://jira.lsstcorp.org/browse/LVV-9752}{LVV-9752}) satisfies
the constraint on maximum fraction of outliers among non-saturated point
sources.

Associated element DMS-REQ-0359-V-04
(\href{https://jira.lsstcorp.org/browse/LVV-9753}{LVV-9753}) satisfies
the accuracy of zero-point for colors that use the u-band.

Associated element DMS-REQ-0359-V-05
(\href{https://jira.lsstcorp.org/browse/LVV-9754}{LVV-9754}) satisfies
the repeatability outlier limit in g, r, and i-bands.

Associated element DMS-REQ-0359-V-06
(\href{https://jira.lsstcorp.org/browse/LVV-9755}{LVV-9755}) satisfies
the constraint on the accuracy of the transformation from internal to
physical photometric scales.

Associated element DMS-REQ-0359-V-07
(\href{https://jira.lsstcorp.org/browse/LVV-9756}{LVV-9756}) satisfies
the rms width of the internal photometric zero-point in u-band.

Associated element DMS-REQ-0359-V-09
(\href{https://jira.lsstcorp.org/browse/LVV-9758}{LVV-9758}) satisfies
the repeatability outlier limit in u, z, and y-bands.

Associated element DMS-REQ-0359-V-10
(\href{https://jira.lsstcorp.org/browse/LVV-9759}{LVV-9759}) satisfies
the rms photometric repeatability in g, r, and i-bands.

Associated element DMS-REQ-0359-V-11
(\href{https://jira.lsstcorp.org/browse/LVV-9760}{LVV-9760}) satisfies
the fraction of zero-point errors that can exceed the outlier limit.

Associated element DMS-REQ-0359-V-12
(\href{https://jira.lsstcorp.org/browse/LVV-9761}{LVV-9761}) satisfies
the maximum fraction of unusable pixels per sensor.

Associated element DMS-REQ-0359-V-13
(\href{https://jira.lsstcorp.org/browse/LVV-9762}{LVV-9762}) satisfies
the maximum allowable precision in the sky brightness determination.

Associated element DMS-REQ-0359-V-14
(\href{https://jira.lsstcorp.org/browse/LVV-9763}{LVV-9763}) satisfies
the rms width of the internal photometric zero-point in g, r, i, z, and
y-bands.

Associated element DMS-REQ-0359-V-15
(\href{https://jira.lsstcorp.org/browse/LVV-9764}{LVV-9764}) satisfies
the percentage of the image area affected by ghosts that exceed the
threshold.

Associated element DMS-REQ-0359-V-16
(\href{https://jira.lsstcorp.org/browse/LVV-9765}{LVV-9765}) satisfies
the accuracy of zero-point for colors that do not include the u-band.

Associated element DMS-REQ-0359-V-17
(\href{https://jira.lsstcorp.org/browse/LVV-9766}{LVV-9766}) satisfies
the maximum RMS of the ratio of the flux measurement error between
resolved/unresolved sources.

{\footnotesize
\begin{longtable}{p{2.5cm}p{13.5cm}}
\hline
\multicolumn{2}{c}{\textbf{Requirement Details}}\\ \hline
Requirement ID & DMS-REQ-0359 \\ \cdashline{1-2}
Requirement Description &
\begin{minipage}[]{13cm}
\textbf{Specification:} The DMS shall include software to enable the
calculation of the photometric performance metrics defined in
OSS-REQ-0387.
\end{minipage}
\\ \cdashline{1-2}
Requirement Parameters & {[}\textbf{GhostAF = 1{{[}percent{]}}} Percentage of image area that can
have ghosts with surface brightness gradient amplitude of more than 1/3
of the sky noise over 1 arcsec., \textbf{PF1 = 10{{[}percent{]}}} The
maximum fraction of isolated non-saturated point source measurements
exceeding the outlier limit., \textbf{PA1gri = 5{{[}millimagnitude{]}}}
The RMS photometric repeatability of bright non-saturated unresolved
point sources in the g, r, and i filters., \textbf{PA3 =
10{{[}millimagnitude{]}}} RMS width of internal photometric zero-point
(precision of system uniformity across the sky) for all bands except
u-band., \textbf{PA4 = 15{{[}millimagnitude{]}}} The zero point error
outlier limit., \textbf{PF2 = 10{{[}percent{]}}} Fraction of zeropoint
errors that can exceed the zero point error outlier limit.,
\textbf{PixFrac = 1{{[}percent{]}}} The maximum fraction of pixels
scientifically unusable per sensor out of the total allowable fraction
of sensors meeting this performance., \textbf{PA1uzy =
7.5{{[}millimagnitude{]}}} The RMS photometric repeatability of bright
non-saturated unresolved point sources in the u, z, and y filters.,
\textbf{PA6 = 10{{[}millimagnitude{]}}} Accuracy of the transformation
of the internal LSST photometry to a physical scale (e.g. AB
magnitudes)., \textbf{Xtalk = 3{{[}sigma{]}}} The maximum local
significance integrated over the PSF of imperfect crosstalk
corrections., \textbf{PA5u = 10{{[}millimagnitude{]}}} Accuracy of
absolute band-to-band color zero-point for colors constructed using the
u-band., \textbf{ResSource = 2{{[}unitless{]}}} Maximum RMS of the ratio
of the error in integrated flux measurement between bright, isolated,
resolved sources less than 10 arcsec in diameter and bright, isolated
unresolved point sources., \textbf{PA2uzy = 22.5{{[}millimagnitude{]}}}
Repeatability outlier limit for isolated bright non-saturated point
sources in the u, z, and y filters., \textbf{SensorFraction =
15{{[}percent{]}}} The maximum allowable fraction of sensors with
PixFrac scientifically unusable pixels., \textbf{PA3u =
20{{[}millimagnitude{]}}} RMS width of internal photometric zero-point
(precision of system uniformity across the sky) in the u-band.,
\textbf{PA2gri = 15{{[}millimagnitude{]}}} Repeatability outlier limit
for isolated bright non-saturated point sources in the g, r, and i
filters., \textbf{SBPrec = 1{{[}percent{]}}} The maximum error in the
precision of the sky brightness determination., \textbf{PA5 =
5{{[}millimagnitude{]}}} Accuracy of absolute band-to-band color
zero-point for all colors constructed from any filter pair, excluding
the u-band.{]} \\ \cdashline{1-2}
Requirement Discussion &
\begin{minipage}[]{13cm}
\textbf{Discussion:} The relevant metrics are listed in the table
photometricPerformance below. The values in the tables are the target
values for LSST but are not verified as part of this requirement.
\end{minipage}
\\ \cdashline{1-2}
Requirement Priority & 1a \\ \cdashline{1-2}
Upper Level Requirement &
\begin{tabular}{cl}
OSS-REQ-0387 & Photometric Performance \\
\end{tabular}
\\ \hline
\end{longtable}
}


\subsubsection{Test Cases Summary}
\begin{longtable}{p{3cm}p{2.5cm}p{2.5cm}p{3cm}p{4cm}}
\toprule
\href{https://jira.lsstcorp.org/secure/Tests.jspa\#/testCase/LVV-T377}{LVV-T377} & \multicolumn{4}{p{12cm}}{ Verify Calculation of Photometric Performance Metrics } \\ \hline
\textbf{Owner} & \textbf{Status} & \textbf{Version} & \textbf{Critical Event} & \textbf{Verification Type} \\ \hline
Leanne Guy & Approved & 1 & false & Test \\ \hline
\end{longtable}
{\scriptsize
\textbf{Objective:}\\
Verify that the DMS system provides software to calculate photometric
performance metrics, and that the algorithms are properly calculating
the desired quantities. Note that because the DMS requirement is that
the software shall be provided (and not on the actual measured values of
the metrics), we verify all of the requirements via a single test case.
}
\begin{longtable}{p{3cm}p{2.5cm}p{2.5cm}p{3cm}p{4cm}}
\toprule
\href{https://jira.lsstcorp.org/secure/Tests.jspa\#/testCase/LVV-T1843}{LVV-T1843} & \multicolumn{4}{p{12cm}}{ Verify calculation of significance of imperfect crosstalk corrections } \\ \hline
\textbf{Owner} & \textbf{Status} & \textbf{Version} & \textbf{Critical Event} & \textbf{Verification Type} \\ \hline
Jeffrey Carlin & Draft & 1 & false & Test \\ \hline
\end{longtable}
{\scriptsize
\textbf{Objective:}\\
Verify that the DM system provides software to assess whether the
maximum local significance integrated over the PSF of imperfect
crosstalk corrections is less than \textbf{Xtalk = 3 sigma}.
}
  
 \newpage 
\subsection{[LVV-9758] DMS-REQ-0359-V-09: Repeatability outlier limit in uzy }\label{lvv-9758}

\begin{longtable}{cccc}
\hline
\textbf{Jira Link} & \textbf{Assignee} & \textbf{Status} & \textbf{Test Cases}\\ \hline
\href{https://jira.lsstcorp.org/browse/LVV-9758}{LVV-9758} &
Leanne Guy & Not Covered &
\begin{tabular}{c}
LVV-T1758 \\
\end{tabular}
\\
\hline
\end{longtable}

\textbf{Verification Element Description:} \\
The repeatability outlier limit for isolated bright non-saturated point
sources in the u, z, and y filters shall be less than \textbf{PA2uzy =
22.5 millimagnitudes}.

Associated element DMS-REQ-0359-V-01
(\href{https://jira.lsstcorp.org/browse/LVV-3401}{LVV-3401}) satisfies
the requirement on photometric repeatability in the u, z, and y-band
filters.

Associated element DMS-REQ-0359-V-02
(\href{https://jira.lsstcorp.org/browse/LVV-9751}{LVV-9751}) satisfies
the requirement on the maximum fraction of sensors with scientifically
unusable pixels.

Associated element DMS-REQ-0359-V-03
(\href{https://jira.lsstcorp.org/browse/LVV-9752}{LVV-9752}) satisfies
the constraint on maximum fraction of outliers among non-saturated point
sources.

Associated element DMS-REQ-0359-V-04
(\href{https://jira.lsstcorp.org/browse/LVV-9753}{LVV-9753}) satisfies
the accuracy of zero-point for colors that use the u-band.

Associated element DMS-REQ-0359-V-05
(\href{https://jira.lsstcorp.org/browse/LVV-9754}{LVV-9754}) satisfies
the repeatability outlier limit in g, r, and i-bands.

Associated element DMS-REQ-0359-V-06
(\href{https://jira.lsstcorp.org/browse/LVV-9755}{LVV-9755}) satisfies
the constraint on the accuracy of the transformation from internal to
physical photometric scales.

Associated element DMS-REQ-0359-V-07
(\href{https://jira.lsstcorp.org/browse/LVV-9756}{LVV-9756}) satisfies
the rms width of the internal photometric zero-point in u-band.

Associated element DMS-REQ-0359-V-08
(\href{https://jira.lsstcorp.org/browse/LVV-9757}{LVV-9757}) satisfies
the maximum local significance of imperfect crosstalk corrections.

Associated element DMS-REQ-0359-V-10
(\href{https://jira.lsstcorp.org/browse/LVV-9759}{LVV-9759}) satisfies
the rms photometric repeatability in g, r, and i-bands.

Associated element DMS-REQ-0359-V-11
(\href{https://jira.lsstcorp.org/browse/LVV-9760}{LVV-9760}) satisfies
the fraction of zero-point errors that can exceed the outlier limit.

Associated element DMS-REQ-0359-V-12
(\href{https://jira.lsstcorp.org/browse/LVV-9761}{LVV-9761}) satisfies
the maximum fraction of unusable pixels per sensor.

Associated element DMS-REQ-0359-V-13
(\href{https://jira.lsstcorp.org/browse/LVV-9762}{LVV-9762}) satisfies
the maximum allowable precision in the sky brightness determination.

Associated element DMS-REQ-0359-V-14
(\href{https://jira.lsstcorp.org/browse/LVV-9763}{LVV-9763}) satisfies
the rms width of the internal photometric zero-point in g, r, i, z, and
y-bands.

Associated element DMS-REQ-0359-V-15
(\href{https://jira.lsstcorp.org/browse/LVV-9764}{LVV-9764}) satisfies
the percentage of the image area affected by ghosts that exceed the
threshold.

Associated element DMS-REQ-0359-V-16
(\href{https://jira.lsstcorp.org/browse/LVV-9765}{LVV-9765}) satisfies
the accuracy of zero-point for colors that do not include the u-band.

Associated element DMS-REQ-0359-V-17
(\href{https://jira.lsstcorp.org/browse/LVV-9766}{LVV-9766}) satisfies
the maximum RMS of the ratio of the flux measurement error between
resolved/unresolved sources.

{\footnotesize
\begin{longtable}{p{2.5cm}p{13.5cm}}
\hline
\multicolumn{2}{c}{\textbf{Requirement Details}}\\ \hline
Requirement ID & DMS-REQ-0359 \\ \cdashline{1-2}
Requirement Description &
\begin{minipage}[]{13cm}
\textbf{Specification:} The DMS shall include software to enable the
calculation of the photometric performance metrics defined in
OSS-REQ-0387.
\end{minipage}
\\ \cdashline{1-2}
Requirement Parameters & {[}\textbf{GhostAF = 1{{[}percent{]}}} Percentage of image area that can
have ghosts with surface brightness gradient amplitude of more than 1/3
of the sky noise over 1 arcsec., \textbf{PF1 = 10{{[}percent{]}}} The
maximum fraction of isolated non-saturated point source measurements
exceeding the outlier limit., \textbf{PA1gri = 5{{[}millimagnitude{]}}}
The RMS photometric repeatability of bright non-saturated unresolved
point sources in the g, r, and i filters., \textbf{PA3 =
10{{[}millimagnitude{]}}} RMS width of internal photometric zero-point
(precision of system uniformity across the sky) for all bands except
u-band., \textbf{PA4 = 15{{[}millimagnitude{]}}} The zero point error
outlier limit., \textbf{PF2 = 10{{[}percent{]}}} Fraction of zeropoint
errors that can exceed the zero point error outlier limit.,
\textbf{PixFrac = 1{{[}percent{]}}} The maximum fraction of pixels
scientifically unusable per sensor out of the total allowable fraction
of sensors meeting this performance., \textbf{PA1uzy =
7.5{{[}millimagnitude{]}}} The RMS photometric repeatability of bright
non-saturated unresolved point sources in the u, z, and y filters.,
\textbf{PA6 = 10{{[}millimagnitude{]}}} Accuracy of the transformation
of the internal LSST photometry to a physical scale (e.g. AB
magnitudes)., \textbf{Xtalk = 3{{[}sigma{]}}} The maximum local
significance integrated over the PSF of imperfect crosstalk
corrections., \textbf{PA5u = 10{{[}millimagnitude{]}}} Accuracy of
absolute band-to-band color zero-point for colors constructed using the
u-band., \textbf{ResSource = 2{{[}unitless{]}}} Maximum RMS of the ratio
of the error in integrated flux measurement between bright, isolated,
resolved sources less than 10 arcsec in diameter and bright, isolated
unresolved point sources., \textbf{PA2uzy = 22.5{{[}millimagnitude{]}}}
Repeatability outlier limit for isolated bright non-saturated point
sources in the u, z, and y filters., \textbf{SensorFraction =
15{{[}percent{]}}} The maximum allowable fraction of sensors with
PixFrac scientifically unusable pixels., \textbf{PA3u =
20{{[}millimagnitude{]}}} RMS width of internal photometric zero-point
(precision of system uniformity across the sky) in the u-band.,
\textbf{PA2gri = 15{{[}millimagnitude{]}}} Repeatability outlier limit
for isolated bright non-saturated point sources in the g, r, and i
filters., \textbf{SBPrec = 1{{[}percent{]}}} The maximum error in the
precision of the sky brightness determination., \textbf{PA5 =
5{{[}millimagnitude{]}}} Accuracy of absolute band-to-band color
zero-point for all colors constructed from any filter pair, excluding
the u-band.{]} \\ \cdashline{1-2}
Requirement Discussion &
\begin{minipage}[]{13cm}
\textbf{Discussion:} The relevant metrics are listed in the table
photometricPerformance below. The values in the tables are the target
values for LSST but are not verified as part of this requirement.
\end{minipage}
\\ \cdashline{1-2}
Requirement Priority & 1a \\ \cdashline{1-2}
Upper Level Requirement &
\begin{tabular}{cl}
OSS-REQ-0387 & Photometric Performance \\
\end{tabular}
\\ \hline
\end{longtable}
}


\subsubsection{Test Cases Summary}
\begin{longtable}{p{3cm}p{2.5cm}p{2.5cm}p{3cm}p{4cm}}
\toprule
\href{https://jira.lsstcorp.org/secure/Tests.jspa\#/testCase/LVV-T1758}{LVV-T1758} & \multicolumn{4}{p{12cm}}{ Verify calculation of photometric outliers in uzy bands } \\ \hline
\textbf{Owner} & \textbf{Status} & \textbf{Version} & \textbf{Critical Event} & \textbf{Verification Type} \\ \hline
Jeffrey Carlin & Approved & 1 & false & Test \\ \hline
\end{longtable}
{\scriptsize
\textbf{Objective:}\\
Verify that the DM system has provided the code to calculate the
photometric repeatability in the u, z, and y filters, and assess whether
it meets the requirement that no more than \textbf{PF1 =
10{[}percent{]}} of the repeatability outliers exceed the outlier limit
of \textbf{PA2uzy = 22.5 millimagnitudes}.~
}
  
 \newpage 
\subsection{[LVV-9759] DMS-REQ-0359-V-10: RMS photometric repeatability in gri }\label{lvv-9759}

\begin{longtable}{cccc}
\hline
\textbf{Jira Link} & \textbf{Assignee} & \textbf{Status} & \textbf{Test Cases}\\ \hline
\href{https://jira.lsstcorp.org/browse/LVV-9759}{LVV-9759} &
Leanne Guy & Not Covered &
\begin{tabular}{c}
LVV-T1757 \\
\end{tabular}
\\
\hline
\end{longtable}

\textbf{Verification Element Description:} \\
The RMS photometric repeatability of bright non-saturated unresolved
point sources in the g, r, and i filters must be less
thanÂ~\textbf{PA1gri = 5 millimagnitudes.}

Associated element DMS-REQ-0359-V-01
(\href{https://jira.lsstcorp.org/browse/LVV-3401}{LVV-3401}) satisfies
the requirement on photometric repeatability in the u, z, and y-band
filters.

Associated element DMS-REQ-0359-V-02
(\href{https://jira.lsstcorp.org/browse/LVV-9751}{LVV-9751}) satisfies
the requirement on the maximum fraction of sensors with scientifically
unusable pixels.

Associated element DMS-REQ-0359-V-03
(\href{https://jira.lsstcorp.org/browse/LVV-9752}{LVV-9752}) satisfies
the constraint on maximum fraction of outliers among non-saturated point
sources.

Associated element DMS-REQ-0359-V-04
(\href{https://jira.lsstcorp.org/browse/LVV-9753}{LVV-9753}) satisfies
the accuracy of zero-point for colors that use the u-band.

Associated element DMS-REQ-0359-V-05
(\href{https://jira.lsstcorp.org/browse/LVV-9754}{LVV-9754}) satisfies
the repeatability outlier limit in g, r, and i-bands.

Associated element DMS-REQ-0359-V-06
(\href{https://jira.lsstcorp.org/browse/LVV-9755}{LVV-9755}) satisfies
the constraint on the accuracy of the transformation from internal to
physical photometric scales.

Associated element DMS-REQ-0359-V-07
(\href{https://jira.lsstcorp.org/browse/LVV-9756}{LVV-9756}) satisfies
the rms width of the internal photometric zero-point in u-band.

Associated element DMS-REQ-0359-V-08
(\href{https://jira.lsstcorp.org/browse/LVV-9757}{LVV-9757}) satisfies
the maximum local significance of imperfect crosstalk corrections.

Associated element DMS-REQ-0359-V-09
(\href{https://jira.lsstcorp.org/browse/LVV-9758}{LVV-9758}) satisfies
the repeatability outlier limit in u, z, and y-bands.

Associated element DMS-REQ-0359-V-11
(\href{https://jira.lsstcorp.org/browse/LVV-9760}{LVV-9760}) satisfies
the fraction of zero-point errors that can exceed the outlier limit.

Associated element DMS-REQ-0359-V-12
(\href{https://jira.lsstcorp.org/browse/LVV-9761}{LVV-9761}) satisfies
the maximum fraction of unusable pixels per sensor.

Associated element DMS-REQ-0359-V-13
(\href{https://jira.lsstcorp.org/browse/LVV-9762}{LVV-9762}) satisfies
the maximum allowable precision in the sky brightness determination.

Associated element DMS-REQ-0359-V-14
(\href{https://jira.lsstcorp.org/browse/LVV-9763}{LVV-9763}) satisfies
the rms width of the internal photometric zero-point in g, r, i, z, and
y-bands.

Associated element DMS-REQ-0359-V-15
(\href{https://jira.lsstcorp.org/browse/LVV-9764}{LVV-9764}) satisfies
the percentage of the image area affected by ghosts that exceed the
threshold.

Associated element DMS-REQ-0359-V-16
(\href{https://jira.lsstcorp.org/browse/LVV-9765}{LVV-9765}) satisfies
the accuracy of zero-point for colors that do not include the u-band.

Associated element DMS-REQ-0359-V-17
(\href{https://jira.lsstcorp.org/browse/LVV-9766}{LVV-9766}) satisfies
the maximum RMS of the ratio of the flux measurement error between
resolved/unresolved sources.

{\footnotesize
\begin{longtable}{p{2.5cm}p{13.5cm}}
\hline
\multicolumn{2}{c}{\textbf{Requirement Details}}\\ \hline
Requirement ID & DMS-REQ-0359 \\ \cdashline{1-2}
Requirement Description &
\begin{minipage}[]{13cm}
\textbf{Specification:} The DMS shall include software to enable the
calculation of the photometric performance metrics defined in
OSS-REQ-0387.
\end{minipage}
\\ \cdashline{1-2}
Requirement Parameters & {[}\textbf{GhostAF = 1{{[}percent{]}}} Percentage of image area that can
have ghosts with surface brightness gradient amplitude of more than 1/3
of the sky noise over 1 arcsec., \textbf{PF1 = 10{{[}percent{]}}} The
maximum fraction of isolated non-saturated point source measurements
exceeding the outlier limit., \textbf{PA1gri = 5{{[}millimagnitude{]}}}
The RMS photometric repeatability of bright non-saturated unresolved
point sources in the g, r, and i filters., \textbf{PA3 =
10{{[}millimagnitude{]}}} RMS width of internal photometric zero-point
(precision of system uniformity across the sky) for all bands except
u-band., \textbf{PA4 = 15{{[}millimagnitude{]}}} The zero point error
outlier limit., \textbf{PF2 = 10{{[}percent{]}}} Fraction of zeropoint
errors that can exceed the zero point error outlier limit.,
\textbf{PixFrac = 1{{[}percent{]}}} The maximum fraction of pixels
scientifically unusable per sensor out of the total allowable fraction
of sensors meeting this performance., \textbf{PA1uzy =
7.5{{[}millimagnitude{]}}} The RMS photometric repeatability of bright
non-saturated unresolved point sources in the u, z, and y filters.,
\textbf{PA6 = 10{{[}millimagnitude{]}}} Accuracy of the transformation
of the internal LSST photometry to a physical scale (e.g. AB
magnitudes)., \textbf{Xtalk = 3{{[}sigma{]}}} The maximum local
significance integrated over the PSF of imperfect crosstalk
corrections., \textbf{PA5u = 10{{[}millimagnitude{]}}} Accuracy of
absolute band-to-band color zero-point for colors constructed using the
u-band., \textbf{ResSource = 2{{[}unitless{]}}} Maximum RMS of the ratio
of the error in integrated flux measurement between bright, isolated,
resolved sources less than 10 arcsec in diameter and bright, isolated
unresolved point sources., \textbf{PA2uzy = 22.5{{[}millimagnitude{]}}}
Repeatability outlier limit for isolated bright non-saturated point
sources in the u, z, and y filters., \textbf{SensorFraction =
15{{[}percent{]}}} The maximum allowable fraction of sensors with
PixFrac scientifically unusable pixels., \textbf{PA3u =
20{{[}millimagnitude{]}}} RMS width of internal photometric zero-point
(precision of system uniformity across the sky) in the u-band.,
\textbf{PA2gri = 15{{[}millimagnitude{]}}} Repeatability outlier limit
for isolated bright non-saturated point sources in the g, r, and i
filters., \textbf{SBPrec = 1{{[}percent{]}}} The maximum error in the
precision of the sky brightness determination., \textbf{PA5 =
5{{[}millimagnitude{]}}} Accuracy of absolute band-to-band color
zero-point for all colors constructed from any filter pair, excluding
the u-band.{]} \\ \cdashline{1-2}
Requirement Discussion &
\begin{minipage}[]{13cm}
\textbf{Discussion:} The relevant metrics are listed in the table
photometricPerformance below. The values in the tables are the target
values for LSST but are not verified as part of this requirement.
\end{minipage}
\\ \cdashline{1-2}
Requirement Priority & 1a \\ \cdashline{1-2}
Upper Level Requirement &
\begin{tabular}{cl}
OSS-REQ-0387 & Photometric Performance \\
\end{tabular}
\\ \hline
\end{longtable}
}


\subsubsection{Test Cases Summary}
\begin{longtable}{p{3cm}p{2.5cm}p{2.5cm}p{3cm}p{4cm}}
\toprule
\href{https://jira.lsstcorp.org/secure/Tests.jspa\#/testCase/LVV-T1757}{LVV-T1757} & \multicolumn{4}{p{12cm}}{ Verify calculation of photometric repeatability in gri filters } \\ \hline
\textbf{Owner} & \textbf{Status} & \textbf{Version} & \textbf{Critical Event} & \textbf{Verification Type} \\ \hline
Jeffrey Carlin & Approved & 1 & false & Test \\ \hline
\end{longtable}
{\scriptsize
\textbf{Objective:}\\
Verify that the DM system has provided the code to calculate the RMS
photometric repeatability of bright non-saturated unresolved point
sources in the g, r, and i filters, and assess whether it meets the
requirement that it shall be less than \textbf{PA1gri = 5.0
millimagnitudes}.
}
  
 \newpage 
\subsection{[LVV-9760] DMS-REQ-0359-V-11: Fraction of zero point outliers }\label{lvv-9760}

\begin{longtable}{cccc}
\hline
\textbf{Jira Link} & \textbf{Assignee} & \textbf{Status} & \textbf{Test Cases}\\ \hline
\href{https://jira.lsstcorp.org/browse/LVV-9760}{LVV-9760} &
Leanne Guy & Not Covered &
\begin{tabular}{c}
LVV-T377 \\
LVV-T1842 \\
\end{tabular}
\\
\hline
\end{longtable}

\textbf{Verification Element Description:} \\
The fraction of zeropoint errors that can exceed the zero point error
outlier limit is less than \textbf{PF2 = 10 percent.}

Associated element DMS-REQ-0359-V-01
(\href{https://jira.lsstcorp.org/browse/LVV-3401}{LVV-3401}) satisfies
the requirement on photometric repeatability in the u, z, and y-band
filters.

Associated element DMS-REQ-0359-V-02
(\href{https://jira.lsstcorp.org/browse/LVV-9751}{LVV-9751}) satisfies
the requirement on the maximum fraction of sensors with scientifically
unusable pixels.

Associated element DMS-REQ-0359-V-03
(\href{https://jira.lsstcorp.org/browse/LVV-9752}{LVV-9752}) satisfies
the constraint on maximum fraction of outliers among non-saturated point
sources.

Associated element DMS-REQ-0359-V-04
(\href{https://jira.lsstcorp.org/browse/LVV-9753}{LVV-9753}) satisfies
the accuracy of zero-point for colors that use the u-band.

Associated element DMS-REQ-0359-V-05
(\href{https://jira.lsstcorp.org/browse/LVV-9754}{LVV-9754}) satisfies
the repeatability outlier limit in g, r, and i-bands.

Associated element DMS-REQ-0359-V-06
(\href{https://jira.lsstcorp.org/browse/LVV-9755}{LVV-9755}) satisfies
the constraint on the accuracy of the transformation from internal to
physical photometric scales.

Associated element DMS-REQ-0359-V-07
(\href{https://jira.lsstcorp.org/browse/LVV-9756}{LVV-9756}) satisfies
the rms width of the internal photometric zero-point in u-band.

Associated element DMS-REQ-0359-V-08
(\href{https://jira.lsstcorp.org/browse/LVV-9757}{LVV-9757}) satisfies
the maximum local significance of imperfect crosstalk corrections.

Associated element DMS-REQ-0359-V-09
(\href{https://jira.lsstcorp.org/browse/LVV-9758}{LVV-9758}) satisfies
the repeatability outlier limit in u, z, and y-bands.

Associated element DMS-REQ-0359-V-10
(\href{https://jira.lsstcorp.org/browse/LVV-9759}{LVV-9759}) satisfies
the rms photometric repeatability in g, r, and i-bands.

Associated element DMS-REQ-0359-V-12
(\href{https://jira.lsstcorp.org/browse/LVV-9761}{LVV-9761}) satisfies
the maximum fraction of unusable pixels per sensor.

Associated element DMS-REQ-0359-V-13
(\href{https://jira.lsstcorp.org/browse/LVV-9762}{LVV-9762}) satisfies
the maximum allowable precision in the sky brightness determination.

Associated element DMS-REQ-0359-V-14
(\href{https://jira.lsstcorp.org/browse/LVV-9763}{LVV-9763}) satisfies
the rms width of the internal photometric zero-point in g, r, i, z, and
y-bands.

Associated element DMS-REQ-0359-V-15
(\href{https://jira.lsstcorp.org/browse/LVV-9764}{LVV-9764}) satisfies
the percentage of the image area affected by ghosts that exceed the
threshold.

Associated element DMS-REQ-0359-V-16
(\href{https://jira.lsstcorp.org/browse/LVV-9765}{LVV-9765}) satisfies
the accuracy of zero-point for colors that do not include the u-band.

Associated element DMS-REQ-0359-V-17
(\href{https://jira.lsstcorp.org/browse/LVV-9766}{LVV-9766}) satisfies
the maximum RMS of the ratio of the flux measurement error between
resolved/unresolved sources.

{\footnotesize
\begin{longtable}{p{2.5cm}p{13.5cm}}
\hline
\multicolumn{2}{c}{\textbf{Requirement Details}}\\ \hline
Requirement ID & DMS-REQ-0359 \\ \cdashline{1-2}
Requirement Description &
\begin{minipage}[]{13cm}
\textbf{Specification:} The DMS shall include software to enable the
calculation of the photometric performance metrics defined in
OSS-REQ-0387.
\end{minipage}
\\ \cdashline{1-2}
Requirement Parameters & {[}\textbf{GhostAF = 1{{[}percent{]}}} Percentage of image area that can
have ghosts with surface brightness gradient amplitude of more than 1/3
of the sky noise over 1 arcsec., \textbf{PF1 = 10{{[}percent{]}}} The
maximum fraction of isolated non-saturated point source measurements
exceeding the outlier limit., \textbf{PA1gri = 5{{[}millimagnitude{]}}}
The RMS photometric repeatability of bright non-saturated unresolved
point sources in the g, r, and i filters., \textbf{PA3 =
10{{[}millimagnitude{]}}} RMS width of internal photometric zero-point
(precision of system uniformity across the sky) for all bands except
u-band., \textbf{PA4 = 15{{[}millimagnitude{]}}} The zero point error
outlier limit., \textbf{PF2 = 10{{[}percent{]}}} Fraction of zeropoint
errors that can exceed the zero point error outlier limit.,
\textbf{PixFrac = 1{{[}percent{]}}} The maximum fraction of pixels
scientifically unusable per sensor out of the total allowable fraction
of sensors meeting this performance., \textbf{PA1uzy =
7.5{{[}millimagnitude{]}}} The RMS photometric repeatability of bright
non-saturated unresolved point sources in the u, z, and y filters.,
\textbf{PA6 = 10{{[}millimagnitude{]}}} Accuracy of the transformation
of the internal LSST photometry to a physical scale (e.g. AB
magnitudes)., \textbf{Xtalk = 3{{[}sigma{]}}} The maximum local
significance integrated over the PSF of imperfect crosstalk
corrections., \textbf{PA5u = 10{{[}millimagnitude{]}}} Accuracy of
absolute band-to-band color zero-point for colors constructed using the
u-band., \textbf{ResSource = 2{{[}unitless{]}}} Maximum RMS of the ratio
of the error in integrated flux measurement between bright, isolated,
resolved sources less than 10 arcsec in diameter and bright, isolated
unresolved point sources., \textbf{PA2uzy = 22.5{{[}millimagnitude{]}}}
Repeatability outlier limit for isolated bright non-saturated point
sources in the u, z, and y filters., \textbf{SensorFraction =
15{{[}percent{]}}} The maximum allowable fraction of sensors with
PixFrac scientifically unusable pixels., \textbf{PA3u =
20{{[}millimagnitude{]}}} RMS width of internal photometric zero-point
(precision of system uniformity across the sky) in the u-band.,
\textbf{PA2gri = 15{{[}millimagnitude{]}}} Repeatability outlier limit
for isolated bright non-saturated point sources in the g, r, and i
filters., \textbf{SBPrec = 1{{[}percent{]}}} The maximum error in the
precision of the sky brightness determination., \textbf{PA5 =
5{{[}millimagnitude{]}}} Accuracy of absolute band-to-band color
zero-point for all colors constructed from any filter pair, excluding
the u-band.{]} \\ \cdashline{1-2}
Requirement Discussion &
\begin{minipage}[]{13cm}
\textbf{Discussion:} The relevant metrics are listed in the table
photometricPerformance below. The values in the tables are the target
values for LSST but are not verified as part of this requirement.
\end{minipage}
\\ \cdashline{1-2}
Requirement Priority & 1a \\ \cdashline{1-2}
Upper Level Requirement &
\begin{tabular}{cl}
OSS-REQ-0387 & Photometric Performance \\
\end{tabular}
\\ \hline
\end{longtable}
}


\subsubsection{Test Cases Summary}
\begin{longtable}{p{3cm}p{2.5cm}p{2.5cm}p{3cm}p{4cm}}
\toprule
\href{https://jira.lsstcorp.org/secure/Tests.jspa\#/testCase/LVV-T377}{LVV-T377} & \multicolumn{4}{p{12cm}}{ Verify Calculation of Photometric Performance Metrics } \\ \hline
\textbf{Owner} & \textbf{Status} & \textbf{Version} & \textbf{Critical Event} & \textbf{Verification Type} \\ \hline
Leanne Guy & Approved & 1 & false & Test \\ \hline
\end{longtable}
{\scriptsize
\textbf{Objective:}\\
Verify that the DMS system provides software to calculate photometric
performance metrics, and that the algorithms are properly calculating
the desired quantities. Note that because the DMS requirement is that
the software shall be provided (and not on the actual measured values of
the metrics), we verify all of the requirements via a single test case.
}
\begin{longtable}{p{3cm}p{2.5cm}p{2.5cm}p{3cm}p{4cm}}
\toprule
\href{https://jira.lsstcorp.org/secure/Tests.jspa\#/testCase/LVV-T1842}{LVV-T1842} & \multicolumn{4}{p{12cm}}{ Verify calculation of zeropoint error fraction exceeding the outlier
limit } \\ \hline
\textbf{Owner} & \textbf{Status} & \textbf{Version} & \textbf{Critical Event} & \textbf{Verification Type} \\ \hline
Jeffrey Carlin & Draft & 1 & false & Test \\ \hline
\end{longtable}
{\scriptsize
\textbf{Objective:}\\
Verify that the DM system provides software to calculate the fraction of
zeropoint errors that exceed the zero point error outlier limit, and
confirm that it is less than \textbf{PF2 = 10 percent.}
}
  
 \newpage 
\subsection{[LVV-9761] DMS-REQ-0359-V-12: Max fraction of unusable pixels per sensor }\label{lvv-9761}

\begin{longtable}{cccc}
\hline
\textbf{Jira Link} & \textbf{Assignee} & \textbf{Status} & \textbf{Test Cases}\\ \hline
\href{https://jira.lsstcorp.org/browse/LVV-9761}{LVV-9761} &
Leanne Guy & Not Covered &
\begin{tabular}{c}
LVV-T377 \\
LVV-T1841 \\
\end{tabular}
\\
\hline
\end{longtable}

\textbf{Verification Element Description:} \\
The maximum fraction of pixels scientifically unusable per sensor out of
the total allowable fraction of sensors meeting this performance shall
be \textbf{PixFrac = 1 percent}.

Associated element DMS-REQ-0359-V-01
(\href{https://jira.lsstcorp.org/browse/LVV-3401}{LVV-3401}) satisfies
the requirement on photometric repeatability in the u, z, and y-band
filters.

Associated element DMS-REQ-0359-V-02
(\href{https://jira.lsstcorp.org/browse/LVV-9751}{LVV-9751}) satisfies
the requirement on the maximum fraction of sensors with scientifically
unusable pixels.

Associated element DMS-REQ-0359-V-03
(\href{https://jira.lsstcorp.org/browse/LVV-9752}{LVV-9752}) satisfies
the constraint on maximum fraction of outliers among non-saturated point
sources.

Associated element DMS-REQ-0359-V-04
(\href{https://jira.lsstcorp.org/browse/LVV-9753}{LVV-9753}) satisfies
the accuracy of zero-point for colors that use the u-band.

Associated element DMS-REQ-0359-V-05
(\href{https://jira.lsstcorp.org/browse/LVV-9754}{LVV-9754}) satisfies
the repeatability outlier limit in g, r, and i-bands.

Associated element DMS-REQ-0359-V-06
(\href{https://jira.lsstcorp.org/browse/LVV-9755}{LVV-9755}) satisfies
the constraint on the accuracy of the transformation from internal to
physical photometric scales.

Associated element DMS-REQ-0359-V-07
(\href{https://jira.lsstcorp.org/browse/LVV-9756}{LVV-9756}) satisfies
the rms width of the internal photometric zero-point in u-band.

Associated element DMS-REQ-0359-V-08
(\href{https://jira.lsstcorp.org/browse/LVV-9757}{LVV-9757}) satisfies
the maximum local significance of imperfect crosstalk corrections.

Associated element DMS-REQ-0359-V-09
(\href{https://jira.lsstcorp.org/browse/LVV-9758}{LVV-9758}) satisfies
the repeatability outlier limit in u, z, and y-bands.

Associated element DMS-REQ-0359-V-10
(\href{https://jira.lsstcorp.org/browse/LVV-9759}{LVV-9759}) satisfies
the rms photometric repeatability in g, r, and i-bands.

Associated element DMS-REQ-0359-V-11
(\href{https://jira.lsstcorp.org/browse/LVV-9760}{LVV-9760}) satisfies
the fraction of zero-point errors that can exceed the outlier limit.

Associated element DMS-REQ-0359-V-13
(\href{https://jira.lsstcorp.org/browse/LVV-9762}{LVV-9762}) satisfies
the maximum allowable precision in the sky brightness determination.

Associated element DMS-REQ-0359-V-14
(\href{https://jira.lsstcorp.org/browse/LVV-9763}{LVV-9763}) satisfies
the rms width of the internal photometric zero-point in g, r, i, z, and
y-bands.

Associated element DMS-REQ-0359-V-15
(\href{https://jira.lsstcorp.org/browse/LVV-9764}{LVV-9764}) satisfies
the percentage of the image area affected by ghosts that exceed the
threshold.

Associated element DMS-REQ-0359-V-16
(\href{https://jira.lsstcorp.org/browse/LVV-9765}{LVV-9765}) satisfies
the accuracy of zero-point for colors that do not include the u-band.

Associated element DMS-REQ-0359-V-17
(\href{https://jira.lsstcorp.org/browse/LVV-9766}{LVV-9766}) satisfies
the maximum RMS of the ratio of the flux measurement error between
resolved/unresolved sources.

{\footnotesize
\begin{longtable}{p{2.5cm}p{13.5cm}}
\hline
\multicolumn{2}{c}{\textbf{Requirement Details}}\\ \hline
Requirement ID & DMS-REQ-0359 \\ \cdashline{1-2}
Requirement Description &
\begin{minipage}[]{13cm}
\textbf{Specification:} The DMS shall include software to enable the
calculation of the photometric performance metrics defined in
OSS-REQ-0387.
\end{minipage}
\\ \cdashline{1-2}
Requirement Parameters & {[}\textbf{GhostAF = 1{{[}percent{]}}} Percentage of image area that can
have ghosts with surface brightness gradient amplitude of more than 1/3
of the sky noise over 1 arcsec., \textbf{PF1 = 10{{[}percent{]}}} The
maximum fraction of isolated non-saturated point source measurements
exceeding the outlier limit., \textbf{PA1gri = 5{{[}millimagnitude{]}}}
The RMS photometric repeatability of bright non-saturated unresolved
point sources in the g, r, and i filters., \textbf{PA3 =
10{{[}millimagnitude{]}}} RMS width of internal photometric zero-point
(precision of system uniformity across the sky) for all bands except
u-band., \textbf{PA4 = 15{{[}millimagnitude{]}}} The zero point error
outlier limit., \textbf{PF2 = 10{{[}percent{]}}} Fraction of zeropoint
errors that can exceed the zero point error outlier limit.,
\textbf{PixFrac = 1{{[}percent{]}}} The maximum fraction of pixels
scientifically unusable per sensor out of the total allowable fraction
of sensors meeting this performance., \textbf{PA1uzy =
7.5{{[}millimagnitude{]}}} The RMS photometric repeatability of bright
non-saturated unresolved point sources in the u, z, and y filters.,
\textbf{PA6 = 10{{[}millimagnitude{]}}} Accuracy of the transformation
of the internal LSST photometry to a physical scale (e.g. AB
magnitudes)., \textbf{Xtalk = 3{{[}sigma{]}}} The maximum local
significance integrated over the PSF of imperfect crosstalk
corrections., \textbf{PA5u = 10{{[}millimagnitude{]}}} Accuracy of
absolute band-to-band color zero-point for colors constructed using the
u-band., \textbf{ResSource = 2{{[}unitless{]}}} Maximum RMS of the ratio
of the error in integrated flux measurement between bright, isolated,
resolved sources less than 10 arcsec in diameter and bright, isolated
unresolved point sources., \textbf{PA2uzy = 22.5{{[}millimagnitude{]}}}
Repeatability outlier limit for isolated bright non-saturated point
sources in the u, z, and y filters., \textbf{SensorFraction =
15{{[}percent{]}}} The maximum allowable fraction of sensors with
PixFrac scientifically unusable pixels., \textbf{PA3u =
20{{[}millimagnitude{]}}} RMS width of internal photometric zero-point
(precision of system uniformity across the sky) in the u-band.,
\textbf{PA2gri = 15{{[}millimagnitude{]}}} Repeatability outlier limit
for isolated bright non-saturated point sources in the g, r, and i
filters., \textbf{SBPrec = 1{{[}percent{]}}} The maximum error in the
precision of the sky brightness determination., \textbf{PA5 =
5{{[}millimagnitude{]}}} Accuracy of absolute band-to-band color
zero-point for all colors constructed from any filter pair, excluding
the u-band.{]} \\ \cdashline{1-2}
Requirement Discussion &
\begin{minipage}[]{13cm}
\textbf{Discussion:} The relevant metrics are listed in the table
photometricPerformance below. The values in the tables are the target
values for LSST but are not verified as part of this requirement.
\end{minipage}
\\ \cdashline{1-2}
Requirement Priority & 1a \\ \cdashline{1-2}
Upper Level Requirement &
\begin{tabular}{cl}
OSS-REQ-0387 & Photometric Performance \\
\end{tabular}
\\ \hline
\end{longtable}
}


\subsubsection{Test Cases Summary}
\begin{longtable}{p{3cm}p{2.5cm}p{2.5cm}p{3cm}p{4cm}}
\toprule
\href{https://jira.lsstcorp.org/secure/Tests.jspa\#/testCase/LVV-T377}{LVV-T377} & \multicolumn{4}{p{12cm}}{ Verify Calculation of Photometric Performance Metrics } \\ \hline
\textbf{Owner} & \textbf{Status} & \textbf{Version} & \textbf{Critical Event} & \textbf{Verification Type} \\ \hline
Leanne Guy & Approved & 1 & false & Test \\ \hline
\end{longtable}
{\scriptsize
\textbf{Objective:}\\
Verify that the DMS system provides software to calculate photometric
performance metrics, and that the algorithms are properly calculating
the desired quantities. Note that because the DMS requirement is that
the software shall be provided (and not on the actual measured values of
the metrics), we verify all of the requirements via a single test case.
}
\begin{longtable}{p{3cm}p{2.5cm}p{2.5cm}p{3cm}p{4cm}}
\toprule
\href{https://jira.lsstcorp.org/secure/Tests.jspa\#/testCase/LVV-T1841}{LVV-T1841} & \multicolumn{4}{p{12cm}}{ Verify calculation of scientifically unusable pixel fraction } \\ \hline
\textbf{Owner} & \textbf{Status} & \textbf{Version} & \textbf{Critical Event} & \textbf{Verification Type} \\ \hline
Jeffrey Carlin & Draft & 1 & false & Test \\ \hline
\end{longtable}
{\scriptsize
\textbf{Objective:}\\
Verify that the DM system provides software to assess whether the
maximum fraction of pixels scientifically unusable per sensor out of the
total allowable fraction of sensors meeting this performance is less
than~\textbf{PixFrac = 1 percent}.
}
  
 \newpage 
\subsection{[LVV-9762] DMS-REQ-0359-V-13: Max sky brightness error }\label{lvv-9762}

\begin{longtable}{cccc}
\hline
\textbf{Jira Link} & \textbf{Assignee} & \textbf{Status} & \textbf{Test Cases}\\ \hline
\href{https://jira.lsstcorp.org/browse/LVV-9762}{LVV-9762} &
Leanne Guy & Not Covered &
\begin{tabular}{c}
LVV-T377 \\
LVV-T1840 \\
\end{tabular}
\\
\hline
\end{longtable}

\textbf{Verification Element Description:} \\
The maximum error in the precision of the sky brightness determination
shall be less than \textbf{SBPrec = 1 percent.}

Associated element DMS-REQ-0359-V-01
(\href{https://jira.lsstcorp.org/browse/LVV-3401}{LVV-3401}) satisfies
the requirement on photometric repeatability in the u, z, and y-band
filters.

Associated element DMS-REQ-0359-V-02
(\href{https://jira.lsstcorp.org/browse/LVV-9751}{LVV-9751}) satisfies
the requirement on the maximum fraction of sensors with scientifically
unusable pixels.

Associated element DMS-REQ-0359-V-03
(\href{https://jira.lsstcorp.org/browse/LVV-9752}{LVV-9752}) satisfies
the constraint on maximum fraction of outliers among non-saturated point
sources.

Associated element DMS-REQ-0359-V-04
(\href{https://jira.lsstcorp.org/browse/LVV-9753}{LVV-9753}) satisfies
the accuracy of zero-point for colors that use the u-band.

Associated element DMS-REQ-0359-V-05
(\href{https://jira.lsstcorp.org/browse/LVV-9754}{LVV-9754}) satisfies
the repeatability outlier limit in g, r, and i-bands.

Associated element DMS-REQ-0359-V-06
(\href{https://jira.lsstcorp.org/browse/LVV-9755}{LVV-9755}) satisfies
the constraint on the accuracy of the transformation from internal to
physical photometric scales.

Associated element DMS-REQ-0359-V-07
(\href{https://jira.lsstcorp.org/browse/LVV-9756}{LVV-9756}) satisfies
the rms width of the internal photometric zero-point in u-band.

Associated element DMS-REQ-0359-V-08
(\href{https://jira.lsstcorp.org/browse/LVV-9757}{LVV-9757}) satisfies
the maximum local significance of imperfect crosstalk corrections.

Associated element DMS-REQ-0359-V-09
(\href{https://jira.lsstcorp.org/browse/LVV-9758}{LVV-9758}) satisfies
the repeatability outlier limit in u, z, and y-bands.

Associated element DMS-REQ-0359-V-10
(\href{https://jira.lsstcorp.org/browse/LVV-9759}{LVV-9759}) satisfies
the rms photometric repeatability in g, r, and i-bands.

Associated element DMS-REQ-0359-V-11
(\href{https://jira.lsstcorp.org/browse/LVV-9760}{LVV-9760}) satisfies
the fraction of zero-point errors that can exceed the outlier limit.

Associated element DMS-REQ-0359-V-12
(\href{https://jira.lsstcorp.org/browse/LVV-9761}{LVV-9761}) satisfies
the maximum fraction of unusable pixels per sensor.

Associated element DMS-REQ-0359-V-14
(\href{https://jira.lsstcorp.org/browse/LVV-9763}{LVV-9763}) satisfies
the rms width of the internal photometric zero-point in g, r, i, z, and
y-bands.

Associated element DMS-REQ-0359-V-15
(\href{https://jira.lsstcorp.org/browse/LVV-9764}{LVV-9764}) satisfies
the percentage of the image area affected by ghosts that exceed the
threshold.

Associated element DMS-REQ-0359-V-16
(\href{https://jira.lsstcorp.org/browse/LVV-9765}{LVV-9765}) satisfies
the accuracy of zero-point for colors that do not include the u-band.

Associated element DMS-REQ-0359-V-17
(\href{https://jira.lsstcorp.org/browse/LVV-9766}{LVV-9766}) satisfies
the maximum RMS of the ratio of the flux measurement error between
resolved/unresolved sources.

{\footnotesize
\begin{longtable}{p{2.5cm}p{13.5cm}}
\hline
\multicolumn{2}{c}{\textbf{Requirement Details}}\\ \hline
Requirement ID & DMS-REQ-0359 \\ \cdashline{1-2}
Requirement Description &
\begin{minipage}[]{13cm}
\textbf{Specification:} The DMS shall include software to enable the
calculation of the photometric performance metrics defined in
OSS-REQ-0387.
\end{minipage}
\\ \cdashline{1-2}
Requirement Parameters & {[}\textbf{GhostAF = 1{{[}percent{]}}} Percentage of image area that can
have ghosts with surface brightness gradient amplitude of more than 1/3
of the sky noise over 1 arcsec., \textbf{PF1 = 10{{[}percent{]}}} The
maximum fraction of isolated non-saturated point source measurements
exceeding the outlier limit., \textbf{PA1gri = 5{{[}millimagnitude{]}}}
The RMS photometric repeatability of bright non-saturated unresolved
point sources in the g, r, and i filters., \textbf{PA3 =
10{{[}millimagnitude{]}}} RMS width of internal photometric zero-point
(precision of system uniformity across the sky) for all bands except
u-band., \textbf{PA4 = 15{{[}millimagnitude{]}}} The zero point error
outlier limit., \textbf{PF2 = 10{{[}percent{]}}} Fraction of zeropoint
errors that can exceed the zero point error outlier limit.,
\textbf{PixFrac = 1{{[}percent{]}}} The maximum fraction of pixels
scientifically unusable per sensor out of the total allowable fraction
of sensors meeting this performance., \textbf{PA1uzy =
7.5{{[}millimagnitude{]}}} The RMS photometric repeatability of bright
non-saturated unresolved point sources in the u, z, and y filters.,
\textbf{PA6 = 10{{[}millimagnitude{]}}} Accuracy of the transformation
of the internal LSST photometry to a physical scale (e.g. AB
magnitudes)., \textbf{Xtalk = 3{{[}sigma{]}}} The maximum local
significance integrated over the PSF of imperfect crosstalk
corrections., \textbf{PA5u = 10{{[}millimagnitude{]}}} Accuracy of
absolute band-to-band color zero-point for colors constructed using the
u-band., \textbf{ResSource = 2{{[}unitless{]}}} Maximum RMS of the ratio
of the error in integrated flux measurement between bright, isolated,
resolved sources less than 10 arcsec in diameter and bright, isolated
unresolved point sources., \textbf{PA2uzy = 22.5{{[}millimagnitude{]}}}
Repeatability outlier limit for isolated bright non-saturated point
sources in the u, z, and y filters., \textbf{SensorFraction =
15{{[}percent{]}}} The maximum allowable fraction of sensors with
PixFrac scientifically unusable pixels., \textbf{PA3u =
20{{[}millimagnitude{]}}} RMS width of internal photometric zero-point
(precision of system uniformity across the sky) in the u-band.,
\textbf{PA2gri = 15{{[}millimagnitude{]}}} Repeatability outlier limit
for isolated bright non-saturated point sources in the g, r, and i
filters., \textbf{SBPrec = 1{{[}percent{]}}} The maximum error in the
precision of the sky brightness determination., \textbf{PA5 =
5{{[}millimagnitude{]}}} Accuracy of absolute band-to-band color
zero-point for all colors constructed from any filter pair, excluding
the u-band.{]} \\ \cdashline{1-2}
Requirement Discussion &
\begin{minipage}[]{13cm}
\textbf{Discussion:} The relevant metrics are listed in the table
photometricPerformance below. The values in the tables are the target
values for LSST but are not verified as part of this requirement.
\end{minipage}
\\ \cdashline{1-2}
Requirement Priority & 1a \\ \cdashline{1-2}
Upper Level Requirement &
\begin{tabular}{cl}
OSS-REQ-0387 & Photometric Performance \\
\end{tabular}
\\ \hline
\end{longtable}
}


\subsubsection{Test Cases Summary}
\begin{longtable}{p{3cm}p{2.5cm}p{2.5cm}p{3cm}p{4cm}}
\toprule
\href{https://jira.lsstcorp.org/secure/Tests.jspa\#/testCase/LVV-T377}{LVV-T377} & \multicolumn{4}{p{12cm}}{ Verify Calculation of Photometric Performance Metrics } \\ \hline
\textbf{Owner} & \textbf{Status} & \textbf{Version} & \textbf{Critical Event} & \textbf{Verification Type} \\ \hline
Leanne Guy & Approved & 1 & false & Test \\ \hline
\end{longtable}
{\scriptsize
\textbf{Objective:}\\
Verify that the DMS system provides software to calculate photometric
performance metrics, and that the algorithms are properly calculating
the desired quantities. Note that because the DMS requirement is that
the software shall be provided (and not on the actual measured values of
the metrics), we verify all of the requirements via a single test case.
}
\begin{longtable}{p{3cm}p{2.5cm}p{2.5cm}p{3cm}p{4cm}}
\toprule
\href{https://jira.lsstcorp.org/secure/Tests.jspa\#/testCase/LVV-T1840}{LVV-T1840} & \multicolumn{4}{p{12cm}}{ Verify calculation of sky brightness precision } \\ \hline
\textbf{Owner} & \textbf{Status} & \textbf{Version} & \textbf{Critical Event} & \textbf{Verification Type} \\ \hline
Jeffrey Carlin & Draft & 1 & false & Test \\ \hline
\end{longtable}
{\scriptsize
\textbf{Objective:}\\
Verify that the DM system provides software to assess whether the
maximum error in the precision of the sky brightness determination is
less than \textbf{SBPrec = 1 percent.}
}
  
 \newpage 
\subsection{[LVV-9763] DMS-REQ-0359-V-14: RMS width of zero point in all bands except u }\label{lvv-9763}

\begin{longtable}{cccc}
\hline
\textbf{Jira Link} & \textbf{Assignee} & \textbf{Status} & \textbf{Test Cases}\\ \hline
\href{https://jira.lsstcorp.org/browse/LVV-9763}{LVV-9763} &
Leanne Guy & Not Covered &
\begin{tabular}{c}
LVV-T377 \\
LVV-T1839 \\
\end{tabular}
\\
\hline
\end{longtable}

\textbf{Verification Element Description:} \\
The RMS width of the internal photometric zero-point (precision of
system uniformity across the sky) for all bands except u-band shall be
less than \textbf{PA3 = 10 millimagnitudes}.

Associated element DMS-REQ-0359-V-01
(\href{https://jira.lsstcorp.org/browse/LVV-3401}{LVV-3401}) satisfies
the requirement on photometric repeatability in the u, z, and y-band
filters.

Associated element DMS-REQ-0359-V-02
(\href{https://jira.lsstcorp.org/browse/LVV-9751}{LVV-9751}) satisfies
the requirement on the maximum fraction of sensors with scientifically
unusable pixels.

Associated element DMS-REQ-0359-V-03
(\href{https://jira.lsstcorp.org/browse/LVV-9752}{LVV-9752}) satisfies
the constraint on maximum fraction of outliers among non-saturated point
sources.

Associated element DMS-REQ-0359-V-04
(\href{https://jira.lsstcorp.org/browse/LVV-9753}{LVV-9753}) satisfies
the accuracy of zero-point for colors that use the u-band.

Associated element DMS-REQ-0359-V-05
(\href{https://jira.lsstcorp.org/browse/LVV-9754}{LVV-9754}) satisfies
the repeatability outlier limit in g, r, and i-bands.

Associated element DMS-REQ-0359-V-06
(\href{https://jira.lsstcorp.org/browse/LVV-9755}{LVV-9755}) satisfies
the constraint on the accuracy of the transformation from internal to
physical photometric scales.

Associated element DMS-REQ-0359-V-07
(\href{https://jira.lsstcorp.org/browse/LVV-9756}{LVV-9756}) satisfies
the rms width of the internal photometric zero-point in u-band.

Associated element DMS-REQ-0359-V-08
(\href{https://jira.lsstcorp.org/browse/LVV-9757}{LVV-9757}) satisfies
the maximum local significance of imperfect crosstalk corrections.

Associated element DMS-REQ-0359-V-09
(\href{https://jira.lsstcorp.org/browse/LVV-9758}{LVV-9758}) satisfies
the repeatability outlier limit in u, z, and y-bands.

Associated element DMS-REQ-0359-V-10
(\href{https://jira.lsstcorp.org/browse/LVV-9759}{LVV-9759}) satisfies
the rms photometric repeatability in g, r, and i-bands.

Associated element DMS-REQ-0359-V-11
(\href{https://jira.lsstcorp.org/browse/LVV-9760}{LVV-9760}) satisfies
the fraction of zero-point errors that can exceed the outlier limit.

Associated element DMS-REQ-0359-V-12
(\href{https://jira.lsstcorp.org/browse/LVV-9761}{LVV-9761}) satisfies
the maximum fraction of unusable pixels per sensor.

Associated element DMS-REQ-0359-V-13
(\href{https://jira.lsstcorp.org/browse/LVV-9762}{LVV-9762}) satisfies
the maximum allowable precision in the sky brightness determination.

Associated element DMS-REQ-0359-V-15
(\href{https://jira.lsstcorp.org/browse/LVV-9764}{LVV-9764}) satisfies
the percentage of the image area affected by ghosts that exceed the
threshold.

Associated element DMS-REQ-0359-V-16
(\href{https://jira.lsstcorp.org/browse/LVV-9765}{LVV-9765}) satisfies
the accuracy of zero-point for colors that do not include the u-band.

Associated element DMS-REQ-0359-V-17
(\href{https://jira.lsstcorp.org/browse/LVV-9766}{LVV-9766}) satisfies
the maximum RMS of the ratio of the flux measurement error between
resolved/unresolved sources.

{\footnotesize
\begin{longtable}{p{2.5cm}p{13.5cm}}
\hline
\multicolumn{2}{c}{\textbf{Requirement Details}}\\ \hline
Requirement ID & DMS-REQ-0359 \\ \cdashline{1-2}
Requirement Description &
\begin{minipage}[]{13cm}
\textbf{Specification:} The DMS shall include software to enable the
calculation of the photometric performance metrics defined in
OSS-REQ-0387.
\end{minipage}
\\ \cdashline{1-2}
Requirement Parameters & {[}\textbf{GhostAF = 1{{[}percent{]}}} Percentage of image area that can
have ghosts with surface brightness gradient amplitude of more than 1/3
of the sky noise over 1 arcsec., \textbf{PF1 = 10{{[}percent{]}}} The
maximum fraction of isolated non-saturated point source measurements
exceeding the outlier limit., \textbf{PA1gri = 5{{[}millimagnitude{]}}}
The RMS photometric repeatability of bright non-saturated unresolved
point sources in the g, r, and i filters., \textbf{PA3 =
10{{[}millimagnitude{]}}} RMS width of internal photometric zero-point
(precision of system uniformity across the sky) for all bands except
u-band., \textbf{PA4 = 15{{[}millimagnitude{]}}} The zero point error
outlier limit., \textbf{PF2 = 10{{[}percent{]}}} Fraction of zeropoint
errors that can exceed the zero point error outlier limit.,
\textbf{PixFrac = 1{{[}percent{]}}} The maximum fraction of pixels
scientifically unusable per sensor out of the total allowable fraction
of sensors meeting this performance., \textbf{PA1uzy =
7.5{{[}millimagnitude{]}}} The RMS photometric repeatability of bright
non-saturated unresolved point sources in the u, z, and y filters.,
\textbf{PA6 = 10{{[}millimagnitude{]}}} Accuracy of the transformation
of the internal LSST photometry to a physical scale (e.g. AB
magnitudes)., \textbf{Xtalk = 3{{[}sigma{]}}} The maximum local
significance integrated over the PSF of imperfect crosstalk
corrections., \textbf{PA5u = 10{{[}millimagnitude{]}}} Accuracy of
absolute band-to-band color zero-point for colors constructed using the
u-band., \textbf{ResSource = 2{{[}unitless{]}}} Maximum RMS of the ratio
of the error in integrated flux measurement between bright, isolated,
resolved sources less than 10 arcsec in diameter and bright, isolated
unresolved point sources., \textbf{PA2uzy = 22.5{{[}millimagnitude{]}}}
Repeatability outlier limit for isolated bright non-saturated point
sources in the u, z, and y filters., \textbf{SensorFraction =
15{{[}percent{]}}} The maximum allowable fraction of sensors with
PixFrac scientifically unusable pixels., \textbf{PA3u =
20{{[}millimagnitude{]}}} RMS width of internal photometric zero-point
(precision of system uniformity across the sky) in the u-band.,
\textbf{PA2gri = 15{{[}millimagnitude{]}}} Repeatability outlier limit
for isolated bright non-saturated point sources in the g, r, and i
filters., \textbf{SBPrec = 1{{[}percent{]}}} The maximum error in the
precision of the sky brightness determination., \textbf{PA5 =
5{{[}millimagnitude{]}}} Accuracy of absolute band-to-band color
zero-point for all colors constructed from any filter pair, excluding
the u-band.{]} \\ \cdashline{1-2}
Requirement Discussion &
\begin{minipage}[]{13cm}
\textbf{Discussion:} The relevant metrics are listed in the table
photometricPerformance below. The values in the tables are the target
values for LSST but are not verified as part of this requirement.
\end{minipage}
\\ \cdashline{1-2}
Requirement Priority & 1a \\ \cdashline{1-2}
Upper Level Requirement &
\begin{tabular}{cl}
OSS-REQ-0387 & Photometric Performance \\
\end{tabular}
\\ \hline
\end{longtable}
}


\subsubsection{Test Cases Summary}
\begin{longtable}{p{3cm}p{2.5cm}p{2.5cm}p{3cm}p{4cm}}
\toprule
\href{https://jira.lsstcorp.org/secure/Tests.jspa\#/testCase/LVV-T377}{LVV-T377} & \multicolumn{4}{p{12cm}}{ Verify Calculation of Photometric Performance Metrics } \\ \hline
\textbf{Owner} & \textbf{Status} & \textbf{Version} & \textbf{Critical Event} & \textbf{Verification Type} \\ \hline
Leanne Guy & Approved & 1 & false & Test \\ \hline
\end{longtable}
{\scriptsize
\textbf{Objective:}\\
Verify that the DMS system provides software to calculate photometric
performance metrics, and that the algorithms are properly calculating
the desired quantities. Note that because the DMS requirement is that
the software shall be provided (and not on the actual measured values of
the metrics), we verify all of the requirements via a single test case.
}
\begin{longtable}{p{3cm}p{2.5cm}p{2.5cm}p{3cm}p{4cm}}
\toprule
\href{https://jira.lsstcorp.org/secure/Tests.jspa\#/testCase/LVV-T1839}{LVV-T1839} & \multicolumn{4}{p{12cm}}{ Verify calculation of RMS width of photometric zeropoint } \\ \hline
\textbf{Owner} & \textbf{Status} & \textbf{Version} & \textbf{Critical Event} & \textbf{Verification Type} \\ \hline
Jeffrey Carlin & Draft & 1 & false & Test \\ \hline
\end{longtable}
{\scriptsize
\textbf{Objective:}\\
Verify that the DM system provides code to assess whether the RMS width
of the internal photometric zero-point (precision of system uniformity
across the sky) for all bands except u-band is less than \textbf{PA3 =
10 millimagnitudes}.
}
  
 \newpage 
\subsection{[LVV-9764] DMS-REQ-0359-V-15: Percentage of image area with ghosts }\label{lvv-9764}

\begin{longtable}{cccc}
\hline
\textbf{Jira Link} & \textbf{Assignee} & \textbf{Status} & \textbf{Test Cases}\\ \hline
\href{https://jira.lsstcorp.org/browse/LVV-9764}{LVV-9764} &
Leanne Guy & Not Covered &
\begin{tabular}{c}
LVV-T377 \\
LVV-T1838 \\
\end{tabular}
\\
\hline
\end{longtable}

\textbf{Verification Element Description:} \\
The percentage of image area that has ghosts with surface brightness
gradient amplitude of more than 1/3 of the sky noise over 1 arcsec shall
be less than \textbf{GhostAF = 1 percent}.

Associated element DMS-REQ-0359-V-01
(\href{https://jira.lsstcorp.org/browse/LVV-3401}{LVV-3401}) satisfies
the requirement on photometric repeatability in the u, z, and y-band
filters.

Associated element DMS-REQ-0359-V-02
(\href{https://jira.lsstcorp.org/browse/LVV-9751}{LVV-9751}) satisfies
the requirement on the maximum fraction of sensors with scientifically
unusable pixels.

Associated element DMS-REQ-0359-V-03
(\href{https://jira.lsstcorp.org/browse/LVV-9752}{LVV-9752}) satisfies
the constraint on maximum fraction of outliers among non-saturated point
sources.

Associated element DMS-REQ-0359-V-04
(\href{https://jira.lsstcorp.org/browse/LVV-9753}{LVV-9753}) satisfies
the accuracy of zero-point for colors that use the u-band.

Associated element DMS-REQ-0359-V-05
(\href{https://jira.lsstcorp.org/browse/LVV-9754}{LVV-9754}) satisfies
the repeatability outlier limit in g, r, and i-bands.

Associated element DMS-REQ-0359-V-06
(\href{https://jira.lsstcorp.org/browse/LVV-9755}{LVV-9755}) satisfies
the constraint on the accuracy of the transformation from internal to
physical photometric scales.

Associated element DMS-REQ-0359-V-07
(\href{https://jira.lsstcorp.org/browse/LVV-9756}{LVV-9756}) satisfies
the rms width of the internal photometric zero-point in u-band.

Associated element DMS-REQ-0359-V-08
(\href{https://jira.lsstcorp.org/browse/LVV-9757}{LVV-9757}) satisfies
the maximum local significance of imperfect crosstalk corrections.

Associated element DMS-REQ-0359-V-09
(\href{https://jira.lsstcorp.org/browse/LVV-9758}{LVV-9758}) satisfies
the repeatability outlier limit in u, z, and y-bands.

Associated element DMS-REQ-0359-V-10
(\href{https://jira.lsstcorp.org/browse/LVV-9759}{LVV-9759}) satisfies
the rms photometric repeatability in g, r, and i-bands.

Associated element DMS-REQ-0359-V-11
(\href{https://jira.lsstcorp.org/browse/LVV-9760}{LVV-9760}) satisfies
the fraction of zero-point errors that can exceed the outlier limit.

Associated element DMS-REQ-0359-V-12
(\href{https://jira.lsstcorp.org/browse/LVV-9761}{LVV-9761}) satisfies
the maximum fraction of unusable pixels per sensor.

Associated element DMS-REQ-0359-V-13
(\href{https://jira.lsstcorp.org/browse/LVV-9762}{LVV-9762}) satisfies
the maximum allowable precision in the sky brightness determination.

Associated element DMS-REQ-0359-V-14
(\href{https://jira.lsstcorp.org/browse/LVV-9763}{LVV-9763}) satisfies
the rms width of the internal photometric zero-point in g, r, i, z, and
y-bands.

Associated element DMS-REQ-0359-V-16
(\href{https://jira.lsstcorp.org/browse/LVV-9765}{LVV-9765}) satisfies
the accuracy of zero-point for colors that do not include the u-band.

Associated element DMS-REQ-0359-V-17
(\href{https://jira.lsstcorp.org/browse/LVV-9766}{LVV-9766}) satisfies
the maximum RMS of the ratio of the flux measurement error between
resolved/unresolved sources.

{\footnotesize
\begin{longtable}{p{2.5cm}p{13.5cm}}
\hline
\multicolumn{2}{c}{\textbf{Requirement Details}}\\ \hline
Requirement ID & DMS-REQ-0359 \\ \cdashline{1-2}
Requirement Description &
\begin{minipage}[]{13cm}
\textbf{Specification:} The DMS shall include software to enable the
calculation of the photometric performance metrics defined in
OSS-REQ-0387.
\end{minipage}
\\ \cdashline{1-2}
Requirement Parameters & {[}\textbf{GhostAF = 1{{[}percent{]}}} Percentage of image area that can
have ghosts with surface brightness gradient amplitude of more than 1/3
of the sky noise over 1 arcsec., \textbf{PF1 = 10{{[}percent{]}}} The
maximum fraction of isolated non-saturated point source measurements
exceeding the outlier limit., \textbf{PA1gri = 5{{[}millimagnitude{]}}}
The RMS photometric repeatability of bright non-saturated unresolved
point sources in the g, r, and i filters., \textbf{PA3 =
10{{[}millimagnitude{]}}} RMS width of internal photometric zero-point
(precision of system uniformity across the sky) for all bands except
u-band., \textbf{PA4 = 15{{[}millimagnitude{]}}} The zero point error
outlier limit., \textbf{PF2 = 10{{[}percent{]}}} Fraction of zeropoint
errors that can exceed the zero point error outlier limit.,
\textbf{PixFrac = 1{{[}percent{]}}} The maximum fraction of pixels
scientifically unusable per sensor out of the total allowable fraction
of sensors meeting this performance., \textbf{PA1uzy =
7.5{{[}millimagnitude{]}}} The RMS photometric repeatability of bright
non-saturated unresolved point sources in the u, z, and y filters.,
\textbf{PA6 = 10{{[}millimagnitude{]}}} Accuracy of the transformation
of the internal LSST photometry to a physical scale (e.g. AB
magnitudes)., \textbf{Xtalk = 3{{[}sigma{]}}} The maximum local
significance integrated over the PSF of imperfect crosstalk
corrections., \textbf{PA5u = 10{{[}millimagnitude{]}}} Accuracy of
absolute band-to-band color zero-point for colors constructed using the
u-band., \textbf{ResSource = 2{{[}unitless{]}}} Maximum RMS of the ratio
of the error in integrated flux measurement between bright, isolated,
resolved sources less than 10 arcsec in diameter and bright, isolated
unresolved point sources., \textbf{PA2uzy = 22.5{{[}millimagnitude{]}}}
Repeatability outlier limit for isolated bright non-saturated point
sources in the u, z, and y filters., \textbf{SensorFraction =
15{{[}percent{]}}} The maximum allowable fraction of sensors with
PixFrac scientifically unusable pixels., \textbf{PA3u =
20{{[}millimagnitude{]}}} RMS width of internal photometric zero-point
(precision of system uniformity across the sky) in the u-band.,
\textbf{PA2gri = 15{{[}millimagnitude{]}}} Repeatability outlier limit
for isolated bright non-saturated point sources in the g, r, and i
filters., \textbf{SBPrec = 1{{[}percent{]}}} The maximum error in the
precision of the sky brightness determination., \textbf{PA5 =
5{{[}millimagnitude{]}}} Accuracy of absolute band-to-band color
zero-point for all colors constructed from any filter pair, excluding
the u-band.{]} \\ \cdashline{1-2}
Requirement Discussion &
\begin{minipage}[]{13cm}
\textbf{Discussion:} The relevant metrics are listed in the table
photometricPerformance below. The values in the tables are the target
values for LSST but are not verified as part of this requirement.
\end{minipage}
\\ \cdashline{1-2}
Requirement Priority & 1a \\ \cdashline{1-2}
Upper Level Requirement &
\begin{tabular}{cl}
OSS-REQ-0387 & Photometric Performance \\
\end{tabular}
\\ \hline
\end{longtable}
}


\subsubsection{Test Cases Summary}
\begin{longtable}{p{3cm}p{2.5cm}p{2.5cm}p{3cm}p{4cm}}
\toprule
\href{https://jira.lsstcorp.org/secure/Tests.jspa\#/testCase/LVV-T377}{LVV-T377} & \multicolumn{4}{p{12cm}}{ Verify Calculation of Photometric Performance Metrics } \\ \hline
\textbf{Owner} & \textbf{Status} & \textbf{Version} & \textbf{Critical Event} & \textbf{Verification Type} \\ \hline
Leanne Guy & Approved & 1 & false & Test \\ \hline
\end{longtable}
{\scriptsize
\textbf{Objective:}\\
Verify that the DMS system provides software to calculate photometric
performance metrics, and that the algorithms are properly calculating
the desired quantities. Note that because the DMS requirement is that
the software shall be provided (and not on the actual measured values of
the metrics), we verify all of the requirements via a single test case.
}
\begin{longtable}{p{3cm}p{2.5cm}p{2.5cm}p{3cm}p{4cm}}
\toprule
\href{https://jira.lsstcorp.org/secure/Tests.jspa\#/testCase/LVV-T1838}{LVV-T1838} & \multicolumn{4}{p{12cm}}{ Verify calculation of image fraction affected by ghosts } \\ \hline
\textbf{Owner} & \textbf{Status} & \textbf{Version} & \textbf{Critical Event} & \textbf{Verification Type} \\ \hline
Jeffrey Carlin & Draft & 1 & false & Test \\ \hline
\end{longtable}
{\scriptsize
\textbf{Objective:}\\
Verify that the DM system provides code to assess whether the percentage
of image area that has ghosts with surface brightness gradient amplitude
of more than 1/3 of the sky noise over 1 arcsec is less than
\textbf{GhostAF = 1 percent}.
}
  
 \newpage 
\subsection{[LVV-9765] DMS-REQ-0359-V-16: Accuracy of zero point for colors without u-band }\label{lvv-9765}

\begin{longtable}{cccc}
\hline
\textbf{Jira Link} & \textbf{Assignee} & \textbf{Status} & \textbf{Test Cases}\\ \hline
\href{https://jira.lsstcorp.org/browse/LVV-9765}{LVV-9765} &
Leanne Guy & Not Covered &
\begin{tabular}{c}
LVV-T377 \\
LVV-T1837 \\
\end{tabular}
\\
\hline
\end{longtable}

\textbf{Verification Element Description:} \\
The accuracy of absolute band-to-band color zero-points for all colors
constructed from any filter pair, excluding the u-band, shall be less
than \textbf{PA5 = 5 millimagnitudes}.

Associated element DMS-REQ-0359-V-01
(\href{https://jira.lsstcorp.org/browse/LVV-3401}{LVV-3401}) satisfies
the requirement on photometric repeatability in the u, z, and y-band
filters.

Associated element DMS-REQ-0359-V-02
(\href{https://jira.lsstcorp.org/browse/LVV-9751}{LVV-9751}) satisfies
the requirement on the maximum fraction of sensors with scientifically
unusable pixels.

Associated element DMS-REQ-0359-V-03
(\href{https://jira.lsstcorp.org/browse/LVV-9752}{LVV-9752}) satisfies
the constraint on maximum fraction of outliers among non-saturated point
sources.

Associated element DMS-REQ-0359-V-04
(\href{https://jira.lsstcorp.org/browse/LVV-9753}{LVV-9753}) satisfies
the accuracy of zero-point for colors that use the u-band.

Associated element DMS-REQ-0359-V-05
(\href{https://jira.lsstcorp.org/browse/LVV-9754}{LVV-9754}) satisfies
the repeatability outlier limit in g, r, and i-bands.

Associated element DMS-REQ-0359-V-06
(\href{https://jira.lsstcorp.org/browse/LVV-9755}{LVV-9755}) satisfies
the constraint on the accuracy of the transformation from internal to
physical photometric scales.

Associated element DMS-REQ-0359-V-07
(\href{https://jira.lsstcorp.org/browse/LVV-9756}{LVV-9756}) satisfies
the rms width of the internal photometric zero-point in u-band.

Associated element DMS-REQ-0359-V-08
(\href{https://jira.lsstcorp.org/browse/LVV-9757}{LVV-9757}) satisfies
the maximum local significance of imperfect crosstalk corrections.

Associated element DMS-REQ-0359-V-09
(\href{https://jira.lsstcorp.org/browse/LVV-9758}{LVV-9758}) satisfies
the repeatability outlier limit in u, z, and y-bands.

Associated element DMS-REQ-0359-V-10
(\href{https://jira.lsstcorp.org/browse/LVV-9759}{LVV-9759}) satisfies
the rms photometric repeatability in g, r, and i-bands.

Associated element DMS-REQ-0359-V-11
(\href{https://jira.lsstcorp.org/browse/LVV-9760}{LVV-9760}) satisfies
the fraction of zero-point errors that can exceed the outlier limit.

Associated element DMS-REQ-0359-V-12
(\href{https://jira.lsstcorp.org/browse/LVV-9761}{LVV-9761}) satisfies
the maximum fraction of unusable pixels per sensor.

Associated element DMS-REQ-0359-V-13
(\href{https://jira.lsstcorp.org/browse/LVV-9762}{LVV-9762}) satisfies
the maximum allowable precision in the sky brightness determination.

Associated element DMS-REQ-0359-V-14
(\href{https://jira.lsstcorp.org/browse/LVV-9763}{LVV-9763}) satisfies
the rms width of the internal photometric zero-point in g, r, i, z, and
y-bands.

Associated element DMS-REQ-0359-V-15
(\href{https://jira.lsstcorp.org/browse/LVV-9764}{LVV-9764}) satisfies
the percentage of the image area affected by ghosts that exceed the
threshold.

Associated element DMS-REQ-0359-V-17
(\href{https://jira.lsstcorp.org/browse/LVV-9766}{LVV-9766}) satisfies
the maximum RMS of the ratio of the flux measurement error between
resolved/unresolved sources.

{\footnotesize
\begin{longtable}{p{2.5cm}p{13.5cm}}
\hline
\multicolumn{2}{c}{\textbf{Requirement Details}}\\ \hline
Requirement ID & DMS-REQ-0359 \\ \cdashline{1-2}
Requirement Description &
\begin{minipage}[]{13cm}
\textbf{Specification:} The DMS shall include software to enable the
calculation of the photometric performance metrics defined in
OSS-REQ-0387.
\end{minipage}
\\ \cdashline{1-2}
Requirement Parameters & {[}\textbf{GhostAF = 1{{[}percent{]}}} Percentage of image area that can
have ghosts with surface brightness gradient amplitude of more than 1/3
of the sky noise over 1 arcsec., \textbf{PF1 = 10{{[}percent{]}}} The
maximum fraction of isolated non-saturated point source measurements
exceeding the outlier limit., \textbf{PA1gri = 5{{[}millimagnitude{]}}}
The RMS photometric repeatability of bright non-saturated unresolved
point sources in the g, r, and i filters., \textbf{PA3 =
10{{[}millimagnitude{]}}} RMS width of internal photometric zero-point
(precision of system uniformity across the sky) for all bands except
u-band., \textbf{PA4 = 15{{[}millimagnitude{]}}} The zero point error
outlier limit., \textbf{PF2 = 10{{[}percent{]}}} Fraction of zeropoint
errors that can exceed the zero point error outlier limit.,
\textbf{PixFrac = 1{{[}percent{]}}} The maximum fraction of pixels
scientifically unusable per sensor out of the total allowable fraction
of sensors meeting this performance., \textbf{PA1uzy =
7.5{{[}millimagnitude{]}}} The RMS photometric repeatability of bright
non-saturated unresolved point sources in the u, z, and y filters.,
\textbf{PA6 = 10{{[}millimagnitude{]}}} Accuracy of the transformation
of the internal LSST photometry to a physical scale (e.g. AB
magnitudes)., \textbf{Xtalk = 3{{[}sigma{]}}} The maximum local
significance integrated over the PSF of imperfect crosstalk
corrections., \textbf{PA5u = 10{{[}millimagnitude{]}}} Accuracy of
absolute band-to-band color zero-point for colors constructed using the
u-band., \textbf{ResSource = 2{{[}unitless{]}}} Maximum RMS of the ratio
of the error in integrated flux measurement between bright, isolated,
resolved sources less than 10 arcsec in diameter and bright, isolated
unresolved point sources., \textbf{PA2uzy = 22.5{{[}millimagnitude{]}}}
Repeatability outlier limit for isolated bright non-saturated point
sources in the u, z, and y filters., \textbf{SensorFraction =
15{{[}percent{]}}} The maximum allowable fraction of sensors with
PixFrac scientifically unusable pixels., \textbf{PA3u =
20{{[}millimagnitude{]}}} RMS width of internal photometric zero-point
(precision of system uniformity across the sky) in the u-band.,
\textbf{PA2gri = 15{{[}millimagnitude{]}}} Repeatability outlier limit
for isolated bright non-saturated point sources in the g, r, and i
filters., \textbf{SBPrec = 1{{[}percent{]}}} The maximum error in the
precision of the sky brightness determination., \textbf{PA5 =
5{{[}millimagnitude{]}}} Accuracy of absolute band-to-band color
zero-point for all colors constructed from any filter pair, excluding
the u-band.{]} \\ \cdashline{1-2}
Requirement Discussion &
\begin{minipage}[]{13cm}
\textbf{Discussion:} The relevant metrics are listed in the table
photometricPerformance below. The values in the tables are the target
values for LSST but are not verified as part of this requirement.
\end{minipage}
\\ \cdashline{1-2}
Requirement Priority & 1a \\ \cdashline{1-2}
Upper Level Requirement &
\begin{tabular}{cl}
OSS-REQ-0387 & Photometric Performance \\
\end{tabular}
\\ \hline
\end{longtable}
}


\subsubsection{Test Cases Summary}
\begin{longtable}{p{3cm}p{2.5cm}p{2.5cm}p{3cm}p{4cm}}
\toprule
\href{https://jira.lsstcorp.org/secure/Tests.jspa\#/testCase/LVV-T377}{LVV-T377} & \multicolumn{4}{p{12cm}}{ Verify Calculation of Photometric Performance Metrics } \\ \hline
\textbf{Owner} & \textbf{Status} & \textbf{Version} & \textbf{Critical Event} & \textbf{Verification Type} \\ \hline
Leanne Guy & Approved & 1 & false & Test \\ \hline
\end{longtable}
{\scriptsize
\textbf{Objective:}\\
Verify that the DMS system provides software to calculate photometric
performance metrics, and that the algorithms are properly calculating
the desired quantities. Note that because the DMS requirement is that
the software shall be provided (and not on the actual measured values of
the metrics), we verify all of the requirements via a single test case.
}
\begin{longtable}{p{3cm}p{2.5cm}p{2.5cm}p{3cm}p{4cm}}
\toprule
\href{https://jira.lsstcorp.org/secure/Tests.jspa\#/testCase/LVV-T1837}{LVV-T1837} & \multicolumn{4}{p{12cm}}{ Verify calculation of band-to-band color zero-point accuracy } \\ \hline
\textbf{Owner} & \textbf{Status} & \textbf{Version} & \textbf{Critical Event} & \textbf{Verification Type} \\ \hline
Jeffrey Carlin & Draft & 1 & false & Test \\ \hline
\end{longtable}
{\scriptsize
\textbf{Objective:}\\
Verify that the DM system provides code to assess whether the accuracy
of absolute band-to-band color zero-points for all colors constructed
from any filter pair, excluding the u-band, is less than \textbf{PA5 = 5
millimagnitudes}.
}
  
 \newpage 
\subsection{[LVV-9766] DMS-REQ-0359-V-17: Max RMS of resolved/unresolved flux ratio }\label{lvv-9766}

\begin{longtable}{cccc}
\hline
\textbf{Jira Link} & \textbf{Assignee} & \textbf{Status} & \textbf{Test Cases}\\ \hline
\href{https://jira.lsstcorp.org/browse/LVV-9766}{LVV-9766} &
Leanne Guy & Not Covered &
\begin{tabular}{c}
LVV-T377 \\
LVV-T1836 \\
\end{tabular}
\\
\hline
\end{longtable}

\textbf{Verification Element Description:} \\
The maximum RMS of the ratio of the error in integrated flux measurement
between bright, isolated, resolved sources less than 10 arcsec in
diameter and bright, isolated unresolved point sources shall be less
than \textbf{ResSource = 2}.

Associated element DMS-REQ-0359-V-01
(\href{https://jira.lsstcorp.org/browse/LVV-3401}{LVV-3401}) satisfies
the requirement on photometric repeatability in the u, z, and y-band
filters.

Associated element DMS-REQ-0359-V-02
(\href{https://jira.lsstcorp.org/browse/LVV-9751}{LVV-9751}) satisfies
the requirement on the maximum fraction of sensors with scientifically
unusable pixels.

Associated element DMS-REQ-0359-V-03
(\href{https://jira.lsstcorp.org/browse/LVV-9752}{LVV-9752}) satisfies
the constraint on maximum fraction of outliers among non-saturated point
sources.

Associated element DMS-REQ-0359-V-04
(\href{https://jira.lsstcorp.org/browse/LVV-9753}{LVV-9753}) satisfies
the accuracy of zero-point for colors that use the u-band.

Associated element DMS-REQ-0359-V-05
(\href{https://jira.lsstcorp.org/browse/LVV-9754}{LVV-9754}) satisfies
the repeatability outlier limit in g, r, and i-bands.

Associated element DMS-REQ-0359-V-06
(\href{https://jira.lsstcorp.org/browse/LVV-9755}{LVV-9755}) satisfies
the constraint on the accuracy of the transformation from internal to
physical photometric scales.

Associated element DMS-REQ-0359-V-07
(\href{https://jira.lsstcorp.org/browse/LVV-9756}{LVV-9756}) satisfies
the rms width of the internal photometric zero-point in u-band.

Associated element DMS-REQ-0359-V-08
(\href{https://jira.lsstcorp.org/browse/LVV-9757}{LVV-9757}) satisfies
the maximum local significance of imperfect crosstalk corrections.

Associated element DMS-REQ-0359-V-09
(\href{https://jira.lsstcorp.org/browse/LVV-9758}{LVV-9758}) satisfies
the repeatability outlier limit in u, z, and y-bands.

Associated element DMS-REQ-0359-V-10
(\href{https://jira.lsstcorp.org/browse/LVV-9759}{LVV-9759}) satisfies
the rms photometric repeatability in g, r, and i-bands.

Associated element DMS-REQ-0359-V-11
(\href{https://jira.lsstcorp.org/browse/LVV-9760}{LVV-9760}) satisfies
the fraction of zero-point errors that can exceed the outlier limit.

Associated element DMS-REQ-0359-V-12
(\href{https://jira.lsstcorp.org/browse/LVV-9761}{LVV-9761}) satisfies
the maximum fraction of unusable pixels per sensor.

Associated element DMS-REQ-0359-V-13
(\href{https://jira.lsstcorp.org/browse/LVV-9762}{LVV-9762}) satisfies
the maximum allowable precision in the sky brightness determination.

Associated element DMS-REQ-0359-V-14
(\href{https://jira.lsstcorp.org/browse/LVV-9763}{LVV-9763}) satisfies
the rms width of the internal photometric zero-point in g, r, i, z, and
y-bands.

Associated element DMS-REQ-0359-V-15
(\href{https://jira.lsstcorp.org/browse/LVV-9764}{LVV-9764}) satisfies
the percentage of the image area affected by ghosts that exceed the
threshold.

Associated element DMS-REQ-0359-V-16
(\href{https://jira.lsstcorp.org/browse/LVV-9765}{LVV-9765}) satisfies
the accuracy of zero-point for colors that do not include the u-band.

{\footnotesize
\begin{longtable}{p{2.5cm}p{13.5cm}}
\hline
\multicolumn{2}{c}{\textbf{Requirement Details}}\\ \hline
Requirement ID & DMS-REQ-0359 \\ \cdashline{1-2}
Requirement Description &
\begin{minipage}[]{13cm}
\textbf{Specification:} The DMS shall include software to enable the
calculation of the photometric performance metrics defined in
OSS-REQ-0387.
\end{minipage}
\\ \cdashline{1-2}
Requirement Parameters & {[}\textbf{GhostAF = 1{{[}percent{]}}} Percentage of image area that can
have ghosts with surface brightness gradient amplitude of more than 1/3
of the sky noise over 1 arcsec., \textbf{PF1 = 10{{[}percent{]}}} The
maximum fraction of isolated non-saturated point source measurements
exceeding the outlier limit., \textbf{PA1gri = 5{{[}millimagnitude{]}}}
The RMS photometric repeatability of bright non-saturated unresolved
point sources in the g, r, and i filters., \textbf{PA3 =
10{{[}millimagnitude{]}}} RMS width of internal photometric zero-point
(precision of system uniformity across the sky) for all bands except
u-band., \textbf{PA4 = 15{{[}millimagnitude{]}}} The zero point error
outlier limit., \textbf{PF2 = 10{{[}percent{]}}} Fraction of zeropoint
errors that can exceed the zero point error outlier limit.,
\textbf{PixFrac = 1{{[}percent{]}}} The maximum fraction of pixels
scientifically unusable per sensor out of the total allowable fraction
of sensors meeting this performance., \textbf{PA1uzy =
7.5{{[}millimagnitude{]}}} The RMS photometric repeatability of bright
non-saturated unresolved point sources in the u, z, and y filters.,
\textbf{PA6 = 10{{[}millimagnitude{]}}} Accuracy of the transformation
of the internal LSST photometry to a physical scale (e.g. AB
magnitudes)., \textbf{Xtalk = 3{{[}sigma{]}}} The maximum local
significance integrated over the PSF of imperfect crosstalk
corrections., \textbf{PA5u = 10{{[}millimagnitude{]}}} Accuracy of
absolute band-to-band color zero-point for colors constructed using the
u-band., \textbf{ResSource = 2{{[}unitless{]}}} Maximum RMS of the ratio
of the error in integrated flux measurement between bright, isolated,
resolved sources less than 10 arcsec in diameter and bright, isolated
unresolved point sources., \textbf{PA2uzy = 22.5{{[}millimagnitude{]}}}
Repeatability outlier limit for isolated bright non-saturated point
sources in the u, z, and y filters., \textbf{SensorFraction =
15{{[}percent{]}}} The maximum allowable fraction of sensors with
PixFrac scientifically unusable pixels., \textbf{PA3u =
20{{[}millimagnitude{]}}} RMS width of internal photometric zero-point
(precision of system uniformity across the sky) in the u-band.,
\textbf{PA2gri = 15{{[}millimagnitude{]}}} Repeatability outlier limit
for isolated bright non-saturated point sources in the g, r, and i
filters., \textbf{SBPrec = 1{{[}percent{]}}} The maximum error in the
precision of the sky brightness determination., \textbf{PA5 =
5{{[}millimagnitude{]}}} Accuracy of absolute band-to-band color
zero-point for all colors constructed from any filter pair, excluding
the u-band.{]} \\ \cdashline{1-2}
Requirement Discussion &
\begin{minipage}[]{13cm}
\textbf{Discussion:} The relevant metrics are listed in the table
photometricPerformance below. The values in the tables are the target
values for LSST but are not verified as part of this requirement.
\end{minipage}
\\ \cdashline{1-2}
Requirement Priority & 1a \\ \cdashline{1-2}
Upper Level Requirement &
\begin{tabular}{cl}
OSS-REQ-0387 & Photometric Performance \\
\end{tabular}
\\ \hline
\end{longtable}
}


\subsubsection{Test Cases Summary}
\begin{longtable}{p{3cm}p{2.5cm}p{2.5cm}p{3cm}p{4cm}}
\toprule
\href{https://jira.lsstcorp.org/secure/Tests.jspa\#/testCase/LVV-T377}{LVV-T377} & \multicolumn{4}{p{12cm}}{ Verify Calculation of Photometric Performance Metrics } \\ \hline
\textbf{Owner} & \textbf{Status} & \textbf{Version} & \textbf{Critical Event} & \textbf{Verification Type} \\ \hline
Leanne Guy & Approved & 1 & false & Test \\ \hline
\end{longtable}
{\scriptsize
\textbf{Objective:}\\
Verify that the DMS system provides software to calculate photometric
performance metrics, and that the algorithms are properly calculating
the desired quantities. Note that because the DMS requirement is that
the software shall be provided (and not on the actual measured values of
the metrics), we verify all of the requirements via a single test case.
}
\begin{longtable}{p{3cm}p{2.5cm}p{2.5cm}p{3cm}p{4cm}}
\toprule
\href{https://jira.lsstcorp.org/secure/Tests.jspa\#/testCase/LVV-T1836}{LVV-T1836} & \multicolumn{4}{p{12cm}}{ Verify calculation of resolved-to-unresolved flux ratio errors } \\ \hline
\textbf{Owner} & \textbf{Status} & \textbf{Version} & \textbf{Critical Event} & \textbf{Verification Type} \\ \hline
Jeffrey Carlin & Draft & 1 & false & Test \\ \hline
\end{longtable}
{\scriptsize
\textbf{Objective:}\\
Verify that the DM system has provided code to assess whether the
maximum RMS of the ratio of the error in integrated flux measurement
between bright, isolated, resolved sources less than 10 arcsec in
diameter and bright, isolated unresolved point sources is less than
\textbf{ResSource = 2}.
}
  
 \newpage 
\subsection{[LVV-9767] DMS-REQ-0360-V-02: Max fraction exceeding limit on 5 arcmin scales }\label{lvv-9767}

\begin{longtable}{cccc}
\hline
\textbf{Jira Link} & \textbf{Assignee} & \textbf{Status} & \textbf{Test Cases}\\ \hline
\href{https://jira.lsstcorp.org/browse/LVV-9767}{LVV-9767} &
Leanne Guy & Not Covered &
\begin{tabular}{c}
LVV-T378 \\
LVV-T1746 \\
\end{tabular}
\\
\hline
\end{longtable}

\textbf{Verification Element Description:} \\
The maximum fraction of relative astrometric measurements on 5 arcminute
scales to exceed the 5 arcminute outlier limit shall be less
than~\textbf{AF1 = 10~percent.}

Associated element DMS-REQ-0360-V-01
(\href{https://jira.lsstcorp.org/browse/LVV-3402}{LVV-3402})~satisfies~the
maximum fraction of astrometric outliers on 5 arcminute scales.

Associated element
DMS-REQ-0360-V-03~(\href{https://jira.lsstcorp.org/browse/LVV-9768}{LVV-9768})~satisfies~the
median astrometric error on 5 arcminute scales.

Associated element
DMS-REQ-0360-V-04~(\href{https://jira.lsstcorp.org/browse/LVV-9769}{LVV-9769})~satisfies~the
median astrometric error in absolute positions.

Associated element
DMS-REQ-0360-V-05~(\href{https://jira.lsstcorp.org/browse/LVV-9770}{LVV-9770})~satisfies~the
astrometric outlier limit on 20 arcminute scales.

Associated element
DMS-REQ-0360-V-06~(\href{https://jira.lsstcorp.org/browse/LVV-9771}{LVV-9771})~satisfies~the
color difference outlier limit relative to r-band.

Associated element
DMS-REQ-0360-V-07~(\href{https://jira.lsstcorp.org/browse/LVV-9773}{LVV-9773})~satisfies
the astrometric outlier limit on 5 arcminute scales.

Associated element
DMS-REQ-0360-V-08~(\href{https://jira.lsstcorp.org/browse/LVV-9774}{LVV-9774})~satisfies~the
median astrometric error on 200 arcminute scales.

Associated element
DMS-REQ-0360-V-09~(\href{https://jira.lsstcorp.org/browse/LVV-9775}{LVV-9775})~satisfies
the astrometric outlier limit on 200 arcminute scales.

Associated element
DMS-REQ-0360-V-10~(\href{https://jira.lsstcorp.org/browse/LVV-9776}{LVV-9776})~satisfies
the maximum fraction of astrometric outliers on 20 arcminute scales.

Associated element
DMS-REQ-0360-V-11~(\href{https://jira.lsstcorp.org/browse/LVV-9777}{LVV-9777})~satisfies
the maximum fraction of r-band color difference outliers.

Associated element
DMS-REQ-0360-V-12~(\href{https://jira.lsstcorp.org/browse/LVV-9778}{LVV-9778})~satisfies~the
RMS difference between separations measured in the r-band and those
measured in any other filter.

Associated element
DMS-REQ-0360-V-13~(\href{https://jira.lsstcorp.org/browse/LVV-9779}{LVV-9779})~satisfies
the maximum fraction of astrometric outliers on 200 arcminute scales.

{\footnotesize
\begin{longtable}{p{2.5cm}p{13.5cm}}
\hline
\multicolumn{2}{c}{\textbf{Requirement Details}}\\ \hline
Requirement ID & DMS-REQ-0360 \\ \cdashline{1-2}
Requirement Description &
\begin{minipage}[]{13cm}
\textbf{Specification:} The DMS shall include software to enable the
calculation of the astrometric performance metrics defined in
OSS-REQ-0388.
\end{minipage}
\\ \cdashline{1-2}
Requirement Parameters & {[}\textbf{AM2 = 10{{[}milliarcsecond{]}}} Median relative astrometric
measurement error on 20 arcminute scales., \textbf{AM1 =
10{{[}milliarcsecond{]}}} Median relative astrometric measurement error
on 5 arcminute scales shall be less than AM1., \textbf{AM3 =
15{{[}milliarcsecond{]}}} Median relative astrometric measurement error
on 200 arcminute scales., \textbf{AA1 = 50{{[}milliarcsecond{]}}} Median
error in absolute position for each axis, RA and DEC, shall be less than
AA1., \textbf{AF1 = 10{{[}percent{]}}} The maximum fraction of relative
astrometric measurements on 5 arcminute scales to exceed 5 arcminute
outlier limit., \textbf{AD3 = 30{{[}milliarcsecond{]}}} 200 arcminute
outlier limit., \textbf{AB1 = 10{{[}milliarcsecond{]}}} RMS difference
between separations measured in the r-band and those measured in any
other filter., \textbf{AD1 = 20{{[}milliarcsecond{]}}} 5 arcminute
outlier limit., \textbf{AB2 = 20{{[}milliarcsecond{]}}} The color
difference outlier limit for separations measured relative the r-band
filter in any other filter., \textbf{AD2 = 20{{[}milliarcsecond{]}}} 20
arcminute outlier limit., \textbf{ABF1 = 10{{[}percent{]}}} Fraction of
separations measured relative to the r-band that can exceed the color
difference outlier limit., \textbf{AF3 = 10{{[}percent{]}}} Fraction of
relative astrometric measurements on 200 arcminute scales to exceed 200
arcminute outlier limit., \textbf{AF2 = 10{{[}percent{]}}} The maximum
fraction of relative astrometric measurements on 20 arcminute scales to
exceed 20 arcminute outlier limit.{]} \\ \cdashline{1-2}
Requirement Discussion &
\begin{minipage}[]{13cm}
\textbf{Discussion:} The relevant metrics are listed in the table below.
The values in the tables are the target values for LSST but are not
verified as part of this requirement.
\end{minipage}
\\ \cdashline{1-2}
Requirement Priority & 1a \\ \cdashline{1-2}
Upper Level Requirement &
\begin{tabular}{cl}
OSS-REQ-0388 & Astrometric Performance \\
\end{tabular}
\\ \hline
\end{longtable}
}


\subsubsection{Test Cases Summary}
\begin{longtable}{p{3cm}p{2.5cm}p{2.5cm}p{3cm}p{4cm}}
\toprule
\href{https://jira.lsstcorp.org/secure/Tests.jspa\#/testCase/LVV-T378}{LVV-T378} & \multicolumn{4}{p{12cm}}{ Verify Calculation of Astrometric Performance Metrics } \\ \hline
\textbf{Owner} & \textbf{Status} & \textbf{Version} & \textbf{Critical Event} & \textbf{Verification Type} \\ \hline
Leanne Guy & Approved & 1 & false & Test \\ \hline
\end{longtable}
{\scriptsize
\textbf{Objective:}\\
Verify that the DMS system provides software to calculate astrometric
performance metrics, and that the algorithms are properly calculating
the desired quantities. Note that because the DMS requirement is that
the software shall be provided (and not on the actual measured values of
the metrics), we verify all of the requirements via a single test case.
}
\begin{longtable}{p{3cm}p{2.5cm}p{2.5cm}p{3cm}p{4cm}}
\toprule
\href{https://jira.lsstcorp.org/secure/Tests.jspa\#/testCase/LVV-T1746}{LVV-T1746} & \multicolumn{4}{p{12cm}}{ Verify calculation of fraction of relative astrometric measurement error
on 5 arcminute scales exceeding outlier limit } \\ \hline
\textbf{Owner} & \textbf{Status} & \textbf{Version} & \textbf{Critical Event} & \textbf{Verification Type} \\ \hline
Jeffrey Carlin & Approved & 1 & false & Test \\ \hline
\end{longtable}
{\scriptsize
\textbf{Objective:}\\
Verify that the DM system has provided the code to calculate the maximum
fraction of relative astrometric measurements on 5 arcminute scales that
exceed the 5 arcminute outlier limit \textbf{AD1 = 20 milliarcseconds},
and assess whether it meets the requirement that it shall be less than
\textbf{AF1 = 10 percent.}
}
  
 \newpage 
\subsection{[LVV-9768] DMS-REQ-0360-V-03: Median astrometric error on 5 arcmin scales }\label{lvv-9768}

\begin{longtable}{cccc}
\hline
\textbf{Jira Link} & \textbf{Assignee} & \textbf{Status} & \textbf{Test Cases}\\ \hline
\href{https://jira.lsstcorp.org/browse/LVV-9768}{LVV-9768} &
Leanne Guy & Not Covered &
\begin{tabular}{c}
LVV-T378 \\
LVV-T1747 \\
\end{tabular}
\\
\hline
\end{longtable}

\textbf{Verification Element Description:} \\
The median relative astrometric measurement error on 5 arcminute scales
shall be less than~\textbf{AM1 = 10~milliarcseconds.}

Associated element DMS-REQ-0360-V-01
(\href{https://jira.lsstcorp.org/browse/LVV-3402}{LVV-3402})~satisfies~the
maximum fraction of astrometric outliers on 5 arcminute scales.

Associated element DMS-REQ-0360-V-02
(\href{https://jira.lsstcorp.org/browse/LVV-9767}{LVV-9767})~satisfies~the
maximum fraction of astrometric outliers on 5 arcminute scales.

Associated element
DMS-REQ-0360-V-04~(\href{https://jira.lsstcorp.org/browse/LVV-9769}{LVV-9769})~satisfies~the
median astrometric error in absolute positions.

Associated element
DMS-REQ-0360-V-05~(\href{https://jira.lsstcorp.org/browse/LVV-9770}{LVV-9770})~satisfies~the
astrometric outlier limit on 20 arcminute scales.

Associated element
DMS-REQ-0360-V-06~(\href{https://jira.lsstcorp.org/browse/LVV-9771}{LVV-9771})~satisfies~the
color difference outlier limit relative to r-band.

Associated element
DMS-REQ-0360-V-07~(\href{https://jira.lsstcorp.org/browse/LVV-9773}{LVV-9773})~satisfies
the astrometric outlier limit on 5 arcminute scales.

Associated element
DMS-REQ-0360-V-08~(\href{https://jira.lsstcorp.org/browse/LVV-9774}{LVV-9774})~satisfies~the
median astrometric error on 200 arcminute scales.

Associated element
DMS-REQ-0360-V-09~(\href{https://jira.lsstcorp.org/browse/LVV-9775}{LVV-9775})~satisfies
the astrometric outlier limit on 200 arcminute scales.

Associated element
DMS-REQ-0360-V-10~(\href{https://jira.lsstcorp.org/browse/LVV-9776}{LVV-9776})~satisfies
the maximum fraction of astrometric outliers on 20 arcminute scales.

Associated element
DMS-REQ-0360-V-11~(\href{https://jira.lsstcorp.org/browse/LVV-9777}{LVV-9777})~satisfies
the maximum fraction of r-band color difference outliers.

Associated element
DMS-REQ-0360-V-12~(\href{https://jira.lsstcorp.org/browse/LVV-9778}{LVV-9778})~satisfies~the
RMS difference between separations measured in the r-band and those
measured in any other filter.

Associated element
DMS-REQ-0360-V-13~(\href{https://jira.lsstcorp.org/browse/LVV-9779}{LVV-9779})~satisfies
the maximum fraction of astrometric outliers on 200 arcminute scales.

{\footnotesize
\begin{longtable}{p{2.5cm}p{13.5cm}}
\hline
\multicolumn{2}{c}{\textbf{Requirement Details}}\\ \hline
Requirement ID & DMS-REQ-0360 \\ \cdashline{1-2}
Requirement Description &
\begin{minipage}[]{13cm}
\textbf{Specification:} The DMS shall include software to enable the
calculation of the astrometric performance metrics defined in
OSS-REQ-0388.
\end{minipage}
\\ \cdashline{1-2}
Requirement Parameters & {[}\textbf{AM2 = 10{{[}milliarcsecond{]}}} Median relative astrometric
measurement error on 20 arcminute scales., \textbf{AM1 =
10{{[}milliarcsecond{]}}} Median relative astrometric measurement error
on 5 arcminute scales shall be less than AM1., \textbf{AM3 =
15{{[}milliarcsecond{]}}} Median relative astrometric measurement error
on 200 arcminute scales., \textbf{AA1 = 50{{[}milliarcsecond{]}}} Median
error in absolute position for each axis, RA and DEC, shall be less than
AA1., \textbf{AF1 = 10{{[}percent{]}}} The maximum fraction of relative
astrometric measurements on 5 arcminute scales to exceed 5 arcminute
outlier limit., \textbf{AD3 = 30{{[}milliarcsecond{]}}} 200 arcminute
outlier limit., \textbf{AB1 = 10{{[}milliarcsecond{]}}} RMS difference
between separations measured in the r-band and those measured in any
other filter., \textbf{AD1 = 20{{[}milliarcsecond{]}}} 5 arcminute
outlier limit., \textbf{AB2 = 20{{[}milliarcsecond{]}}} The color
difference outlier limit for separations measured relative the r-band
filter in any other filter., \textbf{AD2 = 20{{[}milliarcsecond{]}}} 20
arcminute outlier limit., \textbf{ABF1 = 10{{[}percent{]}}} Fraction of
separations measured relative to the r-band that can exceed the color
difference outlier limit., \textbf{AF3 = 10{{[}percent{]}}} Fraction of
relative astrometric measurements on 200 arcminute scales to exceed 200
arcminute outlier limit., \textbf{AF2 = 10{{[}percent{]}}} The maximum
fraction of relative astrometric measurements on 20 arcminute scales to
exceed 20 arcminute outlier limit.{]} \\ \cdashline{1-2}
Requirement Discussion &
\begin{minipage}[]{13cm}
\textbf{Discussion:} The relevant metrics are listed in the table below.
The values in the tables are the target values for LSST but are not
verified as part of this requirement.
\end{minipage}
\\ \cdashline{1-2}
Requirement Priority & 1a \\ \cdashline{1-2}
Upper Level Requirement &
\begin{tabular}{cl}
OSS-REQ-0388 & Astrometric Performance \\
\end{tabular}
\\ \hline
\end{longtable}
}


\subsubsection{Test Cases Summary}
\begin{longtable}{p{3cm}p{2.5cm}p{2.5cm}p{3cm}p{4cm}}
\toprule
\href{https://jira.lsstcorp.org/secure/Tests.jspa\#/testCase/LVV-T378}{LVV-T378} & \multicolumn{4}{p{12cm}}{ Verify Calculation of Astrometric Performance Metrics } \\ \hline
\textbf{Owner} & \textbf{Status} & \textbf{Version} & \textbf{Critical Event} & \textbf{Verification Type} \\ \hline
Leanne Guy & Approved & 1 & false & Test \\ \hline
\end{longtable}
{\scriptsize
\textbf{Objective:}\\
Verify that the DMS system provides software to calculate astrometric
performance metrics, and that the algorithms are properly calculating
the desired quantities. Note that because the DMS requirement is that
the software shall be provided (and not on the actual measured values of
the metrics), we verify all of the requirements via a single test case.
}
\begin{longtable}{p{3cm}p{2.5cm}p{2.5cm}p{3cm}p{4cm}}
\toprule
\href{https://jira.lsstcorp.org/secure/Tests.jspa\#/testCase/LVV-T1747}{LVV-T1747} & \multicolumn{4}{p{12cm}}{ Verify calculation of relative astrometric measurement error on 5
arcminute scales } \\ \hline
\textbf{Owner} & \textbf{Status} & \textbf{Version} & \textbf{Critical Event} & \textbf{Verification Type} \\ \hline
Jeffrey Carlin & Approved & 1 & false & Test \\ \hline
\end{longtable}
{\scriptsize
\textbf{Objective:}\\
Verify that the DM system has provided the code to calculate the
relative astrometric measurement error on 5 arcminute scales, and assess
whether it meets the requirement that it shall be less
than\textbf{~\textbf{AM1 = 10 milliarcseconds.}}
}
  
 \newpage 
\subsection{[LVV-9769] DMS-REQ-0360-V-04: Median absolute error in RA, Dec }\label{lvv-9769}

\begin{longtable}{cccc}
\hline
\textbf{Jira Link} & \textbf{Assignee} & \textbf{Status} & \textbf{Test Cases}\\ \hline
\href{https://jira.lsstcorp.org/browse/LVV-9769}{LVV-9769} &
Leanne Guy & Not Covered &
\begin{tabular}{c}
LVV-T378 \\
LVV-T1748 \\
\end{tabular}
\\
\hline
\end{longtable}

\textbf{Verification Element Description:} \\
The median error in absolute position for each axis, RA and DEC, shall
be less than \textbf{AA1 = 50 milliarcseconds}.

Associated element DMS-REQ-0360-V-01
(\href{https://jira.lsstcorp.org/browse/LVV-3402}{LVV-3402})~satisfies~the
maximum fraction of astrometric outliers on 5 arcminute scales.

Associated element DMS-REQ-0360-V-02
(\href{https://jira.lsstcorp.org/browse/LVV-9767}{LVV-9767})~satisfies~the
maximum fraction of astrometric outliers on 5 arcminute scales.

Associated element
DMS-REQ-0360-V-03~(\href{https://jira.lsstcorp.org/browse/LVV-9768}{LVV-9768})~satisfies~the
median astrometric error on 5 arcminute scales.

Associated element
DMS-REQ-0360-V-05~(\href{https://jira.lsstcorp.org/browse/LVV-9770}{LVV-9770})~satisfies~the
astrometric outlier limit on 20 arcminute scales.

Associated element
DMS-REQ-0360-V-06~(\href{https://jira.lsstcorp.org/browse/LVV-9771}{LVV-9771})~satisfies~the
color difference outlier limit relative to r-band.

Associated element
DMS-REQ-0360-V-07~(\href{https://jira.lsstcorp.org/browse/LVV-9773}{LVV-9773})~satisfies
the astrometric outlier limit on 5 arcminute scales.

Associated element
DMS-REQ-0360-V-08~(\href{https://jira.lsstcorp.org/browse/LVV-9774}{LVV-9774})~satisfies~the
median astrometric error on 200 arcminute scales.

Associated element
DMS-REQ-0360-V-09~(\href{https://jira.lsstcorp.org/browse/LVV-9775}{LVV-9775})~satisfies
the astrometric outlier limit on 200 arcminute scales.

Associated element
DMS-REQ-0360-V-10~(\href{https://jira.lsstcorp.org/browse/LVV-9776}{LVV-9776})~satisfies
the maximum fraction of astrometric outliers on 20 arcminute scales.

Associated element
DMS-REQ-0360-V-11~(\href{https://jira.lsstcorp.org/browse/LVV-9777}{LVV-9777})~satisfies
the maximum fraction of r-band color difference outliers.

Associated element
DMS-REQ-0360-V-12~(\href{https://jira.lsstcorp.org/browse/LVV-9778}{LVV-9778})~satisfies~the
RMS difference between separations measured in the r-band and those
measured in any other filter.

Associated element
DMS-REQ-0360-V-13~(\href{https://jira.lsstcorp.org/browse/LVV-9779}{LVV-9779})~satisfies
the maximum fraction of astrometric outliers on 200 arcminute scales.

{\footnotesize
\begin{longtable}{p{2.5cm}p{13.5cm}}
\hline
\multicolumn{2}{c}{\textbf{Requirement Details}}\\ \hline
Requirement ID & DMS-REQ-0360 \\ \cdashline{1-2}
Requirement Description &
\begin{minipage}[]{13cm}
\textbf{Specification:} The DMS shall include software to enable the
calculation of the astrometric performance metrics defined in
OSS-REQ-0388.
\end{minipage}
\\ \cdashline{1-2}
Requirement Parameters & {[}\textbf{AM2 = 10{{[}milliarcsecond{]}}} Median relative astrometric
measurement error on 20 arcminute scales., \textbf{AM1 =
10{{[}milliarcsecond{]}}} Median relative astrometric measurement error
on 5 arcminute scales shall be less than AM1., \textbf{AM3 =
15{{[}milliarcsecond{]}}} Median relative astrometric measurement error
on 200 arcminute scales., \textbf{AA1 = 50{{[}milliarcsecond{]}}} Median
error in absolute position for each axis, RA and DEC, shall be less than
AA1., \textbf{AF1 = 10{{[}percent{]}}} The maximum fraction of relative
astrometric measurements on 5 arcminute scales to exceed 5 arcminute
outlier limit., \textbf{AD3 = 30{{[}milliarcsecond{]}}} 200 arcminute
outlier limit., \textbf{AB1 = 10{{[}milliarcsecond{]}}} RMS difference
between separations measured in the r-band and those measured in any
other filter., \textbf{AD1 = 20{{[}milliarcsecond{]}}} 5 arcminute
outlier limit., \textbf{AB2 = 20{{[}milliarcsecond{]}}} The color
difference outlier limit for separations measured relative the r-band
filter in any other filter., \textbf{AD2 = 20{{[}milliarcsecond{]}}} 20
arcminute outlier limit., \textbf{ABF1 = 10{{[}percent{]}}} Fraction of
separations measured relative to the r-band that can exceed the color
difference outlier limit., \textbf{AF3 = 10{{[}percent{]}}} Fraction of
relative astrometric measurements on 200 arcminute scales to exceed 200
arcminute outlier limit., \textbf{AF2 = 10{{[}percent{]}}} The maximum
fraction of relative astrometric measurements on 20 arcminute scales to
exceed 20 arcminute outlier limit.{]} \\ \cdashline{1-2}
Requirement Discussion &
\begin{minipage}[]{13cm}
\textbf{Discussion:} The relevant metrics are listed in the table below.
The values in the tables are the target values for LSST but are not
verified as part of this requirement.
\end{minipage}
\\ \cdashline{1-2}
Requirement Priority & 1a \\ \cdashline{1-2}
Upper Level Requirement &
\begin{tabular}{cl}
OSS-REQ-0388 & Astrometric Performance \\
\end{tabular}
\\ \hline
\end{longtable}
}


\subsubsection{Test Cases Summary}
\begin{longtable}{p{3cm}p{2.5cm}p{2.5cm}p{3cm}p{4cm}}
\toprule
\href{https://jira.lsstcorp.org/secure/Tests.jspa\#/testCase/LVV-T378}{LVV-T378} & \multicolumn{4}{p{12cm}}{ Verify Calculation of Astrometric Performance Metrics } \\ \hline
\textbf{Owner} & \textbf{Status} & \textbf{Version} & \textbf{Critical Event} & \textbf{Verification Type} \\ \hline
Leanne Guy & Approved & 1 & false & Test \\ \hline
\end{longtable}
{\scriptsize
\textbf{Objective:}\\
Verify that the DMS system provides software to calculate astrometric
performance metrics, and that the algorithms are properly calculating
the desired quantities. Note that because the DMS requirement is that
the software shall be provided (and not on the actual measured values of
the metrics), we verify all of the requirements via a single test case.
}
\begin{longtable}{p{3cm}p{2.5cm}p{2.5cm}p{3cm}p{4cm}}
\toprule
\href{https://jira.lsstcorp.org/secure/Tests.jspa\#/testCase/LVV-T1748}{LVV-T1748} & \multicolumn{4}{p{12cm}}{ Verify calculation of median error in absolute position for RA, Dec axes } \\ \hline
\textbf{Owner} & \textbf{Status} & \textbf{Version} & \textbf{Critical Event} & \textbf{Verification Type} \\ \hline
Jeffrey Carlin & Approved & 1 & false & Test \\ \hline
\end{longtable}
{\scriptsize
\textbf{Objective:}\\
Verify that the DM system has provided the code to calculate the median
error in absolute position for each axis, RA and DEC, and assess whether
it meets the requirement that it shall be less than \textbf{AA1 = 50
milliarcseconds}.
}
  
 \newpage 
\subsection{[LVV-9770] DMS-REQ-0360-V-05: Outlier limit on 20 arcmin scales }\label{lvv-9770}

\begin{longtable}{cccc}
\hline
\textbf{Jira Link} & \textbf{Assignee} & \textbf{Status} & \textbf{Test Cases}\\ \hline
\href{https://jira.lsstcorp.org/browse/LVV-9770}{LVV-9770} &
Leanne Guy & Not Covered &
\begin{tabular}{c}
LVV-T378 \\
LVV-T1749 \\
\end{tabular}
\\
\hline
\end{longtable}

\textbf{Verification Element Description:} \\
The~20 arcminute outlier limit is~\textbf{AD2 = 20~milliarcseconds}.

Associated element DMS-REQ-0360-V-01
(\href{https://jira.lsstcorp.org/browse/LVV-3402}{LVV-3402})~satisfies~the
maximum fraction of astrometric outliers on 5 arcminute scales.

Associated element DMS-REQ-0360-V-02
(\href{https://jira.lsstcorp.org/browse/LVV-9767}{LVV-9767})~satisfies~the
maximum fraction of astrometric outliers on 5 arcminute scales.

Associated element
DMS-REQ-0360-V-03~(\href{https://jira.lsstcorp.org/browse/LVV-9768}{LVV-9768})~satisfies~the
median astrometric error on 5 arcminute scales.

Associated element
DMS-REQ-0360-V-04~(\href{https://jira.lsstcorp.org/browse/LVV-9769}{LVV-9769})~satisfies~the
median astrometric error in absolute positions.

Associated element
DMS-REQ-0360-V-06~(\href{https://jira.lsstcorp.org/browse/LVV-9771}{LVV-9771})~satisfies~the
color difference outlier limit relative to r-band.

Associated element
DMS-REQ-0360-V-07~(\href{https://jira.lsstcorp.org/browse/LVV-9773}{LVV-9773})~satisfies
the astrometric outlier limit on 5 arcminute scales.

Associated element
DMS-REQ-0360-V-08~(\href{https://jira.lsstcorp.org/browse/LVV-9774}{LVV-9774})~satisfies~the
median astrometric error on 200 arcminute scales.

Associated element
DMS-REQ-0360-V-09~(\href{https://jira.lsstcorp.org/browse/LVV-9775}{LVV-9775})~satisfies
the astrometric outlier limit on 200 arcminute scales.

Associated element
DMS-REQ-0360-V-10~(\href{https://jira.lsstcorp.org/browse/LVV-9776}{LVV-9776})~satisfies
the maximum fraction of astrometric outliers on 20 arcminute scales.

Associated element
DMS-REQ-0360-V-11~(\href{https://jira.lsstcorp.org/browse/LVV-9777}{LVV-9777})~satisfies
the maximum fraction of r-band color difference outliers.

Associated element
DMS-REQ-0360-V-12~(\href{https://jira.lsstcorp.org/browse/LVV-9778}{LVV-9778})~satisfies~the
RMS difference between separations measured in the r-band and those
measured in any other filter.

Associated element
DMS-REQ-0360-V-13~(\href{https://jira.lsstcorp.org/browse/LVV-9779}{LVV-9779})~satisfies
the maximum fraction of astrometric outliers on 200 arcminute scales.

{\footnotesize
\begin{longtable}{p{2.5cm}p{13.5cm}}
\hline
\multicolumn{2}{c}{\textbf{Requirement Details}}\\ \hline
Requirement ID & DMS-REQ-0360 \\ \cdashline{1-2}
Requirement Description &
\begin{minipage}[]{13cm}
\textbf{Specification:} The DMS shall include software to enable the
calculation of the astrometric performance metrics defined in
OSS-REQ-0388.
\end{minipage}
\\ \cdashline{1-2}
Requirement Parameters & {[}\textbf{AM2 = 10{{[}milliarcsecond{]}}} Median relative astrometric
measurement error on 20 arcminute scales., \textbf{AM1 =
10{{[}milliarcsecond{]}}} Median relative astrometric measurement error
on 5 arcminute scales shall be less than AM1., \textbf{AM3 =
15{{[}milliarcsecond{]}}} Median relative astrometric measurement error
on 200 arcminute scales., \textbf{AA1 = 50{{[}milliarcsecond{]}}} Median
error in absolute position for each axis, RA and DEC, shall be less than
AA1., \textbf{AF1 = 10{{[}percent{]}}} The maximum fraction of relative
astrometric measurements on 5 arcminute scales to exceed 5 arcminute
outlier limit., \textbf{AD3 = 30{{[}milliarcsecond{]}}} 200 arcminute
outlier limit., \textbf{AB1 = 10{{[}milliarcsecond{]}}} RMS difference
between separations measured in the r-band and those measured in any
other filter., \textbf{AD1 = 20{{[}milliarcsecond{]}}} 5 arcminute
outlier limit., \textbf{AB2 = 20{{[}milliarcsecond{]}}} The color
difference outlier limit for separations measured relative the r-band
filter in any other filter., \textbf{AD2 = 20{{[}milliarcsecond{]}}} 20
arcminute outlier limit., \textbf{ABF1 = 10{{[}percent{]}}} Fraction of
separations measured relative to the r-band that can exceed the color
difference outlier limit., \textbf{AF3 = 10{{[}percent{]}}} Fraction of
relative astrometric measurements on 200 arcminute scales to exceed 200
arcminute outlier limit., \textbf{AF2 = 10{{[}percent{]}}} The maximum
fraction of relative astrometric measurements on 20 arcminute scales to
exceed 20 arcminute outlier limit.{]} \\ \cdashline{1-2}
Requirement Discussion &
\begin{minipage}[]{13cm}
\textbf{Discussion:} The relevant metrics are listed in the table below.
The values in the tables are the target values for LSST but are not
verified as part of this requirement.
\end{minipage}
\\ \cdashline{1-2}
Requirement Priority & 1a \\ \cdashline{1-2}
Upper Level Requirement &
\begin{tabular}{cl}
OSS-REQ-0388 & Astrometric Performance \\
\end{tabular}
\\ \hline
\end{longtable}
}


\subsubsection{Test Cases Summary}
\begin{longtable}{p{3cm}p{2.5cm}p{2.5cm}p{3cm}p{4cm}}
\toprule
\href{https://jira.lsstcorp.org/secure/Tests.jspa\#/testCase/LVV-T378}{LVV-T378} & \multicolumn{4}{p{12cm}}{ Verify Calculation of Astrometric Performance Metrics } \\ \hline
\textbf{Owner} & \textbf{Status} & \textbf{Version} & \textbf{Critical Event} & \textbf{Verification Type} \\ \hline
Leanne Guy & Approved & 1 & false & Test \\ \hline
\end{longtable}
{\scriptsize
\textbf{Objective:}\\
Verify that the DMS system provides software to calculate astrometric
performance metrics, and that the algorithms are properly calculating
the desired quantities. Note that because the DMS requirement is that
the software shall be provided (and not on the actual measured values of
the metrics), we verify all of the requirements via a single test case.
}
\begin{longtable}{p{3cm}p{2.5cm}p{2.5cm}p{3cm}p{4cm}}
\toprule
\href{https://jira.lsstcorp.org/secure/Tests.jspa\#/testCase/LVV-T1749}{LVV-T1749} & \multicolumn{4}{p{12cm}}{ Verify calculation of fraction of relative astrometric measurement error
on 20 arcminute scales exceeding outlier limit } \\ \hline
\textbf{Owner} & \textbf{Status} & \textbf{Version} & \textbf{Critical Event} & \textbf{Verification Type} \\ \hline
Jeffrey Carlin & Approved & 1 & false & Test \\ \hline
\end{longtable}
{\scriptsize
\textbf{Objective:}\\
Verify that the DM system has provided the code to calculate the maximum
fraction of relative astrometric measurements on 20 arcminute scales
that exceed the 20 arcminute outlier limit \textbf{AD2 = 20
milliarcseconds}, and assess whether it meets the requirement that it
shall be less than \textbf{AF2 = 10 percent.}
}
  
 \newpage 
\subsection{[LVV-9771] DMS-REQ-0360-V-06: Color difference outlier limit relative to r-band }\label{lvv-9771}

\begin{longtable}{cccc}
\hline
\textbf{Jira Link} & \textbf{Assignee} & \textbf{Status} & \textbf{Test Cases}\\ \hline
\href{https://jira.lsstcorp.org/browse/LVV-9771}{LVV-9771} &
Leanne Guy & Not Covered &
\begin{tabular}{c}
LVV-T378 \\
LVV-T1750 \\
\end{tabular}
\\
\hline
\end{longtable}

\textbf{Verification Element Description:} \\
The color difference outlier limit for separations measured relative to
the r-band filter in any other filter is~\textbf{AB2 =
20~milliarcseconds.}

Associated element DMS-REQ-0360-V-01
(\href{https://jira.lsstcorp.org/browse/LVV-3402}{LVV-3402})~satisfies~the
maximum fraction of astrometric outliers on 5 arcminute scales.

Associated element DMS-REQ-0360-V-02
(\href{https://jira.lsstcorp.org/browse/LVV-9767}{LVV-9767})~satisfies~the
maximum fraction of astrometric outliers on 5 arcminute scales.

Associated element
DMS-REQ-0360-V-03~(\href{https://jira.lsstcorp.org/browse/LVV-9768}{LVV-9768})~satisfies~the
median astrometric error on 5 arcminute scales.

Associated element
DMS-REQ-0360-V-04~(\href{https://jira.lsstcorp.org/browse/LVV-9769}{LVV-9769})~satisfies~the
median astrometric error in absolute positions.

Associated element
DMS-REQ-0360-V-05~(\href{https://jira.lsstcorp.org/browse/LVV-9770}{LVV-9770})~satisfies~the
astrometric outlier limit on 20 arcminute scales.

Associated element
DMS-REQ-0360-V-07~(\href{https://jira.lsstcorp.org/browse/LVV-9773}{LVV-9773})~satisfies
the astrometric outlier limit on 5 arcminute scales.

Associated element
DMS-REQ-0360-V-08~(\href{https://jira.lsstcorp.org/browse/LVV-9774}{LVV-9774})~satisfies~the
median astrometric error on 200 arcminute scales.

Associated element
DMS-REQ-0360-V-09~(\href{https://jira.lsstcorp.org/browse/LVV-9775}{LVV-9775})~satisfies
the astrometric outlier limit on 200 arcminute scales.

Associated element
DMS-REQ-0360-V-10~(\href{https://jira.lsstcorp.org/browse/LVV-9776}{LVV-9776})~satisfies
the maximum fraction of astrometric outliers on 20 arcminute scales.

Associated element
DMS-REQ-0360-V-11~(\href{https://jira.lsstcorp.org/browse/LVV-9777}{LVV-9777})~satisfies
the maximum fraction of r-band color difference outliers.

Associated element
DMS-REQ-0360-V-12~(\href{https://jira.lsstcorp.org/browse/LVV-9778}{LVV-9778})~satisfies~the
RMS difference between separations measured in the r-band and those
measured in any other filter.

Associated element
DMS-REQ-0360-V-13~(\href{https://jira.lsstcorp.org/browse/LVV-9779}{LVV-9779})~satisfies
the maximum fraction of astrometric outliers on 200 arcminute scales.

{\footnotesize
\begin{longtable}{p{2.5cm}p{13.5cm}}
\hline
\multicolumn{2}{c}{\textbf{Requirement Details}}\\ \hline
Requirement ID & DMS-REQ-0360 \\ \cdashline{1-2}
Requirement Description &
\begin{minipage}[]{13cm}
\textbf{Specification:} The DMS shall include software to enable the
calculation of the astrometric performance metrics defined in
OSS-REQ-0388.
\end{minipage}
\\ \cdashline{1-2}
Requirement Parameters & {[}\textbf{AM2 = 10{{[}milliarcsecond{]}}} Median relative astrometric
measurement error on 20 arcminute scales., \textbf{AM1 =
10{{[}milliarcsecond{]}}} Median relative astrometric measurement error
on 5 arcminute scales shall be less than AM1., \textbf{AM3 =
15{{[}milliarcsecond{]}}} Median relative astrometric measurement error
on 200 arcminute scales., \textbf{AA1 = 50{{[}milliarcsecond{]}}} Median
error in absolute position for each axis, RA and DEC, shall be less than
AA1., \textbf{AF1 = 10{{[}percent{]}}} The maximum fraction of relative
astrometric measurements on 5 arcminute scales to exceed 5 arcminute
outlier limit., \textbf{AD3 = 30{{[}milliarcsecond{]}}} 200 arcminute
outlier limit., \textbf{AB1 = 10{{[}milliarcsecond{]}}} RMS difference
between separations measured in the r-band and those measured in any
other filter., \textbf{AD1 = 20{{[}milliarcsecond{]}}} 5 arcminute
outlier limit., \textbf{AB2 = 20{{[}milliarcsecond{]}}} The color
difference outlier limit for separations measured relative the r-band
filter in any other filter., \textbf{AD2 = 20{{[}milliarcsecond{]}}} 20
arcminute outlier limit., \textbf{ABF1 = 10{{[}percent{]}}} Fraction of
separations measured relative to the r-band that can exceed the color
difference outlier limit., \textbf{AF3 = 10{{[}percent{]}}} Fraction of
relative astrometric measurements on 200 arcminute scales to exceed 200
arcminute outlier limit., \textbf{AF2 = 10{{[}percent{]}}} The maximum
fraction of relative astrometric measurements on 20 arcminute scales to
exceed 20 arcminute outlier limit.{]} \\ \cdashline{1-2}
Requirement Discussion &
\begin{minipage}[]{13cm}
\textbf{Discussion:} The relevant metrics are listed in the table below.
The values in the tables are the target values for LSST but are not
verified as part of this requirement.
\end{minipage}
\\ \cdashline{1-2}
Requirement Priority & 1a \\ \cdashline{1-2}
Upper Level Requirement &
\begin{tabular}{cl}
OSS-REQ-0388 & Astrometric Performance \\
\end{tabular}
\\ \hline
\end{longtable}
}


\subsubsection{Test Cases Summary}
\begin{longtable}{p{3cm}p{2.5cm}p{2.5cm}p{3cm}p{4cm}}
\toprule
\href{https://jira.lsstcorp.org/secure/Tests.jspa\#/testCase/LVV-T378}{LVV-T378} & \multicolumn{4}{p{12cm}}{ Verify Calculation of Astrometric Performance Metrics } \\ \hline
\textbf{Owner} & \textbf{Status} & \textbf{Version} & \textbf{Critical Event} & \textbf{Verification Type} \\ \hline
Leanne Guy & Approved & 1 & false & Test \\ \hline
\end{longtable}
{\scriptsize
\textbf{Objective:}\\
Verify that the DMS system provides software to calculate astrometric
performance metrics, and that the algorithms are properly calculating
the desired quantities. Note that because the DMS requirement is that
the software shall be provided (and not on the actual measured values of
the metrics), we verify all of the requirements via a single test case.
}
\begin{longtable}{p{3cm}p{2.5cm}p{2.5cm}p{3cm}p{4cm}}
\toprule
\href{https://jira.lsstcorp.org/secure/Tests.jspa\#/testCase/LVV-T1750}{LVV-T1750} & \multicolumn{4}{p{12cm}}{ Verify calculation of separations relative to r-band exceeding color
difference outlier limit } \\ \hline
\textbf{Owner} & \textbf{Status} & \textbf{Version} & \textbf{Critical Event} & \textbf{Verification Type} \\ \hline
Jeffrey Carlin & Approved & 1 & false & Test \\ \hline
\end{longtable}
{\scriptsize
\textbf{Objective:}\\
Verify that the DM system has provided the code to calculate the
separations measured relative to the r-band that exceed the color
difference outlier limit \textbf{AB2 = 20 milliarcseconds}, and assess
whether it meets the requirement that it shall be less than \textbf{ABF1
= 10 percent.~}
}
  
 \newpage 
\subsection{[LVV-9773] DMS-REQ-0360-V-07: Outlier limit on 5 arcmin scales }\label{lvv-9773}

\begin{longtable}{cccc}
\hline
\textbf{Jira Link} & \textbf{Assignee} & \textbf{Status} & \textbf{Test Cases}\\ \hline
\href{https://jira.lsstcorp.org/browse/LVV-9773}{LVV-9773} &
Leanne Guy & Not Covered &
\begin{tabular}{c}
LVV-T378 \\
LVV-T1746 \\
\end{tabular}
\\
\hline
\end{longtable}

\textbf{Verification Element Description:} \\
The 5 arcminute outlier limit is~\textbf{AD1 = 20~milliarcseconds}.

Associated element DMS-REQ-0360-V-01
(\href{https://jira.lsstcorp.org/browse/LVV-3402}{LVV-3402})~satisfies~the
maximum fraction of astrometric outliers on 5 arcminute scales.

Associated element DMS-REQ-0360-V-02
(\href{https://jira.lsstcorp.org/browse/LVV-9767}{LVV-9767})~satisfies~the
maximum fraction of astrometric outliers on 5 arcminute scales.

Associated element
DMS-REQ-0360-V-03~(\href{https://jira.lsstcorp.org/browse/LVV-9768}{LVV-9768})~satisfies~the
median astrometric error on 5 arcminute scales.

Associated element
DMS-REQ-0360-V-04~(\href{https://jira.lsstcorp.org/browse/LVV-9769}{LVV-9769})~satisfies~the
median astrometric error in absolute positions.

Associated element
DMS-REQ-0360-V-05~(\href{https://jira.lsstcorp.org/browse/LVV-9770}{LVV-9770})~satisfies~the
astrometric outlier limit on 20 arcminute scales.

Associated element
DMS-REQ-0360-V-06~(\href{https://jira.lsstcorp.org/browse/LVV-9771}{LVV-9771})~satisfies~the
color difference outlier limit relative to r-band.

Associated element
DMS-REQ-0360-V-08~(\href{https://jira.lsstcorp.org/browse/LVV-9774}{LVV-9774})~satisfies~the
median astrometric error on 200 arcminute scales.

Associated element
DMS-REQ-0360-V-09~(\href{https://jira.lsstcorp.org/browse/LVV-9775}{LVV-9775})~satisfies
the astrometric outlier limit on 200 arcminute scales.

Associated element
DMS-REQ-0360-V-10~(\href{https://jira.lsstcorp.org/browse/LVV-9776}{LVV-9776})~satisfies
the maximum fraction of astrometric outliers on 20 arcminute scales.

Associated element
DMS-REQ-0360-V-11~(\href{https://jira.lsstcorp.org/browse/LVV-9777}{LVV-9777})~satisfies
the maximum fraction of r-band color difference outliers.

Associated element
DMS-REQ-0360-V-12~(\href{https://jira.lsstcorp.org/browse/LVV-9778}{LVV-9778})~satisfies~the
RMS difference between separations measured in the r-band and those
measured in any other filter.

Associated element
DMS-REQ-0360-V-13~(\href{https://jira.lsstcorp.org/browse/LVV-9779}{LVV-9779})~satisfies
the maximum fraction of astrometric outliers on 200 arcminute scales.

{\footnotesize
\begin{longtable}{p{2.5cm}p{13.5cm}}
\hline
\multicolumn{2}{c}{\textbf{Requirement Details}}\\ \hline
Requirement ID & DMS-REQ-0360 \\ \cdashline{1-2}
Requirement Description &
\begin{minipage}[]{13cm}
\textbf{Specification:} The DMS shall include software to enable the
calculation of the astrometric performance metrics defined in
OSS-REQ-0388.
\end{minipage}
\\ \cdashline{1-2}
Requirement Parameters & {[}\textbf{AM2 = 10{{[}milliarcsecond{]}}} Median relative astrometric
measurement error on 20 arcminute scales., \textbf{AM1 =
10{{[}milliarcsecond{]}}} Median relative astrometric measurement error
on 5 arcminute scales shall be less than AM1., \textbf{AM3 =
15{{[}milliarcsecond{]}}} Median relative astrometric measurement error
on 200 arcminute scales., \textbf{AA1 = 50{{[}milliarcsecond{]}}} Median
error in absolute position for each axis, RA and DEC, shall be less than
AA1., \textbf{AF1 = 10{{[}percent{]}}} The maximum fraction of relative
astrometric measurements on 5 arcminute scales to exceed 5 arcminute
outlier limit., \textbf{AD3 = 30{{[}milliarcsecond{]}}} 200 arcminute
outlier limit., \textbf{AB1 = 10{{[}milliarcsecond{]}}} RMS difference
between separations measured in the r-band and those measured in any
other filter., \textbf{AD1 = 20{{[}milliarcsecond{]}}} 5 arcminute
outlier limit., \textbf{AB2 = 20{{[}milliarcsecond{]}}} The color
difference outlier limit for separations measured relative the r-band
filter in any other filter., \textbf{AD2 = 20{{[}milliarcsecond{]}}} 20
arcminute outlier limit., \textbf{ABF1 = 10{{[}percent{]}}} Fraction of
separations measured relative to the r-band that can exceed the color
difference outlier limit., \textbf{AF3 = 10{{[}percent{]}}} Fraction of
relative astrometric measurements on 200 arcminute scales to exceed 200
arcminute outlier limit., \textbf{AF2 = 10{{[}percent{]}}} The maximum
fraction of relative astrometric measurements on 20 arcminute scales to
exceed 20 arcminute outlier limit.{]} \\ \cdashline{1-2}
Requirement Discussion &
\begin{minipage}[]{13cm}
\textbf{Discussion:} The relevant metrics are listed in the table below.
The values in the tables are the target values for LSST but are not
verified as part of this requirement.
\end{minipage}
\\ \cdashline{1-2}
Requirement Priority & 1a \\ \cdashline{1-2}
Upper Level Requirement &
\begin{tabular}{cl}
OSS-REQ-0388 & Astrometric Performance \\
\end{tabular}
\\ \hline
\end{longtable}
}


\subsubsection{Test Cases Summary}
\begin{longtable}{p{3cm}p{2.5cm}p{2.5cm}p{3cm}p{4cm}}
\toprule
\href{https://jira.lsstcorp.org/secure/Tests.jspa\#/testCase/LVV-T378}{LVV-T378} & \multicolumn{4}{p{12cm}}{ Verify Calculation of Astrometric Performance Metrics } \\ \hline
\textbf{Owner} & \textbf{Status} & \textbf{Version} & \textbf{Critical Event} & \textbf{Verification Type} \\ \hline
Leanne Guy & Approved & 1 & false & Test \\ \hline
\end{longtable}
{\scriptsize
\textbf{Objective:}\\
Verify that the DMS system provides software to calculate astrometric
performance metrics, and that the algorithms are properly calculating
the desired quantities. Note that because the DMS requirement is that
the software shall be provided (and not on the actual measured values of
the metrics), we verify all of the requirements via a single test case.
}
\begin{longtable}{p{3cm}p{2.5cm}p{2.5cm}p{3cm}p{4cm}}
\toprule
\href{https://jira.lsstcorp.org/secure/Tests.jspa\#/testCase/LVV-T1746}{LVV-T1746} & \multicolumn{4}{p{12cm}}{ Verify calculation of fraction of relative astrometric measurement error
on 5 arcminute scales exceeding outlier limit } \\ \hline
\textbf{Owner} & \textbf{Status} & \textbf{Version} & \textbf{Critical Event} & \textbf{Verification Type} \\ \hline
Jeffrey Carlin & Approved & 1 & false & Test \\ \hline
\end{longtable}
{\scriptsize
\textbf{Objective:}\\
Verify that the DM system has provided the code to calculate the maximum
fraction of relative astrometric measurements on 5 arcminute scales that
exceed the 5 arcminute outlier limit \textbf{AD1 = 20 milliarcseconds},
and assess whether it meets the requirement that it shall be less than
\textbf{AF1 = 10 percent.}
}
  
 \newpage 
\subsection{[LVV-9774] DMS-REQ-0360-V-08: Median astrometric error on 200 arcmin scales }\label{lvv-9774}

\begin{longtable}{cccc}
\hline
\textbf{Jira Link} & \textbf{Assignee} & \textbf{Status} & \textbf{Test Cases}\\ \hline
\href{https://jira.lsstcorp.org/browse/LVV-9774}{LVV-9774} &
Leanne Guy & Not Covered &
\begin{tabular}{c}
LVV-T378 \\
LVV-T1751 \\
\end{tabular}
\\
\hline
\end{longtable}

\textbf{Verification Element Description:} \\
The median relative astrometric measurement error on 200 arcminute
scales is less than~\textbf{AM3 = 15~milliarcseconds.}

Associated element DMS-REQ-0360-V-01
(\href{https://jira.lsstcorp.org/browse/LVV-3402}{LVV-3402})~satisfies~the
maximum fraction of astrometric outliers on 5 arcminute scales.

Associated element DMS-REQ-0360-V-02
(\href{https://jira.lsstcorp.org/browse/LVV-9767}{LVV-9767})~satisfies~the
maximum fraction of astrometric outliers on 5 arcminute scales.

Associated element
DMS-REQ-0360-V-03~(\href{https://jira.lsstcorp.org/browse/LVV-9768}{LVV-9768})~satisfies~the
median astrometric error on 5 arcminute scales.

Associated element
DMS-REQ-0360-V-04~(\href{https://jira.lsstcorp.org/browse/LVV-9769}{LVV-9769})~satisfies~the
median astrometric error in absolute positions.

Associated element
DMS-REQ-0360-V-05~(\href{https://jira.lsstcorp.org/browse/LVV-9770}{LVV-9770})~satisfies~the
astrometric outlier limit on 20 arcminute scales.

Associated element
DMS-REQ-0360-V-06~(\href{https://jira.lsstcorp.org/browse/LVV-9771}{LVV-9771})~satisfies~the
color difference outlier limit relative to r-band.

Associated element
DMS-REQ-0360-V-07~(\href{https://jira.lsstcorp.org/browse/LVV-9773}{LVV-9773})~satisfies
the astrometric outlier limit on 5 arcminute scales.

Associated element
DMS-REQ-0360-V-09~(\href{https://jira.lsstcorp.org/browse/LVV-9775}{LVV-9775})~satisfies
the astrometric outlier limit on 200 arcminute scales.

Associated element
DMS-REQ-0360-V-10~(\href{https://jira.lsstcorp.org/browse/LVV-9776}{LVV-9776})~satisfies
the maximum fraction of astrometric outliers on 20 arcminute scales.

Associated element
DMS-REQ-0360-V-11~(\href{https://jira.lsstcorp.org/browse/LVV-9777}{LVV-9777})~satisfies
the maximum fraction of r-band color difference outliers.

Associated element
DMS-REQ-0360-V-12~(\href{https://jira.lsstcorp.org/browse/LVV-9778}{LVV-9778})~satisfies~the
RMS difference between separations measured in the r-band and those
measured in any other filter.

Associated element
DMS-REQ-0360-V-13~(\href{https://jira.lsstcorp.org/browse/LVV-9779}{LVV-9779})~satisfies
the maximum fraction of astrometric outliers on 200 arcminute scales.

{\footnotesize
\begin{longtable}{p{2.5cm}p{13.5cm}}
\hline
\multicolumn{2}{c}{\textbf{Requirement Details}}\\ \hline
Requirement ID & DMS-REQ-0360 \\ \cdashline{1-2}
Requirement Description &
\begin{minipage}[]{13cm}
\textbf{Specification:} The DMS shall include software to enable the
calculation of the astrometric performance metrics defined in
OSS-REQ-0388.
\end{minipage}
\\ \cdashline{1-2}
Requirement Parameters & {[}\textbf{AM2 = 10{{[}milliarcsecond{]}}} Median relative astrometric
measurement error on 20 arcminute scales., \textbf{AM1 =
10{{[}milliarcsecond{]}}} Median relative astrometric measurement error
on 5 arcminute scales shall be less than AM1., \textbf{AM3 =
15{{[}milliarcsecond{]}}} Median relative astrometric measurement error
on 200 arcminute scales., \textbf{AA1 = 50{{[}milliarcsecond{]}}} Median
error in absolute position for each axis, RA and DEC, shall be less than
AA1., \textbf{AF1 = 10{{[}percent{]}}} The maximum fraction of relative
astrometric measurements on 5 arcminute scales to exceed 5 arcminute
outlier limit., \textbf{AD3 = 30{{[}milliarcsecond{]}}} 200 arcminute
outlier limit., \textbf{AB1 = 10{{[}milliarcsecond{]}}} RMS difference
between separations measured in the r-band and those measured in any
other filter., \textbf{AD1 = 20{{[}milliarcsecond{]}}} 5 arcminute
outlier limit., \textbf{AB2 = 20{{[}milliarcsecond{]}}} The color
difference outlier limit for separations measured relative the r-band
filter in any other filter., \textbf{AD2 = 20{{[}milliarcsecond{]}}} 20
arcminute outlier limit., \textbf{ABF1 = 10{{[}percent{]}}} Fraction of
separations measured relative to the r-band that can exceed the color
difference outlier limit., \textbf{AF3 = 10{{[}percent{]}}} Fraction of
relative astrometric measurements on 200 arcminute scales to exceed 200
arcminute outlier limit., \textbf{AF2 = 10{{[}percent{]}}} The maximum
fraction of relative astrometric measurements on 20 arcminute scales to
exceed 20 arcminute outlier limit.{]} \\ \cdashline{1-2}
Requirement Discussion &
\begin{minipage}[]{13cm}
\textbf{Discussion:} The relevant metrics are listed in the table below.
The values in the tables are the target values for LSST but are not
verified as part of this requirement.
\end{minipage}
\\ \cdashline{1-2}
Requirement Priority & 1a \\ \cdashline{1-2}
Upper Level Requirement &
\begin{tabular}{cl}
OSS-REQ-0388 & Astrometric Performance \\
\end{tabular}
\\ \hline
\end{longtable}
}


\subsubsection{Test Cases Summary}
\begin{longtable}{p{3cm}p{2.5cm}p{2.5cm}p{3cm}p{4cm}}
\toprule
\href{https://jira.lsstcorp.org/secure/Tests.jspa\#/testCase/LVV-T378}{LVV-T378} & \multicolumn{4}{p{12cm}}{ Verify Calculation of Astrometric Performance Metrics } \\ \hline
\textbf{Owner} & \textbf{Status} & \textbf{Version} & \textbf{Critical Event} & \textbf{Verification Type} \\ \hline
Leanne Guy & Approved & 1 & false & Test \\ \hline
\end{longtable}
{\scriptsize
\textbf{Objective:}\\
Verify that the DMS system provides software to calculate astrometric
performance metrics, and that the algorithms are properly calculating
the desired quantities. Note that because the DMS requirement is that
the software shall be provided (and not on the actual measured values of
the metrics), we verify all of the requirements via a single test case.
}
\begin{longtable}{p{3cm}p{2.5cm}p{2.5cm}p{3cm}p{4cm}}
\toprule
\href{https://jira.lsstcorp.org/secure/Tests.jspa\#/testCase/LVV-T1751}{LVV-T1751} & \multicolumn{4}{p{12cm}}{ Verify calculation of median relative astrometric measurement error on
200 arcminute scales } \\ \hline
\textbf{Owner} & \textbf{Status} & \textbf{Version} & \textbf{Critical Event} & \textbf{Verification Type} \\ \hline
Jeffrey Carlin & Approved & 1 & false & Test \\ \hline
\end{longtable}
{\scriptsize
\textbf{Objective:}\\
Verify that the DM system has provided the code to calculate the median
relative astrometric measurement error on 200 arcminute scales and
assess whether it meets the requirement that it shall be no more than
AM3 = 15 milliarcseconds.
}
  
 \newpage 
\subsection{[LVV-9775] DMS-REQ-0360-V-09: Outlier limit on 200 arcmin scales }\label{lvv-9775}

\begin{longtable}{cccc}
\hline
\textbf{Jira Link} & \textbf{Assignee} & \textbf{Status} & \textbf{Test Cases}\\ \hline
\href{https://jira.lsstcorp.org/browse/LVV-9775}{LVV-9775} &
Leanne Guy & Not Covered &
\begin{tabular}{c}
LVV-T378 \\
\end{tabular}
\\
\hline
\end{longtable}

\textbf{Verification Element Description:} \\
The 200 arcminute outlier limit is~\textbf{AD3 = 30~milliarcseconds}.

Associated element DMS-REQ-0360-V-01
(\href{https://jira.lsstcorp.org/browse/LVV-3402}{LVV-3402})~satisfies~the
maximum fraction of astrometric outliers on 5 arcminute scales.

Associated element DMS-REQ-0360-V-02
(\href{https://jira.lsstcorp.org/browse/LVV-9767}{LVV-9767})~satisfies~the
maximum fraction of astrometric outliers on 5 arcminute scales.

Associated element
DMS-REQ-0360-V-03~(\href{https://jira.lsstcorp.org/browse/LVV-9768}{LVV-9768})~satisfies~the
median astrometric error on 5 arcminute scales.

Associated element
DMS-REQ-0360-V-04~(\href{https://jira.lsstcorp.org/browse/LVV-9769}{LVV-9769})~satisfies~the
median astrometric error in absolute positions.

Associated element
DMS-REQ-0360-V-05~(\href{https://jira.lsstcorp.org/browse/LVV-9770}{LVV-9770})~satisfies~the
astrometric outlier limit on 20 arcminute scales.

Associated element
DMS-REQ-0360-V-06~(\href{https://jira.lsstcorp.org/browse/LVV-9771}{LVV-9771})~satisfies~the
color difference outlier limit relative to r-band.

Associated element
DMS-REQ-0360-V-07~(\href{https://jira.lsstcorp.org/browse/LVV-9773}{LVV-9773})~satisfies
the astrometric outlier limit on 5 arcminute scales.

Associated element
DMS-REQ-0360-V-08~(\href{https://jira.lsstcorp.org/browse/LVV-9774}{LVV-9774})~satisfies~the
median astrometric error on 200 arcminute scales.

Associated element
DMS-REQ-0360-V-10~(\href{https://jira.lsstcorp.org/browse/LVV-9776}{LVV-9776})~satisfies
the maximum fraction of astrometric outliers on 20 arcminute scales.

Associated element
DMS-REQ-0360-V-11~(\href{https://jira.lsstcorp.org/browse/LVV-9777}{LVV-9777})~satisfies
the maximum fraction of r-band color difference outliers.

Associated element
DMS-REQ-0360-V-12~(\href{https://jira.lsstcorp.org/browse/LVV-9778}{LVV-9778})~satisfies~the
RMS difference between separations measured in the r-band and those
measured in any other filter.

Associated element
DMS-REQ-0360-V-13~(\href{https://jira.lsstcorp.org/browse/LVV-9779}{LVV-9779})~satisfies
the maximum fraction of astrometric outliers on 200 arcminute scales.

{\footnotesize
\begin{longtable}{p{2.5cm}p{13.5cm}}
\hline
\multicolumn{2}{c}{\textbf{Requirement Details}}\\ \hline
Requirement ID & DMS-REQ-0360 \\ \cdashline{1-2}
Requirement Description &
\begin{minipage}[]{13cm}
\textbf{Specification:} The DMS shall include software to enable the
calculation of the astrometric performance metrics defined in
OSS-REQ-0388.
\end{minipage}
\\ \cdashline{1-2}
Requirement Parameters & {[}\textbf{AM2 = 10{{[}milliarcsecond{]}}} Median relative astrometric
measurement error on 20 arcminute scales., \textbf{AM1 =
10{{[}milliarcsecond{]}}} Median relative astrometric measurement error
on 5 arcminute scales shall be less than AM1., \textbf{AM3 =
15{{[}milliarcsecond{]}}} Median relative astrometric measurement error
on 200 arcminute scales., \textbf{AA1 = 50{{[}milliarcsecond{]}}} Median
error in absolute position for each axis, RA and DEC, shall be less than
AA1., \textbf{AF1 = 10{{[}percent{]}}} The maximum fraction of relative
astrometric measurements on 5 arcminute scales to exceed 5 arcminute
outlier limit., \textbf{AD3 = 30{{[}milliarcsecond{]}}} 200 arcminute
outlier limit., \textbf{AB1 = 10{{[}milliarcsecond{]}}} RMS difference
between separations measured in the r-band and those measured in any
other filter., \textbf{AD1 = 20{{[}milliarcsecond{]}}} 5 arcminute
outlier limit., \textbf{AB2 = 20{{[}milliarcsecond{]}}} The color
difference outlier limit for separations measured relative the r-band
filter in any other filter., \textbf{AD2 = 20{{[}milliarcsecond{]}}} 20
arcminute outlier limit., \textbf{ABF1 = 10{{[}percent{]}}} Fraction of
separations measured relative to the r-band that can exceed the color
difference outlier limit., \textbf{AF3 = 10{{[}percent{]}}} Fraction of
relative astrometric measurements on 200 arcminute scales to exceed 200
arcminute outlier limit., \textbf{AF2 = 10{{[}percent{]}}} The maximum
fraction of relative astrometric measurements on 20 arcminute scales to
exceed 20 arcminute outlier limit.{]} \\ \cdashline{1-2}
Requirement Discussion &
\begin{minipage}[]{13cm}
\textbf{Discussion:} The relevant metrics are listed in the table below.
The values in the tables are the target values for LSST but are not
verified as part of this requirement.
\end{minipage}
\\ \cdashline{1-2}
Requirement Priority & 1a \\ \cdashline{1-2}
Upper Level Requirement &
\begin{tabular}{cl}
OSS-REQ-0388 & Astrometric Performance \\
\end{tabular}
\\ \hline
\end{longtable}
}


\subsubsection{Test Cases Summary}
\begin{longtable}{p{3cm}p{2.5cm}p{2.5cm}p{3cm}p{4cm}}
\toprule
\href{https://jira.lsstcorp.org/secure/Tests.jspa\#/testCase/LVV-T378}{LVV-T378} & \multicolumn{4}{p{12cm}}{ Verify Calculation of Astrometric Performance Metrics } \\ \hline
\textbf{Owner} & \textbf{Status} & \textbf{Version} & \textbf{Critical Event} & \textbf{Verification Type} \\ \hline
Leanne Guy & Approved & 1 & false & Test \\ \hline
\end{longtable}
{\scriptsize
\textbf{Objective:}\\
Verify that the DMS system provides software to calculate astrometric
performance metrics, and that the algorithms are properly calculating
the desired quantities. Note that because the DMS requirement is that
the software shall be provided (and not on the actual measured values of
the metrics), we verify all of the requirements via a single test case.
}
  
 \newpage 
\subsection{[LVV-9776] DMS-REQ-0360-V-10: Max fraction exceeding limit on 20 arcmin scales }\label{lvv-9776}

\begin{longtable}{cccc}
\hline
\textbf{Jira Link} & \textbf{Assignee} & \textbf{Status} & \textbf{Test Cases}\\ \hline
\href{https://jira.lsstcorp.org/browse/LVV-9776}{LVV-9776} &
Leanne Guy & Not Covered &
\begin{tabular}{c}
LVV-T378 \\
LVV-T1749 \\
\end{tabular}
\\
\hline
\end{longtable}

\textbf{Verification Element Description:} \\
The maximum fraction of relative astrometric measurements on 20
arcminute scales that exceeds the 20 arcminute outlier limit
is~\textbf{AF2 = 10~percent.}

Associated element DMS-REQ-0360-V-01
(\href{https://jira.lsstcorp.org/browse/LVV-3402}{LVV-3402})~satisfies~the
maximum fraction of astrometric outliers on 5 arcminute scales.

Associated element DMS-REQ-0360-V-02
(\href{https://jira.lsstcorp.org/browse/LVV-9767}{LVV-9767})~satisfies~the
maximum fraction of astrometric outliers on 5 arcminute scales.

Associated element
DMS-REQ-0360-V-03~(\href{https://jira.lsstcorp.org/browse/LVV-9768}{LVV-9768})~satisfies~the
median astrometric error on 5 arcminute scales.

Associated element
DMS-REQ-0360-V-04~(\href{https://jira.lsstcorp.org/browse/LVV-9769}{LVV-9769})~satisfies~the
median astrometric error in absolute positions.

Associated element
DMS-REQ-0360-V-05~(\href{https://jira.lsstcorp.org/browse/LVV-9770}{LVV-9770})~satisfies~the
astrometric outlier limit on 20 arcminute scales.

Associated element
DMS-REQ-0360-V-06~(\href{https://jira.lsstcorp.org/browse/LVV-9771}{LVV-9771})~satisfies~the
color difference outlier limit relative to r-band.

Associated element
DMS-REQ-0360-V-07~(\href{https://jira.lsstcorp.org/browse/LVV-9773}{LVV-9773})~satisfies
the astrometric outlier limit on 5 arcminute scales.

Associated element
DMS-REQ-0360-V-08~(\href{https://jira.lsstcorp.org/browse/LVV-9774}{LVV-9774})~satisfies~the
median astrometric error on 200 arcminute scales.

Associated element
DMS-REQ-0360-V-09~(\href{https://jira.lsstcorp.org/browse/LVV-9775}{LVV-9775})~satisfies
the astrometric outlier limit on 200 arcminute scales.

Associated element
DMS-REQ-0360-V-11~(\href{https://jira.lsstcorp.org/browse/LVV-9777}{LVV-9777})~satisfies
the maximum fraction of r-band color difference outliers.

Associated element
DMS-REQ-0360-V-12~(\href{https://jira.lsstcorp.org/browse/LVV-9778}{LVV-9778})~satisfies~the
RMS difference between separations measured in the r-band and those
measured in any other filter.

Associated element
DMS-REQ-0360-V-13~(\href{https://jira.lsstcorp.org/browse/LVV-9779}{LVV-9779})~satisfies
the maximum fraction of astrometric outliers on 200 arcminute scales.

{\footnotesize
\begin{longtable}{p{2.5cm}p{13.5cm}}
\hline
\multicolumn{2}{c}{\textbf{Requirement Details}}\\ \hline
Requirement ID & DMS-REQ-0360 \\ \cdashline{1-2}
Requirement Description &
\begin{minipage}[]{13cm}
\textbf{Specification:} The DMS shall include software to enable the
calculation of the astrometric performance metrics defined in
OSS-REQ-0388.
\end{minipage}
\\ \cdashline{1-2}
Requirement Parameters & {[}\textbf{AM2 = 10{{[}milliarcsecond{]}}} Median relative astrometric
measurement error on 20 arcminute scales., \textbf{AM1 =
10{{[}milliarcsecond{]}}} Median relative astrometric measurement error
on 5 arcminute scales shall be less than AM1., \textbf{AM3 =
15{{[}milliarcsecond{]}}} Median relative astrometric measurement error
on 200 arcminute scales., \textbf{AA1 = 50{{[}milliarcsecond{]}}} Median
error in absolute position for each axis, RA and DEC, shall be less than
AA1., \textbf{AF1 = 10{{[}percent{]}}} The maximum fraction of relative
astrometric measurements on 5 arcminute scales to exceed 5 arcminute
outlier limit., \textbf{AD3 = 30{{[}milliarcsecond{]}}} 200 arcminute
outlier limit., \textbf{AB1 = 10{{[}milliarcsecond{]}}} RMS difference
between separations measured in the r-band and those measured in any
other filter., \textbf{AD1 = 20{{[}milliarcsecond{]}}} 5 arcminute
outlier limit., \textbf{AB2 = 20{{[}milliarcsecond{]}}} The color
difference outlier limit for separations measured relative the r-band
filter in any other filter., \textbf{AD2 = 20{{[}milliarcsecond{]}}} 20
arcminute outlier limit., \textbf{ABF1 = 10{{[}percent{]}}} Fraction of
separations measured relative to the r-band that can exceed the color
difference outlier limit., \textbf{AF3 = 10{{[}percent{]}}} Fraction of
relative astrometric measurements on 200 arcminute scales to exceed 200
arcminute outlier limit., \textbf{AF2 = 10{{[}percent{]}}} The maximum
fraction of relative astrometric measurements on 20 arcminute scales to
exceed 20 arcminute outlier limit.{]} \\ \cdashline{1-2}
Requirement Discussion &
\begin{minipage}[]{13cm}
\textbf{Discussion:} The relevant metrics are listed in the table below.
The values in the tables are the target values for LSST but are not
verified as part of this requirement.
\end{minipage}
\\ \cdashline{1-2}
Requirement Priority & 1a \\ \cdashline{1-2}
Upper Level Requirement &
\begin{tabular}{cl}
OSS-REQ-0388 & Astrometric Performance \\
\end{tabular}
\\ \hline
\end{longtable}
}


\subsubsection{Test Cases Summary}
\begin{longtable}{p{3cm}p{2.5cm}p{2.5cm}p{3cm}p{4cm}}
\toprule
\href{https://jira.lsstcorp.org/secure/Tests.jspa\#/testCase/LVV-T378}{LVV-T378} & \multicolumn{4}{p{12cm}}{ Verify Calculation of Astrometric Performance Metrics } \\ \hline
\textbf{Owner} & \textbf{Status} & \textbf{Version} & \textbf{Critical Event} & \textbf{Verification Type} \\ \hline
Leanne Guy & Approved & 1 & false & Test \\ \hline
\end{longtable}
{\scriptsize
\textbf{Objective:}\\
Verify that the DMS system provides software to calculate astrometric
performance metrics, and that the algorithms are properly calculating
the desired quantities. Note that because the DMS requirement is that
the software shall be provided (and not on the actual measured values of
the metrics), we verify all of the requirements via a single test case.
}
\begin{longtable}{p{3cm}p{2.5cm}p{2.5cm}p{3cm}p{4cm}}
\toprule
\href{https://jira.lsstcorp.org/secure/Tests.jspa\#/testCase/LVV-T1749}{LVV-T1749} & \multicolumn{4}{p{12cm}}{ Verify calculation of fraction of relative astrometric measurement error
on 20 arcminute scales exceeding outlier limit } \\ \hline
\textbf{Owner} & \textbf{Status} & \textbf{Version} & \textbf{Critical Event} & \textbf{Verification Type} \\ \hline
Jeffrey Carlin & Approved & 1 & false & Test \\ \hline
\end{longtable}
{\scriptsize
\textbf{Objective:}\\
Verify that the DM system has provided the code to calculate the maximum
fraction of relative astrometric measurements on 20 arcminute scales
that exceed the 20 arcminute outlier limit \textbf{AD2 = 20
milliarcseconds}, and assess whether it meets the requirement that it
shall be less than \textbf{AF2 = 10 percent.}
}
  
 \newpage 
\subsection{[LVV-9777] DMS-REQ-0360-V-11: Max fraction of r-band color difference outliers }\label{lvv-9777}

\begin{longtable}{cccc}
\hline
\textbf{Jira Link} & \textbf{Assignee} & \textbf{Status} & \textbf{Test Cases}\\ \hline
\href{https://jira.lsstcorp.org/browse/LVV-9777}{LVV-9777} &
Leanne Guy & Not Covered &
\begin{tabular}{c}
LVV-T378 \\
LVV-T1750 \\
\end{tabular}
\\
\hline
\end{longtable}

\textbf{Verification Element Description:} \\
The fraction of separations measured relative to the r-band that can
exceed the color difference outlier limit is~\textbf{ABF1 = 10~percent.}

Associated element DMS-REQ-0360-V-01
(\href{https://jira.lsstcorp.org/browse/LVV-3402}{LVV-3402})~satisfies~the
maximum fraction of astrometric outliers on 5 arcminute scales.

Associated element DMS-REQ-0360-V-02
(\href{https://jira.lsstcorp.org/browse/LVV-9767}{LVV-9767})~satisfies~the
maximum fraction of astrometric outliers on 5 arcminute scales.

Associated element
DMS-REQ-0360-V-03~(\href{https://jira.lsstcorp.org/browse/LVV-9768}{LVV-9768})~satisfies~the
median astrometric error on 5 arcminute scales.

Associated element
DMS-REQ-0360-V-04~(\href{https://jira.lsstcorp.org/browse/LVV-9769}{LVV-9769})~satisfies~the
median astrometric error in absolute positions.

Associated element
DMS-REQ-0360-V-05~(\href{https://jira.lsstcorp.org/browse/LVV-9770}{LVV-9770})~satisfies~the
astrometric outlier limit on 20 arcminute scales.

Associated element
DMS-REQ-0360-V-06~(\href{https://jira.lsstcorp.org/browse/LVV-9771}{LVV-9771})~satisfies~the
color difference outlier limit relative to r-band.

Associated element
DMS-REQ-0360-V-07~(\href{https://jira.lsstcorp.org/browse/LVV-9773}{LVV-9773})~satisfies
the astrometric outlier limit on 5 arcminute scales.

Associated element
DMS-REQ-0360-V-08~(\href{https://jira.lsstcorp.org/browse/LVV-9774}{LVV-9774})~satisfies~the
median astrometric error on 200 arcminute scales.

Associated element
DMS-REQ-0360-V-09~(\href{https://jira.lsstcorp.org/browse/LVV-9775}{LVV-9775})~satisfies
the astrometric outlier limit on 200 arcminute scales.

Associated element
DMS-REQ-0360-V-10~(\href{https://jira.lsstcorp.org/browse/LVV-9776}{LVV-9776})~satisfies
the maximum fraction of astrometric outliers on 20 arcminute scales.

Associated element
DMS-REQ-0360-V-12~(\href{https://jira.lsstcorp.org/browse/LVV-9778}{LVV-9778})~satisfies~the
RMS difference between separations measured in the r-band and those
measured in any other filter.

Associated element
DMS-REQ-0360-V-13~(\href{https://jira.lsstcorp.org/browse/LVV-9779}{LVV-9779})~satisfies
the maximum fraction of astrometric outliers on 200 arcminute scales.

{\footnotesize
\begin{longtable}{p{2.5cm}p{13.5cm}}
\hline
\multicolumn{2}{c}{\textbf{Requirement Details}}\\ \hline
Requirement ID & DMS-REQ-0360 \\ \cdashline{1-2}
Requirement Description &
\begin{minipage}[]{13cm}
\textbf{Specification:} The DMS shall include software to enable the
calculation of the astrometric performance metrics defined in
OSS-REQ-0388.
\end{minipage}
\\ \cdashline{1-2}
Requirement Parameters & {[}\textbf{AM2 = 10{{[}milliarcsecond{]}}} Median relative astrometric
measurement error on 20 arcminute scales., \textbf{AM1 =
10{{[}milliarcsecond{]}}} Median relative astrometric measurement error
on 5 arcminute scales shall be less than AM1., \textbf{AM3 =
15{{[}milliarcsecond{]}}} Median relative astrometric measurement error
on 200 arcminute scales., \textbf{AA1 = 50{{[}milliarcsecond{]}}} Median
error in absolute position for each axis, RA and DEC, shall be less than
AA1., \textbf{AF1 = 10{{[}percent{]}}} The maximum fraction of relative
astrometric measurements on 5 arcminute scales to exceed 5 arcminute
outlier limit., \textbf{AD3 = 30{{[}milliarcsecond{]}}} 200 arcminute
outlier limit., \textbf{AB1 = 10{{[}milliarcsecond{]}}} RMS difference
between separations measured in the r-band and those measured in any
other filter., \textbf{AD1 = 20{{[}milliarcsecond{]}}} 5 arcminute
outlier limit., \textbf{AB2 = 20{{[}milliarcsecond{]}}} The color
difference outlier limit for separations measured relative the r-band
filter in any other filter., \textbf{AD2 = 20{{[}milliarcsecond{]}}} 20
arcminute outlier limit., \textbf{ABF1 = 10{{[}percent{]}}} Fraction of
separations measured relative to the r-band that can exceed the color
difference outlier limit., \textbf{AF3 = 10{{[}percent{]}}} Fraction of
relative astrometric measurements on 200 arcminute scales to exceed 200
arcminute outlier limit., \textbf{AF2 = 10{{[}percent{]}}} The maximum
fraction of relative astrometric measurements on 20 arcminute scales to
exceed 20 arcminute outlier limit.{]} \\ \cdashline{1-2}
Requirement Discussion &
\begin{minipage}[]{13cm}
\textbf{Discussion:} The relevant metrics are listed in the table below.
The values in the tables are the target values for LSST but are not
verified as part of this requirement.
\end{minipage}
\\ \cdashline{1-2}
Requirement Priority & 1a \\ \cdashline{1-2}
Upper Level Requirement &
\begin{tabular}{cl}
OSS-REQ-0388 & Astrometric Performance \\
\end{tabular}
\\ \hline
\end{longtable}
}


\subsubsection{Test Cases Summary}
\begin{longtable}{p{3cm}p{2.5cm}p{2.5cm}p{3cm}p{4cm}}
\toprule
\href{https://jira.lsstcorp.org/secure/Tests.jspa\#/testCase/LVV-T378}{LVV-T378} & \multicolumn{4}{p{12cm}}{ Verify Calculation of Astrometric Performance Metrics } \\ \hline
\textbf{Owner} & \textbf{Status} & \textbf{Version} & \textbf{Critical Event} & \textbf{Verification Type} \\ \hline
Leanne Guy & Approved & 1 & false & Test \\ \hline
\end{longtable}
{\scriptsize
\textbf{Objective:}\\
Verify that the DMS system provides software to calculate astrometric
performance metrics, and that the algorithms are properly calculating
the desired quantities. Note that because the DMS requirement is that
the software shall be provided (and not on the actual measured values of
the metrics), we verify all of the requirements via a single test case.
}
\begin{longtable}{p{3cm}p{2.5cm}p{2.5cm}p{3cm}p{4cm}}
\toprule
\href{https://jira.lsstcorp.org/secure/Tests.jspa\#/testCase/LVV-T1750}{LVV-T1750} & \multicolumn{4}{p{12cm}}{ Verify calculation of separations relative to r-band exceeding color
difference outlier limit } \\ \hline
\textbf{Owner} & \textbf{Status} & \textbf{Version} & \textbf{Critical Event} & \textbf{Verification Type} \\ \hline
Jeffrey Carlin & Approved & 1 & false & Test \\ \hline
\end{longtable}
{\scriptsize
\textbf{Objective:}\\
Verify that the DM system has provided the code to calculate the
separations measured relative to the r-band that exceed the color
difference outlier limit \textbf{AB2 = 20 milliarcseconds}, and assess
whether it meets the requirement that it shall be less than \textbf{ABF1
= 10 percent.~}
}
  
 \newpage 
\subsection{[LVV-9778] DMS-REQ-0360-V-12: RMS difference between r-band and other filter
separation }\label{lvv-9778}

\begin{longtable}{cccc}
\hline
\textbf{Jira Link} & \textbf{Assignee} & \textbf{Status} & \textbf{Test Cases}\\ \hline
\href{https://jira.lsstcorp.org/browse/LVV-9778}{LVV-9778} &
Leanne Guy & Not Covered &
\begin{tabular}{c}
LVV-T378 \\
LVV-T1753 \\
\end{tabular}
\\
\hline
\end{longtable}

\textbf{Verification Element Description:} \\
The RMS difference between separations measured in the r-band and those
measured in any other filter shall be less than~\textbf{AB1 =
10~milliarcseconds.}~

Associated element DMS-REQ-0360-V-01
(\href{https://jira.lsstcorp.org/browse/LVV-3402}{LVV-3402})~satisfies~the
maximum fraction of astrometric outliers on 5 arcminute scales.

Associated element DMS-REQ-0360-V-02
(\href{https://jira.lsstcorp.org/browse/LVV-9767}{LVV-9767})~satisfies~the
maximum fraction of astrometric outliers on 5 arcminute scales.

Associated element
DMS-REQ-0360-V-03~(\href{https://jira.lsstcorp.org/browse/LVV-9768}{LVV-9768})~satisfies~the
median astrometric error on 5 arcminute scales.

Associated element
DMS-REQ-0360-V-04~(\href{https://jira.lsstcorp.org/browse/LVV-9769}{LVV-9769})~satisfies~the
median astrometric error in absolute positions.

Associated element
DMS-REQ-0360-V-05~(\href{https://jira.lsstcorp.org/browse/LVV-9770}{LVV-9770})~satisfies~the
astrometric outlier limit on 20 arcminute scales.

Associated element
DMS-REQ-0360-V-06~(\href{https://jira.lsstcorp.org/browse/LVV-9771}{LVV-9771})~satisfies~the
color difference outlier limit relative to r-band.

Associated element
DMS-REQ-0360-V-07~(\href{https://jira.lsstcorp.org/browse/LVV-9773}{LVV-9773})~satisfies
the astrometric outlier limit on 5 arcminute scales.

Associated element
DMS-REQ-0360-V-08~(\href{https://jira.lsstcorp.org/browse/LVV-9774}{LVV-9774})~satisfies~the
median astrometric error on 200 arcminute scales.

Associated element
DMS-REQ-0360-V-09~(\href{https://jira.lsstcorp.org/browse/LVV-9775}{LVV-9775})~satisfies
the astrometric outlier limit on 200 arcminute scales.

Associated element
DMS-REQ-0360-V-10~(\href{https://jira.lsstcorp.org/browse/LVV-9776}{LVV-9776})~satisfies
the maximum fraction of astrometric outliers on 20 arcminute scales.

Associated element
DMS-REQ-0360-V-11~(\href{https://jira.lsstcorp.org/browse/LVV-9777}{LVV-9777})~satisfies
the maximum fraction of r-band color difference outliers.

Associated element
DMS-REQ-0360-V-13~(\href{https://jira.lsstcorp.org/browse/LVV-9779}{LVV-9779})~satisfies
the maximum fraction of astrometric outliers on 200 arcminute scales.

{\footnotesize
\begin{longtable}{p{2.5cm}p{13.5cm}}
\hline
\multicolumn{2}{c}{\textbf{Requirement Details}}\\ \hline
Requirement ID & DMS-REQ-0360 \\ \cdashline{1-2}
Requirement Description &
\begin{minipage}[]{13cm}
\textbf{Specification:} The DMS shall include software to enable the
calculation of the astrometric performance metrics defined in
OSS-REQ-0388.
\end{minipage}
\\ \cdashline{1-2}
Requirement Parameters & {[}\textbf{AM2 = 10{{[}milliarcsecond{]}}} Median relative astrometric
measurement error on 20 arcminute scales., \textbf{AM1 =
10{{[}milliarcsecond{]}}} Median relative astrometric measurement error
on 5 arcminute scales shall be less than AM1., \textbf{AM3 =
15{{[}milliarcsecond{]}}} Median relative astrometric measurement error
on 200 arcminute scales., \textbf{AA1 = 50{{[}milliarcsecond{]}}} Median
error in absolute position for each axis, RA and DEC, shall be less than
AA1., \textbf{AF1 = 10{{[}percent{]}}} The maximum fraction of relative
astrometric measurements on 5 arcminute scales to exceed 5 arcminute
outlier limit., \textbf{AD3 = 30{{[}milliarcsecond{]}}} 200 arcminute
outlier limit., \textbf{AB1 = 10{{[}milliarcsecond{]}}} RMS difference
between separations measured in the r-band and those measured in any
other filter., \textbf{AD1 = 20{{[}milliarcsecond{]}}} 5 arcminute
outlier limit., \textbf{AB2 = 20{{[}milliarcsecond{]}}} The color
difference outlier limit for separations measured relative the r-band
filter in any other filter., \textbf{AD2 = 20{{[}milliarcsecond{]}}} 20
arcminute outlier limit., \textbf{ABF1 = 10{{[}percent{]}}} Fraction of
separations measured relative to the r-band that can exceed the color
difference outlier limit., \textbf{AF3 = 10{{[}percent{]}}} Fraction of
relative astrometric measurements on 200 arcminute scales to exceed 200
arcminute outlier limit., \textbf{AF2 = 10{{[}percent{]}}} The maximum
fraction of relative astrometric measurements on 20 arcminute scales to
exceed 20 arcminute outlier limit.{]} \\ \cdashline{1-2}
Requirement Discussion &
\begin{minipage}[]{13cm}
\textbf{Discussion:} The relevant metrics are listed in the table below.
The values in the tables are the target values for LSST but are not
verified as part of this requirement.
\end{minipage}
\\ \cdashline{1-2}
Requirement Priority & 1a \\ \cdashline{1-2}
Upper Level Requirement &
\begin{tabular}{cl}
OSS-REQ-0388 & Astrometric Performance \\
\end{tabular}
\\ \hline
\end{longtable}
}


\subsubsection{Test Cases Summary}
\begin{longtable}{p{3cm}p{2.5cm}p{2.5cm}p{3cm}p{4cm}}
\toprule
\href{https://jira.lsstcorp.org/secure/Tests.jspa\#/testCase/LVV-T378}{LVV-T378} & \multicolumn{4}{p{12cm}}{ Verify Calculation of Astrometric Performance Metrics } \\ \hline
\textbf{Owner} & \textbf{Status} & \textbf{Version} & \textbf{Critical Event} & \textbf{Verification Type} \\ \hline
Leanne Guy & Approved & 1 & false & Test \\ \hline
\end{longtable}
{\scriptsize
\textbf{Objective:}\\
Verify that the DMS system provides software to calculate astrometric
performance metrics, and that the algorithms are properly calculating
the desired quantities. Note that because the DMS requirement is that
the software shall be provided (and not on the actual measured values of
the metrics), we verify all of the requirements via a single test case.
}
\begin{longtable}{p{3cm}p{2.5cm}p{2.5cm}p{3cm}p{4cm}}
\toprule
\href{https://jira.lsstcorp.org/secure/Tests.jspa\#/testCase/LVV-T1753}{LVV-T1753} & \multicolumn{4}{p{12cm}}{ Verify calculation of RMS difference of separations relative to r-band } \\ \hline
\textbf{Owner} & \textbf{Status} & \textbf{Version} & \textbf{Critical Event} & \textbf{Verification Type} \\ \hline
Jeffrey Carlin & Approved & 1 & false & Test \\ \hline
\end{longtable}
{\scriptsize
\textbf{Objective:}\\
Verify that the DM system has provided the code to calculate the
separations measured relative to the r-band, and assess whether it meets
the requirement that it shall be less than \textbf{AB1 =
10~milliarcseconds.}
}
  
 \newpage 
\subsection{[LVV-9779] DMS-REQ-0360-V-13: Max fraction exceeding limit on 200 arcmin scales }\label{lvv-9779}

\begin{longtable}{cccc}
\hline
\textbf{Jira Link} & \textbf{Assignee} & \textbf{Status} & \textbf{Test Cases}\\ \hline
\href{https://jira.lsstcorp.org/browse/LVV-9779}{LVV-9779} &
Leanne Guy & Not Covered &
\begin{tabular}{c}
LVV-T378 \\
LVV-T1752 \\
\end{tabular}
\\
\hline
\end{longtable}

\textbf{Verification Element Description:} \\
The fraction of relative astrometric measurements on 200 arcminute
scales to exceed the 200 arcminute outlier limit is less
than~\textbf{AF3 = 10~percent.}

Associated element DMS-REQ-0360-V-01
(\href{https://jira.lsstcorp.org/browse/LVV-3402}{LVV-3402})~satisfies~the
maximum fraction of astrometric outliers on 5 arcminute scales.

Associated element DMS-REQ-0360-V-02
(\href{https://jira.lsstcorp.org/browse/LVV-9767}{LVV-9767})~satisfies~the
maximum fraction of astrometric outliers on 5 arcminute scales.

Associated element
DMS-REQ-0360-V-03~(\href{https://jira.lsstcorp.org/browse/LVV-9768}{LVV-9768})~satisfies~the
median astrometric error on 5 arcminute scales.

Associated element
DMS-REQ-0360-V-04~(\href{https://jira.lsstcorp.org/browse/LVV-9769}{LVV-9769})~satisfies~the
median astrometric error in absolute positions.

Associated element
DMS-REQ-0360-V-05~(\href{https://jira.lsstcorp.org/browse/LVV-9770}{LVV-9770})~satisfies~the
astrometric outlier limit on 20 arcminute scales.

Associated element
DMS-REQ-0360-V-06~(\href{https://jira.lsstcorp.org/browse/LVV-9771}{LVV-9771})~satisfies~the
color difference outlier limit relative to r-band.

Associated element
DMS-REQ-0360-V-07~(\href{https://jira.lsstcorp.org/browse/LVV-9773}{LVV-9773})~satisfies
the astrometric outlier limit on 5 arcminute scales.

Associated element
DMS-REQ-0360-V-08~(\href{https://jira.lsstcorp.org/browse/LVV-9774}{LVV-9774})~satisfies~the
median astrometric error on 200 arcminute scales.

Associated element
DMS-REQ-0360-V-09~(\href{https://jira.lsstcorp.org/browse/LVV-9775}{LVV-9775})~satisfies
the astrometric outlier limit on 200 arcminute scales.

Associated element
DMS-REQ-0360-V-10~(\href{https://jira.lsstcorp.org/browse/LVV-9776}{LVV-9776})~satisfies
the maximum fraction of astrometric outliers on 20 arcminute scales.

Associated element
DMS-REQ-0360-V-11~(\href{https://jira.lsstcorp.org/browse/LVV-9777}{LVV-9777})~satisfies
the maximum fraction of r-band color difference outliers.

Associated element
DMS-REQ-0360-V-12~(\href{https://jira.lsstcorp.org/browse/LVV-9778}{LVV-9778})~satisfies~the
RMS difference between separations measured in the r-band and those
measured in any other filter.

{\footnotesize
\begin{longtable}{p{2.5cm}p{13.5cm}}
\hline
\multicolumn{2}{c}{\textbf{Requirement Details}}\\ \hline
Requirement ID & DMS-REQ-0360 \\ \cdashline{1-2}
Requirement Description &
\begin{minipage}[]{13cm}
\textbf{Specification:} The DMS shall include software to enable the
calculation of the astrometric performance metrics defined in
OSS-REQ-0388.
\end{minipage}
\\ \cdashline{1-2}
Requirement Parameters & {[}\textbf{AM2 = 10{{[}milliarcsecond{]}}} Median relative astrometric
measurement error on 20 arcminute scales., \textbf{AM1 =
10{{[}milliarcsecond{]}}} Median relative astrometric measurement error
on 5 arcminute scales shall be less than AM1., \textbf{AM3 =
15{{[}milliarcsecond{]}}} Median relative astrometric measurement error
on 200 arcminute scales., \textbf{AA1 = 50{{[}milliarcsecond{]}}} Median
error in absolute position for each axis, RA and DEC, shall be less than
AA1., \textbf{AF1 = 10{{[}percent{]}}} The maximum fraction of relative
astrometric measurements on 5 arcminute scales to exceed 5 arcminute
outlier limit., \textbf{AD3 = 30{{[}milliarcsecond{]}}} 200 arcminute
outlier limit., \textbf{AB1 = 10{{[}milliarcsecond{]}}} RMS difference
between separations measured in the r-band and those measured in any
other filter., \textbf{AD1 = 20{{[}milliarcsecond{]}}} 5 arcminute
outlier limit., \textbf{AB2 = 20{{[}milliarcsecond{]}}} The color
difference outlier limit for separations measured relative the r-band
filter in any other filter., \textbf{AD2 = 20{{[}milliarcsecond{]}}} 20
arcminute outlier limit., \textbf{ABF1 = 10{{[}percent{]}}} Fraction of
separations measured relative to the r-band that can exceed the color
difference outlier limit., \textbf{AF3 = 10{{[}percent{]}}} Fraction of
relative astrometric measurements on 200 arcminute scales to exceed 200
arcminute outlier limit., \textbf{AF2 = 10{{[}percent{]}}} The maximum
fraction of relative astrometric measurements on 20 arcminute scales to
exceed 20 arcminute outlier limit.{]} \\ \cdashline{1-2}
Requirement Discussion &
\begin{minipage}[]{13cm}
\textbf{Discussion:} The relevant metrics are listed in the table below.
The values in the tables are the target values for LSST but are not
verified as part of this requirement.
\end{minipage}
\\ \cdashline{1-2}
Requirement Priority & 1a \\ \cdashline{1-2}
Upper Level Requirement &
\begin{tabular}{cl}
OSS-REQ-0388 & Astrometric Performance \\
\end{tabular}
\\ \hline
\end{longtable}
}


\subsubsection{Test Cases Summary}
\begin{longtable}{p{3cm}p{2.5cm}p{2.5cm}p{3cm}p{4cm}}
\toprule
\href{https://jira.lsstcorp.org/secure/Tests.jspa\#/testCase/LVV-T378}{LVV-T378} & \multicolumn{4}{p{12cm}}{ Verify Calculation of Astrometric Performance Metrics } \\ \hline
\textbf{Owner} & \textbf{Status} & \textbf{Version} & \textbf{Critical Event} & \textbf{Verification Type} \\ \hline
Leanne Guy & Approved & 1 & false & Test \\ \hline
\end{longtable}
{\scriptsize
\textbf{Objective:}\\
Verify that the DMS system provides software to calculate astrometric
performance metrics, and that the algorithms are properly calculating
the desired quantities. Note that because the DMS requirement is that
the software shall be provided (and not on the actual measured values of
the metrics), we verify all of the requirements via a single test case.
}
\begin{longtable}{p{3cm}p{2.5cm}p{2.5cm}p{3cm}p{4cm}}
\toprule
\href{https://jira.lsstcorp.org/secure/Tests.jspa\#/testCase/LVV-T1752}{LVV-T1752} & \multicolumn{4}{p{12cm}}{ Verify calculation of fraction of relative astrometric measurement error
on 200 arcminute scales exceeding outlier limit } \\ \hline
\textbf{Owner} & \textbf{Status} & \textbf{Version} & \textbf{Critical Event} & \textbf{Verification Type} \\ \hline
Jeffrey Carlin & Approved & 1 & false & Test \\ \hline
\end{longtable}
{\scriptsize
\textbf{Objective:}\\
Verify that the DM system has provided the code to calculate the maximum
fraction of relative astrometric measurements on 200 arcminute scales
that exceed the 200 arcminute outlier limit \textbf{AD3 = 30
milliarcseconds}, and assess whether it meets the requirement that it
shall be less than \textbf{AF3 = 10 percent.}
}
  
 \newpage 
\subsection{[LVV-9780] DMS-REQ-0362-V-02: Max fraction of excess ellipticity residuals on 1 and
5 arcmin scales }\label{lvv-9780}

\begin{longtable}{cccc}
\hline
\textbf{Jira Link} & \textbf{Assignee} & \textbf{Status} & \textbf{Test Cases}\\ \hline
\href{https://jira.lsstcorp.org/browse/LVV-9780}{LVV-9780} &
Leanne Guy & Not Covered &
\begin{tabular}{c}
LVV-T376 \\
\end{tabular}
\\
\hline
\end{longtable}

\textbf{Verification Element Description:} \\
The maximum fraction of PSF ellipticity correlation residuals that
exceed the outlier limits shall be no greater than~\textbf{TEF = 15
percent.}

Associated element DMS-REQ-0362-V-01
(\href{https://jira.lsstcorp.org/browse/LVV-3404}{LVV-3404}) satisfies
the median residual PSF ellipticity correlations on 5 arcmin scales.

Associated element
DMS-REQ-0362-V-03~(\href{https://jira.lsstcorp.org/browse/LVV-9781}{LVV-9781})~satisfies
the outlier limit on the PSF ellipticity correlation residuals on 5
arcmin scales.

Associated element DMS-REQ-0362-V-04
(\href{https://jira.lsstcorp.org/browse/LVV-9782}{LVV-9782}) satisfies
the median residual PSF ellipticity correlations on 1 arcmin scales.

Associated element DMS-REQ-0362-V-05
(\href{https://jira.lsstcorp.org/browse/LVV-9783}{LVV-9783})
satisfies~the outlier limit on the PSF ellipticity correlation residuals
on 1 arcmin scales.

{\footnotesize
\begin{longtable}{p{2.5cm}p{13.5cm}}
\hline
\multicolumn{2}{c}{\textbf{Requirement Details}}\\ \hline
Requirement ID & DMS-REQ-0362 \\ \cdashline{1-2}
Requirement Description &
\begin{minipage}[]{13cm}
\textbf{Specification:} The DMS shall include software to enable the
calculation of the elipticity correlations metrics defined in
OSS-REQ-0403, OSS-REQ-0404, and OSS-REQ-0405.
\end{minipage}
\\ \cdashline{1-2}
Requirement Parameters & {[}\textbf{TE3 = 4.0e-5{{[}unitless (angular correlation){]}}} Per-image
limit on the median residual ellipticity correlations at scales less
than 5 arcmin., \textbf{TE4 = 2.0e-7{{[}unitless (angular
correlation){]}}} Per-image limit on the median residual ellipticity
correlations at scales greater than or equal to 5 arcmin., \textbf{TE2 =
1.0e-7{{[}unitless (angular correlation){]}}} Maximum full-survey median
for residual ellipticity correlations at scales greater than or equal to
5 arcmin., \textbf{TEF = 15{{[}percent{]}}} Maximum fraction of visit
images that may exceed the TE3 or TE4 limits., \textbf{TE1 =
2.0e-5{{[}unitless (angular correlation){]}}} Maximum full-survey median
for residual ellipticity correlations at scales less than or equal to 1
arcmin.{]} \\ \cdashline{1-2}
Requirement Discussion &
\begin{minipage}[]{13cm}
\textbf{Discussion:} The relevant metrics are listed in the table below.
The values in the tables are the target values for LSST but are not
verified as part of this requirement.
\end{minipage}
\\ \cdashline{1-2}
Requirement Priority & 1b \\ \cdashline{1-2}
Upper Level Requirement &
\begin{tabular}{cl}
OSS-REQ-0403 & Ellipticity Correlation Function Distribution per Image \\
OSS-REQ-0404 & Ellipticity Correlation Function Distribution for Full Survey
(medians) \\
OSS-REQ-0405 & Ellipticity Correlation Function Distribution for Full Survey
(continuity) \\
\end{tabular}
\\ \hline
\end{longtable}
}


\subsubsection{Test Cases Summary}
\begin{longtable}{p{3cm}p{2.5cm}p{2.5cm}p{3cm}p{4cm}}
\toprule
\href{https://jira.lsstcorp.org/secure/Tests.jspa\#/testCase/LVV-T376}{LVV-T376} & \multicolumn{4}{p{12cm}}{ Verify the Calculation of Ellipticity Residuals and Correlations } \\ \hline
\textbf{Owner} & \textbf{Status} & \textbf{Version} & \textbf{Critical Event} & \textbf{Verification Type} \\ \hline
Leanne Guy & Approved & 1 & false & Test \\ \hline
\end{longtable}
{\scriptsize
\textbf{Objective:}\\
Verify that the DMS includes software to enable the calculation of the
ellipticity residuals and correlation metrics defined in the OSS.~
}
  
 \newpage 
\subsection{[LVV-9781] DMS-REQ-0362-V-03: Outlier limit on 5 arcmin scales - ellipticity }\label{lvv-9781}

\begin{longtable}{cccc}
\hline
\textbf{Jira Link} & \textbf{Assignee} & \textbf{Status} & \textbf{Test Cases}\\ \hline
\href{https://jira.lsstcorp.org/browse/LVV-9781}{LVV-9781} &
Leanne Guy & Not Covered &
\begin{tabular}{c}
\end{tabular}
\\
\hline
\end{longtable}

\textbf{Verification Element Description:} \\
Residuals of PSF ellipticity correlations on 5 arcmin scales shall be no
greater than~\textbf{TE4 = 2.0e-7{{[}arcminuteOutlierLimit{]}}.}

Associated element DMS-REQ-0362-V-01
(\href{https://jira.lsstcorp.org/browse/LVV-3404}{LVV-3404}) satisfies
the median residual PSF ellipticity correlations on 5 arcmin scales.

Associated element DMS-REQ-0362-V-02
(\href{https://jira.lsstcorp.org/browse/LVV-9780}{LVV-9780}) satisfies
the maximum fraction of ellipticity residuals exceeding the outlier
limits.~

Associated element DMS-REQ-0362-V-04
(\href{https://jira.lsstcorp.org/browse/LVV-9782}{LVV-9782}) satisfies
the median residual PSF ellipticity correlations on 1 arcmin scales.

Associated element DMS-REQ-0362-V-05
(\href{https://jira.lsstcorp.org/browse/LVV-9783}{LVV-9783})
satisfies~the outlier limit on the PSF ellipticity correlation residuals
on 1 arcmin scales.

{\footnotesize
\begin{longtable}{p{2.5cm}p{13.5cm}}
\hline
\multicolumn{2}{c}{\textbf{Requirement Details}}\\ \hline
Requirement ID & DMS-REQ-0362 \\ \cdashline{1-2}
Requirement Description &
\begin{minipage}[]{13cm}
\textbf{Specification:} The DMS shall include software to enable the
calculation of the elipticity correlations metrics defined in
OSS-REQ-0403, OSS-REQ-0404, and OSS-REQ-0405.
\end{minipage}
\\ \cdashline{1-2}
Requirement Parameters & {[}\textbf{TE3 = 4.0e-5{{[}unitless (angular correlation){]}}} Per-image
limit on the median residual ellipticity correlations at scales less
than 5 arcmin., \textbf{TE4 = 2.0e-7{{[}unitless (angular
correlation){]}}} Per-image limit on the median residual ellipticity
correlations at scales greater than or equal to 5 arcmin., \textbf{TE2 =
1.0e-7{{[}unitless (angular correlation){]}}} Maximum full-survey median
for residual ellipticity correlations at scales greater than or equal to
5 arcmin., \textbf{TEF = 15{{[}percent{]}}} Maximum fraction of visit
images that may exceed the TE3 or TE4 limits., \textbf{TE1 =
2.0e-5{{[}unitless (angular correlation){]}}} Maximum full-survey median
for residual ellipticity correlations at scales less than or equal to 1
arcmin.{]} \\ \cdashline{1-2}
Requirement Discussion &
\begin{minipage}[]{13cm}
\textbf{Discussion:} The relevant metrics are listed in the table below.
The values in the tables are the target values for LSST but are not
verified as part of this requirement.
\end{minipage}
\\ \cdashline{1-2}
Requirement Priority & 1b \\ \cdashline{1-2}
Upper Level Requirement &
\begin{tabular}{cl}
OSS-REQ-0403 & Ellipticity Correlation Function Distribution per Image \\
OSS-REQ-0404 & Ellipticity Correlation Function Distribution for Full Survey
(medians) \\
OSS-REQ-0405 & Ellipticity Correlation Function Distribution for Full Survey
(continuity) \\
\end{tabular}
\\ \hline
\end{longtable}
}


  
 \newpage 
\subsection{[LVV-9782] DMS-REQ-0362-V-04: Median residual PSF ellipticity correlations on 1
arcmin scales }\label{lvv-9782}

\begin{longtable}{cccc}
\hline
\textbf{Jira Link} & \textbf{Assignee} & \textbf{Status} & \textbf{Test Cases}\\ \hline
\href{https://jira.lsstcorp.org/browse/LVV-9782}{LVV-9782} &
Leanne Guy & Not Covered &
\begin{tabular}{c}
LVV-T1755 \\
\end{tabular}
\\
\hline
\end{longtable}

\textbf{Verification Element Description:} \\
Median residual PSF ellipticity correlations averaged over an arbitrary
field of view for separations less than 1 arcmin shall be no greater
than~\textbf{TE1 = 2.0e-5{{[}arcminuteSeparationCorrelation{]}}.}

Associated element DMS-REQ-0362-V-01
(\href{https://jira.lsstcorp.org/browse/LVV-3404}{LVV-3404}) satisfies
the median residual PSF ellipticity correlations on 5 arcmin scales.

Associated element DMS-REQ-0362-V-02
(\href{https://jira.lsstcorp.org/browse/LVV-9780}{LVV-9780}) satisfies
the maximum fraction of ellipticity residuals exceeding the outlier
limits.~

Associated element
DMS-REQ-0362-V-03~(\href{https://jira.lsstcorp.org/browse/LVV-9781}{LVV-9781})~satisfies
the outlier limit on the PSF ellipticity correlation residuals on 5
arcmin scales.

Associated element DMS-REQ-0362-V-05
(\href{https://jira.lsstcorp.org/browse/LVV-9783}{LVV-9783})
satisfies~the outlier limit on the PSF ellipticity correlation residuals
on 1 arcmin scales.

{\footnotesize
\begin{longtable}{p{2.5cm}p{13.5cm}}
\hline
\multicolumn{2}{c}{\textbf{Requirement Details}}\\ \hline
Requirement ID & DMS-REQ-0362 \\ \cdashline{1-2}
Requirement Description &
\begin{minipage}[]{13cm}
\textbf{Specification:} The DMS shall include software to enable the
calculation of the elipticity correlations metrics defined in
OSS-REQ-0403, OSS-REQ-0404, and OSS-REQ-0405.
\end{minipage}
\\ \cdashline{1-2}
Requirement Parameters & {[}\textbf{TE3 = 4.0e-5{{[}unitless (angular correlation){]}}} Per-image
limit on the median residual ellipticity correlations at scales less
than 5 arcmin., \textbf{TE4 = 2.0e-7{{[}unitless (angular
correlation){]}}} Per-image limit on the median residual ellipticity
correlations at scales greater than or equal to 5 arcmin., \textbf{TE2 =
1.0e-7{{[}unitless (angular correlation){]}}} Maximum full-survey median
for residual ellipticity correlations at scales greater than or equal to
5 arcmin., \textbf{TEF = 15{{[}percent{]}}} Maximum fraction of visit
images that may exceed the TE3 or TE4 limits., \textbf{TE1 =
2.0e-5{{[}unitless (angular correlation){]}}} Maximum full-survey median
for residual ellipticity correlations at scales less than or equal to 1
arcmin.{]} \\ \cdashline{1-2}
Requirement Discussion &
\begin{minipage}[]{13cm}
\textbf{Discussion:} The relevant metrics are listed in the table below.
The values in the tables are the target values for LSST but are not
verified as part of this requirement.
\end{minipage}
\\ \cdashline{1-2}
Requirement Priority & 1b \\ \cdashline{1-2}
Upper Level Requirement &
\begin{tabular}{cl}
OSS-REQ-0403 & Ellipticity Correlation Function Distribution per Image \\
OSS-REQ-0404 & Ellipticity Correlation Function Distribution for Full Survey
(medians) \\
OSS-REQ-0405 & Ellipticity Correlation Function Distribution for Full Survey
(continuity) \\
\end{tabular}
\\ \hline
\end{longtable}
}


\subsubsection{Test Cases Summary}
\begin{longtable}{p{3cm}p{2.5cm}p{2.5cm}p{3cm}p{4cm}}
\toprule
\href{https://jira.lsstcorp.org/secure/Tests.jspa\#/testCase/LVV-T1755}{LVV-T1755} & \multicolumn{4}{p{12cm}}{ Verify calculation of residual PSF ellipticity correlations for
separations less than 1 arcmin } \\ \hline
\textbf{Owner} & \textbf{Status} & \textbf{Version} & \textbf{Critical Event} & \textbf{Verification Type} \\ \hline
Jeffrey Carlin & Approved & 1 & false & Test \\ \hline
\end{longtable}
{\scriptsize
\textbf{Objective:}\\
Verify that the DM system has provided the code to calculate the median
residual PSF ellipticity correlations averaged over an arbitrary field
of view for separations less than 1 arcmin, and assess whether it meets
the requirement that it shall be no greater than \textbf{TE1 =
2.0e-5{[}arcminuteSeparationCorrelation{]}.}
}
  
 \newpage 
\subsection{[LVV-9783] DMS-REQ-0362-V-05: Outlier limit on 1 arcmin scales }\label{lvv-9783}

\begin{longtable}{cccc}
\hline
\textbf{Jira Link} & \textbf{Assignee} & \textbf{Status} & \textbf{Test Cases}\\ \hline
\href{https://jira.lsstcorp.org/browse/LVV-9783}{LVV-9783} &
Leanne Guy & Not Covered &
\begin{tabular}{c}
\end{tabular}
\\
\hline
\end{longtable}

\textbf{Verification Element Description:} \\
Residuals of PSF ellipticity correlations on 1 arcmin scales shall be no
greater than \textbf{TE3 = 4.0e-5{{[}arcminuteOutlierLimit{]}}.}

Associated element DMS-REQ-0362-V-01
(\href{https://jira.lsstcorp.org/browse/LVV-3404}{LVV-3404}) satisfies
the median residual PSF ellipticity correlations on 5 arcmin scales.

Associated element DMS-REQ-0362-V-02
(\href{https://jira.lsstcorp.org/browse/LVV-9780}{LVV-9780}) satisfies
the maximum fraction of ellipticity residuals exceeding the outlier
limits.~

Associated element
DMS-REQ-0362-V-03~(\href{https://jira.lsstcorp.org/browse/LVV-9781}{LVV-9781})~satisfies
the outlier limit on the PSF ellipticity correlation residuals on 5
arcmin scales.

Associated element DMS-REQ-0362-V-04
(\href{https://jira.lsstcorp.org/browse/LVV-9782}{LVV-9782}) satisfies
the median residual PSF ellipticity correlations on 1 arcmin scales.

{\footnotesize
\begin{longtable}{p{2.5cm}p{13.5cm}}
\hline
\multicolumn{2}{c}{\textbf{Requirement Details}}\\ \hline
Requirement ID & DMS-REQ-0362 \\ \cdashline{1-2}
Requirement Description &
\begin{minipage}[]{13cm}
\textbf{Specification:} The DMS shall include software to enable the
calculation of the elipticity correlations metrics defined in
OSS-REQ-0403, OSS-REQ-0404, and OSS-REQ-0405.
\end{minipage}
\\ \cdashline{1-2}
Requirement Parameters & {[}\textbf{TE3 = 4.0e-5{{[}unitless (angular correlation){]}}} Per-image
limit on the median residual ellipticity correlations at scales less
than 5 arcmin., \textbf{TE4 = 2.0e-7{{[}unitless (angular
correlation){]}}} Per-image limit on the median residual ellipticity
correlations at scales greater than or equal to 5 arcmin., \textbf{TE2 =
1.0e-7{{[}unitless (angular correlation){]}}} Maximum full-survey median
for residual ellipticity correlations at scales greater than or equal to
5 arcmin., \textbf{TEF = 15{{[}percent{]}}} Maximum fraction of visit
images that may exceed the TE3 or TE4 limits., \textbf{TE1 =
2.0e-5{{[}unitless (angular correlation){]}}} Maximum full-survey median
for residual ellipticity correlations at scales less than or equal to 1
arcmin.{]} \\ \cdashline{1-2}
Requirement Discussion &
\begin{minipage}[]{13cm}
\textbf{Discussion:} The relevant metrics are listed in the table below.
The values in the tables are the target values for LSST but are not
verified as part of this requirement.
\end{minipage}
\\ \cdashline{1-2}
Requirement Priority & 1b \\ \cdashline{1-2}
Upper Level Requirement &
\begin{tabular}{cl}
OSS-REQ-0403 & Ellipticity Correlation Function Distribution per Image \\
OSS-REQ-0404 & Ellipticity Correlation Function Distribution for Full Survey
(medians) \\
OSS-REQ-0405 & Ellipticity Correlation Function Distribution for Full Survey
(continuity) \\
\end{tabular}
\\ \hline
\end{longtable}
}


  
 \newpage 
\subsection{[LVV-9867] DMS-PRTL-REQ-0025-V-01: Positional Query: Solar System Object Names\_1 }\label{lvv-9867}

\begin{longtable}{cccc}
\hline
\textbf{Jira Link} & \textbf{Assignee} & \textbf{Status} & \textbf{Test Cases}\\ \hline
\href{https://jira.lsstcorp.org/browse/LVV-9867}{LVV-9867} &
Gregory Dubois-Felsmann & Not Covered &
\begin{tabular}{c}
LVV-T660 \\
\end{tabular}
\\
\hline
\end{longtable}

\textbf{Verification Element Description:} \\
Undefined

{\footnotesize
\begin{longtable}{p{2.5cm}p{13.5cm}}
\hline
\multicolumn{2}{c}{\textbf{Requirement Details}}\\ \hline
Requirement ID & DMS-PRTL-REQ-0025 \\ \cdashline{1-2}
Requirement Description &
\begin{minipage}[]{13cm}
The Portal aspect shall support positional queries based on external
Solar System Object identifiers, including names from, but not limited
to, NASA?s Navigation and Ancillary Information Facility (NAIF), the
Minor Planet Center, and JPL?s Horizons, coupled with a date/time range
specification.
\end{minipage}
\\ \cdashline{1-2}
Requirement Discussion &
\begin{minipage}[]{13cm}
The intent here is to enable a user to enter a solar system object name
(e.g., 25155 van Belle) and, for a given time range, get back a list of
observations that LSST may have made of that object because the survey
has overlapped the position of that object at the appropriate times.\\
This capability is expected to be available for tables that are both
spatially and temporally organized, e.g., Visit or DIASource.
\end{minipage}
\\ \cdashline{1-2}
Requirement Priority &  \\ \cdashline{1-2}
Upper Level Requirement &
\begin{tabular}{cl}
\end{tabular}
\\ \hline
\end{longtable}
}


\subsubsection{Test Cases Summary}
\begin{longtable}{p{3cm}p{2.5cm}p{2.5cm}p{3cm}p{4cm}}
\toprule
\href{https://jira.lsstcorp.org/secure/Tests.jspa\#/testCase/LVV-T660}{LVV-T660} & \multicolumn{4}{p{12cm}}{ Verify positional query based on Solar System object names } \\ \hline
\textbf{Owner} & \textbf{Status} & \textbf{Version} & \textbf{Critical Event} & \textbf{Verification Type} \\ \hline
Jeffrey Carlin & Draft & 1 & false & Test \\ \hline
\end{longtable}
{\scriptsize
\textbf{Objective:}\\
Verify that positional queries can be performed for coordinates based on
a given Solar System object name.
}
  
 \newpage 
\subsection{[LVV-9869] DMS-PRTL-REQ-0026-V-01: Positional Query by Region: Cone-Search\_1 }\label{lvv-9869}

\begin{longtable}{cccc}
\hline
\textbf{Jira Link} & \textbf{Assignee} & \textbf{Status} & \textbf{Test Cases}\\ \hline
\href{https://jira.lsstcorp.org/browse/LVV-9869}{LVV-9869} &
Gregory Dubois-Felsmann & Not Covered &
\begin{tabular}{c}
LVV-T5 \\
LVV-T661 \\
LVV-T1334 \\
\end{tabular}
\\
\hline
\end{longtable}

\textbf{Verification Element Description:} \\
Undefined

{\footnotesize
\begin{longtable}{p{2.5cm}p{13.5cm}}
\hline
\multicolumn{2}{c}{\textbf{Requirement Details}}\\ \hline
Requirement ID & DMS-PRTL-REQ-0026 \\ \cdashline{1-2}
Requirement Description &
\begin{minipage}[]{13cm}
The Portal aspect shall support position-based queries based on a
cone-shaped radial search.
\end{minipage}
\\ \cdashline{1-2}
Requirement Priority &  \\ \cdashline{1-2}
Upper Level Requirement &
\begin{tabular}{cl}
\end{tabular}
\\ \hline
\end{longtable}
}


\subsubsection{Test Cases Summary}
\begin{longtable}{p{3cm}p{2.5cm}p{2.5cm}p{3cm}p{4cm}}
\toprule
\href{https://jira.lsstcorp.org/secure/Tests.jspa\#/testCase/LVV-T5}{LVV-T5} & \multicolumn{4}{p{12cm}}{ LSP-00-15: Execution of basic catalog queries in the Portal } \\ \hline
\textbf{Owner} & \textbf{Status} & \textbf{Version} & \textbf{Critical Event} & \textbf{Verification Type} \\ \hline
Gregory Dubois-Felsmann & Deprecated & 1 & false & Test \\ \hline
\end{longtable}
{\scriptsize
\textbf{Objective:}\\
This test will test the functional requirements to be able to perform a
range of basic queries through the Portal Aspect of the LSP:

\begin{itemize}
\tightlist
\item
  Cone searches on the Object-like, ForcedSource-like, and Source-like
  WISE tables;~
\item
  Multi-target cone searches;
\item
  Form-based searches for exact equality, e.g., for row IDs; and
\item
  Form-based searches for sets of object attributes.
\end{itemize}

In addition, it tests the ability to download tabular query results from
the Portal Aspect.
}
\begin{longtable}{p{3cm}p{2.5cm}p{2.5cm}p{3cm}p{4cm}}
\toprule
\href{https://jira.lsstcorp.org/secure/Tests.jspa\#/testCase/LVV-T661}{LVV-T661} & \multicolumn{4}{p{12cm}}{ Verify query by cone search } \\ \hline
\textbf{Owner} & \textbf{Status} & \textbf{Version} & \textbf{Critical Event} & \textbf{Verification Type} \\ \hline
Jeffrey Carlin & Draft & 1 & false & Test \\ \hline
\end{longtable}
{\scriptsize
\textbf{Objective:}\\
Verify that Portal supports position-based queries based on a
cone-shaped radial search.
}
\begin{longtable}{p{3cm}p{2.5cm}p{2.5cm}p{3cm}p{4cm}}
\toprule
\href{https://jira.lsstcorp.org/secure/Tests.jspa\#/testCase/LVV-T1334}{LVV-T1334} & \multicolumn{4}{p{12cm}}{ LDM-503-10a: Portal Aspect tests for LSP with Authentication and TAP
milestone } \\ \hline
\textbf{Owner} & \textbf{Status} & \textbf{Version} & \textbf{Critical Event} & \textbf{Verification Type} \\ \hline
Gregory Dubois-Felsmann & Approved & 1 & false & Test \\ \hline
\end{longtable}
{\scriptsize
\textbf{Objective:}\\
This test case verifies that the Portal Aspect of the Science Platform
is accessible to authorized users through a login process, and that TAP
searches can be performed from the Portal Aspect UI.\\[2\baselineskip]In
so doing and in conjunction with the other LDM-503-10a test cases
collected under LVV-P48, it addresses all or part of the following
requirements:

\begin{itemize}
\tightlist
\item
  DMS-LSP-REQ-0002, DMS-LSP-REQ-0005, DMS-LSP-REQ-0006,
  DMS-LSP-REQ-0020, DMS-LSP-REQ-0022, DMS-LSP-REQ-0023, DMS-LSP-REQ-0024
\item
  DMS-PRTL-REQ-0001, DMS-PRTL-REQ-0015, DMS-PRTL-REQ-0016,
  DMS-PRTL-REQ-0017, DMS-PRTL-REQ-0020, DMS-PRTL-REQ-0026,
  DMS-PRTL-REQ-0049, and DMS-PRTL-REQ-0095, primarily
\end{itemize}

Note this test was not designed to perform a full verification of the
above requirements, but rather to demonstrate having reached a certain
level of partial capability during construction.
}
  
 \newpage 
\subsection{[LVV-9926] DMS-PRTL-REQ-0085-V-01: Distance Measurement Tool\_1 }\label{lvv-9926}

\begin{longtable}{cccc}
\hline
\textbf{Jira Link} & \textbf{Assignee} & \textbf{Status} & \textbf{Test Cases}\\ \hline
\href{https://jira.lsstcorp.org/browse/LVV-9926}{LVV-9926} &
Gregory Dubois-Felsmann & Not Covered &
\begin{tabular}{c}
LVV-T719 \\
\end{tabular}
\\
\hline
\end{longtable}

\textbf{Verification Element Description:} \\
Undefined

{\footnotesize
\begin{longtable}{p{2.5cm}p{13.5cm}}
\hline
\multicolumn{2}{c}{\textbf{Requirement Details}}\\ \hline
Requirement ID & DMS-PRTL-REQ-0085 \\ \cdashline{1-2}
Requirement Description &
\begin{minipage}[]{13cm}
The Portal aspect shall have the capability to determine the distance
between two positions within an image or 2-dimensional plot in both
image/plot coordinates (electronic or spatial X and Y) and in
astrophysical coordinates (if applicable). Calculations shall be
performed in spherical geometry where appropriate.
\end{minipage}
\\ \cdashline{1-2}
Requirement Discussion &
\begin{minipage}[]{13cm}
The point behind this requirement is to enable distance determinations
of equatorial, galactic, and ecliptic coordinates and make sure
spherical geometry is used.\\
When applied to general two-dimensional plots, distance measurement
should only be supported when a metric and a geometry, or at least the
relative scale between the two displayed coordinates, is known.\\
This means that this capability is particularly dependent on metadata
support.
\end{minipage}
\\ \cdashline{1-2}
Requirement Priority &  \\ \cdashline{1-2}
Upper Level Requirement &
\begin{tabular}{cl}
\end{tabular}
\\ \hline
\end{longtable}
}


\subsubsection{Test Cases Summary}
\begin{longtable}{p{3cm}p{2.5cm}p{2.5cm}p{3cm}p{4cm}}
\toprule
\href{https://jira.lsstcorp.org/secure/Tests.jspa\#/testCase/LVV-T719}{LVV-T719} & \multicolumn{4}{p{12cm}}{ Verify distance measurement tool } \\ \hline
\textbf{Owner} & \textbf{Status} & \textbf{Version} & \textbf{Critical Event} & \textbf{Verification Type} \\ \hline
Jeffrey Carlin & Draft & 1 & false & Inspection \\ \hline
\end{longtable}
{\scriptsize
\textbf{Objective:}\\
Verify that the Portal provides a tool to measure the distance between
two points in an image or a 2-dimensional plot. Distances should be
calculated in both image/plot coordinates (electronic or spatial X and
Y) and in astrophysical coordinates (if applicable). Calculations shall
be performed in spherical geometry where appropriate.
}
  
 \newpage 
\subsection{[LVV-9937] DMS-PRTL-REQ-0097-V-01: Statistical Measurements on Image Data\_1 }\label{lvv-9937}

\begin{longtable}{cccc}
\hline
\textbf{Jira Link} & \textbf{Assignee} & \textbf{Status} & \textbf{Test Cases}\\ \hline
\href{https://jira.lsstcorp.org/browse/LVV-9937}{LVV-9937} &
Gregory Dubois-Felsmann & Not Covered &
\begin{tabular}{c}
LVV-T731 \\
\end{tabular}
\\
\hline
\end{longtable}

\textbf{Verification Element Description:} \\
Undefined

{\footnotesize
\begin{longtable}{p{2.5cm}p{13.5cm}}
\hline
\multicolumn{2}{c}{\textbf{Requirement Details}}\\ \hline
Requirement ID & DMS-PRTL-REQ-0097 \\ \cdashline{1-2}
Requirement Description &
\begin{minipage}[]{13cm}
The Portal aspect shall enable the capability to perform a set of
statistical measurements (e.g., mean, median, RMS, skew, kurtosis) on
user-selected regions in images.
\end{minipage}
\\ \cdashline{1-2}
Requirement Priority &  \\ \cdashline{1-2}
Upper Level Requirement &
\begin{tabular}{cl}
\end{tabular}
\\ \hline
\end{longtable}
}


\subsubsection{Test Cases Summary}
\begin{longtable}{p{3cm}p{2.5cm}p{2.5cm}p{3cm}p{4cm}}
\toprule
\href{https://jira.lsstcorp.org/secure/Tests.jspa\#/testCase/LVV-T731}{LVV-T731} & \multicolumn{4}{p{12cm}}{ Verify statistical measurements on user-selected regions of images } \\ \hline
\textbf{Owner} & \textbf{Status} & \textbf{Version} & \textbf{Critical Event} & \textbf{Verification Type} \\ \hline
Jeffrey Carlin & Draft & 1 & false & Inspection \\ \hline
\end{longtable}
{\scriptsize
\textbf{Objective:}\\
Verify that the Portal aspect enables the capability to perform a set of
statistical measurements (e.g., mean, median, RMS, skew, kurtosis) on
user-selected regions in images.
}
  
 \newpage 
\subsection{[LVV-18225] DMS-REQ-0382-V-01: HiPS Visualization\_1 }\label{lvv-18225}

\begin{longtable}{cccc}
\hline
\textbf{Jira Link} & \textbf{Assignee} & \textbf{Status} & \textbf{Test Cases}\\ \hline
\href{https://jira.lsstcorp.org/browse/LVV-18225}{LVV-18225} &
Leanne Guy & Not Covered &
\begin{tabular}{c}
LVV-T1527 \\
\end{tabular}
\\
\hline
\end{longtable}

\textbf{Verification Element Description:} \\
Undefined

{\footnotesize
\begin{longtable}{p{2.5cm}p{13.5cm}}
\hline
\multicolumn{2}{c}{\textbf{Requirement Details}}\\ \hline
Requirement ID & DMS-REQ-0382 \\ \cdashline{1-2}
Requirement Description &
\begin{minipage}[]{13cm}
\textbf{Specification:} The LSST Science Platform shall support the
visualization of the LSST-generated HiPS image maps as well as other
HiPS maps which satisfy the IVOA HiPS Recommendation, and shall provide
integrated behavior, such as the overplotting of catalog entries,
comparable to that provided for individual source images (e.g., PVIs and
coadd tiles).
\end{minipage}
\\ \cdashline{1-2}
Requirement Discussion &
\begin{minipage}[]{13cm}
\textbf{Discussion:} Further details will be provided in the LSST
Science Platform Requirements, \citeds{LDM-554}.
\end{minipage}
\\ \cdashline{1-2}
Requirement Priority & 1b \\ \cdashline{1-2}
Upper Level Requirement &
\begin{tabular}{cl}
OSS-REQ-0061 & Data Visualization \\
\end{tabular}
\\ \hline
\end{longtable}
}


\subsubsection{Test Cases Summary}
\begin{longtable}{p{3cm}p{2.5cm}p{2.5cm}p{3cm}p{4cm}}
\toprule
\href{https://jira.lsstcorp.org/secure/Tests.jspa\#/testCase/LVV-T1527}{LVV-T1527} & \multicolumn{4}{p{12cm}}{ Verify Support for HiPS Visualization } \\ \hline
\textbf{Owner} & \textbf{Status} & \textbf{Version} & \textbf{Critical Event} & \textbf{Verification Type} \\ \hline
Jeffrey Carlin & Draft & 1 & false & Demonstration \\ \hline
\end{longtable}
{\scriptsize
\textbf{Objective:}\\
Verify that the LSST Science Platform supports the visualization of
LSST-generated HiPS image maps as well as other HiPS maps which satisfy
the IVOA HiPS Recommendation. Also verify that integrated behavior is
available, such as the overplotting of catalog entries, comparable to
that provided for individual source images (e.g., PVIs and coadd tiles).
}
  
 \newpage 
\subsection{[LVV-18227] DMS-REQ-0379-V-01: Produce All-Sky HiPS Map\_1 }\label{lvv-18227}

\begin{longtable}{cccc}
\hline
\textbf{Jira Link} & \textbf{Assignee} & \textbf{Status} & \textbf{Test Cases}\\ \hline
\href{https://jira.lsstcorp.org/browse/LVV-18227}{LVV-18227} &
Leanne Guy & Not Covered &
\begin{tabular}{c}
LVV-T1529 \\
\end{tabular}
\\
\hline
\end{longtable}

\textbf{Verification Element Description:} \\
Undefined

{\footnotesize
\begin{longtable}{p{2.5cm}p{13.5cm}}
\hline
\multicolumn{2}{c}{\textbf{Requirement Details}}\\ \hline
Requirement ID & DMS-REQ-0379 \\ \cdashline{1-2}
Requirement Description &
\begin{minipage}[]{13cm}
\textbf{Specification:} Data Release Production shall include the
production of an all-sky image map for the existing coadded image area
in each filter band, and at least one pre-defined all-sky color image
map, following the IVOA HiPS Recommendation.
\end{minipage}
\\ \cdashline{1-2}
Requirement Discussion &
\begin{minipage}[]{13cm}
\textbf{Discussion:} The maximum resolution of the image maps is TBD;
however, it would be desirable for it to be at least close to the
underlying coadded image resolution, in order not to give a poor
impression of the data quality. ~It is possible that the
highest-resolution HiPS tiles could be provided on-demand from the LSST
cutout service. ~It is expected that the HiPS tiles will be generated by
resampling the existing coadds, not by performing an independent
coaddition. ~This requires work from Science Pipelines on resolving the
ambiguities in overlap regions. ~Whether the lower-resolution levels of
the HiPS tiles will be generated by existing community tools (i.e.,
hipsgen) or by LSST code is also TBD. ~The color map being
``pre-defined'' means that the choice of bands will be made by the LSST
Project as part of the configuration of a Data Release. ~This does not
preclude the Science Platform additionally providing means for
interactive generation of other colorizations from the single-band HiPS
maps.\\
By the terms of the HiPS Recommendation, a HiPS image map should include
a corresponding MOC. ~This may or may not be the same as the MOCs for
the survey envisioned under DMS-REQ-0383 elsewhere in this document,
depending on choices made for data selection.\\
The Project should produce a technical note, during the construction
era, detailing which of the optional components of the HiPS standard
will be supported.\\
This requirement specifically calls for making HiPS maps from the
standard coadds and therefore whatever policies are used for the
inclusion of Special Programs data in the standard coadds will also
automatically apply here. ~If there are both main-survey-depth and
full-depth coadds for the deep drilling fields, then, it is a separate
question as to whether HiPS maps will be generated for those fields.
\end{minipage}
\\ \cdashline{1-2}
Requirement Priority & 1b \\ \cdashline{1-2}
Upper Level Requirement &
\begin{tabular}{cl}
OSS-REQ-0391 & Data Product Conventions \\
OSS-REQ-0136 & Co-added Exposures \\
\end{tabular}
\\ \hline
\end{longtable}
}


\subsubsection{Test Cases Summary}
\begin{longtable}{p{3cm}p{2.5cm}p{2.5cm}p{3cm}p{4cm}}
\toprule
\href{https://jira.lsstcorp.org/secure/Tests.jspa\#/testCase/LVV-T1529}{LVV-T1529} & \multicolumn{4}{p{12cm}}{ Verify Production of All-Sky HiPS Map } \\ \hline
\textbf{Owner} & \textbf{Status} & \textbf{Version} & \textbf{Critical Event} & \textbf{Verification Type} \\ \hline
Jeffrey Carlin & Draft & 1 & false & Demonstration \\ \hline
\end{longtable}
{\scriptsize
\textbf{Objective:}\\
Verify that Data Release Production includes the production of an
all-sky image map for the existing coadded image area in each filter
band, and at least one pre-defined all-sky color image map, following
the IVOA HiPS Recommendation.
}
  
 \newpage 
\subsection{[LVV-18228] DMS-REQ-0383-V-01: Produce MOC Maps\_1 }\label{lvv-18228}

\begin{longtable}{cccc}
\hline
\textbf{Jira Link} & \textbf{Assignee} & \textbf{Status} & \textbf{Test Cases}\\ \hline
\href{https://jira.lsstcorp.org/browse/LVV-18228}{LVV-18228} &
Leanne Guy & Not Covered &
\begin{tabular}{c}
LVV-T1530 \\
\end{tabular}
\\
\hline
\end{longtable}

\textbf{Verification Element Description:} \\
Undefined

{\footnotesize
\begin{longtable}{p{2.5cm}p{13.5cm}}
\hline
\multicolumn{2}{c}{\textbf{Requirement Details}}\\ \hline
Requirement ID & DMS-REQ-0383 \\ \cdashline{1-2}
Requirement Description &
\begin{minipage}[]{13cm}
\textbf{Specification:} Data Release Production shall include the
production of Multi-Order Coverage maps for the survey data, conformant
with the IVOA MOC recommendation. ~A separate MOC shall be produced for
each filter band for the main survey. ~Additional MOCs SHOULD be
produced to represent special-programs datasets and other collections of
on-sky data.
\end{minipage}
\\ \cdashline{1-2}
Requirement Discussion &
\begin{minipage}[]{13cm}
\textbf{Discussion:} It is likely to be useful to produce quite a large
number of MOCs as part of releasing the data and documenting its
quality. ~For example, it may be useful to produce both a MOC for all
the data from a band and for only that part of the sky for which the SRD
requirements in that band have been met. ~It also seems useful to
produce MOCs for the deep drilling fields, etc. ~It may also be useful
to produce MOCs on, for instance, a nightly basis, reflecting that part
of the sky for which coverage was obtained in that night. ~The LSST
project should engage in the work begun in 2018 on the development of
standards and tools for spatiotemporal MOCs.\\
The tile resolution chosen for these MOCs should be fine enough to
represent the dither pattern of the survey as well as the shape of the
focal plane. ~Some testing should be done to determine a suitable
scale.\\
The Project should produce a technical note, during the construction
era, detailing the specific plans for creation of MOCs.
\end{minipage}
\\ \cdashline{1-2}
Requirement Priority & 1b \\ \cdashline{1-2}
Upper Level Requirement &
\begin{tabular}{cl}
OSS-REQ-0391 & Data Product Conventions \\
OSS-REQ-0033 & Survey Planning and Performance Monitoring \\
\end{tabular}
\\ \hline
\end{longtable}
}


\subsubsection{Test Cases Summary}
\begin{longtable}{p{3cm}p{2.5cm}p{2.5cm}p{3cm}p{4cm}}
\toprule
\href{https://jira.lsstcorp.org/secure/Tests.jspa\#/testCase/LVV-T1530}{LVV-T1530} & \multicolumn{4}{p{12cm}}{ Verify Production of Multi-Order Coverage Maps for Survey Data } \\ \hline
\textbf{Owner} & \textbf{Status} & \textbf{Version} & \textbf{Critical Event} & \textbf{Verification Type} \\ \hline
Jeffrey Carlin & Draft & 1 & false & Demonstration \\ \hline
\end{longtable}
{\scriptsize
\textbf{Objective:}\\
Verify that Data Release Production includes the production of
Multi-Order Coverage maps for the survey data, conformant with the IVOA
MOC recommendation. ~Confirm that separate MOC are produced for each
filter band for the main survey, and additional MOCs are produced to
represent special-programs datasets and other collections of on-sky
data.
}
  
 \newpage 
\subsection{[LVV-18229] DMS-REQ-0344-V-01: Time to L1 public release }\label{lvv-18229}

\begin{longtable}{cccc}
\hline
\textbf{Jira Link} & \textbf{Assignee} & \textbf{Status} & \textbf{Test Cases}\\ \hline
\href{https://jira.lsstcorp.org/browse/LVV-18229}{LVV-18229} &
Melissa Graham & Not Covered &
\begin{tabular}{c}
LVV-T1865 \\
\end{tabular}
\\
\hline
\end{longtable}

\textbf{Verification Element Description:} \\
This is 3 distinct requirements. OTT1 can be tested with simulated data.
L1 Data Products can be created with precursor data but requires that we
include some ``worst case'' datasets (in terms of density and night
length). SSObject orbit determination can be done to a certain extent
with simulated data. Will need to be verified again during
commissioning.

Associated element
(\href{https://jira.lsstcorp.org/browse/LVV-9740}{LVV-9740}) satisfies
the latency of reporting transients.\\
Associated element
(\href{https://jira.lsstcorp.org/browse/LVV-9803}{LVV-9803}) satisfies
the availability of Solar System Object orbits.

Associated element
(\href{https://jira.lsstcorp.org/browse/LVV-9744}{LVV-9744}) satisfies
the latency of reporting optical transients.

{\footnotesize
\begin{longtable}{p{2.5cm}p{13.5cm}}
\hline
\multicolumn{2}{c}{\textbf{Requirement Details}}\\ \hline
Requirement ID & DMS-REQ-0344 \\ \cdashline{1-2}
Requirement Description &
\begin{minipage}[]{13cm}
\textbf{Specification:} The publishing of Level 1 data products from
Special Programs shall be subject to the same performance requirements
of the standard Level 1 system. In particular \textbf{L1PublicT} and
\textbf{OTT1}.
\end{minipage}
\\ \cdashline{1-2}
Requirement Parameters & {[}\textbf{OTT1 = 1{{[}minute{]}}} The latency of reporting optical
transients following the completion of readout of the last image of a
visit, \textbf{L1PublicT = 24{{[}hour{]}}} Maximum time from the
acquisition of science data to the release of associated Level 1 Data
Products (except alerts){]} \\ \cdashline{1-2}
Requirement Priority & 2 \\ \cdashline{1-2}
Upper Level Requirement &
\begin{tabular}{cl}
OSS-REQ-0392 & Data Products Handling for Special Programs \\
\end{tabular}
\\ \hline
\end{longtable}
}


\subsubsection{Test Cases Summary}
\begin{longtable}{p{3cm}p{2.5cm}p{2.5cm}p{3cm}p{4cm}}
\toprule
\href{https://jira.lsstcorp.org/secure/Tests.jspa\#/testCase/LVV-T1865}{LVV-T1865} & \multicolumn{4}{p{12cm}}{ Verify implementation of time to L1 public release for Special Programs } \\ \hline
\textbf{Owner} & \textbf{Status} & \textbf{Version} & \textbf{Critical Event} & \textbf{Verification Type} \\ \hline
Jeffrey Carlin & Draft & 1 & false & Test \\ \hline
\end{longtable}
{\scriptsize
\textbf{Objective:}\\
~Verify that data from Special Programs are made available via public
release within \textbf{L1PublicT = 24{[}hour{]}} from the acquisition of
science data.
}
  
 \newpage 
\subsection{[LVV-18233] DMS-REQ-0390-V-01: Re-Runs on Other Systems\_1 }\label{lvv-18233}

\begin{longtable}{cccc}
\hline
\textbf{Jira Link} & \textbf{Assignee} & \textbf{Status} & \textbf{Test Cases}\\ \hline
\href{https://jira.lsstcorp.org/browse/LVV-18233}{LVV-18233} &
Leanne Guy & Not Covered &
\begin{tabular}{c}
LVV-T1563 \\
\end{tabular}
\\
\hline
\end{longtable}

\textbf{Verification Element Description:} \\
Undefined

{\footnotesize
\begin{longtable}{p{2.5cm}p{13.5cm}}
\hline
\multicolumn{2}{c}{\textbf{Requirement Details}}\\ \hline
Requirement ID & DMS-REQ-0390 \\ \cdashline{1-2}
Requirement Description &
\begin{minipage}[]{13cm}
\textbf{Specification:} A re-run based on provenance, if run on a
different system (but whose configuration still meets established LSST
requirements), shall produce results which are the same to the extent
computationally feasible (with the exception of provenance data or other
execution records that depend on the wall-clock time or on variable
system loads).
\end{minipage}
\\ \cdashline{1-2}
Requirement Discussion &
\begin{minipage}[]{13cm}
\textbf{Discussion:} ``To the extent computationally feasible'' refers
primarily to the possibility that different implementations of the IEEE
floating-point standards may produce different results in the least
significant figures, and that under some circumstances these variations
can be amplified by algorithms and by choices made by optimizing
compilers. ~It is expected that normal ``best practices'' for writing
floating point code will be followed to minimize the effects of these
hardware differences, but they cannot be avoided altogether.
\end{minipage}
\\ \cdashline{1-2}
Requirement Priority & 1b \\ \cdashline{1-2}
Upper Level Requirement &
\begin{tabular}{cl}
OSS-REQ-0122 & Provenance \\
OSS-REQ-0169 & Data Products \\
OSS-REQ-0123 & Reproducibility \\
OSS-REQ-0172 & Provenance Archiving \\
\end{tabular}
\\ \hline
\end{longtable}
}


\subsubsection{Test Cases Summary}
\begin{longtable}{p{3cm}p{2.5cm}p{2.5cm}p{3cm}p{4cm}}
\toprule
\href{https://jira.lsstcorp.org/secure/Tests.jspa\#/testCase/LVV-T1563}{LVV-T1563} & \multicolumn{4}{p{12cm}}{ Verify re-run on different system produces the same results } \\ \hline
\textbf{Owner} & \textbf{Status} & \textbf{Version} & \textbf{Critical Event} & \textbf{Verification Type} \\ \hline
Jeffrey Carlin & Draft & 1 & false & Demonstration \\ \hline
\end{longtable}
{\scriptsize
\textbf{Objective:}\\
Verify that tools are provided to use the archived provenance data to
re-run a data processing operation on different systems, and that the
results produced are the same to the extent computationally feasible.~
}
  
 \newpage 
\subsection{[LVV-18234] DMS-REQ-0389-V-01: Re-Runs on Similar Systems\_1 }\label{lvv-18234}

\begin{longtable}{cccc}
\hline
\textbf{Jira Link} & \textbf{Assignee} & \textbf{Status} & \textbf{Test Cases}\\ \hline
\href{https://jira.lsstcorp.org/browse/LVV-18234}{LVV-18234} &
Leanne Guy & Not Covered &
\begin{tabular}{c}
LVV-T1564 \\
\end{tabular}
\\
\hline
\end{longtable}

\textbf{Verification Element Description:} \\
Undefined

{\footnotesize
\begin{longtable}{p{2.5cm}p{13.5cm}}
\hline
\multicolumn{2}{c}{\textbf{Requirement Details}}\\ \hline
Requirement ID & DMS-REQ-0389 \\ \cdashline{1-2}
Requirement Description &
\begin{minipage}[]{13cm}
\textbf{Specification:} A re-run based on provenance, if run on the same
system or a system with identically configured hardware and system
software, shall produce the same results (with the exception of
provenance data or other execution records that depend on the wall-clock
time or on variable system loads).
\end{minipage}
\\ \cdashline{1-2}
Requirement Discussion &
\begin{minipage}[]{13cm}
\textbf{Discussion:} ``System software'' refers to the substrate of
operating systems, device drivers, language standard libraries, and the
like, not to the higher-level software written by LSST or imported into
the LSST code base; the latter are constrained by DMS-REQ-0388 to be the
same for a provenance-based re-run.
\end{minipage}
\\ \cdashline{1-2}
Requirement Priority & 1b \\ \cdashline{1-2}
Upper Level Requirement &
\begin{tabular}{cl}
OSS-REQ-0122 & Provenance \\
OSS-REQ-0169 & Data Products \\
OSS-REQ-0123 & Reproducibility \\
OSS-REQ-0172 & Provenance Archiving \\
\end{tabular}
\\ \hline
\end{longtable}
}


\subsubsection{Test Cases Summary}
\begin{longtable}{p{3cm}p{2.5cm}p{2.5cm}p{3cm}p{4cm}}
\toprule
\href{https://jira.lsstcorp.org/secure/Tests.jspa\#/testCase/LVV-T1564}{LVV-T1564} & \multicolumn{4}{p{12cm}}{ Verify re-run on similar system produces the same results } \\ \hline
\textbf{Owner} & \textbf{Status} & \textbf{Version} & \textbf{Critical Event} & \textbf{Verification Type} \\ \hline
Jeffrey Carlin & Draft & 1 & false & Demonstration \\ \hline
\end{longtable}
{\scriptsize
\textbf{Objective:}\\
Verify that a provenance-based re-run that is run on the same system, or
a system with identically configured hardware and system software,
produces the same results.~
}
  
 \newpage 
\subsection{[LVV-18295] DMS-REQ-0394-V-01: Data Management Nightly Reporting\_1 }\label{lvv-18295}

\begin{longtable}{cccc}
\hline
\textbf{Jira Link} & \textbf{Assignee} & \textbf{Status} & \textbf{Test Cases}\\ \hline
\href{https://jira.lsstcorp.org/browse/LVV-18295}{LVV-18295} &
Leanne Guy & Not Covered &
\begin{tabular}{c}
LVV-T1831 \\
\end{tabular}
\\
\hline
\end{longtable}

\textbf{Verification Element Description:} \\
Undefined

{\footnotesize
\begin{longtable}{p{2.5cm}p{13.5cm}}
\hline
\multicolumn{2}{c}{\textbf{Requirement Details}}\\ \hline
Requirement ID & DMS-REQ-0394 \\ \cdashline{1-2}
Requirement Description &
\begin{minipage}[]{13cm}
\textbf{Specification:} The LSST Data Management subsystem shall produce
a searchable - interactive nightly report(s), from information published
in the EFD by each subsystem, summarizing performance and behavior over
a user defined period of time (e.g. the previous 24 hours).
\end{minipage}
\\ \cdashline{1-2}
Requirement Priority & 1a \\ \cdashline{1-2}
Upper Level Requirement &
\begin{tabular}{cl}
OSS-REQ-0406 & Subsystem Nightly Reporting \\
\end{tabular}
\\ \hline
\end{longtable}
}


\subsubsection{Test Cases Summary}
\begin{longtable}{p{3cm}p{2.5cm}p{2.5cm}p{3cm}p{4cm}}
\toprule
\href{https://jira.lsstcorp.org/secure/Tests.jspa\#/testCase/LVV-T1831}{LVV-T1831} & \multicolumn{4}{p{12cm}}{ Verify Implementation of Data Management Nightly Reporting } \\ \hline
\textbf{Owner} & \textbf{Status} & \textbf{Version} & \textbf{Critical Event} & \textbf{Verification Type} \\ \hline
Jeffrey Carlin & Draft & 1 & false & Demonstration \\ \hline
\end{longtable}
{\scriptsize
\textbf{Objective:}\\
Verify that the LSST Data Management subsystem produces a searchable -
interactive nightly report(s), from information published in the EFD by
each subsystem, summarizing performance and behavior over a user defined
period of time (e.g. the previous 24 hours).
}
  
 \newpage 
\subsection{[LVV-18298] DMS-REQ-0392-V-01: Fraction of Alerts Transmitted }\label{lvv-18298}

\begin{longtable}{cccc}
\hline
\textbf{Jira Link} & \textbf{Assignee} & \textbf{Status} & \textbf{Test Cases}\\ \hline
\href{https://jira.lsstcorp.org/browse/LVV-18298}{LVV-18298} &
Leanne Guy & Not Covered &
\begin{tabular}{c}
LVV-T2091 \\
\end{tabular}
\\
\hline
\end{longtable}

\textbf{Verification Element Description:} \\
This verification element corresponds to the fraction of alerts
distributed, OTR1 = 98{{[}percent{]}}.

Related requirements are addressed in the sibling verification elements.

{\footnotesize
\begin{longtable}{p{2.5cm}p{13.5cm}}
\hline
\multicolumn{2}{c}{\textbf{Requirement Details}}\\ \hline
Requirement ID & DMS-REQ-0392 \\ \cdashline{1-2}
Requirement Description &
\begin{minipage}[]{13cm}
\textbf{Specification:} The system shall reliably produce alerts for
standard science visits read out in the camera {{[}and specified to be
analyzed by Data Management{]}} such that no more than
\textbf{sciVisitAlertDelay} per cent of visits will fail to have at
least \textbf{OTR1} per cent of its alerts distributed via the LSST
alert distribution system within \textbf{OTT1}, and no more than
\textbf{sciVisitAlertFailure} per cent of visits will fail to generate
and distribute alerts (integrated over all stages of data handling)
\end{minipage}
\\ \cdashline{1-2}
Requirement Parameters & {[}\textbf{OTR1 = 98{{[}percent{]}}} Fraction of detectable alerts for
which an alert is actually transmitted within latency OTT1 (see
LSR-REQ-0101)., \textbf{sciVisitAlertFailure = 0.1{{[}percent{]}}}
Maximum fraction of visits for which alerts are not generated or
distributed., \textbf{OTT1 = 1{{[}minute{]}}} The latency of reporting
optical transients following the completion of readout of the last image
of a visit, \textbf{nAlertVisitPeak = 40000{{[}integer{]}}} The
instantaneous peak number of alerts per standard visit.,
\textbf{sciVisitAlertDelay = 1{{[}percent{]}}} Maximum fraction of
science visits with less than OTR1 percent of the alerts distributed
within OTT1.{]} \\ \cdashline{1-2}
Requirement Discussion &
\begin{minipage}[]{13cm}
\textbf{Discussion:} As with DMS-REQ-0004, this specification applies to
visits which should have resulted in fewer than
\textbf{nAlertVisitPeak}.
\end{minipage}
\\ \cdashline{1-2}
Requirement Priority & 2 \\ \cdashline{1-2}
Upper Level Requirement &
\begin{tabular}{cl}
OSS-REQ-0112 & Science Visit Alert Generation Reliability \\
\end{tabular}
\\ \hline
\end{longtable}
}


\subsubsection{Test Cases Summary}
\begin{longtable}{p{3cm}p{2.5cm}p{2.5cm}p{3cm}p{4cm}}
\toprule
\href{https://jira.lsstcorp.org/secure/Tests.jspa\#/testCase/LVV-T2091}{LVV-T2091} & \multicolumn{4}{p{12cm}}{ Verify Fraction of Alerts Transmitted Within Latency Threshold } \\ \hline
\textbf{Owner} & \textbf{Status} & \textbf{Version} & \textbf{Critical Event} & \textbf{Verification Type} \\ \hline
Eric Bellm & Draft & 1 & false & Test \\ \hline
\end{longtable}
{\scriptsize
\textbf{Objective:}\\
Verify that at least \textbf{OTR1 = 98{[}percent{]}} of detectable
alerts are actually transmitted within latency \textbf{OTT1 =
1{[}minute{]}}.~
}
  
 \newpage 
\subsection{[LVV-18299] DMS-REQ-0393-V-01: Average Number of Alerts Per Visit }\label{lvv-18299}

\begin{longtable}{cccc}
\hline
\textbf{Jira Link} & \textbf{Assignee} & \textbf{Status} & \textbf{Test Cases}\\ \hline
\href{https://jira.lsstcorp.org/browse/LVV-18299}{LVV-18299} &
Leanne Guy & Not Covered &
\begin{tabular}{c}
LVV-T2097 \\
\end{tabular}
\\
\hline
\end{longtable}

\textbf{Verification Element Description:} \\
This verification element corresponds to the average number of alerts
per standard visit, nAlertVisitAvg = 10000{{[}integer{]}}.

The related sibling verification element satisfies the additional
constraint on the peak number of alerts per visit.

{\footnotesize
\begin{longtable}{p{2.5cm}p{13.5cm}}
\hline
\multicolumn{2}{c}{\textbf{Requirement Details}}\\ \hline
Requirement ID & DMS-REQ-0393 \\ \cdashline{1-2}
Requirement Description &
\begin{minipage}[]{13cm}
\textbf{Specification:} The system shall be able to identify and
distribute an average of at least \textbf{nAlertVisitAvg} alerts per
standard visit during a given night, and at least
\textbf{nAlertVisitPeak} for a single standard visit.\\
\hspace*{0.333em}\\
\textbf{Specification:} Performance shall degrade gracefully beyond
\textbf{nAlertVisitAvg}.
\end{minipage}
\\ \cdashline{1-2}
Requirement Parameters & {[}\textbf{nAlertVisitAvg = 10000{{[}integer{]}}} The nightly minimum
average number of alerts per standard visit., \textbf{nAlertVisitPeak =
40000{{[}integer{]}}} The instantaneous peak number of alerts per
standard visit.{]} \\ \cdashline{1-2}
Requirement Discussion &
\begin{minipage}[]{13cm}
\textbf{Discussion:} The term `degrade gracefully' means that visits
with an excess of difference-image sources should not cause any DMS
downtime; i.e., the system does not crash and is able to distribute
alerts from that visit, potentially with greater latency.
\end{minipage}
\\ \cdashline{1-2}
Requirement Priority & 2 \\ \cdashline{1-2}
Upper Level Requirement &
\begin{tabular}{cl}
LSR-REQ-0101 & Data Processing for Single Visits and Transients \\
OSS-REQ-0193 & Alerts per Visit \\
\end{tabular}
\\ \hline
\end{longtable}
}


\subsubsection{Test Cases Summary}
\begin{longtable}{p{3cm}p{2.5cm}p{2.5cm}p{3cm}p{4cm}}
\toprule
\href{https://jira.lsstcorp.org/secure/Tests.jspa\#/testCase/LVV-T2097}{LVV-T2097} & \multicolumn{4}{p{12cm}}{ Verify Handling of Average Number of Alerts } \\ \hline
\textbf{Owner} & \textbf{Status} & \textbf{Version} & \textbf{Critical Event} & \textbf{Verification Type} \\ \hline
Eric Bellm & Draft & 1 & false & Test \\ \hline
\end{longtable}
{\scriptsize
\textbf{Objective:}\\
Verify that the system can identify and distribute an average
of~\textbf{nAlertVisitAvg~= 10000{[}integer{]}~}alerts per standard
visit over a given night.
}
  
 \newpage 
\subsection{[LVV-18339] DMS-REQ-0359-V-18: Outlier limit on zero points }\label{lvv-18339}

\begin{longtable}{cccc}
\hline
\textbf{Jira Link} & \textbf{Assignee} & \textbf{Status} & \textbf{Test Cases}\\ \hline
\href{https://jira.lsstcorp.org/browse/LVV-18339}{LVV-18339} &
Leanne Guy & Not Covered &
\begin{tabular}{c}
\end{tabular}
\\
\hline
\end{longtable}

\textbf{Verification Element Description:} \\
The distribution width (rms) of the internal photometric zero-point
error (the system stability across the sky) will not exceed PA3
millimag, and no more than PF2 \% of the distribution will exceed
\textbf{PA4 = 15 millimag}. (This VE is for PA4.)

{\footnotesize
\begin{longtable}{p{2.5cm}p{13.5cm}}
\hline
\multicolumn{2}{c}{\textbf{Requirement Details}}\\ \hline
Requirement ID & DMS-REQ-0359 \\ \cdashline{1-2}
Requirement Description &
\begin{minipage}[]{13cm}
\textbf{Specification:} The DMS shall include software to enable the
calculation of the photometric performance metrics defined in
OSS-REQ-0387.
\end{minipage}
\\ \cdashline{1-2}
Requirement Parameters & {[}\textbf{GhostAF = 1{{[}percent{]}}} Percentage of image area that can
have ghosts with surface brightness gradient amplitude of more than 1/3
of the sky noise over 1 arcsec., \textbf{PF1 = 10{{[}percent{]}}} The
maximum fraction of isolated non-saturated point source measurements
exceeding the outlier limit., \textbf{PA1gri = 5{{[}millimagnitude{]}}}
The RMS photometric repeatability of bright non-saturated unresolved
point sources in the g, r, and i filters., \textbf{PA3 =
10{{[}millimagnitude{]}}} RMS width of internal photometric zero-point
(precision of system uniformity across the sky) for all bands except
u-band., \textbf{PA4 = 15{{[}millimagnitude{]}}} The zero point error
outlier limit., \textbf{PF2 = 10{{[}percent{]}}} Fraction of zeropoint
errors that can exceed the zero point error outlier limit.,
\textbf{PixFrac = 1{{[}percent{]}}} The maximum fraction of pixels
scientifically unusable per sensor out of the total allowable fraction
of sensors meeting this performance., \textbf{PA1uzy =
7.5{{[}millimagnitude{]}}} The RMS photometric repeatability of bright
non-saturated unresolved point sources in the u, z, and y filters.,
\textbf{PA6 = 10{{[}millimagnitude{]}}} Accuracy of the transformation
of the internal LSST photometry to a physical scale (e.g. AB
magnitudes)., \textbf{Xtalk = 3{{[}sigma{]}}} The maximum local
significance integrated over the PSF of imperfect crosstalk
corrections., \textbf{PA5u = 10{{[}millimagnitude{]}}} Accuracy of
absolute band-to-band color zero-point for colors constructed using the
u-band., \textbf{ResSource = 2{{[}unitless{]}}} Maximum RMS of the ratio
of the error in integrated flux measurement between bright, isolated,
resolved sources less than 10 arcsec in diameter and bright, isolated
unresolved point sources., \textbf{PA2uzy = 22.5{{[}millimagnitude{]}}}
Repeatability outlier limit for isolated bright non-saturated point
sources in the u, z, and y filters., \textbf{SensorFraction =
15{{[}percent{]}}} The maximum allowable fraction of sensors with
PixFrac scientifically unusable pixels., \textbf{PA3u =
20{{[}millimagnitude{]}}} RMS width of internal photometric zero-point
(precision of system uniformity across the sky) in the u-band.,
\textbf{PA2gri = 15{{[}millimagnitude{]}}} Repeatability outlier limit
for isolated bright non-saturated point sources in the g, r, and i
filters., \textbf{SBPrec = 1{{[}percent{]}}} The maximum error in the
precision of the sky brightness determination., \textbf{PA5 =
5{{[}millimagnitude{]}}} Accuracy of absolute band-to-band color
zero-point for all colors constructed from any filter pair, excluding
the u-band.{]} \\ \cdashline{1-2}
Requirement Discussion &
\begin{minipage}[]{13cm}
\textbf{Discussion:} The relevant metrics are listed in the table
photometricPerformance below. The values in the tables are the target
values for LSST but are not verified as part of this requirement.
\end{minipage}
\\ \cdashline{1-2}
Requirement Priority & 1a \\ \cdashline{1-2}
Upper Level Requirement &
\begin{tabular}{cl}
OSS-REQ-0387 & Photometric Performance \\
\end{tabular}
\\ \hline
\end{longtable}
}


  
 \newpage 
\subsection{[LVV-18465] DMS-REQ-0395-V-01: Scientific Visualization of Camera Image Data\_1 }\label{lvv-18465}

\begin{longtable}{cccc}
\hline
\textbf{Jira Link} & \textbf{Assignee} & \textbf{Status} & \textbf{Test Cases}\\ \hline
\href{https://jira.lsstcorp.org/browse/LVV-18465}{LVV-18465} &
Leanne Guy & Not Covered &
\begin{tabular}{c}
LVV-T1830 \\
\end{tabular}
\\
\hline
\end{longtable}

\textbf{Verification Element Description:} \\
Undefined

{\footnotesize
\begin{longtable}{p{2.5cm}p{13.5cm}}
\hline
\multicolumn{2}{c}{\textbf{Requirement Details}}\\ \hline
Requirement ID & DMS-REQ-0395 \\ \cdashline{1-2}
Requirement Description &
\begin{minipage}[]{13cm}
\textbf{Specification:} All scientific visualization of camera image
data shall use the coordinate systems defined in \citeds{LSE-349}.
\end{minipage}
\\ \cdashline{1-2}
Requirement Priority & 1a \\ \cdashline{1-2}
Upper Level Requirement &
\begin{tabular}{cl}
OSS-REQ-0408 & Scientific Visualization of Camera Image Data \\
\end{tabular}
\\ \hline
\end{longtable}
}


\subsubsection{Test Cases Summary}
\begin{longtable}{p{3cm}p{2.5cm}p{2.5cm}p{3cm}p{4cm}}
\toprule
\href{https://jira.lsstcorp.org/secure/Tests.jspa\#/testCase/LVV-T1830}{LVV-T1830} & \multicolumn{4}{p{12cm}}{ Verify Implementation of Scientific Visualization of Camera Image Data } \\ \hline
\textbf{Owner} & \textbf{Status} & \textbf{Version} & \textbf{Critical Event} & \textbf{Verification Type} \\ \hline
Jeffrey Carlin & Draft & 1 & false & Inspection \\ \hline
\end{longtable}
{\scriptsize
\textbf{Objective:}\\
Verify that all scientific visualization of camera image data uses the
coordinate systems defined in \href{https://lse-349.lsst.io/}{LSE-349}.
}
  
 \newpage 
\subsection{[LVV-18881] DMS-REQ-0282-V-02: Dark Current Correction Frame Effectiveness }\label{lvv-18881}

\begin{longtable}{cccc}
\hline
\textbf{Jira Link} & \textbf{Assignee} & \textbf{Status} & \textbf{Test Cases}\\ \hline
\href{https://jira.lsstcorp.org/browse/LVV-18881}{LVV-18881} &
Leanne Guy & Not Covered &
\begin{tabular}{c}
LVV-T1862 \\
\end{tabular}
\\
\hline
\end{longtable}

\textbf{Verification Element Description:} \\
Undefined

{\footnotesize
\begin{longtable}{p{2.5cm}p{13.5cm}}
\hline
\multicolumn{2}{c}{\textbf{Requirement Details}}\\ \hline
Requirement ID & DMS-REQ-0282 \\ \cdashline{1-2}
Requirement Description &
\begin{minipage}[]{13cm}
\textbf{Specification:} The DMS shall produce on an as-needed basis a
dark current correction image, which is constructed from multiple,
closed-shutter exposures of appropriate duration. The effectiveness of
the Dark Correction shall be verified in production processing on
science data.
\end{minipage}
\\ \cdashline{1-2}
Requirement Discussion &
\begin{minipage}[]{13cm}
\textbf{Discussion:} The need for a dark current correction will have to
be quantified during Commissioning. Collecting closed-dome dark
exposures may be deemed necessary to monitor the health of the
detectors, even if not used in calibration processing.
\end{minipage}
\\ \cdashline{1-2}
Requirement Priority & 1a \\ \cdashline{1-2}
Upper Level Requirement &
\begin{tabular}{cl}
OSS-REQ-0271 & Supported Image Types \\
OSS-REQ-0046 & Calibration \\
\end{tabular}
\\ \hline
\end{longtable}
}


\subsubsection{Test Cases Summary}
\begin{longtable}{p{3cm}p{2.5cm}p{2.5cm}p{3cm}p{4cm}}
\toprule
\href{https://jira.lsstcorp.org/secure/Tests.jspa\#/testCase/LVV-T1862}{LVV-T1862} & \multicolumn{4}{p{12cm}}{ Verify determining effectiveness of dark current frame } \\ \hline
\textbf{Owner} & \textbf{Status} & \textbf{Version} & \textbf{Critical Event} & \textbf{Verification Type} \\ \hline
Jeffrey Carlin & Draft & 1 & false & Test \\ \hline
\end{longtable}
{\scriptsize
\textbf{Objective:}\\
Verify that the DMS can determine the effectiveness of a dark correction
and determine how often it should be updated.
}
  

\newpage
\appendix
\section{Traceability}
\label{sec:trace}

\begin{longtable}{ccc}
\hline
\textbf{Requirements} & \textbf{Verification Elements} & \textbf{Test Cases} \\ \hline
 DMS-REQ-0002  &
 LVV-3  &
LVV-T101 \\
 &
 &
LVV-T217 \\
\hline
 DMS-REQ-0009  &
 LVV-6  &
LVV-T125 \\
\hline
 DMS-REQ-0010  &
 LVV-7  &
LVV-T18 \\
 &
 &
LVV-T20 \\
 &
 &
LVV-T36 \\
\hline
 DMS-REQ-0018  &
 LVV-8  &
LVV-T29 \\
 &
 &
LVV-T283 \\
 &
 &
LVV-T284 \\
 &
 &
LVV-T1549 \\
 &
 &
LVV-T1550 \\
 &
 &
LVV-T1556 \\
 &
 &
LVV-T1934 \\
\hline
 DMS-REQ-0020  &
 LVV-9  &
LVV-T30 \\
 &
 &
LVV-T283 \\
 &
 &
LVV-T284 \\
 &
 &
LVV-T1549 \\
 &
 &
LVV-T1556 \\
\hline
 DMS-REQ-0024  &
 LVV-11  &
LVV-T32 \\
 &
 &
LVV-T283 \\
 &
 &
LVV-T284 \\
 &
 &
LVV-T1549 \\
 &
 &
LVV-T1550 \\
 &
 &
LVV-T1556 \\
 &
 &
LVV-T1934 \\
\hline
 DMS-REQ-0029  &
 LVV-12  &
LVV-T15 \\
 &
 &
LVV-T19 \\
 &
 &
LVV-T39 \\
\hline
 DMS-REQ-0030  &
 LVV-13  &
LVV-T15 \\
 &
 &
LVV-T19 \\
 &
 &
LVV-T40 \\
 \cdashline{2-3}  &
 LVV-9741  &
LVV-T1240 \\
\hline
 DMS-REQ-0032  &
 LVV-14  &
LVV-T126 \\
\hline
 DMS-REQ-0033  &
 LVV-15  &
LVV-T127 \\
 &
 &
LVV-T362 \\
\hline
 DMS-REQ-0034  &
 LVV-16  &
LVV-T61 \\
\hline
 DMS-REQ-0042  &
 LVV-17  &
LVV-T128 \\
\hline
 DMS-REQ-0043  &
 LVV-18  &
LVV-T21 \\
 &
 &
LVV-T22 \\
 &
 &
LVV-T129 \\
\hline
 DMS-REQ-0046  &
 LVV-19  &
LVV-T68 \\
\hline
 DMS-REQ-0047  &
 LVV-20  &
LVV-T16 \\
 &
 &
LVV-T62 \\
 &
 &
LVV-T62 \\
\hline
 DMS-REQ-0052  &
 LVV-21  &
LVV-T130 \\
\hline
 DMS-REQ-0059  &
 LVV-22  &
LVV-T83 \\
\hline
 DMS-REQ-0060  &
 LVV-23  &
LVV-T84 \\
 &
 &
LVV-T368 \\
 &
 &
LVV-T368 \\
\hline
 DMS-REQ-0061  &
 LVV-24  &
LVV-T85 \\
\hline
 DMS-REQ-0062  &
 LVV-25  &
LVV-T86 \\
\hline
 DMS-REQ-0063  &
 LVV-26  &
LVV-T87 \\
\hline
 DMS-REQ-0065  &
 LVV-27  &
LVV-T134 \\
 DMS-REQ-0065  &
 LVV-27  &
Verified By LVV-10004 \\
 &
 &
Verified By LVV-10016 \\
 &
 &
Verified By LVV-10017 \\
 &
 &
Verified By LVV-10018 \\
\hline
 DMS-REQ-0068  &
 LVV-28  &
LVV-T33 \\
 &
 &
LVV-T283 \\
 &
 &
LVV-T284 \\
 &
 &
LVV-T286 \\
 &
 &
LVV-T1549 \\
 &
 &
LVV-T1550 \\
 &
 &
LVV-T1556 \\
\hline
 DMS-REQ-0069  &
 LVV-29  &
LVV-T15 \\
 &
 &
LVV-T18 \\
 &
 &
LVV-T19 \\
 &
 &
LVV-T38 \\
 &
 &
LVV-T362 \\
\hline
 DMS-REQ-0070  &
 LVV-30  &
LVV-T15 \\
 &
 &
LVV-T19 \\
 &
 &
LVV-T41 \\
\hline
 DMS-REQ-0072  &
 LVV-31  &
LVV-T15 \\
 &
 &
LVV-T19 \\
 &
 &
LVV-T42 \\
\hline
 DMS-REQ-0075  &
 LVV-33  &
LVV-T149 \\
 &
 &
LVV-T1085 \\
 &
 &
LVV-T1086 \\
 &
 &
LVV-T1087 \\
\hline
 DMS-REQ-0096  &
 LVV-38  &
LVV-T103 \\
\hline
 DMS-REQ-0097  &
 LVV-39  &
LVV-T45 \\
\hline
 DMS-REQ-0099  &
 LVV-41  &
LVV-T46 \\
\hline
 DMS-REQ-0101  &
 LVV-43  &
LVV-T47 \\
\hline
 DMS-REQ-0106  &
 LVV-46  &
LVV-T11 \\
 &
 &
LVV-T64 \\
\hline
 DMS-REQ-0120  &
 LVV-48  &
LVV-T118 \\
\hline
 DMS-REQ-0121  &
 LVV-49  &
LVV-T119 \\
\hline
 DMS-REQ-0124  &
 LVV-52  &
LVV-T206 \\
\hline
 DMS-REQ-0130  &
 LVV-57  &
LVV-T88 \\
\hline
 DMS-REQ-0132  &
 LVV-59  &
LVV-T89 \\
\hline
 DMS-REQ-0158  &
 LVV-62  &
LVV-T11 \\
 DMS-REQ-0158  &
 LVV-62  &
Verified By LVV-136 \\
 &
 &
Verified By LVV-137 \\
 &
 &
Verified By LVV-138 \\
\hline
 DMS-REQ-0161  &
 LVV-64  &
LVV-T172 \\
\hline
 DMS-REQ-0265  &
 LVV-96  &
LVV-T34 \\
 &
 &
LVV-T283 \\
 &
 &
LVV-T284 \\
\hline
 DMS-REQ-0266  &
 LVV-97  &
LVV-T48 \\
\hline
 DMS-REQ-0267  &
 LVV-98  &
LVV-T12 \\
 &
 &
LVV-T13 \\
 &
 &
LVV-T65 \\
 &
 &
LVV-T362 \\
\hline
 DMS-REQ-0268  &
 LVV-99  &
LVV-T12 \\
 &
 &
LVV-T66 \\
\hline
 DMS-REQ-0269  &
 LVV-100  &
LVV-T18 \\
 &
 &
LVV-T21 \\
 &
 &
LVV-T49 \\
\hline
 DMS-REQ-0270  &
 LVV-101  &
LVV-T21 \\
 &
 &
LVV-T50 \\
\hline
 DMS-REQ-0271  &
 LVV-102  &
LVV-T18 \\
 &
 &
LVV-T22 \\
 &
 &
LVV-T51 \\
 \cdashline{2-3} DMS-REQ-0271 & LVV-9743 &  \\ \hline
\hline
 DMS-REQ-0272  &
 LVV-103  &
LVV-T22 \\
 &
 &
LVV-T52 \\
 DMS-REQ-0272  &
 LVV-103  &
Verified By LVV-10990 \\
 &
 &
Verified By LVV-10990 \\
\hline
 DMS-REQ-0273  &
 LVV-104  &
LVV-T53 \\
\hline
 DMS-REQ-0274  &
 LVV-105  &
LVV-T54 \\
\hline
 DMS-REQ-0275  &
 LVV-106  &
LVV-T12 \\
 &
 &
LVV-T14 \\
 &
 &
LVV-T67 \\
 DMS-REQ-0275  &
 LVV-106  &
Verified By DM-9953 \\
 &
 &
Verified By DM-13058 \\
\hline
 DMS-REQ-0276  &
 LVV-107  &
LVV-T69 \\
\hline
DMS-REQ-0277 & LVV-108 &  \\ \hline
\hline
 DMS-REQ-0278  &
 LVV-109  &
LVV-T16 \\
 &
 &
LVV-T72 \\
\hline
 DMS-REQ-0279  &
 LVV-110  &
LVV-T12 \\
 &
 &
LVV-T16 \\
 &
 &
LVV-T73 \\
\hline
 DMS-REQ-0280  &
 LVV-111  &
LVV-T74 \\
\hline
 DMS-REQ-0281  &
 LVV-112  &
LVV-T75 \\
\hline
 DMS-REQ-0282  &
 LVV-113  &
LVV-T90 \\
 \cdashline{2-3}  &
 LVV-18881  &
LVV-T1862 \\
\hline
 DMS-REQ-0283  &
 LVV-114  &
LVV-T91 \\
\hline
 DMS-REQ-0285  &
 LVV-116  &
LVV-T22 \\
 &
 &
LVV-T108 \\
 &
 &
LVV-T550 \\
\hline
 DMS-REQ-0286  &
 LVV-117  &
LVV-T109 \\
\hline
 DMS-REQ-0287  &
 LVV-118  &
LVV-T110 \\
 \cdashline{2-3} DMS-REQ-0287 & LVV-9746 &  \\ \hline
 \cdashline{2-3} DMS-REQ-0287 & LVV-9747 &  \\ \hline
\hline
 DMS-REQ-0288  &
 LVV-119  &
LVV-T111 \\
\hline
 DMS-REQ-0289  &
 LVV-120  &
LVV-T115 \\
 &
 &
LVV-T1935 \\
\hline
 DMS-REQ-0290  &
 LVV-121  &
LVV-T122 \\
\hline
 DMS-REQ-0291  &
 LVV-122  &
LVV-T96 \\
\hline
 DMS-REQ-0292  &
 LVV-123  &
LVV-T97 \\
\hline
 DMS-REQ-0293  &
 LVV-124  &
LVV-T11 \\
 &
 &
LVV-T98 \\
\hline
 DMS-REQ-0294  &
 LVV-125  &
LVV-T12 \\
 &
 &
LVV-T99 \\
\hline
 DMS-REQ-0296  &
 LVV-127  &
LVV-T132 \\
 &
 &
LVV-T362 \\
\hline
 DMS-REQ-0299  &
 LVV-130  &
LVV-T137 \\
 &
 &
LVV-T374 \\
 &
 &
LVV-T1934 \\
 &
 &
LVV-T1935 \\
\hline
 DMS-REQ-0301  &
 LVV-132  &
LVV-T147 \\
\hline
 DMS-REQ-0307  &
 LVV-138  &
LVV-T148 \\
\hline
 DMS-REQ-0311  &
 LVV-142  &
LVV-T156 \\
\hline
 DMS-REQ-0317  &
 LVV-148  &
LVV-T55 \\
\hline
 DMS-REQ-0319  &
 LVV-150  &
LVV-T56 \\
\hline
 DMS-REQ-0321  &
 LVV-152  &
LVV-T93 \\
\hline
 DMS-REQ-0323  &
 LVV-154  &
LVV-T57 \\
\hline
 DMS-REQ-0324  &
 LVV-155  &
LVV-T58 \\
\hline
 DMS-REQ-0325  &
 LVV-156  &
LVV-T59 \\
\hline
 DMS-REQ-0326  &
 LVV-157  &
LVV-T23 \\
\hline
 DMS-REQ-0327  &
 LVV-158  &
LVV-T15 \\
 &
 &
LVV-T19 \\
 &
 &
LVV-T43 \\
\hline
 DMS-REQ-0328  &
 LVV-159  &
LVV-T44 \\
\hline
 DMS-REQ-0329  &
 LVV-160  &
LVV-T76 \\
\hline
 DMS-REQ-0330  &
 LVV-161  &
LVV-T77 \\
\hline
 DMS-REQ-0331  &
 LVV-162  &
LVV-T13 \\
 &
 &
LVV-T14 \\
 &
 &
LVV-T21 \\
 &
 &
LVV-T22 \\
 &
 &
LVV-T24 \\
 DMS-REQ-0331  &
 LVV-162  &
Verified By DM-9953 \\
 &
 &
Verified By DM-13058 \\
\hline
 DMS-REQ-0332  &
 LVV-163  &
LVV-T25 \\
\hline
 DMS-REQ-0333  &
 LVV-164  &
LVV-T26 \\
\hline
 DMS-REQ-0335  &
 LVV-166  &
LVV-T79 \\
\hline
 DMS-REQ-0336  &
 LVV-167  &
LVV-T159 \\
\hline
 DMS-REQ-0337  &
 LVV-168  &
LVV-T80 \\
\hline
 DMS-REQ-0338  &
 LVV-169  &
LVV-T81 \\
\hline
 DMS-REQ-0339  &
 LVV-170  &
LVV-T82 \\
\hline
 DMS-REQ-0347  &
 LVV-178  &
LVV-T13 \\
 &
 &
LVV-T14 \\
 &
 &
LVV-T21 \\
 &
 &
LVV-T22 \\
 &
 &
LVV-T28 \\
 &
 &
LVV-T1946 \\
 &
 &
LVV-T1947 \\
 DMS-REQ-0347  &
 LVV-178  &
Verified By DM-9953 \\
 &
 &
Verified By DM-13058 \\
\hline
 DMS-REQ-0348  &
 LVV-179  &
LVV-T114 \\
 &
 &
LVV-T218 \\
\hline
 DMS-REQ-0349  &
 LVV-180  &
LVV-T71 \\
\hline
 DMS-REQ-0350  &
 LVV-181  &
LVV-T116 \\
\hline
 DMS-REQ-0351  &
 LVV-182  &
LVV-T133 \\
\hline
DMS-REQ-0378 & LVV-3399 &  \\ \hline
\hline
 DMS-REQ-0359  &
 LVV-3401  &
LVV-T1756 \\
 \cdashline{2-3}  &
 LVV-9751  &
LVV-T377 \\
 &
 &
LVV-T1847 \\
 \cdashline{2-3}  &
 LVV-9752  &
LVV-T1758 \\
 &
 &
LVV-T1759 \\
 \cdashline{2-3}  &
 LVV-9753  &
LVV-T377 \\
 &
 &
LVV-T1846 \\
 \cdashline{2-3}  &
 LVV-9754  &
LVV-T1759 \\
 \cdashline{2-3}  &
 LVV-9755  &
LVV-T377 \\
 &
 &
LVV-T1845 \\
 \cdashline{2-3}  &
 LVV-9756  &
LVV-T377 \\
 &
 &
LVV-T1844 \\
 \cdashline{2-3}  &
 LVV-9757  &
LVV-T377 \\
 &
 &
LVV-T1843 \\
 \cdashline{2-3}  &
 LVV-9758  &
LVV-T1758 \\
 \cdashline{2-3}  &
 LVV-9759  &
LVV-T1757 \\
 \cdashline{2-3}  &
 LVV-9760  &
LVV-T377 \\
 &
 &
LVV-T1842 \\
 \cdashline{2-3}  &
 LVV-9761  &
LVV-T377 \\
 &
 &
LVV-T1841 \\
 \cdashline{2-3}  &
 LVV-9762  &
LVV-T377 \\
 &
 &
LVV-T1840 \\
 \cdashline{2-3}  &
 LVV-9763  &
LVV-T377 \\
 &
 &
LVV-T1839 \\
 \cdashline{2-3}  &
 LVV-9764  &
LVV-T377 \\
 &
 &
LVV-T1838 \\
 \cdashline{2-3}  &
 LVV-9765  &
LVV-T377 \\
 &
 &
LVV-T1837 \\
 \cdashline{2-3}  &
 LVV-9766  &
LVV-T377 \\
 &
 &
LVV-T1836 \\
 \cdashline{2-3} DMS-REQ-0359 & LVV-18339 &  \\ \hline
\hline
 DMS-REQ-0360  &
 LVV-3402  &
LVV-T363 \\
 &
 &
LVV-T1745 \\
 \cdashline{2-3}  &
 LVV-9767  &
LVV-T378 \\
 &
 &
LVV-T1746 \\
 \cdashline{2-3}  &
 LVV-9768  &
LVV-T378 \\
 &
 &
LVV-T1747 \\
 \cdashline{2-3}  &
 LVV-9769  &
LVV-T378 \\
 &
 &
LVV-T1748 \\
 \cdashline{2-3}  &
 LVV-9770  &
LVV-T378 \\
 &
 &
LVV-T1749 \\
 \cdashline{2-3}  &
 LVV-9771  &
LVV-T378 \\
 &
 &
LVV-T1750 \\
 \cdashline{2-3}  &
 LVV-9773  &
LVV-T378 \\
 &
 &
LVV-T1746 \\
 \cdashline{2-3}  &
 LVV-9774  &
LVV-T378 \\
 &
 &
LVV-T1751 \\
 \cdashline{2-3}  &
 LVV-9775  &
LVV-T378 \\
 \cdashline{2-3}  &
 LVV-9776  &
LVV-T378 \\
 &
 &
LVV-T1749 \\
 \cdashline{2-3}  &
 LVV-9777  &
LVV-T378 \\
 &
 &
LVV-T1750 \\
 \cdashline{2-3}  &
 LVV-9778  &
LVV-T378 \\
 &
 &
LVV-T1753 \\
 \cdashline{2-3}  &
 LVV-9779  &
LVV-T378 \\
 &
 &
LVV-T1752 \\
\hline
 DMS-REQ-0362  &
 LVV-3404  &
LVV-T376 \\
 &
 &
LVV-T1754 \\
 \cdashline{2-3}  &
 LVV-9780  &
LVV-T376 \\
 \cdashline{2-3} DMS-REQ-0362 & LVV-9781 &  \\ \hline
 \cdashline{2-3}  &
 LVV-9782  &
LVV-T1755 \\
 \cdashline{2-3} DMS-REQ-0362 & LVV-9783 &  \\ \hline
\hline
DM-TS-CON-ICD-0011 & LVV-5640 &  \\ \hline
 \cdashline{2-3} DM-TS-CON-ICD-0011 & LVV-5641 &  \\ \hline
\hline
DM-TS-CON-ICD-0002 & LVV-5646 &  \\ \hline
 \cdashline{2-3} DM-TS-CON-ICD-0002 & LVV-5647 &  \\ \hline
\hline
 DMS-REQ-0131  &
 LVV-9745  &
LVV-T1277 \\
\hline
 DMS-PRTL-REQ-0025  &
 LVV-9867  &
LVV-T660 \\
\hline
 DMS-PRTL-REQ-0026  &
 LVV-9869  &
LVV-T5 \\
 &
 &
LVV-T661 \\
 &
 &
LVV-T1334 \\
\hline
 DMS-PRTL-REQ-0085  &
 LVV-9926  &
LVV-T719 \\
\hline
 DMS-PRTL-REQ-0097  &
 LVV-9937  &
LVV-T731 \\
\hline
 DMS-REQ-0382  &
 LVV-18225  &
LVV-T1527 \\
\hline
 DMS-REQ-0379  &
 LVV-18227  &
LVV-T1529 \\
\hline
 DMS-REQ-0383  &
 LVV-18228  &
LVV-T1530 \\
\hline
 DMS-REQ-0344  &
 LVV-18229  &
LVV-T1865 \\
\hline
 DMS-REQ-0390  &
 LVV-18233  &
LVV-T1563 \\
\hline
 DMS-REQ-0389  &
 LVV-18234  &
LVV-T1564 \\
\hline
 DMS-REQ-0394  &
 LVV-18295  &
LVV-T1831 \\
\hline
 DMS-REQ-0392  &
 LVV-18298  &
LVV-T2091 \\
\hline
 DMS-REQ-0393  &
 LVV-18299  &
LVV-T2097 \\
\hline
 DMS-REQ-0395  &
 LVV-18465  &
LVV-T1830 \\
\hline
\end{longtable}

Note that some of the requirements listed in this traceability table may be related with additional
Verification Elements not in the scope of \textit{ Science } Verification,
and therefore not listed here.
